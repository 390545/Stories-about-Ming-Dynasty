\fancyhead[LO]{{\scriptsize {\FA \ }明朝那些事儿 {\FA } \textbf{\rightmark}}}%奇數頁眉的左邊
\fancyhead[RO]{{\tiny{\textcolor{Gray}{\FA \ }}}\thepage}
\fancyhead[LE]{{\tiny{\textcolor{Gray}{\FA \ }}}\thepage}
\fancyhead[RE]{{\scriptsize {\FA \ }明朝那些事儿 {\FA } \textbf{\rightmark}}}%偶數頁眉的右邊
\fancyfoot[LE,RO]{}
\fancyfoot[LO,CE]{}
\fancyfoot[CO,RE]{}
\setcounter{part}{6}
\setcounter{chapter}{7}
\setcounter{section}{0}
\part{大结局}
\chapter*{朱由校篇}
\addcontentsline{toc}{chapter}{朱由校篇}
\section{皇太极}
\ifnum\theparacolNo=2
	\begin{multicols}{\theparacolNo}
\fi
失败的努尔哈赤悲愤了几个月后,终于笑了——含笑九泉。\\

老头笑着走了,有些人就笑不出来了——比如他的几个儿子。\\

当时,具备继承资格的人,有八个。\\

这八个人分别是四大贝勒:代善、阿敏、莽古尔泰、皇太极;\\

四小贝勒:阿济格、多尔衮、济尔哈朗、多铎。\\

位置只有一个。\\

拜许多“秘史”类电视剧所赐,这个连史学研究者都未必重视的问题,竟然妇孺皆知,且说法众多,什么努尔哈赤讨厌皇太极,喜欢多尔衮,皇太极使坏,干掉了多尔衮他妈,抢了多尔衮的汗位等等等等。\\

以上讲法,在菜市场等地遇熟人时随便说说,是可以的,正式场合,就别扯了。\\

事实上,打努尔哈赤含笑那天起,汗位就已注定,它只属于一个人——皇太极。\\

因为除这位仁兄外,别人都有问题。\\

努尔哈赤确实很喜欢多尔衮,可是问题在于,多尔衮同志当时还是小屁孩,游牧民族比较实在,谁更能打、更能抢,谁就是老大,要搞任人唯亲,广大后金人民是不答应的。\\

四小贝勒里的其他三人,那更别提了,年龄小不说,老头还不待见,以上四人可以全部淘汰。\\

而四大贝勒里,阿敏是努尔哈赤的侄子,没资格,排除;莽古尔泰比较蠢,性情暴躁,排除;能排上号的,只有代善和皇太极。\\

但是代善也有问题——生活作风,这个问题还相当麻烦,因为据说和他传绯闻的,是努尔哈赤的后妃。\\

代善是聪明人,有这个前科,汗位是不敢指望了,他相当宽容地表示,自己就不争这个位置了,让皇太极干吧。\\

于是,在众人的一致推举下,天启六年(1626)九月初一,皇太极登基。\\

在后金将领中,论军事天赋,能与袁崇焕相比的,只有三个人:努尔哈赤、代善、皇太极(多尔衮比较小,不算)。\\

但要论政治水平,能摆上台面的,只有皇太极。\\

因为一个月后,他做了一件努尔哈赤绝不可能做到的事。\\

天启六年(1626)十月,袁崇焕代表团来到了后金首都沈阳,他们此来的目的是吊丧,同时祝贺皇太极上任。\\

在很多书籍里,宁远战役后的袁崇焕是很悲惨的,战绩无人认可,也没有封赏,所有的功劳都被魏忠贤抢走,孤苦伶仃,悲惨世界。\\

可以肯定的是,这些说法是未经史籍,也未经大脑的,因为就在宁远胜利后的几天,袁崇焕就得到了皇帝的表扬,兵部尚书王永光跟袁崇焕不大对劲,也大发感慨:\\

八年来贼始一挫,乃知中国有人矣!\\

总之,捷报传来,全国欢腾,唯一不欢腾的人,就是高第。\\

这位兄弟实在太不争气,所以连阉党都不保他,被干净利落地革职赶回了家。\\

除口头表扬外,明朝也相当实在,正月底打胜,2月初就提了,先是都察院右佥都御史,一个月后又加辽东巡抚,然后是兵部右侍郎,两个月内就到了副部级。\\

部下们也没有白干,满桂、赵率教、左辅、朱梅、祖大寿都升了官,连他的孙承宗老师也论功行赏了。\\

当然,领导的功劳是少不了的,比如魏忠贤公公,顾秉谦大人等等,虽说没去打仗,但整日忙着阴人,也是很辛苦的。\\

无论如何,袁崇焕出头了,虽说他是孙承宗的学生,东林党的成员,但边界得有人守吧,所以阉党不难为他,反正好人坏人都不管他,任他在那倒腾。\\

几个月后,得知努尔哈赤死讯后,他派出了代表团。\\

这就倒腾大了。\\

在明朝看来,后金就是以努尔哈赤为首的强盗团伙,压根不是政权,堂堂天朝怎么能和团伙头目谈判呢?\\

所以多年以来,都是只打不谈。\\

但问题是,打来打去都没个结果,正好这次把团伙头目憋屈死了,趁机去谈谈,也没坏处。\\

当然,作为一名文官出身的将领,袁崇焕还有点政治头脑,谈判之前,先请示了皇帝,才敢开路。\\

憋死(打伤致死)了人家老爹,还派人来吊丧,是很不地道的,如此行径,是让人难以忍受的。\\

然而皇太极忍了。\\

他不但忍了,还作出了出人意料的回应。\\

他用最高标准接待了袁崇焕的使者,好吃好喝招待,还搞了个阅兵式,让他们玩了一个多月,走的时候还送了几匹马、几十只羊,并热情地向自己杀父仇人的使者微笑挥手告别。\\

这意味着,一个比努尔哈赤更为可怕的敌人出现了。\\

懂得暴力的人,是强壮的,懂得克制暴力的人,才是强大的。\\

在下次战争到来之前,必须和平,这就是皇太极的真实想法。\\

袁崇焕也并非善类,对于这次谈判,他在给皇帝的报告中,做出了充分的解释:\\

“奴死之耗,与奴子情形,我已备得,尚复何求?”\\

这句话的意思是,努尔哈赤的死讯,他儿子的情况,我都知道了,还有什么要求呢?\\

谈来谈去,就谈出了这么个玩意。\\

谈判还是继续,到第二年(天启七年)正月,皇太极又派人来了。\\

可这人明显不上道,谈判书上还附了一篇文章——当年他爹写的七大恨。\\

但你要说皇太极有多恨,似乎也说不上,因为,就在七大恨后面,他还列上了谈判的条件,比如金银财宝,比如土地等等。\\

也就是想多要点东西嘛,还把死去的老爷子搬出来,实在辛苦。\\

袁崇焕是很幽默的,他在回信中,很有耐心地逐条批驳了努尔哈赤的著作,同时表示,拒绝你的一切要求。这意思是,虽然你爸憋屈死了,我表示同情,但谈归谈,死人我也不买账。\\

过了一月,皇太极又来信了,这哥们明显是玩上瘾了,他竟把袁崇焕批驳七大恨的理由,又逐条批驳了一次,当然正事他也没忘了谈,这次他的胃口小了点,要的东西也减了半。\\

文字游戏玩玩是可以的,但具体工作还要干,在这一点上,皇太极同志的表现相当不错,就在给袁崇焕送信的同时,他发动了新的进攻,目标是朝鲜。\\

天启七年(1627)正月初八,阿敏出兵朝鲜,朝军的表现相当稳定,依然是一如以往地不经打,一个月后平壤就失陷了,再过一个月,朝鲜国王就签了结盟书,表示愿意服从后金。\\

朝鲜失陷,明朝是不高兴的,但不高兴也没办法,今天不同往日了,家里比较困难,实在没法拉兄弟一把,失陷,就失陷了吧。\\

一边谈判,一边干这种事,实在太过分了,所以在来往的文书中,袁崇焕愤怒地谴责了对方的行径,痛斥皇太极没有谈判的诚意。\\

话这么说,袁崇焕也没闲着,他也很忙,忙着砌砖头。\\

自打宁远之战结束后,他就开始修墙了,打坏的重砌,没坏的加固。他还把几万民工直接拉到锦州,抢工期抓进度,短短几个月,锦州再度成为坚城。\\

此外,他还重新占领了之前放弃的大凌河、前屯、中后所、中右所,修筑堡垒,全面恢复关宁防线。\\

光修墙是不够的,为把皇太极彻底恶心死,他大量召集农民,只要来人就分地,一文钱都不要,白送,开始大规模屯田,积累军粮。\\

一边谈判,一边干这种事,实在太过分了,所以在来往的文书中,皇太极愤怒地谴责了对方的行径,痛斥袁崇焕没有谈判的诚意。\\

到了天启七年(1627)五月,老头子的身后事办完了,朝鲜打下来了,锦州修起来了,防线都恢复了,屯田差不多了,双方都满意了。\\

打吧。\\

天启七年(1627)五月六日,皇太极率六万大军,自沈阳出发,进攻锦州,“宁锦大战”就此揭开序幕。\\

此时出战,并非皇太极的本意,老头子才挂了几个月,遗产分割、追悼会刚刚搞完,朝鲜又打了仗,实在不是进攻的好时候,但没办法,不打不行——家里闹灾荒了。\\

天启七年,辽东受了天灾,袁崇焕和皇太极都遭了灾,紧缺粮食。\\

为解决粮食问题,袁崇焕决定,去关内调粮,补充军需。\\

为解决粮食问题,皇太极决定,去关内抢粮,补充军需。\\

没办法,吃不上饭啊,又没处调粮食,眼看着要闹事,与其闹腾我不如闹你们,索性就带他们去抢吧。\\

对于皇太极的这个打算,袁崇焕是有思想准备的,所以他擦亮了大炮,备齐了炮弹,静静等待着后金抢粮队到来。\\

宁远之战后,袁崇焕顺风顺水,官也升了,权也大了,声势如日中天,威信很高,属下十分服气。\\

不服气的人也是有的,比如满桂。\\

其实满桂和袁崇焕的关系是不错的,他之所以不服气,是因为另一个人——赵率教。\\

在宁远之战时,赵率教驻守前屯,打得最激烈的时候,满桂感觉要撑不住了,就派人给赵率教传令,让他赶紧派人增援。\\

可赵率教不去。\\

因为你吃不消,我也吃不消,一共这么多人,你的兵比我还多,谁增援谁?\\

所以不去。\\

当时情况危急,满桂倒也没有计较,仗打完了,想起这茬了,回头要跟赵率教算帐。\\

于是袁崇焕出场了,现在他是辽东巡抚,遇到这种事情,自然是要和稀泥的。\\

但是他万没有想到,这把稀泥非但没有和成,还把自己给和进去了。\\

因为满桂根本不买账,非但不肯了事,还把袁崇焕拉下了水,说他拉偏架。\\

原因在于,宁远之战前,满桂是宁远总兵,袁崇焕,是宁前道。满桂的级别比袁崇焕高,但根据以文制武惯例,袁崇焕的地位要略高于满桂。\\

战后,满桂升到了右都督,袁崇焕升到兵部侍郎兼辽东巡抚,按级别,袁崇焕依然不如满桂,但论地位,他依然比满桂高。\\

这就相当麻烦了,要知道,满桂光打仗就打了二三十年,他砍人头攒钱(一个五十两)的时候,袁举人还在考进士,且他级别一直比袁崇焕高,现在又是一品武官,你个三品文官,我服从管理就不错了,瞎搅和什么?\\

外加他又是蒙古人,为人比较直爽,毫不虚伪,说打,操家伙就上。至于袁崇焕,他本人曾自我介绍过:“你道本部院是个书生,本部院却是个将首!”\\

于是来来往往,火花四射,袁崇焕随即表示,满桂才堪大用,希望朝廷加以重用(随你怎么用,不要在这儿用)。\\

满桂气得不行,又干不过袁崇焕(巡抚有实权),就告到了袁崇焕的上司,新任辽东督师王之臣那里。\\

王之臣也是文官,所以也和稀泥,表示满桂也是个人才,你们都消停吧,都在关外为国效力。\\

按说和稀泥也就行了,但王督师似乎不甘寂寞,顺道还训袁崇焕几句,于是袁大人也火了,当即上书表示自己很累,要退休(乞休)。\\

王督师顿时火冒三丈,也上了奏疏,说自己要引退(引避)。\\

问题闹大了,朝廷亲自出马,使出了杀手锏——还是和稀泥。\\

但朝廷毕竟是朝廷,这把稀泥的质量十分之高。\\

先是下了封文书,给两人上了堂历史课,说此前经抚不和(指熊廷弼和王化贞),丢掉很多地方,你们要吸取教训,不要再闹了。\\

然后表示,你们两个都是人才,都不要走,但为防你们两个在一起会互相死磕,特划定范围,王之臣管关内,袁崇焕管关外,有功一起赏,有黑锅也一起背,舒坦了吧!\\

命令下来后,袁崇焕和王之臣都相当识趣,当即做出反应,表示愿意留任,并且同意满桂留任,继续共同工作。\\

不久之后,袁崇焕任命满桂镇守山海关,风波就此平息——至少他自己这样认为。\\

然而这件小事,最终也影响了他的命运。\\

但不管有什么后遗症,至少在当时,形势是很好的,一片大好。\\

满桂守山海关,袁崇焕守宁远、锦州,所有的堡垒都已修复完毕,所有的城墙都已加固,弹药充足,粮草齐备,剩下的只有一件事——张开怀抱等你。\\

五月十一日,皇太极一头扎进了怀抱。\\

他的六万大军分为三路,中路由他亲率,左路指挥莽古尔泰,右路指挥代善、阿敏,于同日在锦州城下会师,完成合围。\\

消息传到宁远城的时候,袁崇焕慌张了。他虽然做好了准备,预料到了进攻,却没有料到,会来得这么快。\\

锦州城的守将是赵率教。\\

袁崇焕尚且没有准备,赵率教就不用说了,看城下黑压压一片,实在有点心虚,思考片刻后,他镇定下来,派两个人爬出城墙(不能开门),去找皇太极谈判。\\

这两个人的到来把皇太极彻底搞迷糊了,老子兵都到城下了,你要么就打,要么投降,谈什么判?\\

但愿意谈判,也不是坏事,他随即写了封回信,希望赵率教早日出城投降,奔向光明。\\

使者拿着书信回去了,皇太极就此开始了等待,下午没信,晚上没信,到了第二天,还是没信。\\

于是他向城头瞭望,看到明军在抢修防御工事。\\

这场战役中,赵率教是比较无辜的,其实他压根就不是锦州守将,只不过是恰好呆在那里,等守将到任,就该走人了,没想到皇太极来得太突然,没来得及走,被围在锦州了。四下一打量,官最大的也就是自己了,无可奈何,锦州守将赵率教就此出场。\\

但细一分析,问题来了,辽东兵力总共有十多万,山海关有五万人,宁远有四万人,锦州只有一两万,兵力不足且不说,连出门求援的人都还没到宁远,怎么能开打呢?\\

所以他决定,派人出城谈判,跟皇太极玩太极。\\

皇太极果然名不副实,对太极一窍不通,白等了一天,到五月十三日,想明白了,攻城。\\

六万后金军集结完毕,锣鼓喧天,鞭炮齐鸣,军旗招展,人山人海,等待着皇太极的指令。\\

皇太极沉默片刻,终于下达了指令:停止进攻。\\

皇太极是一个不折不扣的好汉,好汉是不吃眼前亏的。\\

面对着城头黑洞洞的大炮,他决定,暂不进攻——谈判。\\

他主动派出使者,要求城内守军投降,第一次没人理他,第二次也没人理,到第三批使者的时候,赵率教估计是烦得不行,就站到城头,对准下面一声大吼:\\

“要打就打,光说不顶用!(可攻不可说也)”\\

皇太极知道,忽悠是不行了,只能硬拼,后金军随即蜂拥而上,攻击城池。\\

但宁远战役的后遗症实在太过严重,后金军看见大炮就眼晕,没敢玩命,冲了几次就退了,任上级骂遍三代亲属,就是不动。\\

皇太极急了,于是他坐了下来,写了一封劝降信,派人送到城门口,被射死了,又写一封,再让人去送,没人送。\\

无奈之下,他派人把这封劝降信射进了城里,毫无回音。\\

傻子都明白,你压根就攻不下来,你攻不下来,我干嘛投降?\\

但皇太极似乎不明白这个道理,第二天,他又派了几批使者到锦州城谈判,皇天不负有心人,终于有了回应,守军说,你要谈判,使者是不算数的,必须派使臣来,才算正规。\\

皇太极欣喜若狂,连忙选了两个人,准备进城谈判。\\

可是这两位仁兄走到门口,原本说好开门的,偏偏不开,向上喊话,又没人答应,总而言之无人理会,只好打转回家。\\

皇太极很愤怒,因为他被人涮了,但问题是,涮了他,他也没办法。\\

皇太极度过了失望的一天,而即将到来的第二天,却会让他绝望。\\

清晨,正当皇太极准备动员军队攻城的时候,城内的使者来了,不但来了,还解释了昨天没开门的原因:不是我们不热情,实在天色太晚,不方便开门,您多见谅,今天白天再派人来,我们一定接待。\\

皇太极很高兴,又派出了使臣,可是到了城下,明军依然不给开门。\\

这批使臣还比较负责,赖在城下就不走了,于是过了一会,赵率教又出来喊了一嗓子:\\

“你们退兵吧,我大明给赏钱!(自有赏)”\\

就在皇太极被弄得几乎精神失常,气急败坏的时候,城内突然又派出了使者,表示谈可以,但不能到城里,愿意到皇太极的大营去谈判。\\

差点被整疯的皇太极接待了使者,并且写下了一封十分有趣的书信。\\

这封书信并不是劝降信,而是挑战信,他在信中表示,你们龟缩在城里,不是好汉,有种就出来打,你们出一千人,我这里只出十个人,谁打赢了,谁就算胜。你要是敢,咱们就打,要是不敢,就献出城内的所有财物,我就退兵。\\

所谓一千人打不过十个人,比如一千个手无寸铁的傻子打不过十个拿机枪的特种兵,一千个平民打不过十个超人,都是很可能的。\\

在这点上,皇太极体现出游牧民族的狡猾,联系到他爹喜欢玩阴的,这个提议的真正目的,不过是引明军出战。\\

但书信送入城后,却迟迟没有反应,连平时出来吼一嗓子的赵率教也没有踪影,无人搭理。\\

究其原因,还是招数太低级,这种摆明从《三国演义》上抄来的所谓激将法(《三国演义》是后金将领的标准兵书,人手一本),只有在《三国演义》上才能用。\\

皇太极崩溃了,要么就打,要么就谈,要谈又不给开门,送信你又不回,你他娘到底想怎么样?\\

其实赵率教是有苦衷的,他本不想耍皇太极玩,可是无奈,谁让你来这么早,搞得老子走也走不掉,投降又说不过去,只好等援兵了,可是空等实在不太像话,闲来无事谈谈判,当作消遣仅此而已。\\

正月十六日,消遣结束,因为就在这一天,援兵到达锦州。\\

得到锦州被围的消息后,袁崇焕十分焦急,他随即调派兵力,由满桂率领,前往锦州会战。援军的数量很少,只有一万人。\\

六年前,在辽阳战役中,守将袁应泰以五万明军,列队城外,与数量少于自己的后金军决战,结果一塌糊涂,连自己都搭了进去。\\

六年后,满桂带一万人,去锦州打六万后金军。\\

他毫无畏惧,因为他所率领的,是辽东最为精锐的部队——关宁铁骑。\\

经过几年不懈的努力,这支由辽人为主的骑兵训练有素,并配备精良的多管火器,作战极为勇猛,具有极强的冲击力,成为明末最强悍的武装力量。\\

在满桂带领下,关宁铁骑日夜兼程,于十六日抵达塔山附近的笊篱山。\\

按照战前的部署,援军应赶到锦州附近,判明形势发动突袭,击破包围。\\

然而这个构想被无情地打破了,因为就在那天,一位后金将领正在笊篱山巡视——莽古尔泰。\\

这次偶遇完全打乱了双方的计划,片刻惊讶后,满桂率先发动冲锋。\\

后金军毫无提防,前锋被击溃,莽古尔泰虽说比较蠢,打仗还算凑合,很快反应过来,倚仗人多,发动了反击,你来我往几个回合,不打了。\\

因为大家都很忙,莽古尔泰来巡视,差不多也该回去了,满桂来解围,但按目前形势,自己不被围进去就算不错,所以在短暂接触后,双方撤退,各回各家。\\

几乎就在满桂受挫的同一时刻,袁崇焕使出了新的招数。\\

他写好了一封信,并派人秘密送往锦州城,交给赵率教。\\

然而不幸的是,这封信被后金军半路截获,并送到了皇太极的手中。\\

信的内容,让皇太极极为震惊:\\

“锦州被围,但我已调集水师援军以及山海关、宣府等地军队,全部至宁远集结,蒙古援军也即将到来,合计七万余人,耐心等待,必可里应外合,击破包围。”\\

至此,皇太极终于知道了袁崇焕的战略,确切地说,是诡计。\\

锦州被围,援军就这么多,所以只能忽悠,但辽东总共就这么多人,大家心知肚明,所以忽悠必须从外地着手,什么宣府兵、蒙古兵等等,你说多少就多少,在这点上,袁崇焕干得相当好,因为皇太极信了。\\

五月十七日,他更改了部署。\\

三分之一的后金军撤除包围,在外城驻防,因为据“可靠情报”,来自全国四面八方(蒙古、宣府等)的援军,过几天就到。\\

六万人都没戏,剩下这四万就可以休息了,在明军的大炮面前,后金军除了尸体,没有任何收获。\\

第二天,皇太极再次停止了进攻。\\

他又写了封信,用箭射入锦州,再次劝降。\\

对于他的这一举动,我也无语,明知不可能的事,还要几次三番去做,且乐此不疲,到底什么心态,实在难以理解。\\

估计城内的赵率教也被他搞烦了,原本还出来骂几嗓子,现在也不动弹了,连忽悠都懒得忽悠他。\\

五月十九日,皇太极确信,自己上当了。\\

很明显,除了三天前和莽古尔泰交战的那拨人外,再也没有任何援兵。\\

但问题是,锦州还是攻不下来,即使皇太极写信写到手软,射箭射到眼花,还是攻不下来。\\

这样的失败是不能被接受的,所以皇太极决定,改变计划,攻击第二目标。\\

但在此之前,他打算再试一次。\\

五月二十日,后金军发动了最后的猛攻。\\

在这几天里,日程是大致相同的,进攻,大炮,点火,轰隆,死人,撤走,抬尸体,火化,再进攻,再大炮,再点火,再轰隆,再死人,以此类推。\\

五月二十五日,皇太极再也无法忍受,使出最后的杀手锏——撤退。\\

但他的撤退相当有特点,因为他撤退的方向,不是向后,而是向前。\\

他决定越过锦州,前往宁远,因为宁远,就是他的第二攻击目标。\\

经过审慎的思考,皇太极正确地认识到,自己面对的,是一条严密的防线,锦州不过是这条防线上的一点。\\

所有的防线,都有核心,要彻底攻破它,必须找到这个核心——宁远。\\

只要攻破宁远,就能彻底切断锦州与关内的联系,明军将永远地失去辽东。\\

皇太极决定孤注一掷,派遣少量兵力监视锦州,率大队人马直扑宁远,他坚信,自己将在那里迎来辉煌的胜利。\\
\ifnum\theparacolNo=2
	\end{multicols}
\fi
\newpage
\section{宁远决战}
\ifnum\theparacolNo=2
	\begin{multicols}{\theparacolNo}
\fi
五月二十八日,皇太极抵达宁远。\\

一年前,他的父亲在这里倒下,现在,他将在这里再次站立起来——反正他自己是这么想的。\\

然而当他靠近宁远城的时候,却看见了一幕奇特的场景。\\

按照惯例,进攻是这样开始的,明军守在城头,架设大炮,后金军架好营帐,准备云梯、弓箭,然后开始攻城。\\

但这一次,他看到的,是整齐的明军——站在城外。\\

总兵孙祖寿率军,驻守西门,满桂、祖大寿率军,驻守西门,其余兵力驻守南、北方向。宁远守军共三万五千余人,位列城外,准备迎战。\\

现在的袁崇焕,是一个很有自信的人,他相信,凭借自己的实力,可以击败纵横天下的后金骑兵,不用龟缩城内,不用固守城池,击败他们,就在他们的面前,用他们自己的方式!\\

皇太极的神经被彻底搞乱了,这个阵势已经超越了他的理解能力,于是他下达命令,暂停进攻,等等看看先。\\

看了半天,他明白了——这是挑衅,随即发出了怒吼:\\

“当年皇考太祖(努尔哈赤)攻击宁远,没有攻克,今天我打锦州,又没攻克,现在敌人在外布阵,如果还不能胜,我国威何存?!”\\

皇太极认为,不打太没面子,必须且一定要打,但有人认为,不能打。\\

所谓有人,是指大贝勒代善、二贝勒阿敏、三贝勒莽古尔泰。换句话说,四大贝勒里,三个都不同意。\\

虽说皇太极是拍板的,但毕竟是少数派,双方陷入僵持。\\

于是皇太极说,你们都回去吧,我再考虑考虑。\\

三个人撤了,然而没过多久,他们就听见了进攻的号角。\\

对这三位大哥级人物,皇太极还是给面子的:至少把他们忽悠走了再动手。\\

一向只敢躲在城里打炮的明军,竟然站出来单干,实在太嚣张了,他再也无法遏制自己的愤怒,率全军发动了总攻。\\

很多时候,愤怒者往往是弱者。\\

三位贝勒毫无提防,事已至此,只能跟着冲了。\\

但当他们冲到城边时,才终于发现,明军敢来单干,是有原因的。\\

皇太极发动进攻,是打过算盘的,骑兵作战,明军不是后金军的对手,放弃拿手的大炮,偏要打马战,不占这个便宜实在不好意思。\\

袁崇焕之所以摆这个阵势,是因为他认定,关宁铁骑的战斗力,足以与后金骑兵抗衡,但更重要的是,他也没说不用大炮。\\

皇太极认为,当双方骑兵交战时,城头的大炮是无法发射的,因为那样可能误伤自己的军队。\\

袁崇焕知道这一点,但他认为,大炮是可以发射的,具体使用方法是,双方骑兵展开厮杀时,用大炮轰后金的后继部队。\\

换句话说,就是引诱皇太极的骑兵进攻,等上钩的人差不多了,就用大炮攻击他们的后队,截断增援,始终保持人多打人少。\\

在大炮的轰鸣声中,满桂率领骑兵,向蜂拥前来的后金军发动了冲锋。\\

一直以来,在后金军的眼里,明军骑兵很好欺负,一打就散,一散就跑,一跑就死,很明显,眼前的这帮对手也是如此。\\

但自第一次交锋开始时起,自信就变成了绝望。\\

首先,这帮人使用的不是马刀,而是铁制大棒,抡起来呼呼作响,撞上就皮开肉绽,更可怕的是,这种大棒还能发射火器,打着打着冷不丁就开枪,实在太过缺德。\\

而且这帮人的精神状态明显不正常,跟打了鸡血似的,一点不害怕,且战斗力极强,见人就往死里打,身中数箭数刀,依然死战不退。\\

在这群恐怖的对手面前,战无不胜的后金军,终于体验到了一种前所未有的经历——崩溃。\\

当后金军如潮水般涌来的时候,满桂知道,胜利的时刻到了。\\

关宁铁骑是一群不太正常的人,他们和以往的明军骑兵不同,不但是因为他们经过长期训练,且装备先进武器三眼火铳(即当枪打,又当棒使),更为重要的原因在于,他们是既得利益者。\\

根据袁崇焕的原则“以辽人守辽土”,关宁铁骑的主要成员都是辽东人,因为根据以往长期实践,外地人到辽东打仗,一般都没什么积极性,爱打不打,反正丢了就丢了,正好回老家。\\

而对于关宁铁骑来说,他们已经无家可归,这里就是他们唯一的家。\\

但最终决定他们拼命精神的,是袁崇焕的第二条原则:“以辽土养辽人”。\\

和当年的李成梁一样,袁崇焕很明白,要人卖命,就要给人好处。在这一点上,他毫不含糊,只要打仗就给军饷,此外还分地,打回来的地都能分,反正是抢来的,也没谁去管,爱怎么分怎么分。更有甚者,据说每次打仗,抢回来的战利品,他都敢分,没给朝廷报帐。\\

这么一算就明白了,拼死打仗,往光明了说,是保卫家园,保卫大明江山,往黑了说,打仗有工资拿,有土地分,还能分战利品。\\

国仇家恨外加工资外快,要不拼命,实在没有天理。\\

所以每次打仗的时候,关宁铁骑都格外激动,所谓保家卫国,对他们而言,绝不是一个空洞的口号,因为踩在脚底下那块土,没准就是他自己的家和地(地契为证)。\\

所以这场战斗的结局也就不难预料了,关宁铁骑如同疯子一般冲入后金骑兵队,大砍大杀,时不时还射两枪,威慑力极大,后金军损失惨重,只能收缩等待后续部队。\\

而与此同时,城头的大炮开始怒吼,伴随着后金军后队的惨叫声,宣告着残酷的事实:他们的攻击已经失败。\\

皇太极并没有气馁,死人嘛,很正常的事情,死光拉倒,把城攻下来就行。\\

在他的指挥下,后金军略加整顿,向宁远城发起更猛烈的进攻。\\

战斗持续到中午,在关宁铁骑的强大冲击力下,后金军损失极大,却依然没有退却。\\

然而就在此时,皇太极得知了一个让他震惊的消息。\\

锦州出事了。\\

自五月十二日进攻开始,就一直呆在城里不露头的赵率教终于出现了,他没有出来喊话,而是带着一群人,冲进了锦州城边的后金大营,一阵乱砍乱杀之后,又冲了出来,回到了城中。\\

这招实在太狠,城下的后金军做梦都想不到,城里这帮人竟然还敢冲出来,以致于人家砍完、杀完、跑完了,看着眼前的尸体,还以为是在做梦。\\

当赵率教看见城下的后金军绕开锦州,前往宁远那一刻起,他就知道,战役的结局已经注定。\\

宁远的骑兵和大炮,将彻底打碎皇太极的梦想,这是毫无疑问的,而对城下的这些留守人员,是可以趁机打几下的,当然,要等他们的主力走远点。\\

这次进攻导致后金军伤亡近五百人,更重要的是,它让皇太极认识到,锦州不是安全的后方,那个死不出头的赵率教可能随时出头,将自己置于死地。\\

他打算放弃了,但按照以往的习惯,临走前,他还要再试一把。\\

后金军对宁远发动了最猛烈也是最后一次进攻,凭借着坚强的意志,尽管未能攻破关宁铁骑,部分后金军依然冲到了宁远城边。\\

然后,他们看到了一道沟,很深的沟。\\

挖这条沟的,是袁崇焕手下的一支特殊部队——车营。\\

车营,是为应对后金的骑兵冲击组建的战斗团体,由步兵和战车组成,作战时推出战车,挖掘战壕,阻挡骑兵冲击,并使用火枪和弓箭反击,攻击说不上,防守是没问题的。\\

没戏了,毕竟马不是坦克,开不过去,在被赶过来的关宁铁骑一顿猛打后,后金军彻底放弃,退出了战斗。\\

五月二十九日,皇太极离开宁远,向锦州撤退。\\

宁远之战,明军方面,出城迎战的满桂身中数箭(没死),他和将领尤世威的坐骑也被射死。\\

但在后金方面,死得就不只是马了,其伤亡极为惨重,贝勒济尔哈朗重伤,大贝勒代善的两个儿子萨哈廉和瓦克达重伤,将领觉罗拜山、备御巴希战死,仅仅一天,后金损失高达四千余人。\\

皇太极走了,他原本以为能超越他的父亲,攻克这座不起眼的城市,然而事实是,上一次,他爹还在墙上刨了几个洞,这一次,他连城墙都没摸着。\\

回去吧,皇太极同志,宁远是无法攻克的,回家消停几年再来。\\

偏不消停。\\

皇太极并不较真,但这次例外,因为他刚刚上任,面子实在是丢大了,没点业绩,将来如何服众呢?\\

所以在回家的路上,他又有了一个想法,攻击锦州。\\

这是一个将大败变成惨败的想法。\\

五月三十日,皇太极到达锦州,再次合围。\\

他整肃队伍派出骑兵,击鼓、鸣号,呐喊示威,可就是不打。\\

非但不打,他还把大营设在离城五里外的地方。五里,是明军大炮的最远射程。\\

就这样,白天派人去城边吼,晚上躲在营帐发抖,一连五天,天天如此。\\

六月四日,皇太极决定,发动进攻。\\

进攻的重点是锦州南城,后金军动用大量云梯,冒死攻城。\\

接下来的事情我不大想讲了,因为皇太极是个很烦人的家伙,啥新意都没有,攻城的程序,从他爹开始,一直到他,这么多年,都没什么长进,后金军一批批上,一批批死,又一批批火化,毫无进展。\\

赵率教这边也差不多,他虽然进攻不大行,打防守还是不成问题的,守着城池,用大炮,看准人多的地方就轰,按照程序操作,十分轻松。\\

而且趁着后金军撤走的这几天,赵率教还在城边修了几条壕沟,以保证后金军在进攻时,能在这里停上一会,为大炮提供固定的打击地点。\\

战斗继续着,确切地说,不是战斗,而是屠杀。\\

后金军根本没法靠近城墙,每到沟边,就有定点爆破,不是被轰上天,就是被打下沟,尸横遍野。不过客观地讲,赵率教挖这几条沟也方便了后金军,人打死就直接进了沟,管杀,也管埋。\\

就这样,高效率的定点爆破进行了半日,后金军伤亡极大,按赵率教的报告,打死不下三千,打伤不计其数。\\

明军的伤亡人数不明,但很有可能是零,因为在整个战斗中,后金军最远才到壕沟(包括沟里),以弓箭的射程,要打死城头明军,似乎可能性不大。\\

打仗也是要计算成本的,这次战役,皇太极带上了全部家当,而他的全部家当,也就七万多人,按一天损失三千人的打法,他还能打二十多天。\\

这生意不能再做了。\\

六月五日,皇太极撤军,算是彻底撤了。\\

第二天,他率军路过大凌河城,此处空无一人,于是皇太极下令——拆了。\\

泄愤需要,理解万岁。\\

战役至此结束,五月十一日至六月五日,在长达二十余天中,后金与大明在锦州、宁远一线展开大战,最终以后金惨败告终,史称“宁锦大捷”。\\

在这场战役中,后金军伤亡极大。据保守估计,应该在一万人左右,多名牛录战死,退回沈阳。\\

该结果充分说明,明朝只要自己不捣腾自己,后金是没戏的。\\

六月六日,就在皇太极撤退的第二天,袁崇焕向朝廷报捷:\\

“十年来尽天下之兵,未尝敢于奴战,合马交锋,今始一刀一枪拼命,不知有夷之凶狠剽悍……诸军愤恨此贼,一战挫之。”\\

天启皇帝回应:\\

“十年之积弱,今日一旦挫其狂峰!”\\

皇帝很高兴,大臣很高兴,整个朝廷,包括魏忠贤在内,都很高兴。\\

现在是天启七年(1627)六月,很明显,形势还是一片大好。\\

天启七年(1627)七月初一,兵部侍郎、辽东巡抚袁崇焕提出,身体有病,辞职。\\

一般说来,辞职的原因只有一个:如果不辞职,会遇到比辞职更倒霉的事。\\

袁崇焕的情况更复杂一点,首先是有人告他,且告得比较狠。\\

宁锦大捷后几天,御史李应荐上书,弹劾袁崇焕,说他在战役中,不援助锦州,是作战不积极的表现,还用了个专用名词——“暮气”。\\

“暮气”大致就是晚上的气,跟没气也差不了多少,用这个词损人,足见中华文化之博大精深。\\

如果你觉得这个弹劾太扯淡,那说明你还没见过世面。明代的言官,从没有想不到,也没有做不到,只有想不想做,啥理由都能找,啥人物都敢碰,相比以往的张居正、李如松等等,这只是小儿科。\\

此外,不服气应该也是他辞职的原因之一。\\

宁锦大战后,论功行赏,最大的功劳自然是魏忠贤,头功;其次是监军太监;再其次是太监(什么都没干的);再再其次是阉党大臣,如顾秉谦、崔呈秀等等等等;再再再其次,是魏忠贤的从孙(时年四岁,学龄前儿童),封侯爵。\\

袁崇焕的奖励是:升一级,赏银三十两。\\

如果是个老实人,也就罢了,袁崇焕的性格,要让他服气,那是梦想。\\

而最重要,也最关键的原因在于,再干下去,就没意思了。\\

说到底,要想干出点成绩,自己努力是不够的,还得有人罩着,按此标准,袁崇焕只能算个体户。\\

许多书上说,袁崇焕之所以离职,是因为他是东林党,所以阉党容不下他,把他赶走了。\\

这个说法有部分不是胡扯,也就是说,有部分是胡扯,袁崇焕虽然职务不低,但在东林党里,实在是个不起眼的角色,也没什么影响力,既不是首犯,也不算从犯,你要明白,阉党也是人,事情也多,也没功夫见人就灭,像袁崇焕这类人物,睁只眼闭只眼就过了。\\

但干不下去也是实情,袁崇焕的档案实在太黑,比如,他中进士时,录取他的人是韩旷(东林党大学士),提拔他的人是侯恂(东林党御史),培养他的人是孙承宗(模范东林党),如此背景,没抓起来就算是奇迹了,虽说他本人比较乖巧,但要魏公公买他的帐,也不太现实。\\

基于以上原因,他提出辞职,基于同样原因,他的辞职被批准。\\

死了上万人,折腾几十天,连块砖头都没挖到的皇太极永远不会想到,袁崇焕就这么失败了,败在一个连大字都不识的人妖手里。\\

\subsection{妖风}
魏忠贤已经是名副其实的人妖了,不是人,而是妖。\\

解决掉东林党,没有敌人了,就开始四处闹腾刮妖风了。\\

最先刮出来的,是那个妇孺皆知的称号——九千岁,但事实上,这只是个简称,全称是“九千九百岁爷爷”。\\

阉党的贵孙们尽力了,由于天生缺少部件和职位的稀缺性,魏人妖当不上万岁,所以只能九千九百了,用数学的角度讲,应该算极限接近。\\

除称号外,魏公公丝毫不放松对自己的要求,还有个很牛的官衔,就不列出来了,因为我算了一下,总计两百多字,全写出来比较麻烦。\\

光有称号和官衔是不够的,人也得实在点,吃穿住行,还得买房子。\\

简单点说,除了不穿龙袍,魏公公的待遇和皇帝基本是一样的,至于房子,魏公公也不怎么挑,只是比较执着——看中了就要。\\

而且他还有个不好的习惯:只要,不怎么买。\\

比如参政米万钟,在北京郊区有套房子(园林别墅),魏忠贤看中了,象征性地出了个价,要买,米万钟不卖。\\

魏忠贤同意了,他免了米万钟的官职,直接占了他的房子,一分钱都没花。\\

在强买强卖这个问题上,魏忠贤是讲究平等的,无论平民百姓还是皇亲国戚,全都一视同仁。如某位权贵有座大院子,魏忠贤想要,人家没给,魏忠贤随即编了个罪名,把他绕了进去,还打了几十棍。\\

除了自己住的地方外,魏忠贤也没忘了家乡。他的老家河北肃宁,一向很穷,以出太监闻名,现在终于也露了脸。为了让肃宁人民时刻感受到魏公公的光辉,他专门拨款(朝廷出),重新整修了肃宁城,一个小县城,挖了几条护城河,还修了三十座敌楼,城楼十二栋,大炮就安了上百门,实在有够夸张。\\

问题在于,魏公公不忘家乡,却忘了老乡,肃宁的穷光蛋们还是穷光蛋,除了隔三差五被拉去砌墙,生活质量没啥改善。\\

肃宁是个县城,且战略地位极其不重要,修得跟碉堡似的,这么穷的地方,请人来抢人家都未必来,搞得南来北往的强盗们哭笑不得。\\

搞笑的是,十几年后,后金军入侵河北,经过这里,本来没打算抢肃宁,但这城墙修得实在太好,忍不住好奇心,就攻了一下,想打进去看看里面有多少钱。而更搞笑的是,肃宁太过坚固,任他们死攻活攻,竟然没能够攻进去(进了白进)。\\

这件事告诉我们,一个人,即使是魏公公这样的人,如果下定决心要做点事,也是可以做成的。\\

吃喝不愁了,有房子了,光宗耀祖了,官位称号都有了,还缺吗?\\

还缺。\\

自古以来,人类追求的东西不外乎以下几种:金钱、权力、地位,这些魏忠贤全都有了。\\

但最重要的那件东西,他并没有得到。\\

那是无数帝王将相梦寐以求,却终究梦断的奢望——入圣。\\

成为圣贤,成为像老子、孔子、孟子一样的人,为万民景仰,为青史称颂!\\

问题是,魏公公不识字,也写不出《论语》、《道德经》之类的玩意,现在还镇得住,再过个几十年就没辙了。\\

为保证长治久安,数百年如一日地当圣人,魏忠贤干了这样几件事:\\

第一件是修书,虽然他不识字,但他的龟孙还是比较在行的,经过仔细钻研,一本专著随即出版发行,名为《三朝要典》。\\

这是一本很有趣的书,在这本书里,讲了三个故事。\\

第一个故事叫梃击,讲述疯子张差误闯宫廷,被王之寀诱供,以达到东林党不可告人的目的。\\

第二个故事叫红丸,说的是明光宗体弱多病,服用营养品“红丸”,后因体弱死去,无辜的医生李可灼被诬陷。\\

第三个故事移宫,是最让人气愤的,一群以杨涟为首的东林党人恶霸,趁皇帝死去,闯入宫中,欺负弱小,赶走了善良的寡妇李选侍。\\

为弘扬正义,澄清事实,特作本书,由于瞎编时间短,作者水平有限,有错漏之处,敬请指正。\\

从这本书里,我看到了愤怒,很多人的愤怒,浙党、楚党、方从哲,以及所有政治斗争的失败者,还有那个拉住轿子,被杨涟喝斥的小人物李进忠。\\

为圆满完成对东林党人的总清算,除此书外,魏忠贤还弄出了一份别出心裁的名单——东林点将录。\\

几年前,为了抓住伊拉克的头头们,美军特制了一副扑克牌,把人都印在上面,抓人之余还能打牌,创意备受称赞。\\

但和几百年前的魏公公比起来,美军就差的太远了,他的敌人们统统按照水浒传一百单八将归类编印成册,每个人都有对应外号,读来琅琅上口,而且按牌数算,美军只有一副扑克,只能打斗地主,魏公公能做两副打拖拉机。\\

这份东林点将录的内容相当精彩,排第一的托塔天王,是南京户部尚书李三才,第二男主角及时雨宋江,由大学士叶向高扮演。\\

戏中其余主角,以排名为序,不分姓氏笔画:\\

玉麒麟卢俊义——吏部尚书赵南星饰演\\

入云龙公孙胜——左都御史高攀龙饰演\\

智多星吴用——左谕德缪昌期饰演\\

鉴于以下一百余人中没有路人甲、宋兵乙之流,全部有名有姓有外号有官职,篇幅太长,故省略。\\

值得一提的是,在之前斗争中给魏人妖留下深刻印象的杨涟和左光斗,都得到了重要的角色,其中杨涟扮演的,是大刀关胜,而左光斗,是豹子头林冲。\\

当然了,创意并不是魏公公首创的,灵感爆发的撰写者是王绍徽,时任吏部尚书,这位王尚书并非等闲之辈,据说他虽然惟命是从,毫无道德,人品低劣,但相当女性化,长相柔美,还特别喜欢给人起外号,所以江湖上的朋友给他也取了个响亮的外号——王媳妇。\\

王媳妇向来尊重长辈,特别是对魏公公,他知道自己的公公不识字,写得太复杂看不懂,但《水浒》还是听过的,所以想了这么个招。\\

魏公公很高兴,因为他终于看到了一本自己能够看懂的书,兴奋之余,他跑去找皇帝,展示这个文化成果。\\

可是当皇帝拿到这份东林点将录的时候,却问出了一个足以让魏公公跳河的问题:\\

“什么是《水浒》?”\\

魏公公热泪盈眶了,他终于遇到了知音:在这世上,要找到一个文化比他还低的人,是太不容易了。\\

本着扫除文盲的决心和责任,魏文盲对朱文盲详细解说了水浒的意义和内容。\\

皇帝满意了,他翻开首页,看到了托塔天王李三才,随即问了第二个让魏公公崩溃的问题:\\

“谁是托塔天王?”\\

如此朋友实在难寻,有生以来,魏公公第一次有机会展示自己的学问,他马上将自己听来的托塔天王晁盖的故事和盘托出,从生平、入行当强盗、智取生辰纲,梁山结义等等,娓娓道来。\\

然而他还没有讲完,皇帝大人就用一声大喝打断了他:\\

“好!托塔天王,有勇有谋!”\\

讲坏话竟然讲出这个效果,那一刻,魏忠贤觉得自己的人生非常失败。\\

他闭上了嘴,收回了这本书,再没有提过,至于他回去后有没有找王媳妇算帐,就不知道了。\\

除著书立言外,魏公公成为圣贤的另一个标志,是修祠堂。\\

所谓祠堂,是用来祭奠祖先的,换句话说,供在里面的都是死人,而魏公公是唯一一个供在里面,却又活着的人。\\

修祠这个事,是浙江巡抚潘汝桢先弄出来的,为表尊重,他把魏公公的祠堂修在西湖边上,住在他旁边的也是位名人——岳飞(岳庙)。\\

这个由头一出来,就不得了了,全国各地只要有点钱的,就修祠堂,据说袁崇焕同志也干过这活。\\

为显示对魏公公的尊重,祠堂选址还专挑黄金地段,比如凤阳的祠堂,就修在朱元璋祖宗皇陵的旁边。南京的祠堂,竟然修在了朱元璋的坟头,重八兄在天有灵,知道一个死太监竟敢跟自己抢地盘,说不定会把棺材啃穿。\\

但最猛的还是江西,江西巡抚杨邦宪要修祠堂,唯恐地段不好,竟然把朱圣贤(朱熹)的祠堂给砸了,然后在遗址上重建,以表明不破不立的决心。\\

书写完了,祠堂修了,魏人妖当圣人的日子不远了,各种妖魔鬼怪就跳出来了。\\

最能闹腾的,是国子监监生陆万龄,他公然提出,要在国子监里给魏忠贤修祠堂。他还说,当年孔子写了《春秋》,现在魏公公写了《三朝要典》,孔子是圣贤,所以魏公公也应该是圣贤。\\

无耻的人读过书后,往往会变得更加无耻。\\

由于这个人的恶心程度超越了人类的极限,搞得跟魏忠贤关系不错的一位国子监司业(副校长)也受不了了,表示无法忍受,辞职走人。\\

面对如此光辉的荣誉,魏忠贤的内心没有一丝不安,他很高兴,也希望大家都高兴。\\

但这实在有点难,因为他并不是圣贤,而是死太监,是无恶不作、无耻至极的死太监。要想普天同庆,万民敬仰,只能到梦里忽悠自己了。\\

捧他的人越多,骂他的人也就越多,朝廷不给骂,就在民间骂,传到魏公公耳朵里,魏公公很不高兴。\\

可是国家这么大,人这么多,背后骂你两句,你能如何?\\

魏公公说,我能。\\

他自信的来源,就是特务。\\

作为东厂提督太监,魏忠贤对阴人一向很有心得,在他的领导下,东厂特务遍布全国,四下刺探。\\

比如在江西,有一个人到书店买书,看到《三朝要典》,就拿起来看,觉得不爽,就说了两句。\\

结果旁边一人突然爆起,跑过来揪住他,说自己是特务,要把他抓走,好在那人地头熟,找朋友说了几句话,又送了点钱,总算没出事。\\

这个故事虽然悲剧开头,好歹喜剧结尾,下一个故事既不是悲剧,也不是喜剧,而是恐怖电影。\\

这个故事是我十多年前读古书时看到的,一直到今天,都没能忘记。\\

故事发生在一个深夜,四周无人,四个人在密室(或是地下室)交谈,大家兴致很高,边喝边谈,慢慢地,有一个人喝多了。\\

酒壮胆,这位胆大的仁兄就开始骂魏忠贤,越骂越起劲,然而奇怪的是,旁边的三个人竟然沉默了,一言不发,在密室里,静静地听着他开骂。\\

突然,门被人踢破了,几个人在夜色中冲了进来,把那位骂人的兄弟抓走,却没有为难那三个旁听者(请注意这句话)。\\

这意味着,在那天夜里,这几人的门外,有人在耐心地倾听着里面的声音。\\

他们不但听清了屋内的谈话,还分清了每个发言的人,以及他说话的内容。\\

这倒没什么,当年朱重八也干过这种事。\\

但最为可怕的是,这几个人,只是小人物,不是大臣,不是权贵,只是小人物。\\

深夜里,趴在不知名的小人物家门口,认真仔细地听着每一句话,随时准备破门而入。\\

周厉王的时候,但凡说他坏话的,都要被干掉,所以人们在路上遇到,只能使个眼色,不敢说话,时人称为暴政。\\

然而魏公公说,在家说我坏话,就以为我不知道吗,幼稚。\\

周厉王实行政策后没几年,百姓渐渐不满,没过几年,他就被赶到山里去了。\\

魏公公搞了几年,什么事都没有。\\

严嵩在的时候,严党不可一世,也拿徐阶没办法;张居正在的时候,内有冯保,外有爪牙,依然有言官跟他捣乱,魏公公当政时期,这个世界很清净。\\

因为他搞定了所有人,包括皇帝在内。\\

除了皇帝,他可以干掉任何人。\\

包括皇帝的儿子和老婆。\\

事实上,他也搞到了皇帝的头上。\\

对于天启皇帝,魏忠贤是很有好感的,这人文化比他还低,干活比他还懒,业务比他还差,如此难得的废柴,哪里去找?\\

所以魏忠贤认定,在自己的这块自留地上,只能有这根废柴,任何敢于长出来的野草,都必须被连根铲除。\\

所谓野草,就是皇帝的儿子。\\

天启皇帝虽然素质差点,但生儿子还是有两把刷子的,到天启六年,他已经先后生了三个儿子。\\

一个都没有活下来。\\

天启三年十月,皇后生下一子,早产,夭折。\\

十余天后,慧妃生下第二子,母子平安,皇帝大喜,大赦天下,九个月后,夭折。\\

天启五年十月,容妃生子,八个月后,夭折。\\

我相信,明代坐月子的水平就算比不上今天,也差不到哪去,搞出这么个百分百死亡率,要归功于魏忠贤同志的艰苦努力。\\

比如第一个皇子,由于是皇后生的,大肚子时直接下手似乎有点麻烦,但要等她生下来,估计更麻烦,经过反复思考后,魏忠贤使用了一个独特的方法,除掉这个孩子。\\

我确信,该方法的专利不属于魏忠贤(多半是客氏),因为只有女人,才能想出如此专业,如此匪夷所思的解决方案。\\

按某些史料的说法,事情是这样的,皇后腰痛,要找人治,魏公公随即体贴地推荐了一个人帮她按摩,这个人在按摩时使用了一种奇特的手法,伤了胎儿,并直接导致皇后早产,是名副其实的无痛“人”流。\\

如此杀人不见血之神功,实在让人叹为观止,如果这一招数流传下来,无数药厂、医院估计就要关门大吉了。\\

这件事情虽然流得相当利索,但传得相当快,没过多久,宫廷内外都知道了,以至于杨涟在写那封魏忠贤二十四大罪时,把这条也列进去。\\

但皇帝不知道,估计就算知道,也不信。\\

此后,皇帝大人的两个儿子,虽然平安出生,但几个月后就都去见列祖列宗了。\\

可惜,关于这两起死亡事件,没有证据显示跟魏公公有关,充其量只是嫌疑犯。问题在于,他是唯一的嫌疑犯,所以只能委屈他,反正他身上的烂帐多了去了,也不在乎这一件。\\

除了皇帝的儿子外,皇帝的老婆也没能保住。\\

比如裕妃,原本很受皇帝宠信,但由于怀了孕,魏忠贤决定整整她,联合客氏,把她发配到冷宫。\\

更恶劣的是,他还调走了裕妃身边的宫女,让她单独在宫里进行生存训练,连水都没给,最后终于饥渴而死。\\

此外,慧妃、容妃、甚至皇后,只要是皇帝宠信的,能生儿子的,全部都挨过整。\\

魏忠贤的努力,最终换来了胜利的成果:登基六年的天启皇帝,虽然竭尽全力,身心健康,依然毫无收获。\\

魏忠贤的动机很简单,他并不想当皇帝,只是害怕生出了太子,长大后比他爹聪明,不受自己控制,就不好混了。\\

这个算盘没有打错,毕竟皇帝大人才二十二岁,还有很多时间,再享个十几年的福,让他生儿子也不迟。\\

更何况从大臣到太监,一切都在控制之中,即使新皇帝即位,也是自己说了算,世间已没有敌人了。\\

天启六年(1626),情况大抵如此。\\

但事实上,这两个假设都是错误的,首先,皇帝大人今年确实只有二十二岁,不过历史记载,他临终时,也只有二十三岁。\\

其次,魏公公是有敌人的,和以往不同的是,这个敌人虽不起眼,却将置他于死地。\\

我知道,所有的场景,荒唐的,奇异的,不可理解的,都在上天的眼里,六年前,他送来了一个女人,把魏忠贤送上了至高无上的宝座,创造了传奇。\\

现在,他决定终结这个传奇,把那个当年的无赖打回原形,而承担这个任务的,也是一个女人。\\

这个女人叫做张嫣。\\

就在六年前,当客氏和魏忠贤打得火热,太监事业蒸蒸日上的时候,十五岁的张嫣进入了皇宫。\\

作为河南选送的后妃人选,她受到了皇帝的召见。\\

面试结果十分之好,张嫣年级很小,却很漂亮,皇帝很喜欢,并记下了她的名字。\\

而当客氏见到她时,却感受到了一种极致的惊恐,她的直觉告诉她,她所苦心经营的一切,都将毁在这个女孩的手上。\\

于是她去向皇帝哭诉,执意反对,要把这个小女孩送回去。\\

一贯对他言听计从的皇帝,第一次违背了奶妈的意愿,无论客氏哭天抢地,置若罔闻。\\

非但如此,十几天后,他竟然把这个女孩封了皇后,史称懿安皇后。\\

客氏是个相当精明的人,她认为,这个女孩太过漂亮,会影响她在皇帝心中的地位,但是她错了。\\

这个女孩不但漂亮,而且精明,她不但抢走了皇帝的宠信,还将夺走她所有的一切。\\

虽然张皇后才十五,但她的心智年龄应该是五十多,自打入宫起,就开始跟客氏干仗,且丝毫无惧,时常还把魏公公拉进宫来骂几句,完全不把魏大人当外人,九千岁恨得咬牙切齿,没有办法。\\

到天启三年(1623),张皇后怀孕了,客氏无计可施,让人按摩时做了人工流产。\\

这件事情让客氏高兴了很久,然而她想不到的是,短暂的得意换来的,将是永远的毁灭。\\

在失去孩子的那一天,张皇后发誓,客氏和魏忠贤将为此付出惨重的代价。\\

双方矛盾开始激化,由一本书开始。\\

此后不久的一天,皇帝来到了张皇后的寝宫,发现她正在看书,于是发问:\\

“你在看什么书?”\\

“《赵高传》。”\\

皇后这样回答。\\

皇帝没有说话,他虽然不知道托塔天王,却知道赵高。\\

很快,魏忠贤就知道了这件事,他十分愤怒,决定反击。\\

第二天,皇帝在宫里闲逛的时候,意外发现了几个素未谋面的生人,大惊失色,立刻召集侍卫,经过搜查,这些人的身上都带有武器。\\

此事非同小可,相关嫌疑人立即被送往东厂,进行严密审查。\\

这是魏忠贤的诡计,他在宫中埋伏士兵,伪装成刺客,故意被皇帝发现,而这些刺客必定会被送到东厂审问,在东厂里,刺客们一定会坦白从宽,说出指使人,想坑谁,就坑谁。\\

魏忠贤想坑的人,叫做张国纪——张皇后的父亲。\\

这是一条相当毒辣的计策,泰山也好,岳父也罢,扯上这个罪名,上火星也跑不掉。\\

然而就在他准备实施这个计划时,一个人出面阻止了他。\\

这个人表示,即使死,他也绝不同意这种诬陷行为。\\

不过这位仁兄并不是什么善人,他就是魏忠贤的忠实走狗,司礼监掌印太监王体乾。\\

只用一句话,他就说服了魏忠贤:\\

“皇上凡事都不怎么管,但对兄弟老婆是很好的,你要是告状,有个三长两短,我们就没命了!”\\

魏忠贤到底是老江湖,立刻打消主意,为了信息安全,他干掉了那几个被他安排扮演刺客的兄弟。\\

皇后是干不倒了,那就一心一意跟着皇帝混吧。\\

可是皇帝已经混不下去了。\\

天启七年(1627)八月,天启皇帝病危。\\

病危,自然不是勤于政务,估计是做木匠太过操劳,也算是倒在了工作岗位上。\\

魏忠贤很伤心,真的很伤心,他很明白,如果皇帝大人就此挂掉,以后就难办了。\\

拜自己所赐,皇帝的几个儿子都被干掉了,所以垂帘听政、欺负小孩之类的把戏没法玩了,而皇位继承者,将是天启皇帝的弟弟。\\

明光宗虽然只当了一个月皇帝,但生儿子的能力却相当了得,足足有七个。\\

不过很可惜,七个儿子活到现在的只剩两个,一个是天启皇帝朱由校。\\

而另一个,是信王朱由检,当时十七岁,他后来的称呼,叫做崇祯。\\

对于朱由检,魏忠贤并不了解,但他明白,十七岁的人,如果不是天启这样的极品,要想控制,难度是很大的。\\

废柴难得,所以当务之急,必须保住皇帝的命。\\

他随即公告天下,为皇帝寻找名医偏方,兵部尚书霍维华不负众望,仅用了几天,就找到了一个药方。\\

他说,用此药方,有起死回生之效。\\

出于好奇,我找到了这个药方。\\

药名:仙方灵露饮,配方如下:\\

优良小米少许,加入木筒蒸煮,木筒底部镂空,安放金瓶一个,边煮边加水,煮好的米汁流入银瓶,煮到一定时间,换新米再煮,直到银瓶满了为止。\\

金瓶中的液体,就是灵露,据说有长寿之功效。\\

事实证明,灵露确实是有效果的,天启皇帝服用后,感觉很好,连吃几天后,却又不吃了——病情加重,吃不下去。\\

其实对此药物,我也有所了解,按以上配方及制作方法,该灵露还有个更为通俗的称呼——米汤。\\

用米汤,去抢救一个生命垂危,即将歇菜的人,这充分反映了魏公公大无畏的人道主义精神。\\

真是蠢到家了。\\

皇帝大人喝下了米汤,然后依然头都不回地朝黄泉路上一路狂奔,拉都拉不住。\\

痛定思痛,魏忠贤决定放弃自己的医学事业,转向专业行当——阴谋。\\

当皇帝将死未死之时,他找到了第一号心腹崔呈秀,问他,大事可行否?\\

狡猾透顶的崔呈秀自然知道是什么大事,于是他立刻做出了反应——沉默。\\

魏忠贤再问,崔呈秀再沉默,直到魏大人生气了,他才发了句话:我怕有人闹事。\\

直到现在,魏忠贤才明白,自己收进来的,都是些胆小怕死的货,都靠不住,只能靠自己了。\\

他找到客氏,经过仔细商议,决定从宫外找几个孕妇进宫当宫女,等皇帝走人,就搞个狸猫换太子,说是皇帝的遗腹子。反正宫里的事是他说了算,他说是,就是,不是也是。\\

为万无一失,他还找到了张皇后,托人告诉她,我找好了孕妇,等到那个谁死了,就生下来直接当你的儿子,接着做皇帝,你挂个名就能当太后,不用受累。\\

这是文明的说法,流氓的讲法自然也有,比如宫里的事我管,你要不听话,皇帝死后怎么样就不好说了。\\

皇后回答:如听从你的话,必死,不听你的话,也必死,同样是死,还不如不听,死后可以见祖宗在天之灵!\\

说完,她就跑去找皇帝,报告此事。\\

按常理,这种事情,只要让皇帝知道了,是必定完蛋的。\\

然而当皇后见到奄奄一息的皇帝,对他说出这件事时,皇帝陛下却只说了三个字:我知道。\\

魏忠贤并不怕皇后打小报告,在发出威胁之前,他就已经找到了皇帝,本着对社稷人民负责的态度,准备给皇后贡献一个儿子,以保证后继有人。\\

皇帝非常高兴。\\

这很正常,皇帝大人智商本不好使,加上病得稀里糊涂,脑袋也就只剩一团浆糊了。\\

所以魏忠贤相信,自己的目的一定能够实现。\\

但他终究还是犯了一个错误,和当年东林党人一样的错误:低估女人。\\

今天的张皇后,就是当年的客氏,且有过之而无不及。\\

她不但有心眼,而且很有耐心,经过和皇帝长达几个时辰的长谈,她终于让这个人相信,传位给弟弟,才是最好的选择。\\

很快,住在信王府里的朱由检得到消息,皇帝要召见他。\\

在当时的朝廷里,朱由检这个名字的意义,就是没有意义。\\

朱由检,生于万历三十八年,自打出生以来,一直悄无声息,什么梃击、红丸、移宫、三党、东林党、六君子,统统没有关系。\\

他一直很低调,从不发表意见,当然,也没人征求他的意见。\\

但他是个明白人,至少他明白,此时此刻召他觐见,是个什么意思。\\

就快断气的皇帝哥哥没有丝毫客套,一见面就拉住了弟弟的手,说了这样一句话:\\

“来,吾弟当为尧舜。”\\

尧舜是什么人,大家应该知道。\\

朱由检惊呆了,像这种事,多少要开个会,大家探讨探讨,现在一点思想准备都没有,突然收这么大份礼,怎么好意思呢?\\

而且他一贯知道,自己的这位哥哥比较迟钝,没准是魏忠贤设的圈套,所以,他随即做出了答复:\\

“臣死罪!”\\

意思是,我不敢答应。\\

这一天,是天启七年(1627)八月十一日。\\

皇帝已经撑不了多久,他决心,把自己的皇位传给眼前的这个人,但这一切,眼前的人并不知道,他只知道,这可能是个圈套,非常危险,绝不能答应。\\

两个人陷入了沉默。\\

在这关键时刻,一个人从屏风后面站了出来,打破了僵局,并粉碎了魏忠贤的梦想。\\

张皇后对跪在地上的朱由检说,事情紧急,不可推辞。\\

朱由检顿时明白,这件事情是靠谱的,他马上答应了。\\

八月二十二日,足足玩了七年的木匠朱由校驾崩,年二十三。\\

就在那一天,得知噩耗的魏忠贤没有发丧,他立即封锁了消息。\\
\ifnum\theparacolNo=2
	\end{multicols}
\fi
\newpage

\chapter*{朱由检篇}
\addcontentsline{toc}{chapter}{朱由检篇}
\section{疑惑}
\ifnum\theparacolNo=2
	\begin{multicols}{\theparacolNo}
\fi
魏忠贤的意图很明显,在彻底控制政局前,绝不能出现下一个继任者。\\

但就在那天,他见到了匆匆闯进宫的英国公张维迎:\\

“你进宫干什么?”\\

“皇上驾崩了,你不知道?”\\

“谁告诉你的?”\\

“皇后。”\\

魏忠贤确信,女人是不能得罪的。\\

皇帝刚刚驾崩,皇后就发布了遗诏,召集英国公张维迎入宫。\\

在朝廷里,唯一不怕魏忠贤的,也只有张维迎了,这位仁兄是世袭公爵,无数人来了又走了,他还在那里。\\

张维迎接到的第一个使命,就是迎接信王即位。\\

事已至此,魏忠贤明白,没法再海选了,十七岁的朱由检,好歹就是他了。\\

他随即见风使舵,派出亲信太监前去迎接。\\

朱由检终于进宫了,战战兢兢地进来了。\\

按照以往程序,要先读遗诏,然后是劝进三次。\\

所谓劝进,就是如果继任者不愿意当皇帝,必须劝他当。\\

之所以劝进三次,是因为继任者必须不愿当皇帝,必须劝三次,才当。\\

虽然这种礼仪相当无聊,但上千年流传下来,也就图个乐吧。\\

和无数先辈一样,朱由检苦苦推辞了三次,才勉为其难地答应做皇帝。\\

接受了群臣的朝拜后,张皇后走到他的面前,在他的耳边,对他说出了诚挚的话语:\\

“不要吃宫里的东西(勿食宫中食)!”\\

这就是新皇帝上任后,听到的第一句祝词。\\

他会意地点了点头。\\

事实上,张皇后有点杞人忧天,因为皇帝大人早有准备:他是有备而来的。照某些史料的说法,他登基的时候,随身带着干粮(大饼),就藏在袖子里。\\

天启七年(1627)八月二十四日,朱由检举行登基大典,正式即位。\\

在登基前,他收到了一份文书,上面有四个拟好的年号,供他选择:\\

明代每个皇帝,只有一个年号,就好比开店,得取个好名字,才好往下干,所以选择时,必须谦虚谨慎。\\

第一个年号是兴福,朱由检说不好;\\

第二个是咸嘉,朱由检也说不好;\\

第三个是乾圣,朱由检还说不好;\\

最后一个是崇祯。\\

朱由检说,就这个吧。\\

自1368年第一任老板朱元璋开店以来,明朝这家公司已经开了二百五十九年,换过十几个店名,而崇祯,将是它最后的名字。\\

和以往许多皇帝一样,入宫后的第一个夜晚,崇祯没有睡着。他点着蜡烛,坐了整整一夜,不是因为兴奋,而是恐惧,极度的恐惧。\\

因为他很清楚,在这座宫里,所有的人都是魏忠贤的爪牙,他随时都可能被人干掉。\\

每个经过他身边的人,都可能是谋杀者,他不认识任何人,也不了解任何人,在空旷而阴森的宫殿里,没有任何地方是安全的。\\

于是那天夜里,他坐在烛火旁,想出了一个办法,度过这惊险的一夜。\\

他拦住了一个经过的太监,对他说:\\

“你等一等。”\\

太监停住了,崇祯顺手取走了对方腰间的剑,说道:\\

“好剑,让我看看。”\\

但他并没有看,而是直接放在了桌上,并当即宣布,奖赏这名太监。\\

太监很高兴,也很纳闷,然后,他听到了一个让他更纳闷的命令:\\

“召集所有的侍卫和太监,到这里来!”\\

当所有人来到宫中的时候,他们看到了丰盛的酒菜,并被告知,为犒劳他们的辛苦,今天晚上就呆在这里,皇帝请吃饭。\\

人多的地方总是安全的。\\

第一天度过了,然后是第二天、第三天,崇祯静静地等待着,他知道,魏忠贤绝不会放过他。\\

但事实上,魏忠贤不想杀掉崇祯,他只想控制这个人。\\

而要控制他,就必须掌握他的弱点。所谓不怕你清正廉洁,就怕你没有爱好,魏忠贤相信,崇祯是人,只要是人,就有弱点。\\

几天后,他给皇帝送上了一份厚礼。\\

这份礼物是四个女人,确切地说,是四个漂亮的女人。\\

男人的弱点,往往是女人,这就是魏忠贤的心得。\\

这个理论是比较准确的,但对皇帝,就要打折扣了,毕竟皇帝大人君临天下,要什么女人都行,送给他还未必肯要。\\

对此,魏忠贤相当醒目,所以他在送进女人的同时,还附送了副产品——迷魂香。\\

所谓迷魂香,是香料的一种,据说男人接触迷魂香后,会性欲大增,看老母牛都是双眼皮。就此而言,魏公公是很体贴消费者的,管送还管销。\\

但他万万想不到,这套近乎完美的营销策略,却毫无市场效果。据内线报告,崇祯压根就没动过那几个女人。\\

因为四名女子入宫的那一天,崇祯对她们进行了仔细的搜查,找到了那颗隐藏在腰带里的药丸。\\

在许多的史书中,崇祯皇帝应该是这么个形象:很勤奋,很努力,就是人比较傻,死干死干往死了干,干死也白干。\\

这是一种为达到不可告人目的,用心险恶的说法。\\

真正的崇祯,是这样的人:敏感、镇定、冷静、聪明绝顶。\\

其实魏忠贤对崇祯的印象很好。天启执政时,崇祯对他就很客气,见面就喊“厂公”(东厂),称兄道弟,相当激动,魏忠贤觉得,这个人相当够意思。\\

经过长期观察,魏忠贤发现,崇祯是不拘小节的人,衣冠不整,不见人,不拉帮结派,完全搞不清状况。\\

这样的一个人,似乎没什么可担心的。\\

然而魏忠贤并不这样看。\\

几十年混社会的经验告诉他,越是低调的敌人,就越危险。\\

为证实自己的猜想,他决定使用一个方法。\\

天启七年(1627)九月初一,魏忠贤突然上书,提出自己年老体弱,希望辞去东厂提督的职务,回家养老。\\

皇帝已死,靠山没了,主动辞职,这样的机会,真正的敌人是不会放过的。\\

就在当天,他得到了回复。\\

崇祯亲自召见了他,并告诉了他一个秘密。\\

他对魏忠贤说,天启皇帝在临死前,曾对自己交代遗言:\\

要想江山稳固,长治久安,必须信任两个人,一个是张皇后,另一个,就是魏忠贤。\\

崇祯说,这句话,他从来不曾忘记过,所以,魏公公的辞呈,我绝不接受。\\

魏忠贤非常感动,他没有想到,崇祯竟然如此坦诚,如此和善,如此靠谱。\\

就在那天,魏忠贤打消了图谋不轨的念头,既然这是一个听招呼的人,就没有必要撕破脸。\\

崇祯没有撒谎,天启确实对他说过那句话,他也确实没有忘记,只是每当他想起这句话时,都禁不住冷笑。\\

天启认为,崇祯是他的弟弟,一个听话的弟弟;而崇祯认为,天启是他的哥哥,一个白痴的哥哥。\\

虽然只比天启小六岁,但从个性到智商,崇祯都要高出一截,魏忠贤是什么东西,他是很清楚的。\\

而他对魏公公的情感,也是很明确的——干掉这个死人妖,把他千刀万剐,掘坟刨尸!\\

每当看到这个不知羞耻的太监耀武扬威,鱼肉天下的时候,他就会产生极度的厌恶感,没有治国的能力,没有艰辛的努力,却占据了权位,以及无上的荣耀。\\

一切应该恢复正常了。\\

他不过是皇帝的一条狗,有皇帝罩着,谁也动不了他。\\

现在皇帝换人了,没人再管这条狗,却依然动不了他。\\

因为这条狗,已经变成了狼。\\

崇祯很精明,他知道眼前的这个敌人有多么强大。\\

除自己外,他搞定了朝廷里所有的人,从大臣到侍卫,都是他的爪牙,身边没有盟友,没有亲信,没有人可以信任,他将独自面对狼群。\\

如果冒然动手,被撕成碎片的,只有自己。\\

所以要对付这个人,必须有点耐心,不用着急,游戏才刚刚开始。\\

目标,最合适的对象\\

魏忠贤开始相信,崇祯是他的新朋友。\\

于是,天启七年(1627)九月初三,另一个人提出了辞呈。\\

这个人是魏忠贤的老搭档客氏。\\

她不能不辞职,因为她的工作是奶妈。\\

这份工作相当辛苦,从万历年间开始,历经三朝,从天启出生一直到结婚、生子,她都是奶妈。\\

现在喂奶的对象死了,想当奶妈也没辙了。\\

当然,她不想走,但做做样子总是要的,更何况魏姘头已经探过路了,崇祯是不会同意辞职的。\\

一天后,她得到了答复——同意。\\

这一招彻底打乱了魏忠贤的神经,既然不同意我辞职,为什么同意客氏呢?\\

崇祯的理由很无辜,她是先皇的奶妈,现在先皇死了,我也用不着,应该回去了吧。其实我也不好意思,前任刚死就去赶人,但这是她提出来的,我也没办法啊。\\

于是在宫里混了二十多年的客大妈终于走到了终点,她穿着丧服,离开了皇宫,走的时候还烧掉了一些东西:包括天启皇帝小时候的胎发、手脚指甲等,以示留念。\\

魏忠贤身边最得力的助手走了,这引起了他极大的恐慌,他开始怀疑,崇祯是一只披着羊皮的狼,正逐渐将自己推入深渊。\\

还不晚,现在还有反击的机会。\\

但皇帝毕竟是皇帝,能不翻脸就不要翻脸,所以动手之前,必须证实这个判断。\\

第二天(九月初四),司礼监掌印太监王体乾提出辞职。\\

这是一道精心设计的题目。\\

客氏被赶走,还可能是误会,毕竟她没有理由留下来,又是自己提出来的。而王体乾是魏忠贤的死党,对于这点,魏忠贤知道,崇祯也知道。换句话说,如果崇祯同意,魏忠贤将彻底了解对方的真实意图。\\

那时,他将毫不犹豫地采取行动。\\

一天后,他得到了回复——拒绝。\\

崇祯当即婉拒了王体乾的辞职申请,表示朝廷重臣,不能够随意退休。\\

魏忠贤终于再次放心了,很明显,皇帝并不打算动手。\\

这一天是天启七年(1627)九月初七。\\

两个月后,是十一月初七,地点,北直隶河间府阜城县。\\

那天深夜,在那间阴森的小屋里,魏忠贤独自躺在床上,在寒风中回想着过去,是的,致命的错误,就是这个判断。\\

王体乾没有退休,事实上,这对王太监而言,并非一件好事。\\

而刚舒坦下来的魏公公却惊奇地发现,事情发展变得越发扑朔迷离,九月十五日,皇帝突然下发旨意奖赏太监,而这些太监,大都是阉党成员。\\

他还没来得及高兴,就在第二天,又传来了一个惊人的消息,都察院副都御史杨所修上疏弹劾。\\

杨所修弹劾的并不是魏忠贤,而是四个人,分别是兵部尚书崔呈秀,太仆寺少卿陈殷,巡抚朱童蒙,工部尚书李养德。\\

这四个人的唯一共同点是,都是阉党,都是骨干,都很无耻。\\

虽然四个人贪污受贿,无恶不作,把柄满街都是,杨所修却分毫没有提及,事实上,他弹劾的理由相当特别——不孝。\\

经杨所修考证,这四个人的父母都去世了,但都未回家守孝,全部“夺情”了,不合孝道。\\

这是一个很合理的理由,当年的张居正就被这件事搞得半死不活,拿出来整这四号小鱼小虾,很有意思。\\

魏忠贤感到了前所未有的恐惧,因为这四个人都是他的心腹,特别是崔呈秀,是他的头号死党,很明显,矛头是对着他来的。\\

让人难以理解的是,自从杨涟、左光斗死后,朝廷就没人敢骂阉党,杨所修跟自己并无过节,现在突然跳出来,必定有人主使。\\

而敢于主使者,只有一个人选。\\

然而接下来的事情,却让魏忠贤陷入了更深的疑惑。一天后,皇帝做出了批复,痛斥杨所修,说他是“率性轻诋”,意思是随便乱骂人。\\

经过仔细观察,魏忠贤发现,杨所修上疏很可能并非皇帝指使,而从皇帝的表现来看,似乎事前也不知道,总之,这只是个偶发事件。\\

但当事人还是比较机灵的,弹劾当天,崔呈秀等人就提出了辞职,表示自己确实违反规定,崇祯安慰一番后,同意几人回家,但出人意料的是,他坚决留下了一个人——崔呈秀。\\

事情解决了,几天后,另一个人却让这件事变得更为诡异。\\

九月二十四日,国子监副校长朱三俊突然发难,弹劾自己的学生,国子监监生陆万龄。\\

这位陆万龄,之前曾介绍过,是国子监的知名人物,什么在国子监里建生祠,魏忠贤应该与孔子并列之类的屁话,都是他说的,连校长都被他气走了。\\

被弹劾并不是怪事,奇怪的是,弹劾刚送上去,就批了,皇帝命令,立即逮捕审问。\\

魏忠贤得到消息极为惊恐,毕竟陆万龄算是他的粉丝,但他到底是老江湖,当即进宫,对皇帝表示,陆万龄是个败类,应该依法处理。\\

皇帝对魏忠贤的态度非常满意,夸奖了他两句,表示此事到此为止。\\

处理完此事后,魏忠贤拖着一身的疲惫回到了家,但他并不知道,这只是个开头。\\

第二天(九月二十五日),他又得知了另一个消息——一个好消息。\\

他的铁杆,江西巡抚杨邦宪向皇帝上书,夸奖魏忠贤,并且殷切期望,能为魏公公再修座祠堂。\\

魏忠贤都快崩溃了,这是什么时候,老子都快完蛋了,这帮孙子还在拍马屁,他立即向皇帝上书,说修生祠是不对的,自己是反对的,希望一律停止。\\

皇帝的态度出乎意料。崇祯表示,如果没修的,就不修了,但已经批准的,不修也不好,还是接着修吧,没事。\\

魏忠贤并不幼稚,他很清楚,这不过是皇帝的权宜之计,故作姿态而已。\\

但接下来皇帝的一系列行动,却让他开始怀疑自己的看法。\\

几天后,崇祯下令,赐给魏忠贤的侄子魏良卿免死铁券。\\

免死铁券这件东西,之前我是介绍过的,用法很简单,不管犯了多大的罪,统统地免死,但有一点我忘了讲,有一种罪状,这张铁券是不能免的——谋逆。\\

没等魏忠贤上门感谢,崇祯又下令了,从九月底一直下令到十月初,半个多月里,封赏了无数人,不是升官,就是封荫职(给儿子的),受赏者全部都是阉党,从魏忠贤到崔呈秀,连已经死掉的老阉党魏广微都没放过,人死了就追认,升到太师职务才罢手。\\

魏忠贤终于放弃了最后的警惕,他确信,崇祯是一个好人。\\

经过一个多月的考察,魏忠贤判定,崇祯不喜欢自己,也无法控制,但作为一个成熟的政治家,只要自己老老实实不碍事,不挡路,崇祯没必要跟自己玩命。\\

这个推理比较合理,却不正确。如魏忠贤之前所料,崇祯是有弱点的,他确实有一样十分渴求的东西,不是女人,而是权力。\\

要获得至高无上的权力,成为君临天下的皇帝,必须除掉魏忠贤。\\

青蛙遇到热水,会很快地跳出去,所以煮熟它的最好方法,是用温水。\\

杨所修的弹劾,以及国子监副校长的弹劾,并不是他安排的,在他的剧本里,只有封赏、安慰,和时有时无的压力。他的目的是制造迷雾,彻底混乱敌人的神经。\\

经过一个多月的你来我往,紧张局势终于缓和下来,至少看上去如此。\\

在这片寂静中,崇祯准备着进攻。\\

几天后,寂静被打破了,打破它的人不是崇祯。\\

吏科给事中陈尔翼突然上疏,大骂杨所修,公然为崔呈秀辩护,而且还上纲上线,说这是东林余党干的,希望皇帝严查。\\

和杨所修的那封上疏一样,此时上疏者,必定有幕后黑手的指使。\\

和上次一样,敢于主使者,只有一个人选——魏忠贤。\\

也和上次一样,真正的主使者,并不是魏忠贤。\\

杨所修上疏攻击的时候,崇祯很惊讶,陈尔翼上疏反击的时候,魏忠贤也很惊讶,因为他事先并不知道。\\

作为一个政治新手,崇祯表现出了极强的政治天赋,几十年的老江湖魏公公被他耍得团团转。但他并不知道,在这场游戏中,被耍的人,还包括他自己。\\

看上去事情是这样的:杨所修在崇祯的指使下,借攻击崔呈秀来弹劾魏忠贤,而陈尔翼受魏忠贤的指派,为崔呈秀辩护发动反击。\\

然而事情的真相,远比想象中复杂得多:\\

杨所修和陈尔翼上疏开战,确实是有幕后黑手的,但既不是魏忠贤,也不是崇祯。\\

杨所修的指使者,叫陈尔翼,而陈尔翼的指使者,叫杨所修。\\

如果你不明白,我们可以从头解释一下这个复杂的圈套:\\

诡计是这样开始的,有一天,右副都御史杨所修经过对时局的分析,做出了一个肯定的判断:崇祯必定会除掉阉党。\\

看透了崇祯的伪装后,他决定早做打算。顺便说一句,他并不是东林党,而是阉党,但并非骨干。\\

为及早解脱自己,他找到了当年的同事,吏科给事中陈尔翼。\\

两人商议的结果是,由杨所修出面,弹劾崔呈秀。\\

这是条极端狡诈的计谋,是人类智商极致的体现:\\

弹劾崔呈秀,可以给崇祯留下一个深刻的印象,认定自己不是阉党,即使将来秋后算帐,也绝轮不到自己头上。\\

但既然认定崇祯要除掉阉党,要提前立功,为什么不干脆弹劾魏忠贤呢?\\

原因很简单,如果崇祯未必能干得过魏忠贤,到时回头清算,自己也跑不了,而且魏忠贤毕竟是阉党首领,如果首领倒掉,就会全部清盘,彻查阉党,必定会搞到自己头上。\\

崔呈秀是阉党的重要人物,攻击他,可以赢得崇祯的信任,也不会得罪魏忠贤,还能把阉党以往的所有黑锅都让他背上,精彩,真精彩。\\

为了大家,崔先生,你就背了吧。\\

这个近乎完美的计划,几乎得到了一个近乎完美的结局。\\

几乎得到,就是没有得到。\\

因为计划的进行过程中,出现了纰漏:他们忽略了一个人——崔呈秀。\\

杨所修、陈尔翼千算万算,却算漏了崔呈秀本人,能成为阉党的头号人物,崔大人绝非善类,这把戏能骗过魏忠贤,却骗不了崔呈秀。\\

弹劾发生的当天,他就看穿了这个诡计,他意识到,大祸即将临头。\\

但他只用了几天时间,就十分从容地解决了这个问题。\\

他派人找到了杨所修,大骂了对方一顿,最后说,如果你不尽快了结此事,就派人查你。\\

大家同坐一条船,谁的屁股都不干净,敢玩阴的,大家就一起完蛋!\\

这句话相当有效,杨所修当即表示,愿意再次上疏,为崔呈秀辩解。\\

问题是,他已经骂过了,再上疏辩护,实在有点婊子的感觉,所以,这个当婊子的任务,就交给了陈尔翼。\\

问题是,原先把崔呈秀推出来,就是让他背锅的,现在把他拉出来,就必须填个人进去,杨所修不行,魏忠贤不行,崇祯更不行,实在很难办。\\

但陈尔翼不愧是老牌给事中,活人找不到,找到了死人。\\

他把所有的责任,都推到了所谓“东林余孽”的身上,如此一来,杨所修是无知的,崔呈秀是无辜的,世界又和平了。\\

倒腾来,又倒腾去,崔呈秀没错,杨所修没错,陈尔翼当然也没错,所有的错误,都是东林党搞的,就这样,球踢到了崇祯的身上。\\

但最有水平的,还是崇祯,面对陈尔翼的奏疏,他只说了几句话,就把球踢到天上:\\

“大臣之间的问题,先帝(指天启)已经搞清楚了,我刚上台(朕初御极),这些事情不太清楚,也不打算深究,你们不许多事!”\\

结果非常圆满,崔呈秀同志洗清了嫌疑,杨所修和陈尔翼虽说没有收获,也没有损失,完美落幕。\\

但事情的发展,却出现了意想不到的变化。\\

天启七年(1627)十月十五日,云南监察御史杨维垣上疏,弹劾崔呈秀贪权弄私,十恶不赦!\\

在这封文书中,杨维垣表现出极强的正义感,他愤怒地质问阉党,谴责了崔呈秀的恶行。\\

杨维垣是阉党。\\

说起来大家的智商都不低,杨所修的创意不但属于他,也属于无数无耻的阉党同仁们,反正干了也没损失,不干白不干,白干谁不干?\\

形势非常明显,崔呈秀已经成为众矢之的,对于立志搞掉阉党的崇祯而言,这是最好的机会。\\

但崇祯没有动手。他不但没有动手,还骂了杨维垣,说他轻率发言。\\

事实上,他确实不打算动手,虽然他明知现在解决崔呈秀,不但轻而易举,还能有效打击阉党,但他就是不动手。\\

因为他的直觉告诉他,在杨维垣的这封奏疏背后,隐藏着不可告人的秘密。\\

很快,他的直觉得到了证实。\\

几天后,杨维垣再次上疏,弹劾崔呈秀。\\

这是一个怪异的举动,皇帝都发了话,依然豁出去硬干,行动极其反常。\\

而反常的原因,就在他的奏疏里。\\

在这封奏疏里,他不但攻击崔呈秀,还捧了一个人——魏忠贤。\\

照他的说法,长期以来,崔呈秀没给魏忠贤帮忙,净添乱,是不折不扣的罪魁祸首。\\

崇祯的判断很正确,在杨维垣的背后,是魏忠贤的身影。\\

从杨所修的事情中,魏忠贤得到了启示:全身而退绝无可能,要想平安过关,必须给崇祯一个交代。\\

所以他指使杨维垣上书,把责任推给崔呈秀,虽然一直以来,崔呈秀都帮了很多忙,还是他的干儿子。\\

没办法,关键时刻,老子自己都保不住,儿子你就算了吧。\\

但崇祯是不会上当的,在这场残酷的斗争中,目标只有一个,不需要俘虏,也不接受投降。\\
\ifnum\theparacolNo=2
	\end{multicols}
\fi
\newpage
\section{夜半歌声}
\ifnum\theparacolNo=2
	\begin{multicols}{\theparacolNo}
\fi
真正的机会到来了。\\

十月二十三日,工部主事陆澄源上书,弹劾崔呈秀,以及魏忠贤。\\

崇祯决定,开始行动。\\

因为他知道,这个叫陆澄源的人并不是阉党分子,此人职位很小,但名气很大,具体表现为东林党当政,不理东林党,阉党上台,不理阉党,是公认的混不吝,软硬都不吃,他老人家动手,就是真要玩命了。\\

接下来的是例行程序,崇祯照例批评,崔呈秀照例提出辞职。\\

但这一次,崇祯批了,勒令崔呈秀立即滚蛋回家。\\

崔呈秀哭了,这下终于完蛋了。\\

魏忠贤笑了,这下终于过关了。\\

丢了个儿子,保住了命,这笔交易相当划算。\\

但很快,他就知道自己错了。\\

两天后,兵部主事钱元悫上书,痛斥崔呈秀,说崔呈秀竟然还能在朝廷里混这么久,就是因为魏忠贤。\\

然后他又开始痛斥魏忠贤,说魏忠贤竟然还能在朝廷里混这么久,就是因为皇帝。\\

不知钱主事是否过于激动,竟然还稍带了皇帝,但更令人惊讶的是,这封奏疏送上去的时候,皇帝竟然全无反应。\\

几天后,刑部员外郎史躬盛上疏,再次弹劾魏忠贤,在这封奏疏里,他痛责魏忠贤,为表达自己的愤怒,还用上了排比句。\\

魏忠贤终于明白,自己上当了,然而为时已晚。\\

说到底,还是读书太少,魏文盲并不清楚,朝廷斗争从来只有单项选择,不是你死,就是我活。\\

天启皇帝死的那天,他的人生就只剩下一个选择——谋逆。\\

他曾胜券在握,只要趁崇祯立足未稳,及早动手,一切将尽在掌握。\\

然而,那个和善、亲切的崇祯告诉他,自己将继承兄长的遗愿,重用他,信任他,太阳照常升起。\\

于是他相信了。\\

所以他完蛋了。\\

现在反击已不可能,从他抛弃崔呈秀的那一刻开始,他就失去了所有的威信,一个不够意思的领导,绝不会有够意思的员工。\\

阉党就此土崩瓦解,他的党羽纷纷辞职,干儿子、干孙子跟他划清界线,机灵点的,都在家写奏疏,反省自己,痛骂魏公公,告别过去,迎接美好的明天。\\

面对铺天盖地而来的狂风暴雨,魏忠贤决定,使出自己的最后一招。\\

当年他曾用过这一招,效果很好。\\

这招的名字,叫做哭。\\

在崇祯面前,魏忠贤嚎啕大哭,失声痛哭,哭得死去活来。\\

崇祯开始还安慰几句,等魏公公哭到悲凉处,只是不断叹气。\\

眼见哭入佳境,效果明显,魏公公收起眼泪,撤了。\\

哭,特别是无中生有的哭,是一项历史悠久的高难度技术。当年严嵩就凭这一招,哭倒了夏言,最后将其办挺。他也曾凭这一招,扭转了局势,干掉了杨涟。\\

魏公公相信,凭借自己声情并茂的表演,一定能够感动崇祯。\\

崇祯确实很感动。\\

他没有想到,一个人竟然可以恶心到这个程度,都六十的人了,几乎毫无廉耻,眼泪鼻涕说下就下,不要脸,真不要脸。\\

到现在,朝廷内外,就算是扫地的老头,都知道崇祯要动手了。\\

但他就不动手,他还在等一样东西。\\

其实朝廷斗争,就是街头打架斗殴,但斗争的手段和程序比较特别。拿砖头硬干是没办法的,手持西瓜刀杀入敌阵也不是不行的,必须遵守其自身规律,在开打之前,要先放风声,讲明老子是哪帮哪派,要修理谁,能争取的争取,不能争取的死磕,才能动手。\\

崇祯放出了风声,他在等待群臣的响应。\\

可是群臣不响应。\\

截至十月底,敢公开上书弹劾魏忠贤的人只有两三个,这一事实说明,经过魏公公几年来的言传身教,大多数的人已经没种了。\\

没办法,这年头混饭吃不易,等形势明朗点,我们一定出来落井下石。\\

然而崇祯终究等来了一个有种的人。\\

十月二十六日,一位国子监的学生对他的同学,说了这样一句话:\\

“虎狼在前,朝廷竟然无人敢于反抗!我虽一介平民,愿与之决死,虽死无撼!”\\

第二天,国子监监生钱嘉征上书弹劾魏忠贤十大罪。\\

钱嘉征虽然只是学生,但文笔相当不错,内容极狠,态度极硬,把魏忠贤骂得狗血淋头,引起极大反响。\\

魏忠贤得到消息,十分惊慌,立即进宫面见崇祯。\\

遗憾,他没有玩出新意,还是老一套,进去就哭,哭的痛不欲生,感觉差不多了,就收了神功,准备回家。\\

就在此时,崇祯叫住了他:\\

“等一等。”\\

他找来一个太监,交给他一份文书,说:\\

“读。”\\

就这样,魏忠贤亲耳听到了这封要命的文书,每一个字都清清楚楚。\\

他痛苦地抬起头,却只看到了一双冷酷的眼睛和嘲弄的眼神。\\

那一刻,他的威望、自信、以及抵抗的决心,终于彻底崩溃。\\

精神近乎失常的魏忠贤离开了宫殿,但他没有回家,而是去了另一个地方,在那里,还有一个人,能挽救所有的一切。\\

魏忠贤去找的人,叫做徐应元。\\

徐应元的身份,是太监,不同的是,十几年前,他就是崇祯的太监。事到如今,只能求他了。\\

徐应元是很够意思的,他客气地接待了魏忠贤,并给他指出了一条明路:立即辞职,退休回家,可以保全身家性命。\\

魏忠贤思前想后,认了。\\

立即回家,找人写辞职信,当然,临走前,他没有忘记感谢徐应元对他的帮助。\\

徐应元之所以帮助魏忠贤,是想让他死得更快。\\

和魏忠贤一样,大多数太监的习惯是见风使舵,落井下石。\\

一直以来,崇祯都希望,魏忠贤能自动走人(真心实意),毕竟阉党根基太深,这样最省事。\\

在徐应元的帮助下,第二天,魏忠贤提出辞职了,这次他很真诚。\\

同日,崇祯批准了魏忠贤的辞呈,一代巨监就此落马。\\

落马的那天,魏忠贤很高兴。因为他认为,自己已经放弃了争权,无论如何,崇祯都不会也没有必要赶尽杀绝。\\

一年前,东林党人也是这样认为的。\\

应该说,魏忠贤的生活是很不错的,混了这么多年,有钱有房有车,啥都不缺了。特别是他家的房子,就在现在北京的东厂胡同,二环里,黄金地段,交通便利,我常去附近的社科院近代史所开会,曾去看过,园林假山、深宅大院,上千平米,相当气派,但据说这只是当年他家的角落,最多也就六分之一。\\

从河北肃宁的一个小流氓,混到这个份上,也就差不多了,好歹有个留京指标。\\

但这个指标的有效期,也只有三天了。\\

天启七年(1627)十一月一日,崇祯下令,魏忠贤劳苦功高,另有重用——即日出发,去凤阳看坟。\\

得到消息的魏忠贤非常沮丧,但他不知道,崇祯也很沮丧。\\

崇祯是想干掉魏忠贤的,但无论如何,魏公公总算是三朝老监,前任刚死两个月,就干掉他实在不好意思。\\

但接下来发生的事情,却改变了他的决定。\\

当他宣布赶走魏忠贤的时候,有一个人站了出来,反对他的决定,而这个人,是他做梦都想不到的。\\

或许是收了钱,或许是说了情,反正徐应元是站出来了,公然为魏忠贤辩护,希望皇帝给他个面子。\\

面对这个伺候了自己十几年,一向忠心耿耿的老太监,崇祯毫不犹豫地做出了抉择:\\

“奴才!敢与奸臣相通,打一百棍,发南京!”\\

太监不是人啊。\\

顺便说一句,在明代,奴才是朝廷大多数太监的专用蔑呼,而在清代,奴才是朝廷大多数人的尊称(关系不好还不能叫,只能称臣,所谓做奴才而不可得)。\\

这件事情让崇祯意识到,魏忠贤是不会消停的。\\

而下一件事使他明白,魏忠贤是非杀不可的。\\

确定无法挽回,魏公公准备上路了,足足准备了三天。\\

在这三天里,他只干了一件事——打包。\\

既然荣华于我如浮云,那就只要富贵吧。\\

但这是一项相当艰苦的工作,几百个仆人干了六天,清出四十大车,然后光荣上路,前呼后拥,随行的,还有一千名隶属于他本人的骑兵护卫。\\

就算是轻度弱智的白痴,都知道现在是个什么状况,大难当头,竟然如此嚣张,真是活腻了。\\

魏忠贤没有活腻,他活不到九千九百岁,一百岁还是要追求的。\\

事实上,这个大张旗鼓的阵势,是他最后的诡计。\\

这个诡计的来由是历史。\\

历史告诉我们,战国的时候,秦军大将王翦出兵时,一边行军一边给秦王打报告,要官要钱,贪得无厌,有人问他,他说,我军权在手,只有这样,才能让秦王放心。\\

此后,这一招被包括萧何在内的广大仁人志士(识相点的)使用,魏忠贤用这招,说明他虽不识字,却还是懂得历史的。\\

可惜,是略懂。\\

魏公公的用意是,自己已经无权无势,只求回家过几天舒坦日子,这么大排场,只是想告诉崇祯老爷,俺不争了,打算好好过日子。\\

然而,他犯了一个错误——没学过历史唯物主义。\\

所谓历史唯物主义的要点,就是所有的历史事件,都要根据当时的历史环境来考虑。\\

王翦的招数能够奏效,是因为他手中有权,换句话说,他的行为,实际上是跟秦王签合同,我只要钱要官,帮你打江山,绝不动你的权。\\

此时的魏忠贤,已经无权无官,凭什么签合同?\\

所以崇祯很愤怒,他要把魏忠贤余下的都拿走,他的钱,还有他的命。\\

魏忠贤倒没有这个觉悟,他依然得意洋洋地出发了。\\

但聪明人还是有的,比如他的心腹太监李永贞,就曾对他说,低调,低调点好。\\

魏忠贤回答:\\

若要杀我,何须今日?\\

今日之前,还无须杀你。\\

魏忠贤出发后的第三天,崇祯传令兵部,发出了逮捕令。\\

这一天是十一月六日,魏忠贤所在的地点,是直隶河间府阜城县。\\

护卫簇拥的魏公公终于明白了自己的处境,几天来,他在京城的内线不断向他传递着好消息:他的亲信,包括五虎、五彪纷纷落马,老朋友王体乾退了,连费尽心思拉下水的徐应元也被发配去守陵,翻身已无指望。\\

就在他情绪最为低落的时候,京城的快马又告诉他一个最新的消息:皇帝已经派人追上来了。\\

威严的九千九百岁大人当场就晕了过去。\\

追上来,然后呢?逮捕,入狱,定罪,斩首?还是挨剐?\\

天色已晚,无论如何,先找个地方住吧,活过今天再说。\\

魏忠贤进入了眼前的这座小县城:他人生中的最后一站。\\

阜城县是个很小的县城,上千人一拥而入,挤满了所有的客店,当然,魏忠贤住的客店,是其中最好的。\\

为保证九千岁的人有地方住,许多住店的客人都被赶了出去,虽然天气很冷,但这无关紧要,毕竟他们都是无关紧要的人。在这些人中,有个姓白的书生,来自京城。\\

所谓最好的客店,也不过是几间破屋而已,屋内没有辉煌的灯光,十一月的天气非常的冷,无情的北风穿透房屋,发出凄冷的呼啸声。\\

在黑暗和寒冷中,伟大的,无与伦比的,不可一世的九千九百岁蜷缩在那张简陋的床上,回忆着过往的一切。\\

隆庆年间出生的无业游民,文盲,万历年间进宫的小杂役,天启年间的东厂提督,朝廷的掌控者,无数孙子的爷爷,生祠的主人,堪与孔子相比的圣人。\\

到而今,只剩破屋、冷床,孤身一人。\\

荒谬,究竟是自己,还是这个世界?\\

四十年间,不过一场梦幻。\\

不如死了吧。\\

此时,他的窗外,站立着那名姓白的书生。\\

在这个寒冷的夜晚,没有月光,在黑暗和风声中,书生开始吟唱。\\

夜半,歌起。\\

在史料中,这首歌的名字叫做《桂枝儿》,但它还有一个更贴切的名字——五更断魂曲。\\

曲分五段,从一更唱到五更:\\

\begin{quote}
	\begin{spacing}{0.5}  %行間距倍率
		\textit{{\footnotesize
				\begin{description}
					\item[\textcolor{Gray}{\FA }] 一更,愁起
					\item[\textcolor{Gray}{\FA }] 听初更,鼓正敲,心儿懊恼。
					\item[\textcolor{Gray}{\FA }] 想当初,开夜宴,何等奢豪。
					\item[\textcolor{Gray}{\FA }] 进羊羔,斟美酒,笙歌聒噪。
					\item[\textcolor{Gray}{\FA }] 如今寂廖荒店里,只好醉村醪。
					\item[\textcolor{Gray}{\FA }] 又怕酒淡愁浓也,怎把愁肠扫?
					\item[\textcolor{Gray}{\FA }] 二更,凄凉
					\item[\textcolor{Gray}{\FA }] 二更时,展转愁,梦儿难就。
					\item[\textcolor{Gray}{\FA }] 想当初,睡牙床,锦绣衾稠。
					\item[\textcolor{Gray}{\FA }] 如今芦为帷,土为坑,寒风入牖。
					\item[\textcolor{Gray}{\FA }] 壁穿寒月冷,檐浅夜蛩愁。
					\item[\textcolor{Gray}{\FA }] 可怜满枕凄凉也,重起绕房走。
					\item[\textcolor{Gray}{\FA }] 三更,飘零
					\item[\textcolor{Gray}{\FA }] 夜将中,鼓咚咚,更锣三下。
					\item[\textcolor{Gray}{\FA }] 梦才成,又惊觉,无限嗟呀。
					\item[\textcolor{Gray}{\FA }] 想当初,势顷朝,谁人不敬?
					\item[\textcolor{Gray}{\FA }] 九卿称晚辈,宰相为私衙。
					\item[\textcolor{Gray}{\FA }] 如今势去时衰也,零落如飘草。
					\item[\textcolor{Gray}{\FA }] 四更,无望
					\item[\textcolor{Gray}{\FA }] 城楼上,敲四鼓,星移斗转。
					\item[\textcolor{Gray}{\FA }] 思量起,当日里,蟒玉朝天。
					\item[\textcolor{Gray}{\FA }] 如今别龙楼,辞凤阁,凄凄孤馆。
					\item[\textcolor{Gray}{\FA }] 鸡声茅店里,月影草桥烟。
					\item[\textcolor{Gray}{\FA }] 真个目断长途也,一望一回远。
					\item[\textcolor{Gray}{\FA }] 五更,荒凉
					\item[\textcolor{Gray}{\FA }] 闹攘攘,人催起,五更天气。
					\item[\textcolor{Gray}{\FA }] 正寒冬,风凛冽,霜拂征衣。
					\item[\textcolor{Gray}{\FA }] 更何人,效殷勤,寒温彼此。
					\item[\textcolor{Gray}{\FA }] 随行的是寒月影,吆喝的是马声嘶。
					\item[\textcolor{Gray}{\FA }] 似这般荒凉也,真个不如死!
				\end{description}
		}}
	\end{spacing}
\end{quote}

五更已到,曲终,断魂。\\

多年后,史学家计六奇在他的书中记下了这个夜晚发生的一切,但这一段,在后来的史学研究中,是有争议的,就史学研究而言,如此诡异的景象,实在不像历史。\\

但我相信,在那个夜晚,我们所知的一切是真实的。\\

因为历史除了正襟危坐,一丝不苟外,有时也喜欢开开玩笑,算算总账。\\

至于那位姓白的书生,据说是河间府的秀才,之前为图嘴痛快,说了魏忠贤几句坏话,被人告发前途尽墨,于是编曲一首,等候于此不计旧恶,帮其送终。\\

但在那天夜里,魏忠贤听到的,不是这首曲子,而是他的一生。\\

想当初,开夜宴,何等奢豪。想当初,势顷朝,谁人不敬?\\

如今寂廖荒店里,只好醉村醪,如今势去时衰也,零落如飘草。\\

魏忠贤是不相信天道的。当无赖时,他强迫母亲改嫁,卖掉女儿,当太监时,他抢夺朋友的情人,出卖自己的恩人。\\

九千九百岁时,他泯灭一切人性,把铁钉钉入杨涟的脑门,把东林党赶尽杀绝。\\

他没有信仰,没有畏惧,没有顾忌。\\

然而天道是存在的,四十年后,他把魏忠贤送到了阜城县的这所破屋里。\\

这里距离魏公公的老家肃宁,只有几十里。四十年前,他经过这里,踏上了前往京城的路。\\

现在,他回来了,即将失去所有的一切。\\

我认为,这是一种别开生面的折腾,因为得到后再失去,远比一无所有要痛苦得多。\\

魏公公费尽心力,在成功的路上一路狂奔,最终却发现,是他娘的折返跑。\\

似这般荒凉也,真个不如死!\\

真个不如死啊!\\

那就死吧。\\

魏忠贤找到了布带,搭在了房梁上,伸进自己的脖子,离开了这个世界。\\

天道有常,或因人势而迟,然终不误。\\

\subsection{落水狗}
第二天早上,魏忠贤的心腹李朝钦醒来,发现魏忠贤已死,绝望之中,自缢而亡。\\

在魏忠贤的一千多陪同人员,几千朝廷死党里,他是唯一陪死的人。\\

得知魏忠贤的死讯后,一千多名护卫马上行动起来,瓜分了魏公公的财产,四散奔逃而去。\\

魏公公死了,但这场大戏才刚刚开始。\\

\begin{quote}
	\begin{spacing}{0.5}  %行間距倍率
		\textit{{\footnotesize
				\begin{description}
					\item[\textcolor{Gray}{\FA }] 别看今天闹得欢,当心将来拉清单。——小兵张嘎
				\end{description}
		}}
	\end{spacing}
\end{quote}

清单上的第一个人,自然是客氏。\\

虽然她已经离宫,但崇祯下令,把她又拎了进来。\\

进来后先审,但客氏为人极其阴毒,且以耍泼闻名,问什么都骂回去。\\

于是换人,换了个太监审,而且和魏忠贤有仇(估计是专门找来的),由于不算男人,也就谈不上不打女人,加上没文化,不会吵架,二话不说就往死里猛打。\\

客氏实在是个不折不扣的软货,一打就服,害死后妃,让皇后流产,找孕妇入宫冒充皇子,出主意害人等等,统统交代,只求别打。\\

但那位太监似乎心理有点问题,坦白交代还打,直到奄奄一息才罢休。\\

口供报上来,崇祯十分震惊,下令将客氏送往浣衣局做苦工。\\

当然了,这只是个说法,客氏刚进浣衣局,还没分配工作,就被乱棍打死,跟那位被她关入冷宫,活活渴死的后妃相比,这种死法没准还算痛快点。\\

客氏死后,她的儿子被处斩,全家被发配。\\

按身份排,下一个应该是崔呈秀。\\

但是这位兄弟实在太过自觉,自觉到死得比魏公公还要早。\\

得知魏忠贤走人的消息后,崔呈秀下令,准备一桌酒菜,开饭。\\

吃饭的方式很特别,和韦小宝一样,他把自己大小老婆都拉出来,搞了个聚餐,还摆上了多年来四处搜刮的古玩财宝。\\

然后一边吃,一边拿起他的瓶瓶罐罐(古董),砸。\\

吃一口,砸一个,吃完,砸完,就开始哭。\\

哭好,就上吊。\\

按日期推算,这一天,魏忠贤正在前往阜城县的路上。\\

兄弟先走一步。\\

消息传到京城,崇祯非常气愤,老子没让你死,你就敢死?\\

随即批示:\\

“虽死尚有余辜!论罪!”\\

经过刑部商议,崔呈秀应该斩首。\\

虽然人已死了,不要紧,有办法。\\

于是刚死不久的崔呈秀又被挖了出来,被斩首示众,怎么杀是个能力问题,杀不杀是个态度问题。\\

接下来是抄家,无恶不作的崔呈秀,终于为人民做了件有意义的事,由于他多年来勤奋地贪污受贿,存了很多钱,除动产外,还有不动产,光房子就有几千间,等同于替国家攒钱,免去了政府很多麻烦。\\

作为名单上的第三号人物,崔呈秀受到了高标准的接待,以此为基准,一号魏忠贤和二号客氏,接待标准应参照处理。\\

所以,魏忠贤和客氏被翻了出来,客氏的尸体斩首,所谓死无全尸。\\

魏忠贤惨点,按崇祯的处理意见,挖出来后剐了,死后凌迟,割了几千刀。\\

这件事情的实际意义是有限的,最多也就是魏公公进了地府,小鬼认不出他,但教育意义是巨大的,在残缺的尸体面前,明代有史以来最大,最邪恶的政治团体阉党,终于彻底崩盘。\\

接下来的场景,是可以作为喜剧素材的。\\

魏忠贤得势的时候,无数人前来投奔,上至六部尚书,大学士,下到地方知府知县,能拉上关系,就是千恩万谢。\\

现在而今眼目下,没办法了,能撤就撤,不能撤就推,比如蓟辽总督阎鸣泰,有一项绝技——修生祠,据我统计,他修的生祠有十余个,遍布京城一带,有的还修到了关外,估计是打算让皇太极也体验一下魏公公的伟大光辉。\\

凭借此绝活,当年很是风光,现在麻烦了,追查阉党,头一个就查生祠,谁让修的,谁出的钱,生祠上都刻着,跑都跑不掉。\\

为证明自己的清白,阎总督上疏,进行了耐心的说明,虽说生祠很多,但还是可以解释的,如保定的生祠,是顺天巡抚刘诏修的,通州的生祠,是御史梁梦环修的,这些人都是我的下级,作为上级领导,责任是有的,监督不够是有的,检讨是可以的,撤职坐牢是不可以的。\\

但最逗的还是那位国子监的陆万龄同学,本来是一穷孩子,卖力捧魏公公,希望能够混碗饭吃,当年也是风光一时,连国子监的几位校长都争相支持他,陆先生本人也颇为得意。\\

然而学校领导毕竟水平高,魏公公刚走,就翻脸了,立马上疏,表示国子监本与魏忠贤势不两立,出了陆万龄这种败类,实在是教育界的耻辱,将他立即开除出校。\\

据统计,自天启七年(1627)十一月至次年二月,几个月里,朝廷的公文数量增加了数倍,各地奏疏纷至沓来,堪称数十年未有之盛况。\\

这些奏疏字迹相当工整,包装相当精美,内容相当扯淡:上来就痛骂魏忠贤,痛骂阉党,顺便检举某些同事的无耻行径,最后总结:他们的行为让我很愤怒,跟我不相干。\\

心中千言万语化为一句话:我不是阉党,皇帝大人,您就把我们当个屁放了吧。\\

效果很明显,魏忠贤倒台一个月里,崇祯毫无动静,除客氏崔呈秀外,大家过得都还不错。\\

事实上,当时的朝廷,大学士、六部尚书、都察院乃至于全国各级地方机构,都由阉党掌握,所谓法不责众,大家都有份,你能把大家都拉下水吗?把我们都抓了,找谁帮你干活?\\

所以,在阉党同志们看来,该怎么干还怎么干,该怎么活还怎么活。\\

这个看法在大多数人的身上,是管用的。\\

而崇祯,属于少数派。\\

一直以来,崇祯处理问题的理念比较简单,就四个字——斩草除根。所谓法不责众,在他那里是不成问题的,因为他的祖宗有处理这种问题的经验。\\

比如朱元璋,胡惟庸案件,报上来同党一万人,杀,两万人,杀杀,三万人,杀杀杀。无非多说几个杀字,不费劲。\\

时代进步了,社会文明了,道理还一样。\\

六部尚书是阉党,就撤尚书,侍郎是阉党,就撤侍郎,一半人是阉党,就撤一半,全是,就全撤,大明没了你们就不转吗?这年头,看门的狗难找,想当官的人有的是,谁怕谁!\\

值得一提的是,虽然上述奏疏内容雷同,但崇祯的态度是很认真的,他不但看了,而且还保存下来。\\

很简单,真没事的人是不会写这些东西的,原本找不着阉党,照着奏疏抓人,贼准。\\

十一月底,准备工作就绪,正式动手。\\

最先处理的,是魏忠贤的家属,比如他侄子魏良卿,屁都不懂的蠢人,也封到公爵了(宁国公),还有客氏的儿子候国兴(锦衣卫都指挥使),统统拉出去剁了。\\

接下来,是他的亲信太监,毕竟大家生理结构相似,且狼狈为奸,算半亲戚,优先处理。\\

这拨人总共有四个,分别是司礼监掌印太监王体乾,秉笔太监李永贞、李朝钦、刘若愚。\\

作为头等罪犯,这四位按说都该杀头,可到最后,却只死了两个,杀了一个。\\

第一个死的是李朝钦,他是跟着魏忠贤上吊的,并非他杀,算自杀。\\

唯一被他杀的,是李永贞。其实这位兄弟相当机灵,早在九月底,魏公公尚且得意的时候,他就嗅出了风声,连班都不上了,开始在家修碉堡,把院子封得严严实实,只留小洞送饭,每天窝在里面,打死也不出头。\\

坚持到底,就是胜利。\\

李永贞没有看到胜利的一天,到了十月底,他听说魏忠贤走人了,顿时大喜,就把墙拆了,出来放风。\\

刚高兴几天,又听到消息,皇帝要收拾魏公公了,慌了,再修碉堡也没用了。\\

于是他使出了绝招——行贿。\\

当然,行贿崇祯是不管用的,他拿出十余万两银子(以当时市价,合人民币六千万至八千万),送给了崇祯身边的贴身太监,包括徐应元和王体乾。\\

这两人都收了。\\

不久后,他得到消息,徐应元被崇祯免了,而王体乾把他卖了。\\

在名列死亡名单的这四位死太监中,最神秘的,莫过于王体乾了。\\

此人是魏忠贤的铁杆,害死王安,迫害东林党,都有他忙碌的身影,是阉党的首脑人物。\\

但奇怪的是,当我翻阅几百年前那份阉党的最终定罪结果时,却惊奇地发现,以他的丰功劣迹,竟然只排七等(共有八等),罪名是谄附拥戴,连罚款都没交,就给放了。\\

伺候崇祯十几年的徐应元,光说了几句话,定罪比他还高(五等),这个看上去很难理解的现象,有一个简单的答案:王体乾叛变了。\\

据史料分析,王体乾可能很早就“起义”了,所以一直以来,崇祯对魏忠贤的心理活动、斗争策略都了如指掌,当了这么久卧底,也该歇歇了。\\

所以他钱照收,状照告,第二天就汇报了崇祯,李永贞得知后,决定逃跑。\\

跑吧,大明天下,还能跑去非洲不成?\\

十几天后,他被抓捕归案。\\

进了号子,李太监还不安分,打算自杀,他很有勇气地自杀了四次,却很蹊跷地四次都没死成,最后还是被拉到刑场,一刀了断。\\

名单上最后一位,就是刘若愚了。\\

这位仁兄,应该是最有死相的,早年加入阉党,一直是心腹,坏事全干过,不是卧底,不是叛徒,坦白交代,主动退赃之类的法定情节一点没有,不死是不可能的。\\

可他没死。\\

因为刘若愚虽然罪大恶极,但这个人有个特点:能写。\\

在此之前,阉党的大部分文件,全部出于他手,换句话说,他算是个技术人员,而且他知道很多情况,所以崇祯把他留了下来,写交代材料。刘太监很敬业,圆满地完成了这个任务,他所写的《酌中志》,成为后代研究魏忠贤的最重要史料。\\

只要仔细阅读水浒传,就会发现,梁山好汉们招安后,宋江死了,最能打的李逵死了,最聪明的吴用也死了,活下来的,大都是身上有门手艺的,比如神医安道全之流。\\

以上事实清楚地告诉我们,平时学一门技术是多么的重要。\\

处理完人妖后,接下来的就是人渣了,主要是“五虎”和“五彪”。\\

五虎是文臣,分别是(排名分先后):兵部尚书崔呈秀、原兵部尚书田吉、工部尚书吴淳夫、太常寺卿倪文焕、副都御史李燮龙。\\

五彪是武官,分别是:左都督田尔耕、锦衣卫指挥许显纯、都督同知崔应元、右都督孙云鹤、锦衣卫佥事杨寰。\\

关于这十个人,就不多说了,其光辉事迹,不胜枚举,比如田尔耕,是迫害“六君子”的主谋,并杀害了左光斗等人,而许显纯大人,曾亲自把钉子钉进杨涟脑门。用今天的话说,足够枪毙几个来回。\\

因为此十人一贯为非作歹,民愤极大,崇祯下令,将其逮捕,送交司法部门处理。\\

经刑部、都察院调查,并详细会审,结果如下:\\

崔呈秀已死,不再追究,其他九人中,田尔耕、许显纯曾参与调查杨涟、左光斗等人的罪行,结果过失致人死亡,入狱,剩余七人免官为民,就此结案。\\

这份判决只能用一个词来形容——恬不知耻。\\

崇祯很不满意,随即下令,再审。\\

皇帝表态,不敢怠慢,经过再次认真细致的审讯,重新定罪如下:\\

以上十人,除崔呈秀已死外,田尔耕、许显纯因为过失致人死亡,判处死缓,关入监狱,其余七人全部充军,充军地点是离其住处最近的卫所。\\

鉴于有群众反应,以上几人有贪污罪行,为显示威严,震慑罪犯,同时处以大额罚款,分别是倪文焕五千两,吴淳夫三千两,李燮龙、田吉各一千两。结案。\\

报上去后,崇祯怒了。\\

拿钉子钉耳朵,打碎全身肋骨,是过失致人死亡,贪了这么多年,只罚五千、三千,你以为老子好哄是吧。\\

更奇怪的是,案子都判了,有些当事人根本就没到案,比如田吉,每天还出去遛弯,十分逍遥。\\

其实案子审成这样,是再正常不过的事了。\\

审讯此案的,是刑部尚书苏茂相、都察院左都御史曹思诚。\\

苏茂相是阉党,曹思诚也是阉党。\\

让阉党审阉党,确实难为他了。\\

愤怒之余,崇祯换人了,他把查处阉党的任务交给了吏部尚书王永光。\\

可王永光比前两位更逗,命令下来他死都不去,说自己能力有限,无法承担任务。\\

因为王永光同志虽然不是阉党,也不想得罪阉党。\\

按苏茂相、曹思诚、王永光以及无数阉党们的想法,形势是很好的,朝廷内外都是阉党,案子没人敢审,对五虎、五彪的处理,可以慢慢拖,实在不行,就判田尔耕和许显纯死刑,其他的人能放就放,不能放,判个充军也就差不多了。\\

没错,司法部长、监察部长、人事部长都不审,那就只有皇帝审了。\\

几天后,崇祯直接宣布了对五虎五彪的裁定,相比前两次裁决,比较简单:\\

田吉,杀!吴淳夫,杀!倪文焕,杀!田尔耕,杀!许显纯,杀!崔应元,杀!孙云鹤,杀!杨寰,杀!李燮龙,杀!\\

崔呈秀,已死,挖出来,戳尸!\\

以上十人,全部抄家!没收全部财产!\\

什么致人死亡,什么入狱,什么充军,还他娘就近,什么追赃五千两,都去死吧!\\

曹思诚、苏茂相这帮等阉党本来还有点想法,打算说两句,才发现,原来崇祯还没说完。\\

“左都御史曹思诚,阉党,免职查办!”\\

“刑部尚书苏茂相,免职!”\\

跟我玩,玩死你们!\\

随即,崇祯下令,由乔允升接任刑部尚书,大学士韩旷、钱龙锡主办此案,务必追查到底,宁可抓错,不可放过。\\

挑出上面这几个人办事,也算煞费苦心,乔允升和阉党向来势不两立,韩旷这种老牌东林党,不往死里整,实在对不起自己。\\

扫荡,一个不留!\\

几天过去,经过清查,内阁上报了阉党名单,共计五十多人,成果极其丰硕。\\

然而这一次,崇祯先更为愤怒,他当即召集内阁,严厉训斥:人还不够数,老实点!\\

大臣们都很诧异,都五十多了,还不够吗?\\

既然皇上说不够,那就再捞几个吧。\\

第二天,内阁又送上了一份名单,这次是六十几个,该满意了吧。\\

这次皇帝大人没有废话,一拍桌子:人数不对,再敢糊弄我,以抗旨论处!\\

崇祯是正确的,内阁的这几位仁兄,确实糊弄了他。\\

虽然他们跟阉党都有仇,且皇帝支持,但阉党人数太多,毕竟是个得罪人的事,阉党也好,东林党也罢,不过混碗饭吃,何必呢?\\

不管了,接着糊弄:\\

“我们是外臣,宫内的人事并不清楚。”\\

崇祯冷笑:\\

“我看不是不知道,是怕得罪人吧(特畏任怨耳)!”\\

怪事,崇祯初来乍到,他怎么知道人数不对呢?\\

崇祯帮他们解开了这个迷题。\\

他派人抬出了几个包裹,扔到阁臣面前,说:\\

“看看吧。”\\

打开包裹的那一刻,大臣们明白,这次赖都赖不掉了。\\

包裹里的,是无数封跟魏忠贤勾搭的奏疏,很明显,崇祯不但看过,还数过。\\

混不过去,只能玩命干了。\\

就这样,自天启七年(1627)十二月,一直到崇祯元年(1628)三月,足足折腾了四个月,阉党终于被彻底整趴下了。\\

最后的名单,共计二百六十一人,分为八等。\\

特等奖得主两人,魏忠贤,客氏,罪名:首逆,处理:凌迟。\\

一等奖得主六人,以崔呈秀为首,罪名:首逆同谋,处理:斩首。\\

二等奖得主十九人,罪名:结交近侍,处理:秋后处决。\\

三等奖得主十一人,罪名:结交近侍次等,处理:流放。\\

此外,还有四等奖得主(逆孽军犯)三十五人,五等奖得主(谄附拥戴军犯)十六人,六等奖得主(交结近侍又次等)一百二十八人,七等奖得主(祠颂)四十四人,各获得充军、有期徒刑、免职等奖励。\\

以上得奖结果,由大明北京市公证员朱由检同志公证,有效。\\

对此名单,许多史书都颇有微辞,说是人没抓够,放跑了某些阉党,讲这种话的人,脑袋是有问题的。\\

我算了一下,当时朝廷的编制,六部只有一个部长,两个副部长(兵部有四个),每个部有四个司(刑部和户部有十三个),每个司司长(郎中)一人,副司长(员外郎)一人,处长(主事)两人。\\

还有大衙门都察院,加上各地御史,才一百五十人,其余部门人数更少,总共(没算地方政府)大致不会超过八百人。\\

人就这么多,一下子刨走两百六十多,还不算多?\\

其实人家也是有苦衷的,毕竟魏公公当政,不说几句好话,是混不过去的,现在换了领导,承认了错误,也就拉倒了吧。\\

然而崇祯不肯拉倒,不只他不肯,某些人也不肯。\\

这个某些人,是指负责定案的人。\\

大家在朝廷里,平时你来我往,难免有点过节,现在笔在手上,说你是阉党,你就是阉党,大好挖坑机会,不整一下,难免有点说不过去。\\

比如大学士韩旷,清查阉党毫不积极,整人倒是毫不含糊,骂过魏公公的,不一定不是阉党,骂过他的,就一定是阉党,写进去!\\

更搞笑的是,由于人多文书多,某些兄弟被摆了乌龙,明明当年骂的是张居正,竟然被记成了东林党,两笔下去就成了阉党,只能认倒霉。\\

此外,在这份名单上,还有几位有趣的人物。比如那位要在国子监里给魏公公立牌坊的陆万龄同学,屁官都不是,估计连魏忠贤都没见过,由于风头太大,竟然被订为二等,跟五虎五彪一起,被拉出去砍了。\\

那位第一个上疏弹劾魏公公的杨维垣,由于举报有功,被定为三等,拉去充军。\\

而在案中扮演了滑稽角色的陈尔翼、杨所修,也没能跑掉,根据情节,本来没他们什么事,鉴于其双簧演得太过精彩,由皇帝特批六等奖,判处有期徒刑,免官为民。\\

\subsection{复仇}
总体说来,这份名单虽然有点问题,但是相当凑合,弘扬了正气,恶整了恶人,虽然没有不冤枉一个好人,也没有放过大多数坏人,史称“钦定逆案”。\\

其实崇祯和魏忠贤无仇,办案子,无非是魏公公挡道,皇帝看不顺眼,干掉了。\\

但某些人就不同了。干掉是不够的,死了的人挫骨扬灰,活着的人赶尽杀绝,才算够本!\\

黄宗羲就是某些人中的优秀代表。\\

作为“七君子”中黄遵素的长子,黄宗羲可谓天赋异禀,不但精通儒学,还懂得算术、天文。据说天上飞的,地上跑的,没有他不知道的,被称为三百年来学术之集大成者,与顾炎武、王夫之并称。\\

更让人无语的是,黄宗羲还懂得经济学,他经过研究发现,每次农业税法调整,无论是两税法还是一条鞭法,无论动机如何善良,最终都导致税收增加,农民负担加重,换句话说,不管怎么变,最终都是加。\\

这一原理后被社科院教授秦晖总结,命名为“黄宗羲定律”,中华人民共和国国务院经过调研,采纳这一定律,于2006年彻底废除了农业税,打破了这个怪圈。\\

善莫大焉。\\

但这四个字放在当时的黄宗羲身上,是不大恰当的,因为他既不善良,也不大度。\\

当时恰好朝廷审讯许显纯,要找人作证,就找来了黄宗羲。\\

事情就是这么闹起来的。\\

许显纯此人,说是死有余辜,还真是有余辜,拿锤子砸人的肋骨,用钉子钉人耳朵,钉人的脑袋,六君子、七君子,大都死在他的手中,为人恶毒,且有心理变态的倾向。\\

此人向来冷酷无情,没人敢惹,杨涟如此强硬,许先生毫不怯场,敢啃硬骨头,亲自上阵,很有几分硬汉色彩。\\

但让人失望的是,轮到这位变态硬汉入狱,当场就怂了,立即展现出了只会打人,不会被人打的特长。\\

他全然没有之前杨涟的骨气,别说拿钉子顶脑门,给他几巴掌,立马就晕,真是窝囊死了。\\

值得庆幸的是,崇祯的监狱还比较文明,至少比许显纯在的时候文明,打是打,但锤子、钉子之类的东西是不用的,照此情形,审完后一刀了事,算是便宜了他。\\

但便宜不是那么容易找到的。\\

审讯开始,先传许显纯,以及同案犯“五彪”之一的崔应元,然后传黄宗羲。\\

黄宗羲上堂,看见仇人倒不生气,表现得相当平静,回话,作证,整套程序走完,人不走。\\

大家很奇怪,都看着他。\\

别急,先不走,好戏刚刚开场。\\

黄宗羲来的时候,除了他那张作证的嘴外,还带了一件东西——锥子。\\

审讯完毕,他二话不说,操起锥子,就奔许显纯来了。\\

这一刻,许显纯表现出了难得的单纯,他不知道审案期间拿锥子能有啥用,只是呆呆地看着急奔过来的黄宗羲,等待着他的答案。\\

答案是一声惨叫。\\

黄宗羲终于露出了狰狞面目,手持锥子,疯狂地朝许显纯身上戳,而许显纯也不愧孬种本色,当场求饶,并满地打滚,开始放声惨叫。\\

许先生之所以大叫,是有如意算盘的:这里毕竟是刑部大堂,众目睽睽之下,难道你们都能看着他殴打犯人吗?\\

答案是能。\\

无论是主审官还是陪审人员,没有一个人动手,也没有人上前阻拦,大家都饶有兴致的看着眼前的这一幕,黄宗羲不停地扎,许显纯不停地喊,就如同电视剧里最老套的台词:你喊吧,就是喊破喉咙也不会有人来救你!\\

因为所有人都记得,这个人曾经把钢钉扎进杨涟的耳朵和脑门,那时,没有人阻止他。\\

但形势开始变化了,许显纯的声音越来越小,鲜血横流,黄宗羲却越扎越起劲,如此下去,许先生被扎死,黄宗羲是过瘾了,黑锅得大家背。\\

于是许显纯被拉走,黄宗羲被拉开,他的锥子也被没收。\\

审完了,仇报了,气出了,该消停了。\\

黄宗羲却不这么认为,他转头,又奔着崔应元去了。\\

其实这次审讯,崔应元是陪审,无奈碰上了黄恶棍,虽然没挨锥子,却被一顿拳打脚踢,鼻青脸肿。\\

到此境地,主审官终于认定,应该把黄宗羲赶走了,就派人上前把他拉开,但黄宗羲打上了瘾,被人拉走之前,竟然抓住了崔应元的胡子,活生生地拔了下来!\\

当年在狱中狂施暴行的许显纯,终于尝到了暴行的滋味,等待着他的,是最后的一刀。\\

什么样的屠夫,最终也只是懦夫。\\

如许显纯等人,都是钦定名单要死的,而那些没死的,似乎还不如死了的好。\\

比如阉党骨干,太仆寺少卿曹钦程,好不容易捡了条命,回家养老,结果所到之处,都是口水(民争唾其面),实在呆不下去,跑到异地他乡买了个房子住,结果被人打听出来,又是一顿猛打,赶走了。\\

还有老牌阉党顾秉谦,家乡人对他的感情可谓深厚,魏忠贤刚倒台,人民群众就冲进家门,烧光了他家,顾秉谦跑到外地,没人肯接待他,最后在唾骂声中死去。\\

而那些名单上没有,却又应该死的,也没有逃过去。比如黄宗羲,他痛殴许显纯后,又派人找到了当年杀死他父亲的两个看守,把他们干掉了。\\

大明是法制社会,但凡干掉某人,要么有司法部门批准,要么偿命,但黄宗羲自己找人干了这俩看守,似乎也没人管,真是没王法了。\\

黄宗羲这么一闹,接下来就热闹了,所谓“六君子”、“七君子”,都是有儿子的。\\

先是魏大中的儿子魏学濂上书,要为父亲魏大中伸冤,然后是杨涟的儿子杨之易上书,为父亲杨涟伸冤,几天后,周顺昌的儿子周茂兰又上书,为父亲周顺昌伸冤。\\

顺便说一句,以上这几位的上书,所用的并非笔墨,而是一种特别的材料——血。\\

这也是有讲究的,自古以来,但凡奇冤都写血书,不用似乎不够分量。\\

但崇祯同志就不干了,拿上来都是血迹斑斑的东西,实在有点发怵,随即下令:你们的冤情我都知道,但上奏的文书是用墨写的,用血写不合规范,今后严禁再写血书。\\

但他还是讲道理的,崇祯二年(1629)九月,他下令,为殉难的东林党人恢复名誉,追授官职,并加封谥号。\\

杨涟得到的谥号,是忠烈,以此二字,足以慨其一生。\\

至此,为祸七年之久的阉党之乱终于落下帷幕,大明有史以来最强大,最邪恶的势力就此倒台。纵使它曾骄横一时,纵使它曾不可一世。\\

迟来的正义依然是正义。\\

在这个世界上,所谓神灵、天命,对魏忠贤而言,都是放屁,在他的身上,只有一样东西——迷信。\\

不信道德,不信仁义,不信报应,不信邪不胜正。\\

迷信自己,迷信力量,迷信权威,迷信可以为所欲为,迷信将取得永远的胜利。\\

而在遍览史书十余载后,我信了,至少信一样东西——天道。\\

自然界从诞生的那刻起,就有了永恒的规律,春天成长,冬天凋谢,周而复始。\\

人世间也一样,从它的起始,到它的灭亡,规则恒久不变,是为天道。\\

在史书中无数的尸山血河、生生死死背后,我看到了它,它始终在那里,静静地注视着我们,无论兴衰更替,无论岁月流逝。\\

它告诉我,在这个污秽、混乱、肮脏的世界上,公道和正义终究是存在的。\\

天道有常,从它的起始,到它的灭亡,恒久不变。\\
\ifnum\theparacolNo=2
	\end{multicols}
\fi
\newpage
\section{复起}
\ifnum\theparacolNo=2
	\begin{multicols}{\theparacolNo}
\fi
崇祯是一个很有想法的人,很想有番作为,但当他真正站在权力的顶峰时,却没有看到风景,只有一片废墟。\\

史书有云:明之亡,亡于天启。也有史书云:实亡于万历。还有史书云:始亡于嘉靖。\\

应该说,这几句话都是有道理的,经过他哥哥、他爷爷、他爷爷的爷爷几番折腾,已经差不多了,加上又蹦出来个九千岁人妖,里外一顿猛捶,大明公司就剩一口气了。\\

朝廷纷争不断,朝政无人理会,边疆烽火连天,百姓民不聊生,干柴已备,只差一把火。\\

救火员崇祯登场。\\

他浇的第一盆水,叫做袁崇焕。\\

崇祯是很喜欢袁崇焕的,因为他起用袁崇焕的时间,是天启七年(1627)十一月十九日。\\

此时,魏忠贤刚死十三天,尸体都还没烂。\\

几天后,在老家东莞数星星的袁崇焕接到了复起任职通知,大吃一惊。\\

吃惊的不是复起,而是职务。\\

袁崇焕当时的身份是平民,按惯例,复起也得有个级别,先干个主事(处级),过段时间再提,比较合理。\\

然而他接受的第一个职务,是都察院右都御史,兵部左侍郎。\\

兵部右侍郎,是兵部副部长,都察院右都御史,是二品正部级,也就是说,在一天之内,布衣袁崇焕就变成了正部级副部长。\\

袁部长明显没缓过劲来,在家呆了几个月,啥事都没干,却又等来了第二道任职令。\\

这一次,他的职务变成了兵部尚书,督师蓟辽。\\

明代有史以来,最不可思议的任职令诞生了。\\

因为兵部尚书,督师蓟辽,是一个很大的官,很大。\\

所谓兵部尚书就是国防部部长,很牛,但最牛的官职,是后四个字——督师蓟辽。\\

我之前曾经说过,明代的地方官,最大的是布政使、按察使和指挥使,为防互相扯皮,由中央下派特派员统一管理,即为巡抚。\\

鉴于后期经营不善,巡抚只管一个地方,也摆不平,就派高级特派员管理巡抚,即为总督。\\

到了天启崇祯,局势太乱,连总督都搞不定了,就派特级特派员,比总督还大,即为督师。\\

换句话说,督师是明代除皇帝外,管辖地方权力最大的官员。\\

而要当巡抚、总督、督师的条件,也是不同的。\\

要当巡抚,至少混到都察院佥都御史(四品正厅级)或是六部侍郎(副部级),才有资格。\\

而担任总督的,一般都是都察院都御史(二品部级),或是六部尚书(部长)。\\

明代最高级别的干部,就是部级,所以能当上督师的,只剩下一种人——内阁大学士。\\

比如之前的孙承宗,后来的杨嗣昌,都是大学士督师。\\

袁崇焕例外。\\

就在几个月前,他还只是袁百姓,几月后,他就成了袁尚书,还破格当上了督师,而袁督师的管辖范围包括蓟州、辽东、登州、天津、莱州等地,换句话说,袁督师手下,有五六个巡抚。\\

任职令同时告知,立刻启程,赶到京城,皇帝急着见你。\\

崇祯确实急着见袁崇焕,因为此时的辽东,已经出现了一个更为强大的敌人。\\

自从被袁崇焕打跑后,皇太极始终很消停,他没有继续用兵,却开始了不同寻常的举动。\\

皇太极和他老爹不同,从某种角度讲,努尔哈赤算半个野蛮人,打仗,占了地方就杀,不杀的拉回来做奴隶,给贵族当畜牲使,在后金当官的汉人,只能埋头干活,不能骑马,不能养牲口,活着还好,要是死了,老婆就得没收,送到贵族家当奴隶。\\

相比而言,皇太极很文明,他尊重汉族习惯,不乱杀人,讲信用,特别是对汉族前来投奔的官员,那是相当的客气,还经常赏赐财物。\\

总而言之,他很温和。\\

温和文明的皇太极,是一个比野蛮挥刀的努尔哈赤更为可怕的敌人。\\

张牙舞爪的人,往往是脆弱的,因为真正强大的人,是自信的,自信就会温和,温和就会坚定。\\

无需暴力,无需杀戮,因为温和,才是最高层次的暴力。\\

在皇太极的政策指引下,后金领地逐渐安定,经济开始发展稳固,而某些在明朝混不下去的人,也开始跑去讨生活,这其中最典型的人物,就是范文程。\\

每次说到这个人,我都要呸一口,呸。\\

呸完了,接着说。\\

说起汉奸,全国人民就会马上想起吴三桂,但客观地讲,吴三桂当汉奸还算情况所迫,范文程就不同了,他是自动前去投奔,出卖自己同胞的,属于汉奸的最原始,最无耻形态。\\

他原本是个举人(另说是秀才),因为在大明混得不好,就投了皇太极,在此后几十年的汉奸生涯中,他起了极坏的作用,更讽刺的是,据说他还有个光荣的嫡系祖先——范仲淹。\\

想当年,范仲淹同志在宋朝艰苦奋斗,抗击西夏,如在天有灵,估计是要改家谱的。不过自古以来,爷爷是好汉,孙子哭着喊着偏要当汉奸的,实在太多,古代有古代的汉奸,现在有现代的汉奸,此所谓汉奸恒久远,遗臭永流传。\\

在范文程的帮助下,皇太极建立了朝廷(完全仿照明朝),开始组建国家机器,进行奴隶制改造,为进入封建社会而努力。\\

要对付这个可怕的敌人,必须立刻采取行动。\\

在紫禁城里的平台,怀着憧憬和希望,皇帝陛下第一次见到了袁崇焕。\\

这是一次十分重要的召见,史称平台召对。\\

他们见面的那一天,是崇祯元年(1628)七月十四日。\\

顺便说一句,由于本人数学不好,在我以上叙述的所有史实中,日期都是依照原始史料,使用阴历。而如果我没记错的话,阴历七月十四日,是鬼节。\\

七月十四,鬼门大开,阴风四起。\\

那天有没有鬼出来我不知道,但当天的这场谈话,确实比较鬼。\\

谈话开始,崇祯先客套,狠狠地夸奖袁崇焕,把袁督师说得心潮澎湃,此起彼伏,于是,袁督师激动地说出了下面的话:\\

“计五年,全辽可复。”\\

这句话的意思是,五年时间,我就能恢复辽东,彻底解决皇太极。\\

这下吹大发了。\\

百年之后的清朝史官们,在经过时间的磨砺和洗礼后,选出了此时此刻,唯一能够挽救危局的人,并给予了公正的评价。\\

但这个人不是袁崇焕,而是孙承宗。\\

翻阅了上千万字的明代史料后,我认为,这个判断是客观的。\\

袁崇焕是一个优秀的战术实施者,一个坚定的战斗执行者,但他并不是一个卓越的战略制定者。\\

而从他此后的表现看,他也不是一个能正确认识自己的人。\\

所有的悲剧,即由此言而起。\\

崇祯很兴奋,兴奋得连声夸奖袁崇焕,说你只要给我好好干,我也不吝惜赏赐,旁边大臣也猛添柴火,欢呼雀跃,气氛如此热烈,以至于皇帝陛下决定,休会。\\

但脑袋清醒的人还是有的,比如兵科给事中许誉卿。\\

他抱着学习的态度,找到了袁崇焕,向他讨教如何五年平辽。\\

照许先生的想法,袁督师的计划应该非常严密。\\

然而袁崇焕的回答只有四个字:聊慰上意!\\

翻译过来就是,随口说说,安慰皇上的。\\

差点拿笔做笔记的许誉卿当时就傻了。\\

他立刻小声(怕旁边人听见)地对袁崇焕说:\\

“上英明,岂可浪对?异日按期责功,奈何?”\\

这句话意思是,皇上固然不懂业务,但是比较较真,现在忽悠他,到时候他按日期验收工作,你怎么办?\\

袁督师的反应,史书上用了四个字:怃然自失。\\

没事,牛吹过了,就往回拉。\\

于是,当崇祯第二次出场的时候,袁督师就开始提要求了。\\

首先是钱粮,要求户部支持,武器装备,要求工部支持。\\

然后是人事,用兵、选将,吏部、兵部不得干涉,全力支持。\\

最后是言官,我在外打仗,言官唧唧喳喳难免,不要让他们烦我。\\

以上要求全部得到了满足,立即。\\

崇祯是个很认真的人,他马上召集六部尚书,开了现场办公会,逐个落实,保证兑现。\\

会议就此结束,双方各致问候,散伙。\\

在这场召对中,崇祯是很真诚的,袁崇焕是很不真诚的,因为当时的辽东局势已成定论,后金连衙门都修起来了,能够守住就算不错,你看崇祯兄才刚二十,又不懂业务,就敢糊弄他,是很不厚道的。\\

就这样,袁崇焕胸怀五年平辽的口号,在崇祯期望的目光中,走向了辽东。\\

可他刚走到半路,就有人告诉他,你不用去了,去了也没兵。\\

就在他被皇帝召见的十天后,宁远发生了兵变。\\

兵变的原因,是不发工资。\\

我曾翻阅过明代户部记录,惊奇地发现,明朝的财政制度,是非常奇特的,因为几乎所有的地方政府,竟然都没有行政拨款。也就是说,地方办公经费,除老少边穷地区外,朝廷是不管的,自己去挣,挣得多就多花,挣得少就少花,挣不到就滚蛋。\\

而明朝财政收入的百分之八十,都用在了同一个地方——军费。\\

什么军饷、粮草、衣物,打赢了有赏钱,打输了有补偿,打死了有安家费,再加上个别不地道的人吃空额,扣奖金,几乎每年都不够用。\\

宁远的情况大致如此,由于财政困难,已经连续四个月没有发工资。\\

要知道,明朝拖欠军饷和拖欠工钱是不一样的,不给工资,最多就去衙门告你,让你吃官司,不给军饷,就让你吃大刀。\\

最先吃苦头的,是辽东巡抚毕自肃,兵变发生时,他正在衙门审案,还没反应过来,就被绑成了粽子,关进了牢房,和他一起被抓的,还有宁远总兵朱梅。\\

抓起来就一件事,要钱,可惜的是,翻遍巡抚衙门,竟然一文钱没有。\\

其实毕自肃同志,确实是个很自肃的人,为发饷的事情,几次找户部要钱,讽刺的是,户部尚书的名字叫做毕自严,是他的哥哥,关系铁到这个份上,都没要到钱,可见是真没办法了。\\

但苦大兵不管这个,干活就得发工钱,不发工钱就干你,毕大人最先遭殃,被打得遍体鳞伤,奄奄一息,关键时刻部下赶到,说你们把他打死也没用,不如把人留着,我去筹钱。\\

就这样,兵变弄成了绑票,东拼西凑,找来两万银子,当兵的不干,又要闹事,无奈之下,巡抚衙门主动出面,以政府做担保,找人借了五万两银子(要算利息),补了部分工资,这才把人弄出来。\\

毕自肃确实是个好人,出来后没找打他的人,反而跟自己过不去,觉得闹到这个局势,有很大的领导责任,但他实在太过实诚,为负责任,竟然自杀了。\\

毕巡抚是个老实人,袁督师就不同了,听说兵变消息,勃然大怒:竟敢闹事,反了你们了!\\

立刻马不停蹄往地方赶,到了宁远,衙门都不进,直接就奔军营。\\

此时的军营,已彻底失去控制,军官都不敢进,进去就打,闹得不行,袁崇焕进去了,大家都安静了。\\

所谓闹事,也是有欺软怕硬这一说的。\\

袁崇焕首先宣读了皇帝的谕令,让大家散会,回营休息,然后他找到几个心腹,只问了一个问题:\\

“谁带头闹的?”\\

回答:\\

“杨正朝,张思顺。”\\

那就好办了,先抓这两个。\\

两个人抓来,袁崇焕又只问了一个问题:想死,还是想活。\\

不过是讨点钱,犯不着跟自己过不去,想活。\\

想活可以,当叛徒就行。\\

很快,在两人的帮助下,袁崇焕找到了参与叛乱的其余十几个乱党,对这些人,就没有问题,也没有政策了,全部杀头。\\

领头的没有了,自然就不闹了,接下来的,是追究领导责任。\\

负有直接责任的中军部将吴国琦,杀头,其余相关将领,免职的免职,查办的查办,这其中还包括后来把李自成打得满世界乱逃的左良玉。\\

兵变就此平息,但问题没有解决,毕竟物质基础决定上层建筑,老不发工资,玉皇大帝也镇不住。\\

袁崇焕直接找到崇祯,开口就要八十万。\\

八十万两白银,折合崇祯时期米价,大致是人民币六亿多。\\

袁崇焕真敢要,崇祯也真敢给,马上批示户部尚书毕自严,照办。\\

毕自严回复,不办。\\

崇祯大发雷霆,毕自严雷打不动,说来说去就一句话,没钱。\\

毕尚书不怕事,也不怕死,他的弟弟死都没能发出军饷,你袁崇焕算老几?\\

事实确实如此,我查了一下,当时明朝每年的收入,大致是四百万两,而明朝一年的军费,竟然是五百万两!如此下去,必定破产。\\

明朝,其实就是公司,公司没钱要破产,明朝没钱就完蛋,而军费的激增,应归功于努尔哈赤父子这十几年的抢掠带折腾,所谓明亡清兴的必然结局,不过如此。\\

虽说经济紧张,但崇祯还是满足了袁崇焕的要求,只是打了个折——三十万两。\\

钱搞定了,接下来是搞人。首先是辽东巡抚,毕巡抚死后,这个位置一直没人坐,袁崇焕说,干脆别派了,撤了这个职务拉倒。\\

崇祯同意了。\\

然后袁崇焕又说,登州、莱州两地(归他管)干脆也不要巡抚了,都撤了吧。\\

崇祯又同意了。\\

最后袁崇焕还说,为方便调遣,特推荐三人:赵率教、何可纲、祖大寿(他的铁杆),赵率教为山海关总兵,何可纲为宁远总兵,原任总兵满桂、麻登云(非铁杆),另行任用。\\

崇祯还是同意了。\\

值得一提的是,在请示任用这三个人的时候,袁崇焕曾经说过一句话:\\

“臣选此三人,愿与此三人共始终,若到期无果,愿杀此三人,然后自动请死。”\\

此后的事情证明,这个誓言是比较准的,到期无果,三人互相残杀,他却未能请死。\\

至此,袁崇焕人也有了,钱也有了,蓟辽之内,已无人可与抗衡。\\

不,不,还是有一个。\\

近十年来,历任蓟辽总督,无论是袁应泰、熊廷弼,王化贞,都没有管过他,也管不了他。\\

“孤处天涯,为国效命,曲直生死,惟君命是从。”\\

臣左都督,挂将军印领尚方宝剑,总兵皮岛毛文龙泣血上疏。\\

\subsection{决定}
袁崇焕想杀掉毛文龙。\\

这个念头啥时候蹦出来的,实在无法考证,反正不是一天两天了,而杀人动机,只有四个字:看不顺眼。\\

当然,也有些人说,袁崇焕要杀掉毛文龙,是要为投敌做准备,其实这个说法并不新鲜,三百多年前袁崇焕快死那阵,京城里都这么说。\\

但事实上,这是个相当无聊的讲法,因为根据清朝《满文老档》的记载,毛文龙曾经跟皇太极通过信,说要投敌,连进攻路线都商量好了,要这么说,袁崇焕还算是为国除害了。\\

鉴于清朝有乱改史料的习惯,再加上毛文龙一贯的表现,其真实性是值得商榷的。\\

袁崇焕之所以决定干掉毛文龙,只是因为毛文龙不太听话。\\

毛文龙所在的皮岛,位于后金的后方,要传命令过去,要么穿越敌军阵地,要么坐船,如果不是什么惊天剧变,谁也不想费这个事。\\

躲在岛上,长期没人管,交通基本靠走,通讯基本靠吼,想听话也听不了,所以不太听话。\\

更重要的是,毛文龙在皮岛,还是很有点作用的,他位于后金后方,经常派游击队骚扰皇太极,出来弄他一下,又不真打,实在比较恶心。被皇太极视为心腹大患。\\

但这个人也是有问题的,毛总兵驻守皮岛八年,做得最成功的不是军事,而是经济,皮岛也就是个岛,竟然被他做成了经济开发区,招商引资,无数的客商蜂拥而至,大大小小的走私船都从他那儿过,收钱就放行,他还参股。\\

打仗倒也真打,每年都去,就是次数少点——六次,大多数时间,是在岛上列队示威,或者派人去后金那边摸个岗哨,打个闷棍之类。\\

但总体而言,毛文龙还是不错的,一人孤悬海外,把生意做得这么大,还牵制了皇太极,虽说打仗不太积极,但以他的兵力,能固守就及格了。\\

鉴于以上原因,历代总督、巡抚都是睁只眼闭只眼,放他过去了。\\

但袁崇焕是不闭眼的,他的眼里,连粒沙子都不容。\\

几年前,当他只是个四品宁前道的时候,就敢不经请示杀副总兵,现在的袁督师手握重权,小小的皮岛总兵算老几?\\

更恶劣的是,毛文龙有严重的经济问题,八年多账目不清,还从不接受检查,且虚报战功,也不听招呼,实在是罪大恶极,必须干掉!\\

其实毛总兵是有苦衷的,说我捞钱,确是事实,那也是没办法,就这么个荒岛,要不弄点钱,谁跟你干?说我虚报战功,也是事实,但这年头,不打仗的都吹牛,打仗的都虚报,多报点成绩也正常,都照程序走,混个屁啊?\\

我曾查阅明代户部资料及相关史料,毛文龙手下的人数,大致在四万多人左右,按户部拨出的军饷,是铁定不够用的,换句话说,毛总兵做生意赚的钱,很多都贴进了军饷,很够意思。\\

可惜对袁崇焕同志而言,这些都没有意义,在这件事上,他是纯粹的对人不对事。\\

大难即将临头的毛总兵依然天真无邪,直到他得知了那个消息。\\

崇祯二年(1629)四月,蓟辽督师袁崇焕下令:凡运往东江之物资船队,必须先开到宁远觉华岛,然后再运往东江。\\

接到命令后,毛文龙当场晕菜,大呼:\\

“此乃拦喉切我一刀,必定立死!”\\

只是换个地方起运,为什么立死呢?\\

因为毛总兵的船队是有猫腻的,不但里面夹杂私货,还要顺道带商船上岛,袁督师改道,就是断了他的财路,只能散伙。\\

他立即向皇帝上疏,连声诉苦,说自己混不下去了,连哭带吓唬,得到的,却只是皇帝的几个字:从长计议。\\

从长计议?怎么从长,喝西北风?\\

在他最困难的时候,一个最不可能帮助他的人帮助了他。\\

穷得发慌的毛文龙突然收到了十万两军饷,这笔钱是袁崇焕特批的。\\

拿钱的那一刻,毛文龙终于明白了袁崇焕的用意:拿我的钱,就得听我的话。\\

也好,先拿着,到时再慢慢谈。\\

然而袁崇焕的真实用意是:拿我的钱,就要你的命!\\

说起来,毛文龙算是老江湖了,混了好几十年,还是吃了没文化的亏,要论耍心眼,实在不如袁崇焕。\\

他做梦也想不到,很久以前,袁督师就打算干掉他。\\

早在崇祯元年(1628)七月,袁崇焕在京城的时候,曾找到大学士钱龙锡,对他说过这样一句话:\\

“(毛文龙)可用则用之,不可用则杀之。”\\

这还不算,杀的方法都想好了:\\

“入其军,斩其帅!”\\

后来他给皇帝的奏疏上,也明明白白写着:\\

“去年(崇祯元年)十二月,臣安排已定,文龙有死无生矣!”\\

“安排已定”,那还谈个屁。\\

但谈还是要谈,因为毛总兵手下毕竟还有几万人,占据要地,如果把他咔嚓了,他的部下起来跟自己死磕,那就大大不妙了。\\

所以袁崇焕决定,先哄哄他。\\

他先补发了十万两军饷,然后又在毛总兵最困难的时候,送去了许多粮食和慰问品,并写信致问候。\\

毛文龙终于上当了,他十分感激,终于离开了皮岛老巢,亲自前往宁远,拜会袁崇焕。\\

机会来了。\\

在几万重兵的注视下,毛文龙进入了宁远城。\\

他拜会了袁崇焕,并受到了热情的接待,双方把酒言欢,然后……\\

然后他安然无恙地走了。\\

袁崇焕确实想杀掉毛文龙,但绝不是在宁远。\\

这个问题,有点脑子的人就能想明白,如果在宁远把他干掉了,他手下那几万人,要么作鸟兽散,要么索性反出去当土匪,或是投敌,到时这烂摊子怎么收?\\

所以在临走时,袁崇焕对毛文龙说,过一个月,我要去你的地盘阅兵,到时再叙。\\

因为解决问题的最好方法,就是在他自己的地盘上干掉他。\\

崇祯二年(1629)五月二十九日,袁崇焕的船队抵达双岛。\\

双岛距离皮岛很近,是毛文龙的防区,五月三十日,毛文龙到达双岛,与袁崇焕会面。\\

六月初一夜晚,袁崇焕来到毛文龙的营房,和他进行了谈话,双方都很客气,互相勉励,表示时局艰难,要共同努力,度过难关。\\

这是两人三次谈话中的第一次。\\

既然在自己的地盘,自然要威风点,毛文龙带来了三千多士兵,在岛上列队,准备迎接袁崇焕的检阅。\\

六月初三,列队完毕,袁崇焕上岛,开始检阅。\\

出乎意料的是,毛文龙显得很紧张,几十年的战场经验告诉他,这天可能要出事,所以在整个检阅过程中,他的身边都站满了拿刀的侍卫。\\

然而袁崇焕显得很轻松,他的护卫不多,却谈笑自若,搞得毛文龙相当不好意思。\\

或许是袁崇焕的诚意感动了毛文龙,他赶走了护卫,就在当天深夜,来到了袁督师的营帐,和他谈话。\\

这是他们三次谈话中的第二次。\\

第二天,和睦的气氛终于到达了顶点,一整天都在吃吃喝喝中度过,夜晚,好戏终于开场。\\

毛文龙来到袁崇焕的营帐,开始了人生中的最后一次谈话。\\

一般说来,两人密谈,内容是不会外泄的,好比秦朝赵高和李斯的密谋,要想知道,只能靠猜。\\

我不在场,也不猜,却知道这次谈话的内容,因为袁崇焕告诉了我。\\

一个月后,在给皇帝的奏疏中,袁崇焕详细记录了在这个杀戮前的夜晚,他和毛文龙所说的每句话。\\

袁崇焕说:\\

“你在边疆这么久,实在太劳累了,还是你老家杭州西湖好。”\\

毛文龙说:\\

“我也这么想,只是奴(指后金)尚在。”\\

袁崇焕说:\\

“会有人来替你的。”\\

毛文龙说:\\

“此处谁能代得?”\\

袁崇焕没有回答这个问题,接着说:\\

“我此来劳军,你手下兵士每人赏银一两,布一匹,米一石,按人头发放。”\\

毛文龙说:\\

“我这里有三千五百人,明天就去领赏。”\\

讨论了一些细节问题后,谈话正式结束。\\

毛文龙的命运就此结束。\\

他不知道,这个夜晚的这次谈话,是他最后救命的机会,而所有的秘密,就藏在这份看似毫不起眼的记录里。\\

现在,让我来翻译一下这份记录:\\

在谈话的开始,袁崇焕说杭州西湖好,解释:毛文龙你回老家吧,只要你把权力乖乖让出来,可以不杀你。\\

毛文龙说工作任务重,不能走,解释:我在这儿很舒坦,不想走。\\

袁崇焕说,可以找人替你,解释:这里不是缺了你不行,大把人可以代替你。\\

毛文龙说,此处谁代得,解释:都是我的人,谁能替我!\\

这算是谈崩了,接下来的,是袁崇焕的最后一次尝试。\\

袁崇焕说,按人头发放赏赐,解释:把你的家底亮出来,到底有多少人,老实交代。\\

毛文龙说,这里的三千五百人,明天领赏,解释:知道你想查我家底,就是不告诉你!\\

谈不拢,杀吧。\\

六月五日。\\

袁崇焕在山上设置了大帐,准备在那里召见毛文龙。\\

然后他走到路边,等待着毛文龙的到来。\\

毛文龙列队完毕,准备上山。\\

袁崇焕拦住了他,说,不用这么多人,带上你的亲信将领就行了。\\

毛文龙表示同意,带着随从跟着袁崇焕上了山。\\

在上山的路上,袁崇焕突然停住脚步,对着毛文龙身旁的将校们,说了这样一句话:\\

“你们在边疆为国效力,每月的粮饷只有一斛,实在太辛苦了,请受我一拜!”\\

袁督师如此客气,大家受宠若惊,纷纷回拜,所以,在一片忙乱之中,许多人都没有听懂他的下一句话:\\

“你们只要为国家效力,今后不用怕无粮饷。”\\

这句话的意思是,就算你们的毛总兵死了,只要继续干,就有饭吃。\\

一路走,一路聊,袁崇焕很和气,毛文龙很高兴,气氛很好,直到进入营帐的那一刻。\\

“毛文龙!本部院与你谈了三日,只道你回头是迟也不迟,哪晓得你狼子野心总是欺诳,目中无本部院,国法岂能容你!”\\

面对袁崇焕严厉的训斥,毛文龙却依旧满脸堆笑——还没反应过来。\\

太突然了,事情怎么能这样发展呢?\\

袁崇焕到底有备而来,毛总兵脑袋还在运算之中,他就抛出了重量级的武器——十二大罪。\\

这十二大罪包括钱粮不受管辖、冒功、撒泼无礼、走私、干海盗、好色、给魏忠贤立碑、未能收复辽东土地等等。\\

这十二大罪的提出,证明袁崇焕同志的挖坑功夫,还差得太远。\\

类似这种材料公文,骂的是人是鬼不要紧,有没有事实也不要紧,贵在找得准,打得狠,比如杨涟参魏忠贤的二十四大罪,就是该类型公文的典范。\\

但袁崇焕给毛文龙栽的这十二条,实在不太高明,所谓冒功、无礼、好色,只要是人就干过,实在摆不上台。而最有趣的,莫过于给魏忠贤立碑,要知道,当年袁巡抚也干过这出,他曾向朝廷上书,建议在宁远给魏忠贤修生祠,可惜由于提早下课,没能实现。\\

这些都是扯淡,其实说来说去就两个字:办你。\\

文龙兄尚在晕菜之际,袁督师已经派人脱了他的官服,绑起来了。\\

绑成粽子的毛文龙终于清醒过来,大喊一声:\\

“文龙无罪!”\\

敢喊这句话,是有底的,毕竟是自己的地盘,几千人就等在外边,且身为一品武官,总镇总兵,除皇帝外,无人敢杀。\\

但袁崇焕敢,他敢杀毛文龙,有两个原因。\\

第一个原因:他是袁崇焕,四品文官就敢杀副总兵的袁崇焕。\\

第二个原因是一件东西,他拿了出来给毛文龙看。\\

当看到这件东西时,毛文龙终于服软了,这玩意他并不陌生,事实上再熟悉不过了,因为他自己也有一件——尚方宝剑。\\

活到头了。\\

虽说文龙兄手里也有一把尚方宝剑,可惜那是天启皇帝给的,所谓尚方宝剑,是皇帝的象征,不是死皇帝的象征,人都死了,把死人送给你的宝剑拿出来,吓唬鬼还行,跟现任皇帝的剑死磕,只能是找死了。\\

手持尚方宝剑的袁崇焕,此刻终于说出了他的心声和名言:\\

“你道本部院是个书生,本部院是朝廷的将首!”\\

毛文龙明白,今天这关不低头是过不去了,马上开始装孙子:\\

“文龙自知死罪,只求督师恩赦。”\\

统帅认怂了,属下自然不凑热闹,毛文龙的部将毫无反抗,当即跪倒求饶,只求别把自己搭进去。\\

其实事情到此为止,教训教训毛文龙,也就凑合了。\\

然而袁崇焕很执着。\\

局势尽在掌握,胜利就在眼前,这一切的一切冲昏了他的头脑,让他说出了下面的话:\\

“今日杀了毛文龙,本督师若不能恢复全辽,愿试尚方宝剑偿命!”\\

这话很准。\\

然后他面向京城的方向请旨跪拜,将毛文龙拉出营帐,斩首。\\

辽东的重量级风云人物毛文龙,就此结束了他传奇的一生。\\

可惜毛总兵并不知道,他是可以不死的,因为袁崇焕根本就杀不了他,只要他向袁崇焕索要一样东西。\\

这件东西,就是皇帝的旨意。\\

在古往今来的戏台、电视剧里,尚方宝剑都是个很牛的东西,扛着到处走,想杀谁就杀谁。\\

这种观点,基本上是京剧票友的水平。别的朝代且不说,在明朝,所谓尚方宝剑,说起来是代天子执法,但大多数时,也就做个样子,表示皇帝信任我,给我这么个东西,可以狐假虎威一下,算是特别赏赐。\\

一般情况下,真凭这玩意去砍人的,是少之又少,最多就是砍点中低级别的阿猫阿狗,敢杀朝廷一品大员的,也只有袁崇焕这种二杆子。\\

换句话说,袁崇焕要干掉毛文龙,必须有皇帝的旨意,问题在于,毛文龙同志当官多年,肯定也知道这一点,他为什么不提出来呢?\\

对于这个疑问,我曾百思不得其解,经过仔细分析材料,我才发现,原来毛文龙同志之所以认栽,只是出于一个偶然的误会:\\

因为当袁崇焕拿出尚方宝剑,威胁要杀掉毛文龙的时候,曾说过这样一句话,正是这句话,断送了毛文龙的所有期望。\\

他说:我五年平辽,全凭法度,今天不杀你,如何惩戒后人?皇上给我尚方宝剑,就是为此!\\

这是句相当忽悠人的话,特别是最后一句,皇上给我尚方宝剑,就是为此。\\

为此——到底为什么?\\

所谓为此,就是为了维护纪律,也就是客气客气的话,没有特指,因为皇帝并未下令,用此剑杀死毛文龙。\\

但在毛文龙听来,为此,就是皇帝发话,让袁同志拿着家伙,今天上岛来砍自己,所以他没有反抗。\\

换句话说,毛文龙同志之所以束手待毙,是因为他的语法没学好,没搞清主谓宾的指代关系,弄错了行情。\\

从小混社会,有丰富江湖经验的毛总兵就这么被稀里糊涂地干掉了。这就是小时候不好好读书的恶果。\\

人干掉了,接下来的是擦屁股程序。\\

首先是安慰大家,我只杀毛文龙,首恶必办,胁从不问。然后是发钱,袁崇焕随身带着十万两(约六千多万人民币),全都发了,只是这种先杀人,再分钱的方式,实在太像强盗打劫。\\

而最后,也最重要的一步,是安抚。\\

毛文龙手下这几万人,基本都是他的亲信,要保证这些人不跑,也不散伙,袁崇焕很是花了一番心思,先是换了一批将领,安插自己的亲信,然后又任命毛文龙的儿子毛承禄当部将,这意思是,我虽然杀了你爹,但那是公事,跟你没有关系,照用你,别再闹事。\\

几大棒加胡萝卜下去,效果很好,没人闹,也没人反,该干啥还干啥,袁崇焕很高兴。\\

毛文龙就这么死了,似乎什么都没有改变。\\

但后果是有的,且非常非常非常非常严重。\\

最高兴的是皇太极,他可以放心了,因为毛文龙所控制的区域,除皮岛外,还有金州、旅顺等地区,而毛总兵人品虽不咋样,但才能出众,此人一死,这些地盘就算没人管了,他可以放心大胆地进攻京城。\\

而自信的袁督师认定,他的善后工作非常出色。但他不知道的是,在那群被他安抚的毛文龙部下里,有这样三个人,他们的名字分别是尚可喜、耿仲明、孔有德。\\

这三位仁兄就不用多介绍了,都是各类“辫子戏”里的老熟人了,前两位先是造反,折腾明朝,后来又跟着吴三桂造反,折腾清朝,史称“三藩”。\\

而最后这位孔有德更是个极品,他是清朝仅有的两名汉人封王者之一(另一个是吴三桂)。现在北京有个地名叫公主坟,据说里面埋的就是孔有德的女儿。\\

当汉奸能当出这么大成就,实在是因为他的汉奸当得非常彻底。后来镇守桂林时,遇到了明末第一名将李定国,被打得满地找牙,气不过,竟然自焚了。清朝认为这兄弟很够意思,就追认了个王。\\

这三位仁兄原先都是山东的矿工,觉得挣钱没够,就改行当了海盗,后来转正成了毛文龙的部将。事实证明,这三个人只有毛文龙能镇住,因为两年后,他们就都反了。\\

事实还证明,他们是很有点水平的,后来当汉奸时很能打仗,为大清的统一事业做出了卓越贡献。\\

再提一句,那位被袁督师提拔的毛文龙之子毛承禄后来也反了,不过运气差点,没当上汉奸,就被剁了。\\

所谓文龙该死,结果大致如此。\\

但跟上述结果相比,下面这个才是最为致命的。\\

到底是朝廷里混过的,杀死毛文龙后,袁崇焕立刻意识到,这事办大了。\\

所以他立即上书,向皇帝请罪,说这事我办错了,以我的权力,不应该杀死毛文龙,请追究我的责任,等待皇帝处分。\\

袁崇焕认识错误的态度很诚恳,方法却不对。如果要追究责任,处分、撤职、充军都是不够的,唯一能够摆平此事的方法,就是杀人偿命。\\
\ifnum\theparacolNo=2
	\end{multicols}
\fi
\newpage
\section{杀人的必备程序}
\ifnum\theparacolNo=2
	\begin{multicols}{\theparacolNo}
\fi
在明朝,杀一个人很难吗?\\

答案是不难,拍黑砖、打闷棍、路上遇到劫道的,手脚利落的,也就一根烟功夫。\\

但要合法地杀掉一个人,很难。\\

因为大明是法制社会,彻头彻尾的法制社会。\\

这绝不是开玩笑,只要熟读以下攻略,就算你在明朝犯了死罪,要想不死,也是可能的。\\

比如你在明朝犯了法(杀了人),就要定罪,运气要是不好,定了个死罪,就要杀头。\\

但暂时别慌,只要你没干造反之类的特种行当,不会马上被推出去杀掉,一般都是秋后处决。\\

有人会问,秋后处决不一样是处决吗?不过是多活两天而已。\\

确实是多活了,但只要你方式得当,就不只是多活两天,事实上,据记载,最高记录是二十多年。\\

之所以出现这种奇怪的现象,是因为要处决一个人,必须经过复核,而在明朝,复核的人不是地方政府,也不是最高法院大理寺,甚至不是刑部部长。\\

唯一拥有复核权的人,是皇帝。\\

这句话的意思是,无论你在哪里犯罪,市区、县城乃至边远山区,无论你犯的是什么罪,杀人、放火或是砸人家窗户,且无论你是张三、李四、王二麻子,还是王侯将相,只要你犯了死罪,除特殊情况外,都得层层报批,县城报省城,省城报刑部,刑部报皇帝,皇帝批准,才能把你干掉。\\

自古以来,人命关天。\\

批准的方式是打勾,每年刑部的官员,会把判刑定罪的人写成名单,让皇帝去勾,勾一个杀一个。\\

但问题是,如果你的名字在名单上,无非也就让皇帝大人受累勾一笔,秋后就拉出去砍了,怎么可能活二十多年不死呢?\\

不死攻略一:\\

死缓二十多年的奇迹,起源于皇帝大人的某种独特习惯,要知道,皇帝大人在勾人的时候,并不是全勾,每张纸上,他只勾一部分,经常会留几个。\\

此即所谓君临天下,慈悲为怀,皇帝大人是神龙转世,犯不着跟你们平头百姓计较,少杀几个没关系。\\

但要把你的性命寄托在皇帝大人打勾上,实在太悬,万一那天他心情欠佳,全勾了,你也没辙。\\

所以要保证活下来,我们必须另想办法。\\

不死攻略二:\\

相对而言,攻略二的生存机率要高得多。当然,成本也高得多。\\

攻略二同样起源于皇帝大人的某种习惯——日理万机。\\

要打通攻略二,靠运气是没戏的。你必须买通一个人,但这个人不是地方官员(能买通早就买了),也不是刑部(人太多,你买不起),更加不是皇帝(你试试看)。\\

而是太监。\\

皇帝大人从来不清理办公桌,也不整理公文的,每次死刑名单送上来,都是往桌上一放,打完勾再换一张,毕竟我国幅员辽阔,犯罪分子一点不缺,动不动几十张勾决名单,今天勾不完,放在桌上等着明天批。\\

但是皇帝们绝不会想到,明天勾的那张名单,并不是今天眼前的这张。\\

玄机就在这里,既然皇帝只管打勾,名字太多,又记不住,索性就把下面名单挪到上面去,让没出钱的难兄难弟们先死,等过段时间,看着关系户的那张名单又上来了,就再往下放,周而复始,皇帝不批,就不能杀,就在牢里住着,反正管吃管住,每年全家人进牢过个年,吃顿团圆饭,不亦乐乎。\\

而能干这件事的,只有皇帝身边的太监,而且这事没啥风险,也就是把公文换个位置,又没拿走,皇帝发现也没话说。\\

但这件事也不容易。因为能翻皇帝公文的,大都是司礼监,能混到司礼监的,都不是凡人,很难攀上关系,且收费也很贵,就算买通了,万一哪天他忘了,或是下去了,该杀还是得杀。\\

无论费多大功夫,能保住命,还是值得的。\\

不过需要说明的是,以上攻略不适用于某些特殊人物,比如崇祯,工作干劲极大,喜欢打勾,一勾全勾完,且记性极好,又比较讨厌太监,遇到这种皇帝,就别再指望了。\\

综上所述,在明代,要合法干掉一个人,是很难的。\\

之所以说这么多,得出这个结论,只是要告诉你,袁崇焕的行为,有多么严重。\\

杀个老百姓,都要皇帝复核,握有重兵,关系国家安危的一品武官毛文龙,就这么被袁崇焕杀了,连个报告都没有。\\

仅此一条,即可处死袁崇焕。\\

更重要的是,此时已有传言,说袁崇焕杀死毛文龙,是与皇太极配合投敌,因为他做了皇太极想做而做不到的事。\\

这种说法是比较扯的,整个辽东都在袁崇焕的手中,他要投敌,打开关宁防线就行,毛文龙只能在岛上看着。\\

事情闹到这步,只能说他实在太有个性了。\\

在朝廷里,太有个性的人注定是混不长的。\\

但袁崇焕做梦也没想到,他等来的,却是一份嘉奖。\\

崇祯二年(1629)六月十八日,崇祯下令,痛斥毛文龙专横跋扈,目无军法,称赞袁崇焕处理及时,没有防卫过当,加以奖励。\\

这份旨意说明了崇祯对袁崇焕的完全推崇和信任,以及对毛文龙的完全唾弃。\\

他是这样说的,不是这样想的。\\

按照史料的说法,听说此事后,崇祯“惊惶不已”。\\

惊惶是肯定的,好不容易找了个人收拾残局,结果这人一上来,啥都没整,就先干掉了帮自己撑了八年的毛总兵,脑袋进水了不成?\\

但崇祯同志不愧为政治家,关键时刻义无反顾地装了孙子:人你杀了,就是骂你,他也活不了,索性骂他几句,说他死得该再吐上几口唾沫,没问题。\\

袁崇焕非常高兴,杀人还杀出好了,很是欢欣鼓舞了几天,但他并不清楚,他可以越权,可以妄为,却必须满足一个条件。\\

这个条件的名字,叫做办事。\\

要当督师,可以,要取消巡抚,可以,辽东你说了算,可以,杀掉毛文龙,也可以,但前提条件是,你得办事,五年平辽,只要平了,什么都好办,平不了嘛,就办你。\\

袁崇焕很清楚这点,但毕竟还有五年,鬼知道五年后什么样,慢慢来。\\

但两个月后,一个人的一次举动,彻底改变了他的命运,顺便说一句,这人不是故意的。\\

崇祯二年(1629)十月,皇太极准备进攻。\\

虽然之前曾被袁崇焕暴打一顿,狼狈而归,但现实是严峻的,上次抢回来的东西,都用得差不多了,又没有再生产能力,不抢不行啊。\\

可问题是,关宁防线实在太硬,连他爹算在内,都去了两次了,连块砖头都没能敲回来。\\

皇太极进攻的消息,袁崇焕听到过风声,一点不慌。\\

北京,背靠太行山脉和燕山山脉,通往辽东的唯一大道就是山海关,把这道口子一堵,鬼都进不来,所以袁崇焕很安心。\\

关卡是死的,人是活的。\\

冥思苦想的皇太极终于想出了通过关宁防线的唯一方法——不通过关宁防线。\\

中国这么大,不一定非要从辽东去,飞不了,却可以绕路。\\

辽东没法走,那就绕吧,绕到蒙古,从那儿进去,没辙了吧。\\

就这样,皇太极率十万军队(包括蒙古部落),发动了这次决定袁崇焕命运的进攻。\\

这是一次载入军事史册的突袭,皇太极充分展现了他的军事才华,率军以不怕跑路的精神,跑了半个多月,从辽东跑到辽西,再到蒙古。\\

蒙古边界没有坚城,没有大炮,皇太极十分轻松地跨过长城,在地图上画个半圆后,于十月底到达明朝重镇遵化。\\

遵化位于北京西北面,距离仅两百多公里,一旦失守,北京将无险可守。\\

袁崇焕终于清醒了,但大错已经酿成,当务之急,是派人挡住皇太极。\\

估计是欺负皇太极上了瘾,袁崇焕没有亲自上阵,他把这个光荣的任务交给了赵率教。\\

皇太极同志带了十万人,全部家当,以极为认真的态度来抢东西,竟然只派个手下,率这么点人(估计不到一万)来挡,太瞧不起人了。\\

赵率教不愧名将之名,得令后率军连赶三天三夜,于十一月三日到达遵化,很不容易。\\

十一月四日,出去打了一仗,死了。\\

对于赵率教的死,许多史料上说,他是被冷箭射死,部下由于失去指挥,导致崩溃,全军覆没。\\

但我认为赵率教死不死,不是概率问题,是个时间问题,就那么点人,要对抗十万大军,就算手下全变成赵率教,估计也挡不住。\\

赵率教阵亡,十一月五日,遵化失陷。\\

占领遵化后,后金军按照惯例,搞了次屠城,火光冲天,鬼哭狼嚎,再讲一下,不知是为了留个纪念,还是觉得风水好,清军入关后,把遵化当成了清朝皇帝的坟地,包括所谓“千古一帝”的康熙、乾隆以及“名垂青史”的慈禧太后,都埋在这里。\\

几具有名的尸体躺在无数具无名的尸体上,所谓之霸业,如此而已。\\

最后说几句,到了民国时期,土匪出身的孙殿英又跑到遵化,挖了清朝的祖坟,据说把乾隆、慈禧等一干伟大人物的尸体乱踩一通,着实是死不瞑目。当然,由于此事干得不地道,除个别人(冯玉祥)说他是革命行为外,大家都骂,又当然,骂归骂,从坟里掏出来的宝贝,什么乾隆的宝剑,慈禧的玉枕头(据说是宋美龄拿了),还是收归收。\\

几百年折腾来,折腾去,也就那么回事。\\

但遵化怎么样,对当时的袁崇焕而言,已经不重要了。\\

十一月五日,得知消息的袁督师明白,必须出马了。随即亲率大军,前去迎战皇太极。\\

十一月十日,当他到达京城近郊,刚松口气的时候,却得知了一个意外的消息。\\

原任兵部尚书王洽被捕了,而接替的他的人,是孙承宗。\\

王洽刚上任不久就下台,实在是运气太差,突然遇上这么一出,打也打不过,守也守不住,只好撤职,一般说来,老板开除员工,也就罢了,但崇祯老板比较牛,撤职之后又把他给砍了。\\

关键时刻,崇祯决定,请孙承宗出马,任内阁大学士、兵部尚书。\\

在这场史称“己巳之变”的战争中,这是崇祯做出的最英明,也是唯一英明的决定。\\

此时的袁崇焕已经到达遵化附近的蓟州,等待着皇太极的到来,因为根据后金军之前的动向看,这里将是他的下一个目标。\\

这是个错误的判断。\\

皇太极绕开蓟州,继续朝京城挺进。\\

情况万分紧急,因为从种种迹象看,他的最终目的就是京城。\\

但袁崇焕不这么看,他始终认为,皇太极就是个抢劫的,兜圈子也好,绕路也罢,抢一把就走,京城并无危险。\\

其实孙承宗也这样认为,但毕竟是十万人的抢劫团伙,所以他立即下令,袁崇焕应立即率部,赶到京郊昌平、三河一带布防,阻击皇太极。\\

到此为止,事情都很正常。\\

接下来发生的一切,都很不正常。\\

袁崇焕知道了孙承宗的部署,却并未执行,当年的学生,今天的袁督师,已无需服从老师的意见。\\

他召集军队,开始了一种极为诡异的行动方式。\\

十一月十一日,袁崇焕率军对皇太极发动追击。\\

说错了,是只追不击。\\

皇太极绕过蓟州,开始北京近郊旅游,三河、香河、顺义一路过去,所到之处都抢劫留念。袁崇焕一直跟着他,抢到哪里就跟到哪里。\\

就这样,袁崇焕几万人,皇太极十万人,共十多万人在北京周围转悠,从十一日到十五日,五天一仗没打。\\

袁崇焕在这五天里的表现,是有争议的,争议了几百年,到今天都没消停。\\

争议的核心只有一个:他到底想干什么?\\

大敌当前,既不全力进攻,也不部署防守,为什么?\\

当时人民群众的看法比较一致:袁崇焕是叛徒。\\

不攻也不守,跟着人家兜圈子,不是叛徒是什么?\\

更重要的是,皇太极在这五天里没闲着,四处抢劫,抢了又没人做主,郊区居民异常愤怒,都骂袁崇焕。\\

朝廷的许多高级官员也很愤怒,也骂袁崇焕,因为他们也被抢了(北京城土地紧张,园林别墅都在郊区)。\\

民不聊生,官也不聊生,叛徒的名头算是背定了。\\

所以每当翻阅这段史料时,我总会寻找一样东西——动机。\\

叛徒是不对的,要叛变不用等到今天,他手下的关宁军是战斗力最强的部队,将领全都是他的人,只要学习吴三桂同志,把关一交,事情就算结了。\\

失误也不对,凭他的智商和水平,跟着敌人兜圈之类的蠢事,也还干不出来。\\

所以我很费解,费解他的举动为何如此奇怪,直到我想起了在这三年前他对熊廷弼说过的四个字,才终于恍然大悟。\\

“主守,后战。”\\

\subsection{致命漏洞}
袁崇焕很清楚,以战斗力而言,如果与后金军野战,就算是最精锐的关宁铁骑,也只能略占上风,要想彻底击败皇太极,必须用老方法:凭坚城,用大炮。\\

而这里,唯一的坚城,就是北京。\\

为实现这一战略构想,必须故意示弱,引诱皇太极前往北京,然后以京城为依托,发动反击。\\

鉴于袁崇焕同志已经死了,也没时间告诉我他的想法,但事情的发展印证了这一切。\\

十一月十六日,当皇太极终于掉头,冲向北京时,袁崇焕当即下令,向北京进发。\\

袁崇焕坚信,到达京城之时,即是胜利到来之日。\\

但事实上,命令下发的那天,他的死期已然注定。\\

因为在计划中,他忽视了一个十分不起眼,却又至关重要的漏洞。\\

一直以来,袁崇焕的固定战法都是坚守城池,杀伤敌军,待敌疲惫再奋勇出击,从宁远到锦州,屡试不爽。\\

所以这次也一样,将敌军引至城下,诱其攻坚,待其受挫后,全力进攻,可获全胜。\\

很完美,很高明,如此完美高明的计划,大明最伟大的战略家,城里的孙承宗先生竟然没想到。\\

孙承宗想到了。\\

他坚持在北京外围迎敌,不想诱敌深入,不想大获全胜,并不是他愚蠢,而是因为他不但知道袁崇焕的计划,还知道这个计划的致命漏洞。\\

这个漏洞,可以用五个字来概括:这里是北京。\\

无论理论还是实战,这个计划都无懈可击,之前宁远的胜利已经证明,它是行得通的。\\

但是这一次,它注定会失败,因为这里是北京。\\

宁远也好,锦州也罢,都是小城市,里面当兵的比老百姓还多,且位居前线,都是袁督师说了算,让守就守,让撤就撤,不用讨论,不用测评。\\

但在京城里,说话算数的人只有一个,且绝不会是袁崇焕。\\

袁督师这辈子什么都懂,就是不懂政治。皇上坐在京城里,看着敌军跑来跑去,就在眼皮子底下转悠,觉都睡不好,把你叫来护驾,结果你也跑来跑去,就是不动手,把皇帝当猴耍,现在连招呼也没打,就突然冲到北京城下,到底想干什么?!\\

洞悉这一切的人,只有孙承宗。\\

所以谦虚的老师设置了那个无比保守,却也是唯一可行的计划。\\

骄傲的学生拒绝了这个计划,他认为,自己已经超越了所有的人。\\

就在袁崇焕率军到达北京的那一天,孙承宗派出了使者。\\

这位使者前往袁崇焕的军营,只说了一段话:皇上十分赏识你,我也相信你的忠诚,但是你杀掉了毛文龙,现在又把军队驻扎在城外,很多人都怀疑你,希望你尽力为国效力,若有差错,后果不堪设想。\\

虽然在史料上,这段话是使者说的,但很明显,这是一个老师,对他学生的最后告诫。\\

孙承宗的判断一如既往,很准。\\

袁崇焕到北京的那一天,是十一月十七日,很巧,他刚到不久,另一个人就到了——皇太极。\\

跳进黄河都洗不清了。\\

我曾查过当时的布阵方位,皇太极的军队在北城,而袁崇焕在南城的广渠门,虽说比较远,但你刚来,人家就到,实在太像带路的,要人民群众不怀疑你,实在很难。\\

更重要的是,明朝有规定,边防军队,未经皇帝允许,不得驻扎于北京城下。但袁崇焕同志实在很有想法,谁都没请示,就到了南城。\\

到这份上,如果还不怀疑袁崇焕,就不算正常了。\\

京城里大多数人很正常,所以上到朝廷,下到卖菜的,全都认定,袁崇焕有问题。\\

唯一不正常的,是崇祯。\\

他没有骂袁崇焕,只是下令袁崇焕进城,他要亲自召见。\\

召见的地点是平台,一年前,袁崇焕在这里,得到了一切。现在,他将在这里,失去一切。\\

其实袁崇焕本人是有思想准备的,一年过去,寸土未复不说,还让皇太极打到了城下,实在有点说不过去,皇帝召见,大事不妙。\\

如果是叛徒,是不会去的,然而他不是叛徒,所以他去了。\\

跟他一起进去的,还有三个人,分别是总兵满桂、黑云龙、祖大寿。\\

祖大寿是袁崇焕的心腹,而满桂跟袁崇焕有矛盾,黑云龙是他的部下。\\

此前我曾一度纳闷,见袁崇焕,为什么要拉这三个人进去,后来才明白,其中大有奥妙。\\

袁崇焕的政治感觉相当好,预感今天要挨整,所以进去时脱掉了官服,穿着布衣,戴黑帽子以示低调。\\

然而接下来的事情却是他做梦都想不到的。\\

崇祯没有发火,没有训斥,只是做了一个动作:\\

他解下了自己身上的大衣,披到了袁崇焕的身上。\\

袁督师目瞪口呆。\\

一年多啥也没干,敌人都打到城下了,竟然还这么客气,实在太够意思了。\\

在以往众多的史料中,对崇祯同志都有个统一的评价:急躁。\\

然而这件事情充分证明,崇祯,是一个成熟、卓越的政治家。\\

一年前开会,要钱给钱,要粮给粮,看谁顺眼就提谁(比如祖大寿),看谁不顺眼就换谁(比如满桂),无所谓,只要把活干好。\\

一年了,寸土未复,干掉了牵制后金的毛文龙,皇太极来了,也不玩命打,跟他在城边兜圈子,严重违反治安规定,擅自带兵进驻城下,还是那句话,你到底想干什么?\\

在这种情况下,只要是个人,就要解决袁崇焕了。\\

崇祯不是人,他是皇帝,一个有着非凡忍耐力,和政治判断的皇帝。\\

以他的脾气,换在以往,早就把袁崇焕给剁了,现在情况紧急,必须装孙子。\\

所以自打袁崇焕进来,他一直都很客气,除了脱衣服,就是说好话,你如何辛苦,如何忠心,我如何高兴等。\\

其实千言万语就一句话:你的工作干得很不好,我很不高兴,但是现在不能收拾你。\\

到这个份上,还能如此克制,实在难得,如果要给崇祯同志的表现打分的话,应该是十分。\\

而袁崇焕同志之后的表现,应该是负分。\\

说的事情没有做到,做的事情不应该做,又让皇帝大人吃那么多苦头,却得到了这样的嘉奖,袁崇焕受宠若惊。\\

所谓受宠若惊,是受宠后自己吃惊,他接下来的举动,却让别人吃惊。\\

在感谢皇帝大人的恩典后,袁崇焕开始了一场让无数人匪夷所思许多年的演说:\\

他首先描述了敌情,按照他的说法,敌军异常强大,且倾尽全力,准备拿下北京,把皇帝陛下赶出去,连继位的日子都定好了,很难抵挡。\\

这段话是彻头彻尾的胡说,且是故意的胡说,皇帝大人不懂业务,或许还会乱想,袁崇焕是专业人士,明知皇太极是穷的没办法,才来抢一把的,抢完了人家即回去了,竟然还要蒙领导,实在太不像话了。\\

但问题的关键在于,为什么?\\

袁崇焕的这一表现,被当时以及后来的许多人认定,他是跟皇太极勾结的叛徒。\\

从经济学的观点来看,这是不太可能的。所谓勾结,总得有个理由,换句话说,有个价钱,但问题是,当年皇太极同志,可是很穷的。\\

要知道,皇太极之所以来抢,是因为家里没钱,没钱,怎么跟人勾结呢?\\

虽说此前也有李永芳、范文程之类的人前去投奔,但事实上,也都并非什么大人物。比如李永芳,只是个地区总兵,而且就这么个小人物,努尔哈赤同志都送了一个孙女,一个驸马的(额驸)头衔,还有无数金银财宝,才算把他套住。\\

范文程更不用说,大明混不下去,到后金混饭吃的,只是一个举人而已,皇太极都给个大学士,让他当主力参谋。\\

李永芳投降的时候,是地区副总兵,四品武官,努尔哈赤就搭进去一个孙女,按照这个标准,如果要买通明代最大地方官,总管辽东、天津、登州、莱州、蓟州五个巡抚的袁崇焕,估计他就算把女儿、孙女全部打包送过去,估计也是白搭。\\

至于分地盘,就更不用说了,皇太极手里的地方,也就那么大,要分都拿不出手,谁跟你干?\\

当然,如果你非要较真,说他们俩一见如故,不要钱和地盘,老子也豁出去跟你干,我也没办法。\\

所以从经济学的角度讲,只要袁崇焕智商正常,是不会当叛徒的。\\

他糊弄皇帝的唯一原因,是两个字——心虚。\\

没法不心虚,跟皇帝吹了牛,说五年平辽,不到一年,人家就带兵来平你了。之前干掉了毛总兵,在北京城下又跟人兜圈,不经许可冲到城下,这事干得实在太糙。\\

不把敌人说得狠点,不把任务描述得艰巨点,怎么混过去?\\

可他万万没想到,这一糊弄,就糊弄过了。\\

皇帝当场傻眼不说,大臣们都吓得不行,户部尚书毕自严的舌头伸了出来,半天都没收回去。\\

客观地讲,袁督师干了一件相当缺德的事,但精彩的表演还没完,等大家惊讶完后,他又说了这样一句话。\\

我始终认为,这句话让他最终送了命。\\

“我的士兵连日征战,希望能够进城修整。”\\

这孩子没救了。\\

在明朝,边防军队未经许可进驻城下,基本就算造反,竟然还要兵马入城休息,实在太嚣张了。\\

当然,这个要求是有前科的。之前不久,满桂在城外与后金军大战,中途曾经进入德胜门瓮城休息,按袁崇焕的想法,他的地位比满桂高,满桂能进瓮城,他也能进。\\

举动如此可疑,大家本来就猜忌你,还要带兵入城,辽东人参吃多了。\\

所以崇祯立即做出了答复:不行。\\

袁督师倒也不依不饶:那我自己进城。\\

答复:不行。\\

会议就此结束。\\

这一天是崇祯二年(1629)十一月二十三日,根据种种迹象显示,崇祯判定,袁崇焕不可再用。\\

但除掉此人,还需要时间,至少七天。\\

幕后人物\\

袁崇焕的宿命已经注定。\\

但他的悲剧,不在于他最后被杀,而是他直到被杀,也不知道为什么。\\

事实上,致他于死地的那几条罪状里,有一条是很滑稽的。\\

这条滑稽的罪状,来源于三天前的一次偶然事件。\\

三天前,是十一月二十日。\\

在这一天,皇太极率军发动了进攻。\\

这是自于谦保卫战后,京城发生的最大规模的战斗,皇太极以南北对进战术,分别进攻北城的德胜门和南城的广渠门。\\

为保证不白来,皇太极下了血本,北路军五万余人,由他亲率,随同攻击的包括大贝勒代善,济尔哈朗等,而守卫北城的,是满桂。\\

南路军也不白给,共四万人,三贝勒莽古尔泰带队,还包括后来辫子戏里的主要角色多尔衮、多铎,守在这里的,就是袁崇焕。\\

战斗同时开始。\\

袁崇焕率所部九千余人,在城外列阵迎敌。\\

莽古尔泰虽然比较蠢,但算术还是会的,四万对九千,往前冲就是了。\\

但战术还是要讲的,他先率军先冲袁崇焕的左翼,冲不动,退了。\\

过了一会,又率军冲击明军右翼,还是冲不动,又退了。\\

估计是自尊心受到了伤害,第三次,他率领全部主力,直接扑袁崇焕。\\

后果很严重。\\

袁崇焕带来的,是明军最精锐的部队——关宁铁骑。\\

而且据某些史料讲,包括祖大寿、吴襄在内的一干猛人,都在这支部队里。\\

几乎就在莽古尔泰冲锋的同时,袁崇焕发动了反冲锋。\\

此战无需介绍战术,因为基本没有战术,双方骑兵对冲,谁更能砍,谁就能赢。\\

战斗过程极其惨烈,四小贝勒之一的阿济格的坐骑被射死,他身中数箭,差点当场完蛋,莽古尔泰本人被击伤。\\

袁崇焕也很悬,为鼓励士兵,他亲自上阵参加冲锋。据史书记载,他左冲右突如入无人之境,身中数箭,竟然毫发无伤,有如神助。\\

同样身中数箭,阿济格被射得奄奄一息,袁督师还能继续奋斗,秘诀在于四个字——“重甲难透”。\\

这四个字的意思是,袁督师身上的盔甲厚,箭射到他身上,一点事都没有。\\

在关宁铁骑的攻击下,后金军开始败退。\\

但八旗军的战斗力相当强悍,加上莽古尔泰脑子不好用,还有几把力气,再次集结部队,发动了第二次冲锋。\\

死磕的力量是很大的,袁督师的中军被冲散,他在乱军之中被人围攻,差点被剁。好在部下反应快,帮他格了几刀(格之获免),才从鬼门关爬出来。\\

稳住阵脚后,关宁军开始反击,然后又是你打过来,我打过去,一直折腾了八个钟头,直到晚上六点,莽古尔泰终于支持不住,败退,没来得及跑的,都被赶进了护城河。\\

广渠门之战结束,后金累计伤亡一千余人,明军大胜。\\

南城胜利之际,北城的满桂正在苦苦支撑。\\

进攻德胜门的军队,包括皇太极的亲军主力,战斗力非常强,满桂先派部将迎战,没一会就被打回来。关键时刻,满桂同志表现出了高昂的革命斗志,亲自上阵,并指挥城头炮兵开炮支援。\\

在他的光辉榜样映照下,城下明军勇猛作战,城上明军勇猛开炮,后金军死伤惨重。但不知城头上的哪位仁兄,点炮的时候太过勇猛,一哆嗦偏了准头,一炮直奔满桂同志,当场就把他撂倒,遍体负伤,好在捡了条命,被人护着回去养伤了。\\

主帅虽然撤走,但在大炮的掩护下,明军依然奋战不已,付出重大伤亡后,皇太极被迫撤退,德胜门之战就此结束。\\

这一天对袁崇焕而言,是很光荣的,他凭借自己的精兵良将,在京城打败了实力强劲的八旗军。\\

更重要的是,同一天出战的满桂,是他的死敌,当着皇帝的面,一个打出去,一个抬回来,实在很有面子。\\

可是他想不到,满桂同志的这笔帐,最终会算到他的身上,因为在那天战役结束时,一个流言开始在京城流传:\\

开炮打伤满桂的,就是袁崇焕。\\

这个说法是不可信的,因为满桂在德胜门作战,而袁崇焕在广渠门,今天在北京,要跑个来回,估计都要一个钟头,无论如何,袁崇焕都是过不去的。\\

但袁督师背这个黑锅,也不是全无道理,他跟满桂从宁远就开始干仗,后来硬把满总兵挤回关内,从来就不待见这人,现在满桂受伤了,算在他头上也不奇怪。\\

从毛文龙开始,到满桂,再到崇祯,袁崇焕一步步将自己逼入绝境,虽然他自己并不知晓。\\

袁崇焕,广西藤县人,自“蛮夷之地”而起,奋发读书,然资质平平,四次落第,以三甲侥幸登科,后赴辽东,得孙承宗赏识,于辽东溃败之时,以独军守孤城,屹然不倒,先后击溃努尔哈赤、皇太极父子,护卫辽东。\\

后受阉党所迫离职,蒙崇祯器重再起,然性格跋扈,调离满桂,安插亲信,以尚方宝剑杀毛文龙,奉调守京,不顾大局,擅自驻防于城下,致京郊怨声四起,后不惜性命,与皇太极苦战,大破敌军,不顾生死,身先士卒。\\

我想,差不多了。\\

最终命运揭晓之前,袁崇焕的表现大致如此。\\

他并不是一个天赋异禀的人,经过努力和奋斗,还有难得的机遇(比如孙承宗),才最终站上历史的舞台。\\

他并不完美,不守规章,不讲原则,想怎么干就怎么干,私心很重,听话的就提,不听话的就整(或杀)。\\

而某些所谓“专家”的所谓“力挽狂澜”,基本就是扯淡。关于这个问题,我曾在社科院明史学会的例会上,跟明史专家讨论过多次。客观地讲,以他的战略眼光(跟着皇太极绕京城跑圈)和实际表现(擅杀毛文龙),守城出战确属上乘,让他继续镇守辽东,还能闹出什么事来也难说,所谓挽救危局,随便讲几句吧。\\

袁崇焕绝不是叛徒,也绝不是一个关键性人物,他存在与否,并不能决定明朝的兴衰成败。换句话说,以他的才能,无论怎么折腾,该怎么样还怎么样。\\

对于这个悲剧性的结论,我不知道袁崇焕是否知道,他的一生丰富多彩,困守孤城,决死拼杀、遭人排挤、纵横驰骋、身处绝境,人家遇不上的事,他大都遇上了。\\

但无论何时、何地,得意、失意,他一直在努力,他坚信,自己的努力终将改变一切。\\

他始终没有放弃过。\\

崇祯二年(1629)十一月二十七日,京城九门换防,一切准备就绪。\\

最终的结局已经注定,无需改变,也无法改变。\\

就在这天,坚定的袁崇焕开始了自己人生中的最后一战——左安门之战。\\

袁崇焕列队于城外。\\

因为不能入城,只能背城布阵。背对着冰冷的墙砖,在京城凛冽的寒风中,他面对皇太极,展开了波澜壮阔人生的最后一幕。\\

后金军用潮水般的进攻,证明了自己还想进北京抢一把的美好愿景,但关宁铁骑用倒在他们面前的无数尸体证明,你们不行。\\

双方在左安门外持续激战,经过长达五个多小时的拉锯,皇太极终于支持不住,再次败退。左安门之战,以明军获胜告终。\\

结束了,都结束了。\\

\begin{quote}
	\begin{spacing}{0.5}  %行間距倍率
		\textit{{\footnotesize
				\begin{description}
					\item[\textcolor{Gray}{\FA }] 一个将军最好的归宿,就是在最后一场战役中,被最后一颗子弹打死。——巴顿
				\end{description}
		}}
	\end{spacing}
\end{quote}

我原先认为,说这句话的人,应该是吃饱了撑的外加精神失常,现在我明白了,他是对的。\\

崇祯二年(1629)十二月一日,袁崇焕得到指示,皇帝召见立即进城。\\

召见的理由是议饷,换句话说就是发工资。\\

命令还说,部将祖大寿一同觐见。\\

从古到今,领工资这种事都是跑着去的。袁崇焕二话不说,马上往城里跑,所以他忽略了如下问题:既然是议饷,为什么要拉上祖大寿?\\

跑到城下,却没人迎接,也不给开城门,等了半天,丢下来个筐子,让袁督师蹲进去,拉上来。\\

这种入城法虽说比较寒掺,但好歹是进去了,在城内守军的指引下,他来到了平台。\\

满桂和黑云龙也来了,正等待着他。\\

在这个曾带给他无比荣誉和光辉的地方,他第三次见到了崇祯。\\

第一次来,崇祯很客气,对他言听计从,说什么是什么,要什么给什么。第二次来,还是很客气,十一月份了,城头风大(我曾试过),二话不说就脱衣服,很够意思。\\

第三次来,崇祯很直接,他看着袁崇焕,以低沉的声音,问了他三个问题:\\

一、你为什么要杀毛文龙。\\

二、敌军为何能长驱直入,进犯北京。\\

三、你为什么要打伤满桂。\\

袁崇焕没有回答。\\

对于他的这一反应,许多史书上说,是没能反应过来,所以没说话。\\

事实上,他就算反应过来,也很难回答。\\

比如毛文龙同志,实在是不听话外加不顺眼,才剁了的,要跟崇祯明说,估计是不行的。再比如敌军为何长驱直入,这就说来话长了,最好拿张地图来,画几笔,解释一下战术构思,最后再顺便介绍自己的作战特点。\\

至于最后满桂问题,对袁督师而言,是很有点无厘头的,因为他确实不知道这事。\\

总而言之,这三个问题下来,袁督师就傻了。\\

对于袁督师的沉默,崇祯更为愤怒,他当即命令满桂脱下衣服,展示伤疤。\\

其实袁崇焕是比较莫名其妙的,说得好好的,你脱衣服干嘛?又不是我打的,关我屁事。\\

但崇祯就不这么想了,袁崇焕不出声,他就当是默认了,随即下令,脱去袁崇焕的官服,投入大牢。\\

这是一个让在场所有人都很惊讶的举动,虽然有些人已经知道,崇祯今天要整袁崇焕,但万万没想到,这哥们竟然玩大了,当场就把人给拿下。更重要的是,袁崇焕手握兵权,是城外明军总指挥,敌人还在城外呢,你把他办了,谁来指挥?\\

所以内阁大学士成基命、户部尚书毕自严马上提出反对,说了一堆话:大致意思是,敌人还在,不能冲动,冲动是魔鬼。\\

但崇祯实在是个四头牛都拉不回来的人物,老子抓了就不放,袁崇焕军由祖大寿率领,明军总指挥由满桂担任,就这么定了!\\

现在你应该明白,为什么两次平台召见,除袁崇焕外,还要叫上满桂、黑云龙和祖大寿。\\

祖大寿是袁崇焕的心腹,只要他在场,就不怕袁军哗变,而满桂是袁崇焕的死敌,抓了袁崇焕,可以马上接班,如此心计,令人胆寒。\\

综观崇祯的表现,断言如下:但凡说他蠢的,真蠢。\\

但这个滴水不漏的安排,还是漏水。\\

袁崇焕被抓的时候,祖大寿看上去并不吃惊。\\

他没有大声喧哗,也没有高调抗议,甚至连句话都没说。毕竟抓了袁崇焕后,崇祯就马上发了话,此事与其他人无关,该干什么还干什么。\\

但史书依然记下了他的反常举动——发抖,出门的时候迈错步等等。\\

对于这一迹象,大家都认为很正常——领导被抓了,抖几抖没什么。\\

只有一个人发现了其中的玄妙。\\

这个人叫余大成,时任兵部职方司郎中。\\

祖大寿刚走,他就找到了兵部尚书梁廷栋,对他说:\\

“敌军兵临城下,辽军若无主帅,必有大乱!”\\

梁廷栋毫不在意:\\

“有祖大寿在,断不至此!”\\

余大成答:\\

“作乱者必是此人!”\\

梁廷栋没搭理余大成,回头进了内阁。\\

在梁部长看来,余大成说了个笑话。于是,他就把这个笑话讲给了同在内阁里的大学士周延儒。\\

这个笑话讲给一般人听,也就是笑一笑,但周大学士不是一般人。\\

周延儒,字玉绳,常州人,万历四十一年进士。\\

周延儒同志的名气,是很大的。十几年前我第一次翻明史的时候,曾专门去翻他的列传,没有翻到。后几经查找才发现,这位仁兄被归入了特别列传——奸臣传。\\

奸臣还不好说,奸是肯定的。此人天资聪明,所谓万历四十一年进士,那是谦虚的说法。事实上,他是那一年的状元,不但考试第一,连面试(殿试)也第一。\\

听到这句话,嗅觉敏锐的周延儒立即起身,问:\\

“余大成在哪里?”\\

余大成找来了,接着问:\\

“你认为祖大寿会反吗?”\\

余大成回答:\\

“必反。”\\

“几天?”\\

“三天之内。”\\

周延儒立即指示梁廷栋,密切注意辽军动向,异常立即报告。\\

第一天,十二月二日,无事。\\

第二天,十二月三日,无事。\\

第三天,十二月四日,出事。\\

祖大寿未经批示,于当日凌晨率领辽军撤离北京,他没有投敌,临走时留下话,说要回宁远。\\

回宁远,也就是反了。皇帝十分震惊,关宁铁骑是精锐主力,敌人还在,要都跑了,摊子怎么收拾?\\

周延儒很镇定,他立即叫来了余大成,带他去见皇帝谈话。\\

皇帝问:祖大寿率军出走,怎么办?\\

余大成答:袁崇焕被抓,祖大寿心中畏惧,不会投敌。\\

皇帝再问:怎么让他回来?\\

余大成答:只有一件东西,能把他拉回来。\\

这件东西,就是袁崇焕的手谕。\\

好办,马上派人去牢里,找袁督师写信。\\

袁督师不写。\\

可以理解,被人当场把官服收了,关进了号子,有意见难免,加上袁督师本非善男信女,任你说,就不写。\\

急眼了,内阁大学士,外加六部尚书,搞了个探监团,全跑到监狱去,轮流劝说,口水乱飞。袁督师还是不肯,还说出了不肯的理由:\\

“我不是不写,只是写了没用,祖大寿听我的话,是因为我是督师,现我已入狱,他必定不肯就范。”\\

这话糊弄崇祯还行,余大成是懂业务的:什么你是督师,他才听你的话,那崇祯还是皇上呢,他不也跑了吗?\\

但这话说破,就没意思了,所以余大成同志换了个讲法,先捧了捧袁崇焕,然后从民族大义方面,对袁崇焕进行了深刻的教育,说到最后,袁督师欣然拍板,马上就写。\\

拿到信后,崇祯即刻派人,没日没夜地去追,但祖大寿实在跑得太快,追上的时候,人都到锦州了。\\

事实证明,袁督师就算改行去卖油条,说话也是算数的。祖大寿看见书信(还没见人),就当即大哭失声,二话不说就带领部队回了北京。\\

局势暂时稳定,一天后,再度逆转。\\

十二月十七日,皇太极再度发起攻击。\\

这次他选择的目标,是永定门。\\

估计是转了一圈,没抢到多少实在玩意,所以皇太极决定,玩一把大的,他集结了所有兵力,猛攻永定门。\\

明军于城下列阵,由满桂指挥,总兵力约四万,迎战后金。\\

战役的结果再次证明,古代游牧民族在玩命方面,是有优越性的。\\

经过整日激战,明军付出重大伤亡,主将满桂战死,但后金军也损失惨重,未能攻破城门,全军撤退。\\

四年前,籍籍无名的四品文官袁崇焕,站在那座叫宁远的孤城里,面对着只知道攒钱的满桂、当过逃兵的赵率教、消极怠工的祖大寿,说:\\

“独卧孤城,以当虏耳!”\\

在绝境之中,他们始终相信,坚定的信念,必将战胜强大的敌人。\\

之后,他们战胜了努尔哈赤,战胜了皇太极,再之后,是反目、排挤、阵亡、定罪、叛逃。\\

赵率教死了,袁崇焕坐牢了,满桂指认袁崇焕后,也死了,祖大寿终将走上那条不归之路。\\

共患难者,不可共安乐,世上的事情,大致都是如此吧。\\

\subsection{密谋}
永定门之战后,一直没捞到硬货的皇太极终于退兵了——不是真退。\\

他派兵占据了遵化、滦城、永平、迁安,并指派四大贝勒之一的阿敏镇守,以此为据点,等待时机再次发动进攻。\\

战局已经坏到不能再坏的地步,虽然外地勤王的军队已达二十多万,鉴于满桂这样的猛人也战死了,谁都不敢轻举妄动,朝廷跟关外已基本失去联系,辽东如何,山海关如何,鬼才知道,京城人心惶惶,形势极度危险。\\

然后,真正的拯救者出现了。\\

半个月前,草民孙承宗受召进入京城,皇帝对他说:“从今天起,你就是大学士,这是上级对你的信任。”\\

然后皇帝又说,“既然你是孙大学士了,现在出发去通州,敌人马上到。”\\

对于这种平时不待见,临时拉来背锅的欠揍行为,孙承宗没多说什么,在他看来,这是义务。\\

但要说上级一点不支持,也不对,孙草民进京的时候,身边只有一个人,他去通州迎敌的时候,朝廷还是给了孙大学士一些人。\\

一些人的数量是,二十七个。\\

孙大学士就带着二十七个人,从京城冲了出来,前往通州。\\

当时的通州已经是前线了,后金军到处劫掠,杀人放火兼干车匪路霸,孙大学士路上就干了好几仗,还死了五个人,到达通州的时候,只剩二十二个。\\

通州是有兵的,但不到一万人,且人心惶惶,总兵杨国栋本来打算跑路了,孙承宗把他拉住,硬拽上城楼,巡视一周,说明白不走,才把大家稳住。\\

通州稳定后,作为内阁大学士兼兵部尚书,孙承宗开始协调各路军队,组织作战。\\

以级别而言,孙大学士是总指挥,但具体实施起来,却啥也不是。\\

且不说其他地区的勤王军,就连嫡系袁崇焕都不听招呼,孙承宗说,你别绕来绕去,在通州布防,把人挡回去就是了,偏不听,协调来协调去,终于把皇太极协调到北京城下。\\

然后又是噼里啪啦一阵乱打,袁督师进牢房,皇太极也没真走,占着四座城池,随时准备再来。京城附近的二十多万明军,也是看着人多,压根没人出头,关宁铁骑也不可靠,祖大寿都逃过一次了,难保他不逃第二次。\\

据说孙承宗是个水命,所以当救火队员实在再适合不过了。\\

他先找祖大寿。\\

祖大寿是个比较难缠的人,且向来嚣张跋扈,除了袁崇焕,谁的面子都不给。\\

但孙承宗是例外,用今天的话说,当年袁督师都是给他提包的,老领导的老领导,就是领导的平方。\\

孙大学士说:袁督师已经进去了,你要继续为国效力。\\

祖大寿说,袁督师都进去了,我不知哪天也得进去,还效力个屁。\\

孙承宗说:就是因为袁督师进去了,你才别闹腾,赶紧给皇帝写检讨,就说你要立功,为袁督师赎罪。\\

祖大寿同意了,立即给皇帝写信。\\

这边糊弄完了,孙承宗马上再去找皇帝,说祖大寿已经认错了,希望能再有个机会,继续为国效命。\\

话刚说完,祖大寿的信就到了,皇帝大人非常高兴,当即回复,祖大寿同志放心去干,对你的举动,本人完全支持。\\

虽然之前他也曾对袁崇焕说过这句话,但这次他做到了,两年后祖大寿在大凌河与皇太极作战,被人抓了,后来投降又放回来,崇祯问都没问,还接着用。如此铁杆,就是孙承宗糊弄出来的。\\

孙承宗搞定了祖大寿,又去找马世龙。\\

马世龙也是辽东系将领,跟祖大寿关系很好,当时拿着袁崇焕的信去追祖大寿的,即是此人。这人的性格跟祖大寿很类似,极其强横,唯一的不同是,他连袁崇焕的面子都不给,此前有个兵部侍郎刘之纶,带兵出去跟皇太极死磕,命令他带兵救援,结果直到刘侍郎战死,马世龙都没有来。\\

但是孙大学士仍然例外,什么关宁铁骑、关宁防线,还有这帮认人不认组织的武将,都是当年他弄出来的,能压得住阵的,也只有他。\\

但手下出去找了几天,都没找到这人,因为马世龙的部队在西边被后金军隔开,没消息。\\

但孙承宗是有办法的,他出了点钱,找了几个人当敢死队,拿着他的手书,直接冲过后金防线,找到了马世龙。\\

老领导就是老领导,看到孙承宗的信,马世龙当即表示,服从指挥,立即前来会师。\\

至此,孙承宗终于集结了辽东系最强的两支军队,他的下一个目标是:击溃入侵者。\\

皇太极退出关外,并派重兵驻守遵化、永平四城,作为后金驻关内办事处,下次来抢东西也好有个照应。\\

这种未经许可的经营行为,自然是要禁止的,崇祯三年(1630)二月,孙承宗集结辽东军,发起进攻。\\

得知孙承宗进攻的消息时,皇太极并不在意,按年份算,这一年,孙承宗都六十八了,又精瘦,风吹都要摆几摆,看着且没几天蹦头了,实在不值得在意。\\

结果如下:\\

第一天,孙承宗进攻栾城,一天,打下来了。\\

第二天,进攻迁安,一天,打下来了。\\

第三天,皇太极坐不住了,他派出了援兵。\\

带领援兵的,是皇太极的大哥,四大贝勒之一的阿敏。\\

阿敏是皇太极的大哥,在四大贝勒里,是很能打的。派他去,显示了皇太极对孙承宗的重视,但我始终怀疑,皇太极跟阿敏是有点矛盾的。\\

因为战斗结果实在是惨不忍睹。\\

阿敏带了五千多人到了遵化,正赶上孙承宗进攻,但他刚到,看了看阵势,就跑路了。\\

孙承宗并没有派兵攻城,他只是在城下,摆上了所有的大炮。\\

战斗过程十分无聊,孙承宗对炮兵的使用已经炉火纯青,几十炮打完,城墙就轰塌了,阿敏还算机灵,早就跑到了最后一个据点——永平。\\

如果就这么跑回去,实在太不像话,所以阿敏在永平城下摆出了阵势,要跟孙承宗决战。\\

决战的过程就不说了,直接说结果吧,因为从开战起,胜负已无悬念,孙承宗对战场的操控,已经到了炉火纯青的地步,大炮轰完后,骑兵再去砍,真正实现了无缝对接。\\

阿敏久经沙场,但在孙老头面前,军事技术还是小学生水平,连一天都没撑住,白天开打下午就跑了,死伤四千余人,连他自己都负了重伤,差点没能回去。\\

就这样,皇太极固守的关内四城全部失守,整个过程只用五天。\\

消息传到京城,崇祯激动了,他二话不说,立马跑到祖庙向先辈汇报,并认定,从今以后,就靠孙承宗了。\\

事情就这样结束了,自崇祯二年十一月起,皇太极率军进入关内,威胁北京,沿途烧杀抢掠,所过之地实行屠城,尸横遍野,史称“己巳之变”。\\

在这场战争中,无辜百姓被杀戮,经济受到严重破坏,包括满桂在内的几位总兵阵亡,袁崇焕下狱,明朝元气大伤。\\

但一切已经过去,对于崇祯而言,明天比昨天更重要。\\

当然,在处理明天的问题前,必须先处理昨天的问题。\\

这个问题的名字叫做袁崇焕。\\

\subsection{对话}
怎么处理袁崇焕,这是个问题。\\

其实崇祯并不想杀袁崇焕。\\

十二月一日,逮捕袁崇焕的那天,崇祯给了个说法——解职听堪。\\

这四个字的意思是,先把职务免了,再看着办。\\

看着办,也就是说可以不办。\\

事实上,当时帮袁崇焕说话的人很多,看情形关几天没准就放了,将来说不定还能复职。\\

但九个月后,崇祯改变了主意,他已下定决心,处死袁崇焕。\\

为什么?\\

对于这一变化,许多人的解释,都来源于一个故事。\\

故事是这样的:\\

崇祯二年(1629)十一月二十八日,在北京城外无计可施的皇太极,决定玩个阴招。\\

他派人找来了前几天抓住的两个太监,并把他们安排到了一个特定的营帐里,派专人看守。\\

晚上,夜深人静之时,在太监的隔壁营帐,住进了两个人,这两个人用人类能够听见的声音(至少太监能听见),说了一个秘密。\\

秘密的内容是袁崇焕已经和皇太极达成了密约,过几天,皇太极攻击北京,就能直接进城。\\

这两个太监不负众望,听见了这个秘密,第二天,皇太极又派人把他们给送了回去。\\

他们回去之后,就找到了相关部门,把这件事给说了,崇祯大怒,认定袁崇焕是个叛徒,最终把他给办了。\\

故事讲完了。\\

这是个相当智慧且相当胡扯的故事。\\

二十年前,我刚上小学二年级的时候,曾相信过这个故事,后来我长大了,就不信了。\\

但把话说绝了,似乎不太好,所以我更正一下:如果当事人全都是小学二年级水平,故事里的诡计是可以成功的。\\

因为这个故事实在太过幼稚。\\

首先,你要明白,崇祯不是小学二年级学生,他是一个老练成熟的政治家,也是大明的最高领导。\\

三年前,满朝都是阉党,他啥都没说,只凭自己,就摆平了无法无天的魏忠贤;两年前,袁崇焕不经许可,干掉了毛文龙,他还是啥都没说。\\

明朝的言官很有职业道德,喜欢告状,自打袁崇焕上任,他的检举信就没停过,说得有鼻子有眼,某些问题可能还是真的,他仍然没说。\\

敌军兵临城下,大家都骂袁崇焕是叛徒,他脱掉自己的衣服,给袁崇焕披上,打死他都没说。\\

所以最后,他听到了两个从敌营里跑出来的太监的话,终于说了:杀掉袁崇焕。\\

无语,彻底的无语。\\

我曾十分好奇,这个让人无语的故事到底是怎么来的。\\

经过比对记载此事的几十种史料,我确定,这个故事最早出现的地方,是清军入关后,由清朝史官编撰的《清太宗实录》。\\

明白了。\\

记得当年我第一次去看清朝入关前的原始史料,曾经比较烦,因为按照常规,这些由几百年前的人记录的资料,是比较难懂的,而且基本都是满文,我虽认识几个,但要看懂,估计是很难的。\\

结果大吃一惊。\\

我看懂了,至少明白这份资料说些什么,且毫不费力,因为在我翻开的那本史料里,有很多绣像。\\

所谓绣像,用今天的话说就是插图,且画工很好,很详细,打仗、谈事都画出来,是个人就能看明白。\\

后来我又翻过满洲实录,也有很多插图,比如宁远之战、锦州之战,都画得相当好。\\

这是个比较奇怪的现象,古代的插图本图书很多,比如金瓶梅、西游记等等,但通常来讲,类似政治文书、历史记录之类的玩意,为示庄重,是没有插图的,从司马迁、班固,到修明史的张廷玉,二十五史,统统地没有。顺便说句,如果哪位仁兄能够找到司马迁版原始插图史记,或是班固版插图汉书,记得通知我,多少钱我都收。\\

疑惑了很久后,我终于找到了答案——文化。\\

后金是游牧民族,文化比较落后,虽说时不时也有范文程之类的文化人跑过去,但终究是差点,汉字且不说,满文都是刚造出来的,认识的人实在太少。\\

但这么多年,都干过些什么事,必须要记,开个会、谈个话之类的,一个个传达太费劲,写成文字印出去,许多人又看不懂,所以就搞插图版,认字的看字,不认字的就当连环画看,都能明白。\\

而在军事作战上,这点就更为明显了。\\

努尔哈赤、皇太极以及后来的多尔衮,都是卓越的军事家,能征善战,但基本都是野路子练出来的,属于实干派。在这方面,明朝大致相反,孙承宗袁崇焕都是考试考出来的,属于理论派。\\

打仗这个行当,和打架有点类似,被人拍几砖头,下次就知道该拿菜刀还是板砖,朝哪下手更狠,老是当观众,很难有技术上的进步。\\

所以在战场上,卷袖子猛干的实干派往往比读兵书的理论派混得开。\\

但马克思同志告诉我们,理论一旦与实践结合,就会产生巨大的能量,成功范例如孙承宗等,都是旷世名将。\\

皇太极等人及时意识到了自己工作中的不足,于是他们摆事实,找差距,决定普及理论。\\

在明朝找人来教,估计是不行了,所以教育的主要方法,是读兵书。反正兵书也不是违禁品,找人去明朝采购回来,每人发一本,慢慢看。\\

工作进行得十分顺利,托人到关内去买,但采购员到地方,就傻眼了。\\

因为从古至今,兵书很多,什么太公兵法、孙子兵法、六韬三略且不说,光是明代,兵书就有上百种,是出版行业的一支生力军。\\

面对困难,皇太极们没有气馁,他们经过仔细研讨比较,终于确定了最终的兵法教材,并大量采购,保证发到每个高级将领手中。\\

此后无论是行军还是打仗,后金军的高级将领们都带着这本指定兵法教材,早晚阅读。\\

这本书的名字,叫做《三国演义》。\\

其实没必要吃惊,毕竟孙子兵法之类的书,确实比较深奥,到京城街上拉个人回来,都未必会读。要让天天骑马打仗的人读,实在勉为其难,当时《三国演义》里的语言,大致就相当于是白话文了,方便理解,而且我相信,这本书很容易引起后金将领们的共鸣——有插图。\\

没错,答案就在这本书中。\\

所谓反间计的故事,如不知来源,可参考《三国演义》之蒋干中计,综合上述资料,以皇太极们的文化背景,能编出这么个故事,差不多了。\\

但更关键的,是下一个问题——为什么要编这个故事。\\

这个问题困惑了我三年,一次偶然的机会,让我找到了答案——我的答案。\\

我认定,这是一个阴谋,一个蓄谋已久且极其高明的阴谋。\\

关于此阴谋的来龙去脉,鉴于本人为此思考了很久,所以我决定,歇口气,等会再讲。\\

其实改变崇祯主意的,并不是那个幼稚的反间计,而是一次谈话。\\

这次谈话发生在一年前,谈话的两个人,分别是内阁大学士钱龙锡,和刚刚上任的蓟辽督师袁崇焕。\\

谈话内容如下:\\

钱龙锡:平辽方略如何?\\

袁崇焕:东江、关宁而已。\\

钱龙锡:东江何解?\\

袁崇焕:毛文龙者,可用则用之,不可用则除之。\\

翻译一下,意思大致是这样的:钱龙锡问,你上任后准备怎么干。袁崇焕答,安顿东江和关宁两个地方。钱龙锡又问:为什么要安顿东江。\\

袁崇焕答:东江的毛文龙,能用就用,不能用就杀了他。\\

按说这是两人密谈,偏偏就被记入了史料,实在是莫名其妙。\\

而且这份谈话记录看上去似乎也没啥,钱龙锡问袁崇焕的打算,袁崇焕说准备收拾毛文龙,仅此而已。\\

但杀死袁崇焕的,就是这份谈话记录。\\

崇祯二年(1629)十二月七日,御史高捷上疏,弹劾钱龙锡与袁崇焕互相勾结,一番争论之后,钱龙锡被迫辞职。\\

著名史学家孟森曾说过,明朝有两大祸患:第一是太监,其次是言官。\\

我认为,这句话是错的。言官应该排在太监的前面,如太监是流氓,言官就是流氓2.0版本——文化流氓。\\

鉴于明代政治风气实在太过开明,且为了保持政治平衡,打朱元璋起,皇帝就不怎么管这帮人。结果脾气越惯越大,有事说事,没事说人,逮谁骂谁,见谁踩谁(包括皇帝),到了崇祯,基本已经形成了有组织,有系统的流氓集团,许多事情就坏在他们的手里。\\

在这件事上,他们表现得非常积极,此后连续半年,关于袁崇焕同志叛变、投敌乃至于生活作风等多方面问题的黑材料源源不断,一个比一个狠(许多后人认定所谓袁崇焕投敌卖国的铁证,即源自于此)。\\

就这么骂了半年,终于出来个更狠的。\\

崇祯三年(1630)八月,山东御史史范上疏,弹劾钱龙锡收受袁崇焕贿赂几万两,连钱放在哪里,都说得一清二楚。\\

太阴险了。\\

在明代,收点黑钱,捞点外快,基本属于内部问题,不算啥事,但这封奏疏却截然不同。\\

因为他说,送钱的人是袁崇焕。\\

这钱就算是阎王送的,都没问题,惟独不能是袁崇焕。\\

因为袁崇焕是边帅,而钱龙锡是内阁大臣。按照明朝规定,如果边帅勾结近臣,必死无疑(有谋反嫌疑)。\\

十天后,崇祯开会,决定,处死袁崇焕。\\

崇祯二年(1629)十二月袁崇焕入狱,一群人围着骂了八个月,终于,骂死了。\\

事情就是这样吗?\\

不是。\\

在那群看似漫无目的,毫无组织的言官背后,是一双黑手,更正一下,是两双。\\

这两双手的主人,一个叫温体仁,一个叫周延儒。\\

周延儒同志前面已经介绍过了,这里讲一下温体仁同志的简历:男,浙江湖州人,字长卿,万历二十六年进士。\\

这两人后面还要讲,这里就不多说了,对这二位有兴趣的,可以去翻翻明史。顺提一下,很好找,直接翻奸臣传,周延儒同志就在严嵩的后面,接下来就是温体仁。\\

应该说,袁崇焕从“听堪”,变成了“听斩”,基本上就是这二位的功劳。但这件事情,最有讽刺意味的,也就在这里。\\

因为温体仁和周延儒,其实跟袁崇焕没仇,且压根儿就没想干掉袁崇焕。\\

他们真正想要除掉的人,是钱龙锡。\\

有点糊涂了吧,慢慢来。\\

一直以来,温体仁和周延儒都想解决钱龙锡,可是钱龙锡为人谨慎,势力很大,要铲除他非常困难。十分凑巧,他跟袁崇焕的关系很好,这次恰好袁崇焕又出了事,所以只要把袁崇焕的事情扯大,用他的罪名,把钱龙锡拉下水,就能达到目的。\\

袁崇焕之所以被杀,不是因为他自己,而是因为钱龙锡,钱龙锡之所以出事,不是因为他自己,而是因为袁崇焕。\\

幕后操纵,言官上疏,骂声一片,只是为了一个政治目的。\\

接下来要解开的迷题是,他们为什么要除掉钱龙锡。\\

有人认为,这是一个复仇的问题。是由于党争引起的,周延儒和温体仁都是阉党,因为被整,所以借此事打击东林党,报仇雪恨。\\

我认为,这是一个历史基本功问题,是由于史料读得太少引起的。\\

周延儒和温体仁绝不是阉党,虽然他们并非什么好鸟,但这一点我是可以帮他们二位担保的。事实上,阉党要有他们这样的人才,估计也倒不了。\\

崇祯元年(1628),就在崇祯大张旗鼓猛捶阉党的时候,温体仁光荣提任礼部尚书,周延儒荣升礼部侍郎。堂堂阉党,如此顶风作案,公然与严惩阉党的皇帝勾结获得提升,令人发指。\\

在攻击袁崇焕的人中,确实有阉党,但这件事情的幕后策划者,却绝非同类,当一切的伪装去除后,真正的动机始终只有俩字——权力。\\

内阁的权力很大,位置却太少,要把自己挤上去,只有把别人挤下来。事实上,他们确实达到了目的,由于袁崇焕的事太大,钱龙锡当即提出辞职,而跟钱龙锡关系很好的大学士成基命几个月后也下课,温体仁入阁,成为了大学士。\\

而袁崇焕,只是一个无辜的牺牲品。\\

崇祯三年(1630)八月十六日,崇祯在平台召开会议——第四次会议。\\

第一次,他提拔袁崇焕,袁崇焕很高兴;第二次,他脱衣服给袁崇焕,袁崇焕很感动;第三次,他抓了袁崇焕,袁崇焕很意外;第四次,他要杀掉袁崇焕,袁崇焕不在。\\

袁崇焕虽没办法与会(坐牢中),却毫无妨碍会议的盛况,参加会议的各单位有内阁、六部、都察院、大理寺、通政司、五府、六科、锦衣卫等等,连翰林院都来凑了人数。\\

人到齐了,崇祯开始发言,发言的内容,是列举袁崇焕的罪状。主要包括给钱给人给官,啥都没干,且杀掉毛文龙,放纵敌人长驱而入,消极出战等等。\\

讲完了,问:\\

“三法司如何定罪?”\\

没人吱声。\\

弄这么多人来,说这么多,还问什么意见,想怎么办就怎么办吧。\\

于是,崇祯说出了他的裁决:\\

依律,凌迟。\\

现场鸦雀无声。\\

袁崇焕的命运就这样确定了。\\

他是冤枉的。\\

在场的所有人,都是凶手。\\

温体仁、周延儒未必想干掉袁崇焕,崇祯未必不知道袁崇焕是冤枉的,袁崇焕未必知道自己为什么会死。\\

但他就是死了。\\

很滑稽,历史有时候就是这么滑稽。\\

袁崇焕被押赴西市,行刑。\\

或许到人生的最后一刻,他都不知道自己为什么会死,他永远也不会知道,在这个世界上,有着许多或明或暗的规则,必须适应,必须放弃原则,背离良知,和光同尘,否则,无论你有多么伟大的抱负,多么光辉的理想,都终将被湮灭。\\

袁崇焕是不知道和光同尘的,由始至终,他都是一个不上道的人。他有才能,有抱负,有个性,施展自己的才能,实现自己的抱负,彰显自己的个性,如此而已。\\

那天,袁崇焕走出牢房,前往刑场,沿途民众围观,骂声不绝。\\

他最后一次看着这个他曾为之奉献一切的国家,以及那些他用生命护卫,却谩骂指责他的平民。\\

倾尽心力,呕心沥血,只换来了这个结果。\\

我经常在想,那时候的袁崇焕,到底在想些什么。\\

他应该很绝望,很失落,因为他不知道,什么时候他的冤屈才能被洗刷,他的抱负才能被了解,或许永远也没有那一天,他的全部努力,最终也许只是遗臭万年的骂名。\\

然而就在行刑台上,他念出了自己的遗言:\\

\begin{quote}
	\begin{spacing}{0.5}  %行間距倍率
		\textit{{\footnotesize
				\begin{description}
					\item[\textcolor{Gray}{\FA }] 一生事业总成空,
					\item[\textcolor{Gray}{\FA }] 半世功名在梦中。
					\item[\textcolor{Gray}{\FA }] 死后不愁无勇将,
					\item[\textcolor{Gray}{\FA }] 忠魂依旧守辽东。
				\end{description}
		}}
	\end{spacing}
\end{quote}

这是一个被误解、被冤枉、且即将被千刀万剐的人,在人生的最后时刻留下的诗句。\\

所以我知道了,在那一刻,他没有绝望,没有失落,没有委屈,在他的心中,只有两个字——坚持。\\

一直以来,几乎所有的人都告诉我,袁崇焕的一生是一个悲剧。\\

事实并非如此。\\

因为在我看来,他这一生,至少做到了一件事,一件很多人无法做到的事——坚持。\\

蛮荒之地的苦读书生,福建的县令,京城的小小主事,坚守孤城的宁远道,威震天下的蓟辽督师,逮捕入狱的将领,背负冤屈死去的囚犯。\\

无论得意,失意,起或是落,始终坚持。\\

或许不能改变什么,或许并不是扭转乾坤的关键人物,或许所作所为并无意义,但他依然坚定地,毫无退缩地坚持下来。\\

直到生命的最后一刻,他也没有放弃。\\
\ifnum\theparacolNo=2
	\end{multicols}
\fi
\newpage
\section{阴谋}
\ifnum\theparacolNo=2
	\begin{multicols}{\theparacolNo}
\fi
袁崇焕是一个折腾了我很久的人。\\

围绕这位仁兄的是是非非,叛徒也罢,英雄也好,几百年吵下来,毫无消停迹象。\\

但一直以来,对袁崇焕这个人,我都感到很纳闷。因为就历史学而言,历史人物的分类大致分为三级:\\

第一级:关键人物,对历史发展产生过转折性影响的,归于此类。\\

典型代表:张居正。如果没这人,就没有张居正改革,万历同志幼小的心灵没准能茁壮成长,明朝也没准会早日完蛋,总而言之,都没准。再比如秦桧,也是关键人物,他要不干掉岳飞,不跟金朝和谈,后来怎么样,也很难说。总而言之,是能给历史改道的人。\\

第二级:重要人物,对历史产生重大影响的,归于此类。\\

典型代表:戚继光。没有戚继光,东南沿海的倭寇很难平息。但此级人物与一级人物的区别在于,就算没有戚继光,倭寇也会平息,无非是个时间问题。换句话说,这类人没法改道,只能在道上一路狂奔。\\

第三级:鸡肋人物,但凡史书留名,又不属于上述两类人物的,皆归于此类。\\

典型代表:太多,就不扯了,这类人基本都有点用,但不用似乎也没问题,属路人甲乙丙丁型。\\

袁崇焕,是第二级。\\

明末是一个特别乱的年代。朱氏公司已经走到悬崖边,就快掉下去了,还有人往下踹(比如皇太极之流),也有人往上拉(比如崇祯,杨嗣昌)。出场人物很多,但大都是二、三级人物,折腾来折腾去,还是亡了。\\

一级人物也有,只有一个。\\

只有这个人,拥有改变宿命的能力——我说过了,是孙承宗。\\

关宁防线的构建者,袁崇焕、祖大寿、赵率教、满桂的提拔者,收拾烂摊子,收复关内四城,赶走皇太极的护卫者。\\

从头到尾,由始至终,都是他在忙活。\\

其实二级人物袁崇焕和一级人物孙承宗之间的差距并不大,他有坚定的决心,顽强的意志,卓越的战斗能力,只差一样东西——战略眼光。\\

他不知道为什么不能随便杀总兵,为什么不能把皇太极放进来打,为什么自己会成为党争的牺牲品。\\

所以他一辈子,也只能做个二级人物。\\

好了,现在最关键的时刻到了:\\

为什么一个二级人物,会引起这么大的争议呢?不是民族英雄,就是卖国贼。\\

卖国贼肯定不是。所谓指认袁崇焕是卖国贼的资料,大都出自当时言官们的奏疏,要么是家在郊区,被皇太极烧了;要么是跟着温体仁、周延儒混,至少也是看袁崇焕不顺眼。这帮人搞材料,那是很有一套的,什么黑写什么,偶尔几份流传在外,留到今天,还被当成宝贝。\\

其实这种黑材料,如果想看,可以找我。外面找不到的,我这里基本都有,什么政治问题、经济问题、生活作风问题,应有尽有,编本袁崇焕黑材料全集,绰绰有余。\\

至于民族英雄,似乎也有点悬,毕竟他老人家太有个性,干过些不地道的事,就水平而言,也不如孙老师,实在有点勉为其难。\\

所以一直以来,我都在思考这个问题,从未间断,因为我隐约感到,在所谓民族英雄与卖国贼之争的背后,隐藏着不为人知的秘密。\\

直到有一天,我找到了这个秘密的答案:阴谋。\\

那一天,我跟几位史学家聊天,偶尔有人说起,据某些史料及考证,其实弘光皇帝(朱由崧,南明南京政权皇帝)跟崇祯比较类似,也是相当勤政,卖命干没结果。\\

这位弘光同志,在史书上,从来就是皇帝的反面教材,吃喝嫖赌无一不精,所以我很奇怪,问:\\

“若果真如此,为何这么多年,他都是反面形象?”\\

答:\\

“因为他是清朝灭掉的。”\\

都解开了。\\

崇祯很勤政,崇祯并非亡国之君,弘光很昏庸,弘光活该倒霉,几百年来,我们都这样认为。\\

但我们之所以一直这样认为,只是因为有人这样告诉我们。\\

之所以有人这样告诉我们,是因为他们希望我们这样认为。\\

在那一刻,我脑海中的谜团终于解开,所有看似毫不相关的线索,全都连成了一线。\\

崇祯不该死,因为他是被李自成灭掉的,所以李自成在清朝所修明史里面的分类,是流寇。\\

而我依稀记得,清军入关时,他们的口号并非建立大清,而是为崇祯报仇,所以崇祯应该是正义的。\\

弘光之所以该死,因为他是被清军灭掉的,大清王朝所剿灭的对象,必须邪恶,所以,弘光应该是邪恶的。\\

在百花缭乱的历史评论背后,还是只有两字——利益。\\

但凡能争取大明百姓支持的,都要利用,但凡是大清除掉的,都是敌人。只为了同一个目的——维护大清利益,稳固大清统治。\\

掌握这把钥匙,就能解开袁崇焕事件的所有疑团。\\

其实袁崇焕之所以成为几百年都在风口浪尖上转悠,只是因为一个意外事件的发生。\\

由于清军入关时,打出了替崇祯皇帝报仇的口号,所以清朝对这位皇帝的被害,曾表示极度的同情,对邪恶的李自成、张献忠等人,则表示极度的唾弃(具体表现,可参阅明史流寇传)。\\

因此,对于崇祯皇帝,清朝的评价相当之高,后来顺治还跑到崇祯坟上哭了一场,据说还叫了几声大哥,且每次都以兄弟相称,很够哥们,但到康乾时期,日子过安稳了,发现不对劲了。\\

因为崇祯说到底,也是大明公司的最后一任董事长,说崇祯如何好,如何死得憋屈,说到最后,就会出现一个悖论:\\

既然崇祯这么好,为什么还要接受大清的统治呢?\\

所以要搞点绯闻丑闻之类的玩意,把人搞臭才行。\\

但要直接泼污水,是不行的,毕竟夸也夸了,哭也哭了,连兄弟都认了,转头再来这么一出,太没水准。\\

要解决这件事,绝不能挥大锤猛敲,只能用软刀子背后捅人。\\

最好的软刀子,就是袁崇焕。\\

阴谋的来龙去脉大致如上,如果你不明白,答案如下:\\

要诋毁崇祯,无需谩骂,无需污蔑,只需要夸奖一个人——袁崇焕。\\

因为袁崇焕是被崇祯干掉的,所以只要死命地捧袁崇焕,把他说成千古伟人,而如此伟人,竟然被崇祯干掉了,所谓自毁长城,不费吹灰之力,就能把崇祯与历史上宋高宗(杀岳飞)之流归为同类。\\

当然了,安抚大明百姓的工作还是要做,所以该夸崇祯的,还是得夸,只是夸的内容要改一改,要着力宣传他很勤政,很认真,很执着,至于精明能干之类的,可以忽略忽略。总而言之,一定要表现人物的急躁、冲动,想干却没干成的形象。\\

而要树立这个形象,就必须借用袁崇焕。\\

之后的事情就顺理成章了,把袁崇焕树立为英雄,没有缺点,战无不胜,只要有他在,就有大明江山,再适当渲染气氛,编实录,顺便弄个反间计故事,然后,在戏剧的最高潮,伟大的英雄袁崇焕——\\

被崇祯杀掉了。\\

多么愚蠢,多么自寻死路,多么无可救药。\\

就这样,在袁崇焕的叹息声中,崇祯的形象出现了:\\

一个很有想法,很有能力,却没有脑子,没有运气,没有耐心,活活被憋死的皇帝。\\

最后,打出主题语:\\

如此皇帝,大明怎能不亡?\\

收工。\\

袁崇焕就这样变成了明朝的对立面,由于他被捧得太高,所以但凡跟他作对的(特别是崇祯),都成了反面人物。\\

肯定了袁崇焕,就是否定了崇祯,否定了明朝,清朝弄到这么好的挡箭牌,自然豁出去用,所以几百年下来,跟袁督师过不去的人也很多,争来争去,一直争到今天。\\

说到底,这就是个套。\\

几百年来,崇祯和袁崇焕,还有无数的人,都在这个套子里,被翻来覆去,纷争、吵闹,自己却浑然不知。\\

所以,应该戳破它。\\

当然,这一切只是我的看法,不能保证皆为真理,却可确定绝非谬误。\\

其实无论是前世的纷争,还是后代的阴谋,对袁崇焕本人而言,都毫无意义。他竭尽全力,立下战功,成为了英雄,却背负着叛徒的罪名死去。\\

很多人曾问我,对袁崇焕,是喜欢,还是憎恶。\\

对我而言,这是个没有意义的问题,因为我坚信历史的判断和评价,一切的缺陷和荣耀,都将在永恒的时间面前,展现自己的面目,没有伪装,没有掩饰。\\

所以我竭尽所能,去描述一个真实的袁崇焕:并非天才,并非优等生,却运气极好,受人栽培,意志坚定,却又性格急躁,同舟共济,却又难以容人,一个极其单纯,却又极其复杂的人。\\

在这世上,只要是人,都复杂,不复杂的,都不是人。\\

袁崇焕很复杂,他极英明,也极愚蠢,曾经正确,也曾经错误。其实他被争议,并不是他的错,因为他本就如此,他很简单的时候,我们以为他很复杂,他很复杂的时候,我们以为他很简单。\\

事实上,无论叛徒,或是英雄,他都从未变过,变的,只是我们自己。\\

越过几百年的烟云,我看到的袁崇焕,并没有那么复杂,他只是一个普通的人,在那个风云际会的时代,抱持着自己的理想,坚持到底。\\

即使这理想永远无法实现,即使这注定是个悲剧的结尾,即使到人生的最后一刻,也永不放弃。\\

有时候,我会想起这个人,想起他传奇的一生,他的光荣,他的遗憾。\\

有时候,我看见他站在我的面前,对我说:\\

我这一生,从没有放弃。\\

\subsection{抽签}
对袁崇焕而言,一切都结束了,但对崇祯而言,生活还要继续,明天,又是新的一天,当然,未必会更好。\\

他亲手除掉了有史以来最庞大、最邪恶的阉党,却惊奇地发现,另一个更强大的敌人,已经站立在他的面前。\\

这是一个看不见的敌人。\\

崇祯上台不久,就发现了一件奇怪的事:他是皇帝,大家也认这个皇帝,交代下去的事,却总是干不成,工作效率极其低下。\\

因为自登基以来,所有的大臣都在干同一件事——吵架。\\

今天你告我,明天我告你,瞎折腾,开始崇祯还以为这是某些阉党的反扑,但时间长了才发现,这是纯粹的、无组织、无纪律的吵架。\\

一夜之间,朝廷就变了,正事没人干,尽吵,且极其复杂。当年朝廷斗争,虽说残酷,好歹还分个东林党,阉党,带头的也是魏忠贤、杨涟之类的大腕,而今不同了,党争标准极低,只要是个人,哪怕是六部里的一个主事处长,都敢拉帮结伙,逮谁骂谁,搞得崇祯摸不着头脑:是谁弄出来这帮龟孙?\\

就是他自己。\\

这一切乱象的源头,来源自一年前崇祯同志的一个错误决定。\\

解决魏忠贤后,崇祯认为,除恶必须务尽,矫枉必须过正,干人必须彻底,所以开始拉清单,整阉党,但凡跟魏忠贤有关系的,拍马屁的,站过队的,统统滚他娘的。\\

这是一个极其不地道的举动。大家到朝廷来,无非是混,谁当朝就跟谁混,说几句好话,服软低头,也就是混碗饭吃。像杨涟那样的英雄人物,我们都是身不能至,心向往之,起码在精神上支持他,现在反攻倒算,打工一族,何苦呢?\\

但崇祯同志偏要把事做绝,砸掉打工仔的饭碗,那就没办法了。大家都往死里整,当年你说我是阉党,整顿我,没事,过两年我上来,不玩死你不算好汉。\\

特别是东林党,那真不是善人,逮谁灭谁,不听话的,有意见的,就打成阉党,啥事都干不成。\\

比如天启七年(1627),除掉魏忠贤后,崇祯打算重建内阁,挑了十几个人候选,官员就开始骂,这个有问题,那个是特务,搞得崇祯很头疼,选谁都有人骂,都得罪人,抓狂不已。\\

在难题面前,崇祯体现出了天才政治家的本色,闭门几天,想出了一个中国政治史上前所未有的绝招。只要用这招,无论选谁,大家都服气,且毫无怨言——枚卜。\\

天启七年(1627)十二月,在崇祯的亲自主持下,枚卜大典召开。\\

就读音而言,枚卜和没谱是很像的,实际上,效果也差不多,因为所谓枚卜,用今天的话说,就是抓阄。\\

具体方法是,把候选人的名字写在字条上,放进金瓶,然后摇一摇,再拿夹子夹,夹到的上岗,没夹到下课,完事。\\

内阁大学士,大致相当于内阁成员,首辅大学士就是总理,其他大学士就是副总理,是大明帝国除皇帝外的最高领导——抓阄抓出来的。\\

有人曾告诉我,论资排辈是个好政策,我不信,现在我认为,抓阄也是个好政策,你最好相信。\\

抓阄抓出来的,谁也没话说,且防止走后台,告黑状、搞关系等等,好歹就是一抓,都能服气,实为中华传统厚黑学、稀泥学之瑰宝。\\

崇祯同志的首任内阁就此抓齐,总共九人,除之前已经在位的三个,后面六个全是抓的,包括后来被袁崇焕拖下水的钱龙锡同志,也是这次抓出来的。\\

这是明朝有史以来最庞大的内阁之一,具体都是谁就不说了,因为没过一年,除钱龙锡外,基本都下课了。\\

下课的原因不外务以下几种:被骂走,被挤走,被赶走,自己走。\\

不是不想干,实在是环境太恶劣,明朝这帮大臣都不省油,个个开足马力,谁当政,就把谁往死里骂。特别是言官,人送外号“抹布”:干净送别人,肮脏留自己,贴切。\\

但归根结底,还是这帮孙子欠教育,内阁大臣又比较软,好好说话,就是不听,首任内阁刚成立,就一拥而上,弹来骂去,当即干挺五个。\\

这下皇帝也不干了,你们把人赶走,是痛快了,老子找谁干活?\\

所以崇祯元年(1628)十一月,崇祯决定,再抓几个。\\

吏部随即列出候选名单,准备抓阄。\\

在这份名单上,有十一个人,按说抓阄这事没谱,能不能入阁全看运气,但这一次,几乎所有的人都认定,有一个人,必定能够入阁。\\

这个人的名字,叫做钱谦益。\\

《三国演义》到了八十回后,猛人基本都死绝了,稍微有点名的,也就是姜维、刘禅之类的杂鱼。明末倒也凑合,还算名人辈出,特别是干仗的武将,什么袁崇焕、皇太极、张献忠、李自成,知名度都高。\\

文臣方面就差多了,到了明末,特别是崇祯年间,十几年里,文臣无数,光内阁大臣就换了五十个,都是肉包子打狗。就算研究历史的,估计也不认识,而其中唯一的例外,就是钱谦益。\\

钱谦益,字受之,苏州常熟人,万历三十六年进士,名人,超级名人。\\

钱谦益之所以有名,很大原因在于,他有个更有名的老婆——柳如是。\\

关于这个人的是是非非,以后再说,至少在当时,他就很有名了。\\

因为他不但饱读诗书,才华横溢,且是东林党的领导。阉党倒台,东林上台,理所应当,朝廷里从上到下,基本都是东林党,现在领导要入阁,就是探囊取物。\\

所以连钱谦益自己都认为,抓阄只是程序问题,入阁只是时间问题,洗个澡,换件衣服,就准备换单位上班了。\\

可这世上,越是看上去没事的事,就越容易出事。\\

\subsection{作弊}
钱谦益入内阁,一般说来是没有对手的,而他最终没有入阁,是因为遇上了非一般的对手。\\

在崇祯十余年的统治中,总共用过五十个内阁大臣,鉴于皇帝难伺候,下属不好管,大部分都只干了几个月,就光荣下岗。\\

只有两个人,能够延续始终,把革命进行到底,这两个人,一个是周延儒,一个是温体仁。\\

虽然二位兄弟在历史上的名声差点(奸臣传),但要论业务能力和智商,实在无与伦比。\\

不幸的是,钱谦益的对手,就是这两位。\\

之所以要整钱谦益,不是因为他们也在吏部候选名单上,实际上,他们连海选都没入,第一轮干部考察就被刷下来了。\\

海选都没进,为什么要坑决赛选手呢?\\

因为实在太不像话了。\\

海选的时候,钱谦益的职务是礼部右侍郎,而周延儒是礼部左侍郎,温体仁是礼部尚书。\\

同一个部门,副部长入阁,部长连决赛都没进,岂有此理。\\

所以两个岂有此理的人,希望讨一个公道。\\

在后世的史书里,出于某种目的,温体仁和周延儒的归类都是奸臣,也就是坏人。但仔细分析,就会发现,至少在当时,这两位坏人,都是弱势群体。\\

在当时的朝廷,东林党势力极大,内阁和六部,大都是东林派,所以钱谦益基本上算是个没人敢惹的狠角色。\\

但温部长和周副部长认为,让钱副部长就这么上去,实在太不公平,必须闹一闹。\\

于是,他们决定整理钱谦益的黑材料,经过不懈努力,他们找到了一个破绽,七年前的破绽。\\

七年前(天启元年)。\\

作为浙江乡试的主考官,钱谦益来到浙江监考,考试、选拔、出榜,考试顺利完成。\\

几天后,他回到了北京,又几天后,礼部给事中顾其中上疏弹劾钱谦益,罪名,作弊。\\

批判应试教育的人曾说,今日之高考,即是古代之进士科举,罪大恶极。\\

我觉得这句话是不恰当的,因为客观地讲,高考上榜的人,换到明代,最多就是秀才,举人可以想想,进士可以做梦。\\

明代考完,如果没有意外,基本能有官做,且至少是处级(举人除外),高考考完,大学毕业,如果没有意外,且运气好点,基本能有工作。\\

明代的进士考试,每三年一次,每次录取名额,大概是一百五十多人,现在高考,每年两次,每次录取名额……\\

所以总体说来,明代的进士考试,大致相当于今天的高考+公务员考试+高级公务员选拔。\\

只要考中,学历有了,工作有了,连级别都有了,如此好事,自然挤破头,怕挤破头,就要读书,读不过,就要作弊。\\

鉴于科举关系重大,明代规定,但凡作弊查实,是要掉脑袋的。但由于作弊前景太过美妙,所以作弊者层出不穷,作弊招数也推陈出新。由低到高,大致分为四种。\\

最初级的作弊方式,是夹带,所以明朝规定,进入考场时,每人只能携带笔墨,进考场就把门一锁,吃喝拉撒都在里面,考完才给开门。\\

为适应新形势的需要,同学们开动脑筋,比如把毛笔凿空,里面塞上小抄,或是在砚台里面夹藏,更牛一点的,就找人在考场外看准地方,把答案绑在石头上扔进去,据说射箭进去的也有。面对新局面,朝廷规定,毛笔只能用空心笔杆,砚台不能太厚,考场内要派人巡逻等等。\\

这是基本技术,更高级一点的,是第二种方法:枪手代考,明朝的同学们趁着照相技术尚未发明,四处找人代考。当然朝廷不是吃素的,在准考证上,还加上了体貌特征描述,比如面白,无须,高个等等。\\

以上两项技术,都是常用技术,且好用,为广大人民群众喜闻乐见,所以流传至今,且发扬光大。今日之大学,继承前辈遗志者,大有人在。\\

但真正有钱,有办法的,用的是第三种方法——买考题。\\

考试最重要的,就是考题,只要知道考题,不愁考不上,所以出题的考官,都是重点对象。\\

但问题是,明代规定,知情人员如果卖题,基本是先下岗再处理。轻则坐牢,重则杀头,风险太大,而且明朝为了防止作弊,还额外规定,所有获知考题人员,必须住进考场,无论如何,不许外出。\\

所以在明朝,卖考题的生意是不好做的。\\

虽然买不到考题,但天无绝人之路,有权有势的同学们还有最后一招杀手锏,此招一出,必定上榜——买考官。\\

不过,这些考官并不是出题的考官,而是改题的考官。\\

是的,知不知道题目并不重要,就算你交白卷,只要能搞定改题的人,就能金榜题名。\\

但问题是,给钱固然容易,那么多卷子,怎么对上号呢?\\

最原始的方法,是认名字,毕竟跟高考不同,考试的人就那么多,看到名字就录取。\\

魔高一尺,道高一丈,从此以后,试卷开始封名,实行匿名批改。\\

但作弊的同学们是不会甘心失败的,有的做记号,有的故意在考卷里增大字体,只为对改卷的考官说一句话:我就是给钱的那个!\\

这几招相当地有效,且难以禁止,送进去不少人,面对新形势朝廷不等不靠,经过仔细钻研,想出了一个绝妙的对策。\\

具体方法是,所有的考卷收齐后,密封姓名,不直接交给考官,而是转给一个特别的人。\\

这个人并非官员,他收到考卷后,只干一件事——抄。\\

所有的考卷,都由他重新抄写,然后送给考官批改,全程由人监督。\\

这招实在是狠,因为所有的考卷,是统一笔迹,统一形式,考官根本无从判断,且毫不影响考试成绩,可谓万无一失。\\

综上所述,作弊与反作弊的斗争是长期的,艰苦的,没有尽头的,同学们为了前途,虽屡战屡败,但屡败屡战,到明代,斗争达到了高潮。\\

高潮,就发生在天启元年的浙江。\\

在这次科举考试中,监考程序非常严密,并实行了统一抄写制度,按说是不会有问题的。\\

但偏偏就出了问题。\\

因为有人破解了统一抄写制度。\\

虽然笔迹相同,试卷相同,但这个方法,依然有漏洞,依然可以作弊。\\

作弊的具体方法是,考生事前与考官预定密码,比如一首唐诗,或是几个字,故意写在试卷的开头,或是结尾,这样即使格式与字迹改变,依然能够辨别出考卷作者。\\

在这次考试中,有一个叫钱千秋的人,买到了密码。\\

密码是七个字——一朝平步上青云。按照约定,他只要将这七个字,写在每段话的末尾,就能平步青云,金榜题名。\\

事情非常顺利,考试结束,钱千秋录取。\\

这位钱同志也相当守规矩,录取之后,乖乖地给了钱,按说事情就该结了。\\

可是意外发生了。\\

因为这种事情,一个人是做不成的,必须是团伙作案,既然是团伙,就要分赃,既然分赃,就可能不匀,既然不匀,就可能闹事,既然闹事,就必定出事。\\

钱千秋同志的情况如上,由于卖密码给他的那帮人分赃不匀,某些心态不好的同志就把大家都给告了,于是事情败露,捅到了北京。\\

但这件事情说起来,跟钱谦益的关系似乎并不大,虽然他是考官,并没有直接证据证实,他就是卖密码的人,最多也就背个领导责任。\\

不巧的是,当时,他有一个仇人。\\

这个仇人的名字,叫做韩敬,而滑稽的是,他所以跟钱谦益结仇,也是因为作弊。\\

十年前,举人钱谦益从家乡出发,前往北京参加会试,而韩敬,是他同科的同学。\\

在考场上,他们并未相识,但考试结束时,就认识了,以一种极为有趣的方式。\\

跟其他人不同,在考试成绩出来前,钱谦益就准备好当状元了,因为他作弊了。\\

但他作弊的方式,既不是夹带,也不是买考官,甚至不是买密码,而是作弊中的最高技巧——买朝廷。\\

买考题、买考官都太小儿科了,既然横竖要买,还不如直接买通朝廷,让组织考试的人,给自己定个状元,直接到位,省得麻烦。\\

所以在此之前,他已经通过熟人,买通了宫里能说得上话的几个太监,找好了主考官,考完后专门找出他的卷子,给个状元了事。\\

当然,办这种事,成本非常巨大。据说钱同志花了两万两白银,按今天的人民币算,大致是一千二百万。\\

能出得起这个价钱,还要作弊,可见作弊之诚意。\\

两万白银,买个官也行了,钱谦益出这个价,就是奔着状元名头去的,但他万没想到,还有个比他更有诚意的。\\

在考试前,韩敬也很自信,因为他也出了钱,且打了包票,必中状元。\\

可是卷子交上去后,他却得到了一个让人震惊的消息——他的卷子被淘汰了。\\

淘汰是正常的,要真有水平,就不用出钱了。\\

可问题是,人找了钱出了,怎么能收钱不办事呢?\\

韩敬在朝廷里是有关系的,于是连夜找人去查,才知道他的运气不好。偏偏改他卷子的人,是没收过钱的,看完卷子就怒了,觉得如此胡说八道的人,怎么还能考试,就判了落榜。\\

落榜不要紧,找回来再改成上榜就行。\\

韩敬同学毕竟手眼通天,找到了其他考官,帮他找卷子重新改。\\

可是找来找去,竟然没找到。后来才知道,因为那位考官太讨厌他的卷子,直接就给扔废纸堆里了,翻了半天垃圾,才算把卷子给淘回来。\\

按常理,事已至此,重新改个上榜进士,也就差不多了,但韩敬同学对名次的感情实在太深,非要把自己的卷子改成第一名。\\

但名次已经排定,且排名都是出了钱的(比如钱谦益),你要排第一,别人怎么办?\\

关键时刻,韩敬使出了绝招——加钱。\\

钱谦益找太监,出两万两,他找大太监,加价四万两,跟我斗,加死你!\\

四万两,大致是两千四百万人民币,出这个价钱,买个状元,无语。\\

更无语的,是钱谦益,出了这么多钱,都打了水飘,好在太监办事还比较地道,虽然没有状元,也给了个探花(第三名)。\\

花这么多钱,买个状元,并不是吃饱了撑的。要知道,状元不光能当官,还能名垂青史。自古以来,状元都是最高荣誉,且按规定,每次科举的录取者,都刻在石碑上,放在国子监里供后代瞻仰(现在还有),状元的名字就在首位,几万两买个名垂青史,值了。\\

但钱谦益同志是不值的,虽说也是探花,但花了这么多钱,只买了个次品,心理极不平衡,跟韩敬同学就此结下梁子。\\

韩敬是幸运的,也是不幸的,他虽然加了钱,买到了状元,却并不知道得罪钱谦益的后果。\\

因为钱同学虽然钱不够多,关系不够硬,却很能混。进朝廷后没多久就交了几个朋友,分别叫做孙承宗、叶向高、杨涟、左光斗。\\

概括成一句话,他投了东林党。\\

万历末年,东林党是很有点能量的,而钱谦益也并不是个很大方的人,所以没过几年搞京察的时候,韩敬同志就因为业绩不好,被整走了。\\

背负血海深仇的韩敬同志,终于等到了现在的机会,他大肆宣扬,应该追究钱谦益的责任。\\

但是说来说去,毕竟只是领导责任,经过朝廷审查,钱千秋免去举人头衔,充军,主考官(包括钱谦益)罚三个月工资。\\

七年之后。\\

在周延儒和温体仁眼前的,并不是一起无足轻重的陈年旧案,而是一个千载难逢的机会。\\

在很多史书里,这都是一段催人泪下的段落,强大且无耻的温体仁和周延儒,组成了恶毒的同盟,坑害了无辜弱小的钱谦益。\\

我觉得,这个说法,如果倒转过来,是比较符合事实的。\\

首先,温体仁和周延儒无不无耻,还不好讲;钱谦益无辜,肯定不是。\\

温体仁之所以要整钱谦益,是个心态问题。\\

他是当年内阁首辅沈一贯的门生,钱谦益刚入伙的时候,他就是老江湖了,在朝廷里混迹多年,威信很高,而且他还是礼部部长,专管钱谦益,居然还被抢了先,实在郁闷。\\

周延儒则不同,他是真吃亏了,且吃的就是钱谦益的亏。\\

其实原本推选入阁名单时,排在第一的,应该是周延儒,因为他状元出身,且受皇帝信任,但钱谦益感觉此人威胁太大,怕干不过他,就下了黑手,派人找到吏部尚书王永光,做了工作,把周延儒挤了。\\

其次,在当时朝廷里,强大的那个,应该是钱谦益。他是东林党领袖,一呼百应,从上到下,都是他的人,温体仁周延儒基本算是孤军奋战。\\

当时的真实情况大致如此。\\

形势很严峻,但同志们很勇敢,在共同的敌人面前,温体仁、周延儒擦干眼泪,决定跟钱谦益玩命。\\

周延儒问温体仁,打算怎么干。\\

温体仁说,直接上疏弹劾钱谦益。\\

周延儒问,然后呢?\\

温体仁说,没有然后。\\

周延儒很生气,因为他认为,温体仁在拿他开涮,一封奏疏怎么可能干倒钱谦益呢?\\

温体仁没有回答。\\

周延儒告诉温体仁,先找几个人通通气,做些工作,搞好战前准备,别急着上疏。\\

第二天,温体仁上疏了。\\

就文笔而言,这封奏疏非常一般,主要内容是弹劾钱谦益主使作弊。也没玩什么写血书,沐浴更衣之类的花样,也没做工作,没找人,递上去就完了。\\

然后他告诉周延儒,必胜无疑。\\

周延儒认为,温体仁是疯了。\\
\ifnum\theparacolNo=2
	\end{multicols}
\fi
\newpage
\section{斗争技术}
\ifnum\theparacolNo=2
	\begin{multicols}{\theparacolNo}
\fi
\subsection{辩论}
事情的发展,跟周延儒想得差不多,朝廷上下一片哗然,崇祯也震惊了,决定召开御前会议,辩论此事。\\

辩论议题:浙江作弊案,钱谦益有无责任。\\

辩论双方:\\

正方,没有责任,辩论队成员:钱谦益、内阁大学士李标、钱龙锡、刑部尚书乔允升,吏部尚书王永光……(以下省略)\\

反方,有责任,辩论队成员:温体仁、周延儒(以下无省略)。\\

崇祯元年(1628)十一月六日,辩论开始。\\

所有的人,包括周延儒在内,都认定温体仁必败无疑。\\

奇迹,就是所有人都认定不可能发生,却终究发生的事。\\

这场惊天逆转,从皇帝的提问开始:\\

“你说钱谦益受贿,是真的吗?”\\

温体仁回答:是真的。\\

于是崇祯又问钱谦益:\\

“温体仁说的话,是真的吗?”\\

钱谦益回答:不是。\\

辩论陈词就此结束,吵架开始。\\

温体仁先声夺人,说,钱千秋逃了,此案未结。\\

钱谦益说:查了,有案卷为证。\\

温体仁说:没有结案。\\

钱谦益说:结了。\\

刑部尚书乔允升出场。\\

乔允升说:结案了,有案卷。\\

温体仁吃了秤砣:没有结案。\\

吏部尚书王永光出场。\\

王永光说:结案了,我亲眼看过。\\

礼部给事中章允儒出场。\\

章允儒说:结案了,我曾看过口供。\\

温体仁很顽强:没有结案!\\

崇祯做第一次案件总结:\\

“都别废话了,把案卷拿来看!”\\

休会,休息十分钟。\\

再次开场,崇祯问王永光:刑部案卷在哪里?\\

王永光说:我不知道,章允儒知道。\\

章允儒出场,回答:现在没有,原来看过。\\

温体仁骂:王永光和章允儒是同伙,结党营私!\\

章允儒回骂:当年魏忠贤在位时,驱除忠良,也说结党营私!\\

崇祯大骂:胡说!殿前说话,竟敢如此胡扯!抓起来!\\

这句话的对象,是章允儒。\\

章允儒被抓走后,辩论继续。\\

温体仁发言:推举钱谦益,是结党营私!\\

吏部尚书王永光发言:推举内阁人选,出于公心,没有结党。\\

内阁大臣钱龙锡发言:没有结党。\\

内阁大臣李标发言:没有结党。\\

崇祯总结陈词:推举这样的人(指钱谦益),还说出于公心!\\

二次休会。\\

再次开场,钱龙锡发言:钱谦益应离职,听候处理。\\

崇祯发言:我让你们推举人才,竟然推举这样的恶人,今后不如不推。\\

温体仁发言:满朝都是钱谦益的人,我很孤立,恨我的人很多,希望皇上让我告老还乡。\\

崇祯发言:你为国效力,不用走。\\

辩论结束,反方,温体仁获胜,逆转,就此完成。\\

史料记载大致如此,看似平淡,实则暗藏玄机。\\

这是一个圈套,是温体仁设计的完美圈套。\\

这个圈套分三个阶段,共三招。\\

第一招,开始辩论时,无论对方说什么,咬定,没有结案。\\

这个举动毫不明智,许多人被激怒,出来跟他对骂指责他。\\

然而这正是温体仁的目的。\\

很快,奇迹就发生了,章允儒被抓走,崇祯的天平向温体仁倾斜。\\

接下来,温体仁开始实施第二步——挑衅。\\

他直接攻击内阁,攻击所有大臣,说他们结党营私。\\

于是大家都怒了,纷纷出场,驳斥温体仁。\\

这也是温体仁的目的。\\

至此,崇祯认定,钱谦益与作弊案有关,应予罢免。\\

第三阶段开始,内阁的诸位大人终于意识到,今天输定了,所以主动提出,让钱谦益走人,温体仁同志随即使出最后一招——辞职。\\

当然,他是不会辞职的,但走到这一步,摆摆姿态还是需要的。\\

三招用完,大功告成。\\

温体仁没有魔法,这个世界上也没有奇迹,他之所以肯定他必定能胜,是因为他知道一个秘密,崇祯心底的秘密。\\

这个秘密的名字,叫做结党。\\

温体仁老谋深算,他知道,即使朝廷里的所有人,都跟他对立,只要皇帝支持,就必胜无疑,而皇帝最不喜欢的事情,就是结党。\\

崇祯登基以来,干掉了阉党,扶植了东林党,却没能消停,朝廷党争不断,干什么什么都不成,所以最恨结党。\\

换句话说,钱谦益有无作弊,并不重要,只要把他打成结党,就必定完蛋。\\

事实上,钱谦益确实是东林党的领袖,所以在辩论时,务必不断挑事,耍流氓,吸引更多的人来骂自己,都无所谓。\\

因为最后的决断者,只有一个。\\

当崇祯看到这一切时,他必定会认为,钱谦益的势力太大,结党营私,绝不可留。\\

这就是温体仁的诡计,事实证明,他成功了。\\

通过这个圈套,他骗过了崇祯,除掉了钱谦益,所有的人都被他蒙在鼓里,至少他自己这样认为。\\

但事实可能并非如此,这场辩论的背后,真正的胜利者,是另一个人——崇祯。\\

其实温体仁的计谋,崇祯未必不知道,但他之所以如此配合,是因为这是一个千载难逢的机会。\\

当时的朝廷,东林党实力很强,从内阁到言官,都是东林党,虽说就工作业绩而言,比阉党要强得多,但归根结底,也是个威胁,如此下去再不管,就管不住了。\\

现在既然温体仁跳出来,主动背上黑锅,索性就用他一把,敲打一下,提提醒,换几个人,阿猫阿狗都行,只要不是东林党,让你们明白,都是给老子打工的,老实干活!\\

当然明白人也不是没有,比如黄宗羲,就是这么想的,还写进了书里。\\

但搞倒了钱谦益,对温体仁而言,是纯粹的损人不利已,因为他老兄太过讨嫌,没人推举他,闹腾了半天,还是消停了。\\

消停了一年,机会来了,机会的名字,叫袁崇焕。\\

画了一个圈,终于回到了原点。\\

之后的事,之前都讲了,袁督师很不幸,指挥出了点问题,本来没事,偏偏和钱龙锡拉上关系,就这么七搞八搞,自己进去了,钱龙锡也下了水。\\

在很多人眼里,崇祯初年是很乱的,钱谦益、袁崇焕、钱龙锡、作弊、通敌、下课。\\

现在你应该明白,其实一点不乱,事实的真相就是这么简单,只有两个字——利益,周延儒的利益,温体仁的利益,以及崇祯的利益。\\

钱谦益、袁崇焕、还有钱龙锡,都是利益的牺牲品。\\

而这个推论,有一个最好的例证。袁崇焕被杀掉后,钱龙锡按规定,也该干掉,死刑批了,连刑场都备好,家人都准备收尸了,崇祯突然下令:不杀了。\\

关于这件事,许多史书上都说,崇祯皇帝突然觉悟。\\

我觉得,持这种观点的人,确实应该去觉悟一下,其实意思很明白,教训教训你,跟你开个玩笑,临上刑场再拉下来,很有教育意义。\\

周延儒和温体仁终究还是成功了,崇祯三年(1630)二月,周延儒顺利入阁,几个月后,温体仁入阁。\\

温体仁入阁,是周延儒推荐的,因为崇祯最喜欢的,就是周延儒。但周兄还是很讲义气,毕竟当年全靠温兄在前面踩雷,差点被口水淹死,才有了今天的局面,拉兄弟一把,是应该的。\\

其实就能力而言,周延儒和温体仁都是能人,如果就这么干下去,也是不错的,毕竟他们都是恶人,且手下并非善茬,换个人,估计压不住阵。\\

但所谓患难兄弟,基本都有规律,拉兄弟一把后,就该踹兄弟一脚了。\\

最先开踹的,是温体仁。\\

钱龙锡被皇帝赦免后,第一个上门问候的,不是东林党,而是周延儒。\\

周兄此来的目的,是邀功。什么皇上原本很生气,很愤怒,很想干掉你,但是关键时刻,我挺身而出,在皇帝面前帮你说了很多好话,你才终于脱险云云。\\

这种先挖坑,再拉人,既做婊子,又立牌坊的行为,虽很无聊,却很有效,钱龙锡很感动,千恩万谢。\\

周延儒走了,第二个上门问候的来了,温体仁。\\

温体仁的目的,大致也是邀功,然而意外发生了。\\

因为钱龙锡同志刚从鬼门关回来,且经周延儒忽悠,异常激动,温兄还没开口,钱龙锡就如同连珠炮般,把监狱风云,脱离苦海等前因后果全盘托出。\\

特别讲到皇帝愤怒,周延儒挺身而出,力挽狂澜时,钱龙锡同志极为感激,眼泪哗哗地流着。\\

温体仁安静地听完,说了句话。\\

这句话彻底止住了钱龙锡的眼泪:\\

“据我所知,其实皇上不怎么气愤。”\\

啥?不气愤?不气愤你邀什么功?混蛋!\\

所以钱龙锡气愤了。类似这种事情,自然有人去传,周延儒知道后,也很气愤——我拉你,你踹我?\\

温体仁这个人,史书上的评价,大都是八个字:表面温和,深不可测。\\

其实他跟周延儒的区别不大,只有一点:如果周延儒是坏人,他是更坏的坏人。\\

对他而言,敌人的名字是经常换的,之前是钱谦益,之后是周延儒。\\

所以在搞倒周延儒这件事上,他是个很坚定,很有毅力的人。\\

不久之后,他就等到了机会,因为周延儒犯了一个与钱谦益同样的错误——作弊。\\

崇祯四年,周延儒担任主考官,有一个考生跟他家有关系,就找到他,想走走后门,周考官很大方,给了个第一名。\\

应该说,对此类案件,崇祯一向是相当痛恨的,更巧的是,这事温体仁知道了,找了个人写黑材料,准备下点猛药,让周延儒下课。\\

不幸的是,周延儒比钱谦益狡猾得多。听到风声,不慌不忙地做了一件事,把问题搞定了,充分反映了他的厚黑学水平。\\

他把这位考生的卷子,交给了崇祯。\\

应该说,这位作弊的同学还是有点水平的,崇祯看后,十分高兴,连连说好,周延儒趁机添把火,说打算把这份卷子评为第一,皇帝认为没有问题,就批了。\\

皇帝都过了,再找麻烦,就是找抽了,所以这事也就过了。\\

但温体仁这关,终究是过不去的。\\

崇祯年间的十七年里,一共用了五十个内阁大臣,特别是内阁首辅,基本只能干几个月,任期超过两年的,只有两个人。\\

第二名,周延儒,任期三年。\\

第一名,温体仁,任期八年。\\

温首辅能混这么久,只靠两个字,特别。\\

特别能战斗,特别能折腾。\\

在此后的一年里,温体仁无怨无悔、锲而不舍地折腾着,他不断地找人黑周延儒,但皇帝实在很喜欢周首辅,虽屡败屡战,却屡战屡败,直到一年后,他知道了一句话。\\

就是这句话,最终搞定了千言万语都搞不定的周延儒。\\

全文如下:\\

“余有回天之力,今上是羲皇上人。”\\

前半句很好懂,意思是我的能量很大。\\

后半句很不好懂,却很要命。\\

今上,是指崇祯,所谓羲皇上人,具体是谁很难讲,反正是原始社会的某位皇帝,属于七十二帝之一,就不扯了,而他的主要特点,是不管事。\\

翻译过来,意思是,我的能量很大,皇上不管事。\\

这句话是周延儒说的,是跟别人聊天时说的,说时旁边还有人。\\

温体仁把这件事翻了出来,并找到了证人。\\

啥也别说了,下课吧。\\

周延儒终于走了,十年后,他还会再回来,不过,这未必是件好事。\\

朝廷就此进入温体仁时代。\\

按照传统观点,这是一个极其黑暗的时代,在无能的温体仁的带领下,明朝终于走向了不归路。\\

我的观点不太传统,因为我看到的史料告诉我,这并非事实。\\

温体仁能够当八年的内阁首辅,只有一个原因——他能够当八年的内阁首辅。\\

作为内阁首辅,温体仁具备以下条件:首先,他很精明强干。据说一件事情报上来,别人还在琢磨,他就想明白了,而且能很快做出反应。其次,他熟悉政务,而且效率极高,还善于整人(所以善于管人)。\\

最后,他不是个好人。当然,对朝廷官员而言,这一点在某些时候,绝对不是缺点。\\

估计很多人都想不到,这位温体仁还是个清官,不折不扣的清官,做了八年首辅,家里还穷得叮当响,从来不受贿,不贪污。\\

相对而言,流芳千古的钱谦益先生,就有点区别了,除了家产外,也很能挣钱(怎么来的就别说了),经常出没红灯区。六十多岁了,还娶了柳如是。明朝亡时,说要跳河殉国,脚趾头都还没下去,就缩了回来,说水冷,不跳了,就投降了清朝。清朝官员前来拜访,看过他家后,发出了同样的感叹:你家真有钱。\\

温体仁未必是奸臣,钱谦益未必是好人。不需要惊讶,历史往往跟你所想的并不一样。英雄可以写成懦夫,能臣可以写成奸臣,史实并不重要,重要的是,谁来写。\\

温体仁的上任,对崇祯而言,不算是件坏事。就人品而言,他确实很卑劣,很无耻,且工于心计,城府极深,但要镇住朝廷那帮大臣,也只能靠他了。\\

应该说,崇祯是有点想法的,毕竟他手中的,不是烂摊子,而是一个烂得不能再烂的摊子。边关战乱,民不聊生,政治腐败,朝廷混乱,如此下去,只能收摊。\\

崇祯同志一直很担心,如果在他手里收摊,将来下去了,没脸见当年摆摊的朱重八(后来他用一个比较简单的方法办到了)。\\

所以执政以来,他干了几件事,希望力挽狂澜。\\

第一件事,就是肃贪。\\

到崇祯时期,官员已经相当腐败,收钱办事,就算是好人了。对此,崇祯非常地不满,决心肃贪。\\

问题在于,明朝官场,经过二百多年的磨砺,越来越光,越来越滑。潜规则、明规则,基本已经形成一套行之有效的规章,大家都在里边混,就谈不上什么贪不贪了,所谓天下皆贪,即是天下无贪。\\

当然,偶尔也有个把人,是要突破规则,冒冒头的。\\

比如户部给事中韩一良,就是典型代表。\\

当崇祯下令整顿吏治时,他慷慨上书,直言污秽,而且还说得很详细,什么考试作弊内幕,买官卖官内幕,提成、陋规等等。为到达警醒世人的目的,他还坦白,自己身为言官,几个月之内,已经推掉了几百两银子的红包。\\

崇祯感动了,这都什么年月了,还有这样的人啊,感动之余,他决定在平台召开会议,召见韩一良及朝廷百官,并当众嘉奖提升。\\

皇帝很激动,后果很严重。\\

因为韩一良同志本非好鸟,也没有与贪污犯罪死磕到底的决心,只是打算骂几句出出气,没想到皇帝大人反应如此强烈,无奈,事都干了,只能硬着头皮去。\\

在平台,崇祯让人读了韩一良的奏疏,并交给百官传阅,大为赞赏,并叫出韩一良,提升他为都察院右佥都御史。\\

原本只是七品,一转眼,就成了四品。\\

我研读历史,曾总结出一条恒久不变的规律——世上的事,从没有白给的。\\

韩一良同志还没高兴完,就听到了这样一句话:\\

“此文甚好,希望科臣(指韩一良)能指出几个贪污的人,由皇帝惩处,以示惩戒。”\\

说话的人,是吏部尚书王永光。\\

王永光很不爽,自打听到这封奏疏,他就不爽了,因为他是吏部尚书,管理人事,说朝廷贪污成风,也就是说他管得不太好,所以他决定教训韩一良同志。\\

这下韩御史抓瞎了,因为他没法开口。\\

自古以来,所谓集体负责,就是不负责,所以批评集体,就是不批评。韩御史本意,也就是批评集体,反正没有具体对象,没人冒头反驳,可以过过嘴瘾。\\

现在一定要你说出来,是谁贪污,是谁受贿,就不好玩了。\\

但崇祯似乎很有兴趣,当即把韩一良叫了出来,让他指名道姓。\\

韩一良想了半天,说,现在不能讲。\\

崇祯说,现在讲。\\

韩一良说,我写这封奏疏,都是泛指,不知道名字。\\

崇祯怒了:你一个名字都不知道,竟然能写这封奏疏,胡扯!五天之内,把名字报来!\\

事儿大了,照这么搞,别说升官,能保住官就不错,韩一良回去了,在家抓狂了五天,憋得脸通红,终于憋出了一份奏疏。\\

很明显,韩一良是下了功夫的,因为在这份奏疏里,他依然没有说出名字,却列出了几种人的贪污行径,并希望有关部门严查。当然,他也知道,这样是不过了关的,就列出了几个人——已经被处理过的人。\\

反正处理过了,骂绝祖宗十八代,也不要紧。\\

这封极为滑头的奏疏送上去后,崇祯没说什么,只是下令在平台召集群臣,再次开会。\\

刚开始的时候,气氛是很和谐的,崇祯同志对韩一良说,你文章里提到的那几个人,都已经处理了,就不必再提了。\\

然后,他又很和气地提到韩一良的奏疏,比如他曾经拒绝红包,达几百两之多的优秀事迹。\\

戏演完了,说正事:\\

“是谁送钱给你的!说!”\\

韩一良同志懵了,但优秀的自律精神鼓舞了他,秉承着打死也不说的思想,到底也没说。\\

崇祯也很干脆,既然你不说,就不要干了,走人吧。\\

韩一良同志的升官事迹就此结束,御史没捞到,给事中丢了,回家。\\

然而最伤心的,并不是他,是崇祯。\\

他不知道,自己如此坦白,如此真诚,如此想干点事,怎么连句实话都换不到呢?\\

这个问题,没人能回答。\\

但要说他啥事都没干成,也不对,事实上,崇祯二年(1629),他就干过一件大事,且相当成功。\\

这年四月,刑部给事中刘懋上疏,请求清理驿站。\\

所谓驿站,就是招待所,著名的伟大的政治家、军事家、哲学家王守仁先生,就曾经当过招待所的所长。\\

当然,王守仁同志干过的职务很多,这是最差的一个。因为在明代,驿站所长虽说是公务员,论级别,还不到九品,算是不入流,还要负责接待沿途官员,可谓人见人欺。\\

所以一直以来,驿站都没人管。\\

但到崇祯这段,驿站不管都不行了。\\

因为明代规定,驿站接待中央各级官员,由地方代管。\\

这句话不好理解,说白了,就是驿站管各级官员吃喝拉撒睡,但费用自负。\\

因为明代地方政府,并没有办公经费,必须自行解决,所以驿站看起来,级别不高,也没人管。\\

但驿站还是有油水的,因为毕竟是官方招待所,上面来个人没法接待,追究到底,还是地方官吃亏,所以每年地方花在驿站上的钱,数额也很多。\\

而且驿站还有个优势,不但有钱,且有政策——摊派。\\

只要有接待任务,就有名目,就能逼老百姓,上面来个人,招待所所长自然不会自己出钱请人吃饭,就找老百姓摊,你家有钱,就出钱,没钱?无所谓,你们要相信,只要是人,就有用处,什么挑夫、轿夫,都可以干。\\

其实根据规定,过往官员,如要使用驿站,必须是公务,且出示堪合(介绍信),否则,不得随便使用。\\

也就是说说。\\

到崇祯年间,驿站基本上就成了车站,按说堪合用完了,就要上交,但这事也没人管,所以许多人用了,都自己收起来,时不时出去旅游,都用一用,更缺德的,还把这玩意当礼物,送给亲朋好友,让大家都捞点实惠。\\

鉴于驿站好处如此之多,所以但凡过路官员,无论何等妖魔鬼怪,都是能住就住,不住也宰点钱,既不住也不宰的,至少也得找几个人抬轿子,顺便送一程。\\

比如我国古代最伟大的地理学家徐霞客,云游各地(驿站),拿着堪合四处转悠,绝对没少用。\\

刘懋建议,整顿驿站,不但可以节省成本,还能减轻地方负担。\\

但问题是,怎么整顿。\\

刘懋的方法很简单,一个字——裁。\\

裁减驿站,开除富余人员,减开支,严管介绍信,非紧急不得使用。\\

按照他的说法,只要执行这项措施,朝廷一年能省几十万两白银,且地方负担能大大减轻。\\

崇祯很高兴,同意了,并且雷厉风行地执行了。\\

一年之后,上报执行成果,裁减驿站二百余处,全国各省累计减少经费八十万两,成绩显著。\\

不久之后,刘懋就滚蛋了。\\

这世上,有很多事情,看上去是好事,实际上不是,比如这件事。\\

刘懋同志干这件事,基本是“损人不利己”。国家没有好处,地方经费节省了,也省不到老百姓头上,地方吃驿站的那帮人又吃了亏,要跟他拼命,闹来闹去折腾一年,啥都没有,只能走人。\\

崇祯同志很扫兴,好不容易干了件事,又干成这副熊样,好在没有造成严重后果,反正驿站有没有无所谓,就这么着吧。\\

事实上,如果他知道刘懋改革的另一个后果,估计就不会让他走了,他会把刘懋留下来,然后,砍成两截。\\

因为汇报裁减业绩的人,少报了一件事:之所以减掉了八十余万两白银的经费,是因为裁掉驿站的同时,还裁掉了上万名驿卒。\\

崇祯二年(1629),按照规定,银川驿站被撤销,驿卒们统统走人。\\

一个驿卒无奈地离开了,这里已无容身之所。为了养活自己,他决定,去另找一份工作,一份更有前途的工作。\\

这个驿卒的名字,叫做李自成。\\

换句话说,崇祯上台以后,是很想干事的。但有的事,干了也白干,有的事,干了不如不干,朝廷就是这么个朝廷,大臣就是这帮大臣,没法干。\\

所以他很失落,很伤心,但更伤心的事,还在后头。\\

因为上面这些事,最多是不能干,但下面的事情,是不能不干。\\

崇祯四年(1631),辽东总兵祖大寿急报:被围。\\

他被围的地方,叫做大凌河。\\

一年前,孙承宗接替了袁崇焕的位置,成为蓟辽总督。\\

虽然老头已经七十多了,但实在肯靠谱,上任不久,就再次巡视辽东,转了一圈,回来给崇祯打了个报告。\\

报告的主要内容是,关锦防线非常稳固,但锦州深入敌前,孤城难守,建议在锦州附近的大凌河筑城,扩大地盘,稳固锦州。\\

这个报告体现了孙承宗同志卓越的战略思想。七年前,他稳固山海关,恢复了宁远,稳固宁远,恢复了锦州,现在,他稳固锦州,是打算恢复广宁,照这么个搞法,估计是想稳固沈阳,恢复赫图阿拉,把皇太极赶进河里。\\

想法好,做得也很好,被派去砌城的,是总兵祖大寿、副总兵何可纲。\\

在袁崇焕死前,曾向朝廷举荐过三个人,分别是赵率教、祖大寿、何可纲。\\

他在举荐三人时,曾说过:\\

“臣选此三人,愿与此三人共始终,若到期无果,愿杀此三人,然后自动请死。”\\

袁崇焕的意思是,我选了这几个人,工作任务要是完不成,我就先自相残杀,然后自杀。\\

这句话比较准,却也不太准。\\

因为袁崇焕还没死,赵率教就先死了。袁崇焕死的时候,祖大寿也没死,逃了。\\

现在,只剩下了祖大寿和何可纲,他们不会自杀,却将兑现这个诺言的最后一部分——自相残杀。\\
\ifnum\theparacolNo=2
	\end{multicols}
\fi
\newpage
\section{投降?}
\ifnum\theparacolNo=2
	\begin{multicols}{\theparacolNo}
\fi
带了一万多人,祖大寿跟何可纲去砌砖头了,砌到一半,皇太极来了。\\

皇太极之所以来,也是不能不来,因为当他发现明军在大凌河筑城时,就明白,孙老头又使坏了。\\

如果让明军在大凌河站住脚,锦州稳固,照孙承宗的风格,接下来必定是蚕食,慢慢地磨,今天占你十亩地,站住了,明天再来,还是十亩,玩死你。\\

所以,他亲率大军,前往大凌河,准备拆迁。\\

但祖大寿辛苦半年多,自然不让拆,早早收工,把人都撤了回来,准备当钉子户。\\

然而,当皇太极气喘吁吁地赶到大凌河城下时,却又不动手了。\\

他只是远远地扎营,然后在城下开始挖沟。\\

皇太极很卖力,在城下呆了一个多月,也不开打,只是围城挖沟,挖沟围城,经过不懈努力,竟然沿着大凌河城挖了个圈,此外,他还很有诚意地找来木头,围城修了一圈栅栏。\\

如此用功,只因害怕。\\

鉴于此前他在宁远、锦州吃过大亏,看见城头的大炮就哆嗦,所以决定,不攻城,只围城,等围得差不多了,再攻。\\

对于这一举动,祖大寿嗤之以鼻,并不害怕,事实上,得知围城后,他还派人在城头喊话:\\

“我军粮草充足,足以支撑两年,你奈我何?”\\

皇太极听到了,并不生气,想了个很绝的回答,又派了个人去回话:\\

“那就困你三年!”\\

所谓粮食支撑两年,自然是吹牛的,几天倒还成,而且祖大寿当时手下的部队,有一万多人,虽然皇太极的兵力是两万多,但以他的水平,守半个月没问题。\\

更重要的是,他还有个指望——援军。\\

大凌河被围的消息传来后,孙承宗立刻开始组织援军,先派了几拨小部队,由吴襄带头,往大凌河奔。据说后来的著名人物吴三桂也在部队里。\\

可惜,这支部队刚到松山,就被打回去了。\\

皇太极早有准备,因为他的部队,攻城不在行,打野战没问题,反正这破楼拆定了,来几拨打几拨!\\

孙承宗也很硬,这城楼修定了,就是用人挤,也要挤进去!\\

崇祯四年(1631),最大规模的援军出发了。\\

这支援军由大将张春率领,共四万余人,奔袭大凌河,列阵迎敌。\\

大客户上门,皇太极自然亲自迎接,到阵前一看,傻眼了。\\

统帅张春是个不怎么出名,却有点水平的人,他千里迢迢赶到大凌河,却摆出防守的阵势,收缩兵力,广建营寨,然后架起大炮,等皇太极来打。\\

因为就双方军事实力而言,跟皇太极玩骑兵对砍,基本等于自杀。摆好阵势,准备大炮,还能打几天。\\

这是个极为英明的抉择,可惜,还不够。\\

战斗开始,皇太极派出精锐骑兵,以左右对进战术,攻击张春军两翼。\\

但张春同志很有水平,阵势摆的很好,大炮打得很准,几轮下来,后金军队损失惨重。\\

在战场上,英明是不够的,决定战争胜负的,是实力。\\

进攻失败后,皇太极拿出了他的实力——大炮。\\

由于之前被大炮打得太惨,皇太极决定,开发新技术,造大炮。\\

经过刻苦偷学,后金军造出了自己的大炮,共三十门,虽说质量如何不能保证,至少能响。\\

所以当巨大的轰鸣声从后金军队中传出时,张春竟然产生错觉,认为是自己的大炮炸膛,还派人去查,但残酷的事实告诉他,敌人已经马刀换炮了。\\

但张春认定,无论如何,都要顶住,他亲自上阵督战,希望稳住阵脚。\\

这个愿望落空了。\\

为保证此战必胜,张春来的时候,还带上了一员猛将——吴襄。按原先的想法,吴将军是本地人,跟皇太极也打了不少仗,熟悉情况。\\

应该说,这个说法是很对的,吴襄到底了解情况,一看仗打成这样,立马就跑了。\\

这种搞法极其恶心,并直接导致了张春的溃败。\\

明朝四万援军就此覆灭,而城内的祖大寿,基本可以绝望了。\\

但绝望的祖大寿不打算放弃,他决定突围。\\

突围的地点,选在南城,据他观察,南城敌人最为薄弱。\\

按祖大寿的想法,能突出去最好,突不出去就回来,也就是试试。但他万没想到,这一试,竟然解决了一个贝勒。\\

几天后,祖大寿发动突围,与后金军发生激战。\\

围困南城的,是皇太极的哥哥莽古尔泰,此人属于大脑很稀缺,四肢很发达类型,故被称为后金第一猛将(粗人代名词),但这次,他遇上了更猛的祖大寿。\\

战斗非常激烈,祖大寿不愧为名将,带着城里的兵(并非关宁军)往死里冲,重创城南军队。\\

莽古尔泰感觉不对,便向皇太极请求援兵,但出乎意料的是,援兵竟然迟迟不到,莽古尔泰只能亲自督阵,用上所部全部兵力,才挡住了祖大寿的突围,损失极为惨重。\\

莽古尔泰在四大贝勒里,排行第三(皇太极第四),被弟弟忽悠了,实在是气不过,所以他立即找到皇太极,说自己损失过重,要求换防。\\

但皇太极压根不搭理他,莽古尔泰气不过,就把刀抽了出来,要砍皇太极,幸好被人拦住,才没出事。\\

搞笑的是,莽古尔泰同志回去后,居然怂了,且越想越怕,连夜就跑到皇太极那里承认错误。\\

皇太极倒也干脆,直接绑了关进牢房,不久后莽古尔泰就死了,死因不明。\\

这已经不是皇太极第一次耍诈了,他老人家虽然靠兄弟上台,却很信不过兄弟,按照他的想法,四大贝勒是没有必要的,只要一个就够了。\\

为达到这一目的,每到打硬仗时,他都故意安排兄弟上阵,所谓“打死敌人除外患,打死自己除内乱”。\\

比如崇祯三年,他听说孙承宗出兵关内四城,明知敌人很猛,就派二贝勒阿敏出征,被打了个稀里哗啦回来,趁机撤了兄弟的职。\\

这次也差不多,如此说来,他大概还差祖大寿个人情。\\

但祖大寿的情况并未改变,他依然出不去,援军依然没法来,他依然不投降。\\

皇太极想招降祖大寿,很想,所以他费尽心机,先是往城里射箭,夹带信件,可是祖大寿的习惯很不好,总不回。\\

打了个把月,回信了。\\

这也是迫不得已,当初被围的时候,实在太过突然,按照明朝规定,军事部队执行任务时,身边只带三天干粮,现在都三十天了,吃什么?\\

吃人。\\

大凌河城里,除了一万多军队外,还有两万多民工,几千匹马。\\

还好,没有粮食,吃马也能活,过了几十天,马吃完了。\\

没办法,只能吃人了。\\

当兵的开始吃民工,而且很有组织性,今天吃几个,就杀几个,挑好人,组织起来杀掉,分吃。\\

杀掉的人除了肉吃完外,连骨头都没剩,收起来当柴禾烧,用人骨烤人肉,真正是物尽其用。\\

就是这样,也没有投降。\\

但祖大寿已经到极限了,这样下去,没被后金军打死,也被城里的兵给吃了。所以他开始跟皇太极联系。\\

联系的话题很简单,两个字——投降。\\

皇太极知道城里很困难,很缺粮食,但他并不知道,祖大寿很坚韧。\\

祖大寿根本不想投降,他只是拖延时间,等待援军,但时间越来越长,援军却越来越少,于是,经过审慎地思考,祖大寿做出了一个抉择,脱离苦海的抉择。\\

他与皇太极的使者进行了会谈,表示愿意投降。\\

崇祯四年(1631),祖大寿召集众将,宣布决定,投降。\\

所有的人都赞成,只有一个人反对——何可纲。\\

袁崇焕没有看错人,何可纲是一个靠得住的人,他严辞拒绝了祖大寿的提议,即使饿死,绝不投降!\\

袁崇焕也没有说错,他的魔咒最终应验了。\\

大家都投降,你不投降,就只有杀了你了。\\

祖大寿用行动,完成了袁崇焕诺言的最后部分:自相残杀。\\

他命令将拒不投降的何可纲推出城外,斩首示众。\\

何可纲死前,并不惊慌,也不愤怒,只有鄙视,对叛徒祖大寿的鄙视。或许在他看来,这是最后的解脱,他终究没有辜负袁崇焕的期望。\\

但他并不知道,坚持到底的人,并不只他一个,坚持的方式,除死外,还有其它方式,比死更痛苦的方式。\\

杀死何可纲后,祖大寿出城投降。\\

对于祖大寿同志,皇太极显示了最高程度的敬意,比对兄弟还客气,带着所有高级官员出营迎接,连跪拜礼都免了,拉进大营后,管吃管喝,吃完喝完又送土特产,安排休息。\\

祖大寿很感动,随即提出,希望为后金立功,并拟出了一个方案:\\

锦州的守将,都是自己的手下,虽然现在有巡抚丘禾嘉坐镇,但只要能潜入城内,召集部下,就能杀掉丘禾嘉,攻陷锦州。\\

皇太极同意了他的方案,给祖大寿凑了几百人,假装大凌河逃兵,护送他进入锦州,并派出多尔衮率领军队,隐藏在锦州附近,等待祖大寿的信号。\\

信号是炮声,按照约定,祖大寿如顺利入城,应于十一月二日放炮,第二天动手,杀掉丘禾嘉,如一切顺利,就鸣炮通知城外后金军,里应外合,攻克锦州。\\

两天后,在皇太极的注视下,祖大寿率领随从,出发前往锦州。\\

事情非常顺利,十一月一日,在后金军的暗中护送下,祖大寿顺利入城。\\

从某个角度看,皇太极是个生意人。\\

其实他并不相信祖大寿,所以劝降又放走,还客客气气地请客送礼,只是希望得到更大的回报。\\

十一月二日,当他听到锦州城内传来炮声时,他终于放心了,祖大寿传出入城信号,这次生意不会亏本了。\\

但是第二天,他没有听到炮声,很明显,祖大寿还没有动手。\\

第三天,也没有炮声。\\

就在他极度怀疑之刻,却收到了祖大寿的密信。\\

这封信是祖大寿从城中送出的,大致内容是说,由于出发仓促,且锦州军队很多,身边的人又少,暂时无法动手,过两天再说。\\

既然如此,就多等两天。\\

两天,没信。\\

又两天,还没信。\\

到第三个两天,终于有信了。\\

皇太极又收到了祖大寿的信,写得相当客气,首先感谢皇太极同志的耐心等待,然后诉苦,说锦州城内防布森严,难以动手,希望皇太极继续等着,估计到来年,就能办这事了。\\

被人涮了。\\

其实从开始,祖大寿就没打算投降,堂堂大明总兵,怎么能投降呢?\\

但不投降就出不去,所以他决定,投个降,先出去。\\

但是何可纲反对。\\

此时,祖大寿有两种选择,第一,当着大家告诉何可纲,我们不是投降,是忽悠皇太极的,等出去后,我们就找个机会跑路,回家洗了睡。\\

但这么干,难保不被人举报,保密起见最好别讲。且何可纲本是个二杆子,要死就死,投降就投降,投什么假降?\\

第二;杀了他。\\

只能这样。\\

于是何可纲死去了,祖大寿活下来,为了同一个目标。\\

事实上,祖大寿回到锦州后,啥都没干,就说自己跑回来了,继续一心一意地镇守锦州,坚决打击皇太极。\\

但刚涮完人家,就不认账,实在太过缺德,所以他在十一月二日的时候,还是按约定放了几炮,就当是给皇太极同志留个纪念,说声拜拜。\\

至于送信解释情况,说自己暂时无法下手,倒也并非客气,实在是没办法,因为他的许多部下和亲属,还在皇太极那边,自己跑了,还不客气客气,就扯淡了。所以这几封信的意思也很明确,就是说我虽然骗了你,但你也消消气,别把事情做绝,将来没准还能合作。\\

当然,关于这件事,也有争议说祖大寿同志不是诈降,是真降,只不过回锦州后人手不足没法下手,所以才没干。\\

这种说法是不太靠谱的,因为很快,他就接受了锦州防务,镇守锦州,要多少人手有多少人手,也没干。\\

袁崇焕终究没有看错人。\\

但这件事情最奇特的地方,既不是祖大寿忽悠,也不是皇太极被忽悠,而是崇祯。\\

锦州守将,巡抚丘禾嘉是一个极其谨慎的人,虽然祖大寿没说实话,但他已多方查证,确认了祖大寿的投降,并且写成了报告,上报崇祯。\\

奇怪的是,报告送上去了,崇祯也看了,却没有任何反应,压根就没理这事,依然委任祖大寿镇守锦州。\\

在这世上混,大家都不容易,睁只眼闭只眼算了吧。\\

最倒霉的反倒是孙承宗。他开始砌墙的时候,很多人就不服气,现在墙没砌好,就给人拆了,还收拾了施工队,于是又是一片口水铺天盖地而来,孙承宗比较识趣,一个月后就辞职走人了。\\

历经三朝风云,关宁防线的构架者,袁崇焕、祖大寿的提拔者,忠诚的爱国者,力挽狂澜的伟大战略家孙承宗,结束了。\\

但这并不是他的终点,七年之后,他将在另一个舞台上,演出他人生最辉煌的一刻,以最壮烈的方式。\\

\subsection{意外的意外}
大凌河失陷了,皇太极走了,孙承宗也走了,这就是崇祯四年大凌河之战的结果。\\

但还有一个结果,是很多人并不知道,也没有料到的。\\

而这个结果的出现,和袁崇焕同志有莫大的关系。\\

袁崇焕杀掉毛文龙后,皮岛的局势很稳定,过了一年,就开始闹事。\\

闹事的根本原因,还是毛文龙,因为这位兄弟太有才能,以致于他在岛上的时候,大肆招兵,不但招汉人,还招满人。\\

毕竟不管汉人满人,都认钱,而且满人作战勇猛,更好用,加上毛文龙会忽悠,越招越多,许多关外的人还专程坐船来参军,到最后竟然有上千人。\\

但毛文龙死后,继任的人能力差点,没法控制局面,就兵变了,先是士兵互砍,然后是将领互砍,最后总兵黄龙专程带兵上岛,才算把事镇住。\\

但这件事一闹,许多人都不想在岛上呆了。其中有两个人,这两个人是孔有德和耿仲明。\\

但到底去哪里,还是个问题,这二位仁兄都是山东人,原先还是矿工,出来闯关东,现在闯不下去,一合计,还是回老家。\\

当然,回去挖矿是不能的,既然是兵油子,还是当兵合算,找来找去,听说登莱巡抚孙元化那里缺人,就去了。\\

孙元化,明代伟大的科学家,徐光启的学友,特长是炸药学、弹道学,简而言之,是搞大炮的。\\

据说这人不但精通物理、化学,还懂葡萄牙语,当年还上过葡萄牙火炮培训班,属于放炮专家。\\

当时他正跟葡萄牙人搞科学试验(造大炮),手下缺人,孔有德带人跑过来,十分之高兴,当即就把人给收编了。\\

其实孙先生虽说致力于科学研究,也曾打过仗,之前还曾当过宁远副使,给袁崇焕打过工,也见过世面。\\

可惜,知识分子就是知识分子。\\

他并不知道,所谓孔有德、耿仲明,属于有奶便是娘型,是典型的兵油子,给钱就开工,不给钱就打老板,招这么俩员工,只好认倒霉。\\

其实刚开始的时候,这两位矿工兄弟还是很听话的,也服管,估计换了老板,也想好好干两天。\\

然而意外发生了。\\

祖大寿在大凌河筑城,被人围攻,朝廷四处调援兵,孙元化归孙承宗管,孙承宗找他要兵,他就把孔有德派去了。\\

孔有德很听命,立马就出发,前去拯救祖大寿。\\

走到半路,意外的意外发生了。\\

因为此时已经是十月份(阴历),天开始下雪,孔有德估计是走得急了点,不知是粮食没带够,还是当兵的想开小灶,反正是几个人私自到老百姓家打猎,把人家里的鸡给吃了。\\

吃完了,被人发现了。\\

吃了就吃了吧,并非什么大事,大不了赔几只。\\

可问题是,当地的老百姓比较彪悍,且没说赔鸡,把人抓住以后,先修理了一顿,打得很惨。\\

消息传上去,当即炸锅,孔有德怒了,这还了得,后金军老子都没怕过,怕老百姓?二话不说,索性抢你娘的。\\

问题是,抢完了怎么办,毕竟大明是法制社会,犯了法,是要杀头的,所以孔有德破罐子破摔,反了。\\

孔有德同志原本是挖矿的,也没什么政治目标,更不打算替天行道,但既然反了,替天抢一把还是要的。\\

他带领部队,开始沿路抢劫。\\

此时,得到消息的孙元化急得不行,连忙找来山东巡抚余大成商量对策,谈来谈去,谈出一个结果——招安。\\

想出这么个招,原因在于他们认定,孔有德的反叛是出于误会,只要把他拉回来,安慰安慰,没准再给几只鸡,就能解决问题。\\

更重要的是,这件事如果追究起来,黑锅就背定了,趁着现在事情还不大,瞒报情况拉人回来,还能保住官位,所以不能动武,只能招安。\\

事实证明,瞒报注定是要穿帮的。\\

孙元化派出使者,找到孔有德,告诉他,赶紧归队投降,否则就什么什么。\\

孔有德很害怕,当即表示愿意投降,前往登州接受整编。\\

孙元化很满意,坐在城里等着孔有德,几天后,孔有德顺利到达登州,干的第一件事,就是攻城。\\

孙元化同志毕竟是知识分子,他并不知道,像孔有德这种兵油子,本没有道德观念,算是无赖,而能镇得住他的,也只有更无赖的无赖,比如毛文龙。\\

而孙专家最多也就是个技术员,对孔有德而言,不欺负是白不欺负。\\

还好守军反应快,立即出城迎敌。\\

但就战斗力而言,双方差距实在太大,登州城里的部队,平时最多也就打打土匪,跟从皮岛来的孔有德相比,只能算仪仗。\\

所以没过多久,部队就被孔有德军击溃,退回城内。\\

虽然失利,但大体还算不错,因为登州城有大炮,据城坚守,应该没有问题。\\

可惜孙元化同志疏忽了极为重要的一点——他忘记了一个人:耿仲明。\\

耿仲明还在城内,作为孔有德的铁杆、老乡、战友兼同事,如果不拉兄弟一把,是不地道的。\\

耿仲明很地道,所以他连夜打开了城门,放孔有德进城,登州沦陷了。\\

孙元化很有点骨气,听说叛军入城,就准备自杀,但手慢了点,导致自杀未遂,被俘。\\

孔有德到底是混社会的,讲点江湖道义,没有杀孙元化,只是把他扣作人质,同时,他又致信山东巡抚余大成,要求和谈。\\

好在余大成还比较清醒,知道事情闹大了,当即上报朝廷,登州失陷。\\

崇祯大怒,搞这么大的事,现在才来汇报,干什么吃的!\\

他马上下令,免去孙元化、余大成的职务,委派谢涟为新任登莱巡抚,接替孙元化,平定叛乱。\\

很快,孔有德也得知了这个消息,他明白,只能一条路走到黑了。\\

但他对孙元化似乎很有感情,到这份上,都没动他一根指头,竟然给放了。\\

但他做梦都没想到,自己难得干了件好事,也能把孙专家害死。\\

因为这事从头到尾,孙专家的责任太大,所以孙元化千里迢迢投奔朝廷后,就被朝廷逮了,送到京城,审讯完毕,竟然判了死刑,拉出去砍了。\\

现在的孔有德很麻烦,他虽然占据了登州,但也就是个县城,且还在明朝腹地,上天没路,下地没门,渡海没船,基本是歇菜了。\\

但非常难得,孔有德同志很乐观,他非但没有走,还干起了大买卖,找来了当年的同事李九成、耿仲明、陈友时,还拉上毛文龙的儿子毛承禄,并广泛招募各地犯罪分子,扩编军队。\\

更搞笑的是,他们还组织政府,开始封官,封到一半,发现没有官印,还专门抓了几个刻印章的,帮他们刻印,很有点过日子的意思。\\

当然,他们在百忙之中,没有忘记自己的主业——抢劫,原先只抢个把县,现在牛了,统筹抢劫,分兵几路,从登州开始,沿着山东半岛去抢,搞得民不聊生。\\

崇祯决定,解决这个问题。\\

但新任巡抚谢涟刚到任,就发现,在围剿孔有德之前,他必须先突围。\\

孔有德同志手下这帮兵,打后金军,只能算是凑合,但打关内这帮人,实在是绰绰有余,谢涟到达莱州之后,就被围了。\\

但孔有德攻城的水平明显是差点,双方陷入僵持,你进不来,我出不去。\\

朝廷倒真急眼了,听说新到的巡抚又被围住,立即增兵,两万多人,直奔莱州。\\

孔有德听说朝廷援兵到了,也不含糊,加班加点地攻城,现炒现卖,拉出了登州城里的大炮,猛轰城头,竟然轰死了新到任的山东巡抚(谢涟是登莱巡抚)。\\

谢涟虽说打仗没谱,还是比较硬的,死撑,等援兵来。\\

他等来的不是援兵,而是一个做梦也想不到的消息。\\

围城的孔有德派出了使者,交给他一封信,信中表示,希望谢大人开恩,愿意投降。\\

听明白了,不是要谢大人投降,而是要谢大人接受投降。\\

这是个比较搞笑的事,深陷重围还没投降,包围的人倒要投降了,鬼才信。\\

谢涟信了,因为形势摆在眼前,朝廷援兵即刻就到,孔有德是聪明人,投降是他仅存选择。\\

他决定亲自出城,接受投降。\\

谢大人到底还是知识分子,他不知道,孔有德同志虽然是个聪明人,却是个聪明的坏人,从他反叛那天起,就没打算回头。\\

时候到了,孔有德张灯结彩,锣鼓喧天,亲自在城门迎接。谢巡抚很受感动,带着几个随从出城受降。\\

为示庄重,他还去找莱州总兵,让他一起出城。\\

总兵不去。\\

不但不去,还劝谢巡抚,最好别去。\\

跟谢涟不同,这位总兵,是从基层干起来的,比较了解兵油子的特点,认定有诈,坚持不去。\\

保住莱州,就此一举。\\

接下来的过程很有戏剧性,谢涟出城后,受到了孔有德的热情接待,手下纷纷上前,亲密地围住了谢巡抚,把他直接拉到了大营。\\

一进去,就变脸了。\\

孔有德的打算是,先把谢巡抚绑起来,当作人质,然后又把随同的一个知府拉到城下,逼他传话,让里面的人投降。\\

这位知府表示配合,到城下,让喊话,就真喊了:\\

“我死后,你们要好好守城(汝等固守)!”\\

按常规,此时发生的事情,应该是贼兵极其愤怒,残忍地杀害了知府大人。\\

但事情并非如此,因为知府大人固然有种,但更有种的,是那位不肯出城的总兵。\\

他听说巡抚被人劫了,知府在下面喊话,二话不说,就让人装炮弹,看准敌人密集地区,开炮。\\

敌人的密集地,也就是知府大人所在地,几炮打下去,叛军死伤惨重,知府大人也在其中,壮烈捐躯。\\

虽然巡抚够傻,好在知府够硬,总兵够狠,莱州终究守住。\\

但孔有德还是溜了,赶在援军到来之前。\\

这么闹下去,就没完了,崇祯随即下令,出狠招,调兵。\\

照目前情况看,要收拾这帮人,随便找人没有效果,要整,就必须恶整。\\

所以,他调来了两个猛人。\\

第一个,新任山东巡抚朱大典,浙江金华人,文官出身,但此人性格坚毅,饱读兵书,很有军事才能。\\

但更猛的,是第二个。\\

此时的山东半岛,基本算孔有德主管,巡抚的工作,他基本都干,想怎么来怎么来,看样子是打算定居了。\\

而且此时他的手下,已经有四五万人,且很有战斗经验,对付一般部队,绰绰有余。\\

所以派来打他的,是特种部队。\\

崇祯五年(1632)七月,明军先锋抵达莱州近郊,与孔有德军相遇,大败之。\\

孔有德很不服气,决定亲自出马,在沙河附近布下阵势,迎战明军。\\

他迎战的,是明军先锋。明军先锋,是关宁铁骑,统领关宁铁骑的,是吴三桂。\\

猛胜朱大典者,吴三桂也。\\

虽然按年龄推算,此时的吴三桂,还不到二十,但已经很猛,只要开战就往前冲,连他爹都没法管,对付孔有德之流,是比较合适的。\\

战斗的进程可以用一个词形容——杀鸡焉用牛刀。\\

关宁铁骑的战斗力,已经讲过了,这么多年来,能跟皇太极打几场的,也就这支部队。\\

而孔有德的军队,虽然也在辽东转悠,但基本算是游击队,逢年过节跟毛文龙出来打黑枪,实在没法比。\\

反映在战斗力上,效果非常明显。\\

孔有德的军队一触即溃,被吴三桂赶着跑了几十里,死了近万人,才算成功逃走。\\

原本孔有德的战术,是围城打援,围着莱州,援军来一个打一个。\\

但这批援军实在太狠,别说打援,城都别围了,立马就撤。\\

莱州成功解围,但吴三桂的使命并未结束,他接下来的目标,是登州。\\

被彻底打怕的孔有德退回登州,在那里,他纠集了耿仲明、李九成、毛承禄的所有军力,共计三万余人固守城池,他坚信,必定能够守住。\\

其实朱大典也这么想,倒不是孔有德那三万人太多,而是因为登州城太厚。\\

登州,是明代重要的军事基地,往宁远、锦州送粮食,大都由此地起航,所以防御极其坚固。\\

更要命的是,后来孙元化来了,这位兄弟是搞大炮的,所以他修城墙的时候,是按炮弹破坏力来算。\\

换句话说,平常的城墙,也就能抗凿子凿,而登州的城墙,是能扛大炮的,抗击打能力很强。\\

更麻烦的是,孙巡抚是搞理科的,比较较真,把城墙修得贼厚且不说,还充分利用了地形,把登州城扩建到海边,还专门开了个门,即使在城内支持不住,只要打开此门,就能立刻乘船溜号,万无一失。\\

所以朱大典很担心,凭借目前手中的兵力,如果要硬攻,没准一年半载还打不下来。\\

按朱大典的想法,这是一场持久战,所以他筹集了三个月的粮食,准备在登州城过年。\\

到了登州,就后悔了,不用三个月,三天就行。\\

孔有德到底还是文化低,对于登州城的技术含量,完全无知。听说明军到来,跟耿仲明一商量,认为如果龟缩城内,太过认怂,索性出城迎战,以示顽抗到底之决心。\\

这个决心,只维持了一天。\\

率军出城作战的,是跟孔有德共同叛乱的李九成,他威风凛凛地列队出城,摆好阵势,随即,就被干掉了。\\

明军出战的,依然是关宁铁骑,来去如风,管你什么阵势不阵势,就怕你没出来,出来就好办,骑兵反复冲锋,见人就打,叛军四散奔逃,鉴于李九成站在队伍最前面(最威风),所以最快被干掉,没跑掉的全数被歼。\\

此时城里的叛军,还有上万人,但孔有德明显对手下缺乏信心,晚上找耿仲明、毛承禄谈话,经过短时间磋商,决定跑路。\\

说跑就跑,三个人带着部分手下、家属,沿路抢劫成果,连夜坐船,从海边跑了。\\

按孔有德的想法,跑他个冷不防,这里这帮傻人不知道,还能顶会,为自己争取跑路时间。\\

然而意外发生了,他过高估计了自己手下的道德水准,毕竟谁都不傻,孔有德刚跑,消息就传了出去,而类似孔有德这类黑社会团伙,只要打掉领头的,剩下的人用扫把都能干掉。\\

于是还没等城外明军动手,城里就先乱了,登州城门洞开,逃跑的逃跑,投降的投降,跳海的跳海,朱大典随即率军进城,收复登州。\\

事情算是结了,但孔有德这帮人在山东乱搞了半年,不抓回来修理修理太不像话,所以将领们纷纷提议,要率军追击孔有德。\\

但朱大典没有同意。\\

不同意出兵,是因为不需要出兵。\\

逃到海上的孔有德很得意,虽说登州丢了,但半年来东西也没少抢,地主当不成,还能当财主。\\

得意到半路,遇上个人,消停了。\\

他遇上的这个人,名叫黄龙。\\

孔有德跟黄龙算是老熟人,因为黄龙曾经当过皮岛总兵,还管过孔有德。\\

孔有德怕的人比较少,而黄龙就属于少数派之一,孔有德之所以投孙元化,就是因为黄龙太厉害,在他手下太难混。\\

在最不想见人的地方,最不想见人的时候,遇上了最不想见的人,孔有德很伤心。\\

老领导黄龙见到了老部下孔有德,倒也没客气,上去就打,孔先生当即被打懵,部下伤亡过半,连他的亲人都没幸免(他抢劫是带家属的),纷纷堕海而亡。\\

但最不幸的还不是他,而是毛承禄。\\

这位仁兄先是老爹(毛文龙)被杀,朝廷给了个官,也不好好干,被孔有德拉下水搞叛乱,落到这般地步,而关键时刻,孔有德不负众望,毅然抛弃了这位老上级的公子,把他丢给了黄龙。\\

而孔有德和耿仲明不愧干过海盗,虽说打海战差点,但逃命还凑合,拼死杀出血路,保住了性命。\\

毛承禄就不行了,被抓住后送到了京城,被人千刀万剐。\\

黄龙的战役基本上彻底摧毁了叛军,孔有德和耿仲明逃上岸的时候,已经是光杆司令了。山东叛乱就此结束。\\

这次叛乱历时半年,破坏很大,而最关键的是,叛乱造成了两个极为重要的结果——足以影响历史的结果。\\

第一个是坏结果:鉴于生意赔得太大,既没钱,也没人了,回本都回不了。孔有德、耿仲明经过短时间思想斗争,决定去当汉奸,投靠皇太极。\\

其实这两个人投降,倒也没什么,关键在于他们曾在孙元化手下混过,对火炮技术比较了解,且由于一贯打劫,却在海上被人给劫了,很是气愤,不顾知识产权,无私地把技术转让给了皇太极。从此火炮部队成为了后金的固定组成部分,虽说孔有德、耿仲明文化不高,学得不地道,造出来的大炮准头也差点,但好歹是弄出来了。\\

更重要的是,由于他们辛苦折腾半年,弄回来的本钱,连同家属,都被明军赶进海里喂鱼,亏了老本,所以全心全意给后金打工,向明朝复仇。\\

一年后,他们找到了复仇的机会。\\

除锦州、宁远外,明朝在关外的重要据点,大都是海岛,这些海岛有重兵驻守,时不时出来打个游击,是后金的心腹大患,其中实力最强的守岛人,叫做尚可喜。\\

之前我说过,孔有德、耿仲明、尚可喜是山东老乡,且全都是挖矿的,现在孔有德决定改行挖人,劝降尚可喜。\\

一边是国家利益,民族大义,一边是老乡、老同事,尚可喜毫不为难地做出了抉择——当汉奸。\\

当英雄很累,当汉奸很轻松。\\

第二个是好结果,经过这件事,崇祯清楚地认识到,关内的军队,是很废的,关外的军队,是很强的,所以有什么麻烦事,可以找关外军队解决(比如打农民军)。\\
\ifnum\theparacolNo=2
	\end{multicols}
\fi
\newpage
\section{偶然的偶然}
\ifnum\theparacolNo=2
	\begin{multicols}{\theparacolNo}
\fi
山东的叛乱是个麻烦事,但要看跟谁比,要跟西北比,就不算个事。\\

据说朱元璋当年建都的时候,曾经找人算过一卦,大致内容跟现在做生意的差不多,比如这笔生意能做多少年,有什么忌讳等等。\\

据说那位算卦的半仙想了很久,说了八个字:\\

始于东南,终于西北。\\

朱元璋建都南京,就是东南,按照这句话的指示,最后收拾他的人,是从西北过来的。\\

这句话看起来很玄,实际上倒未必。这位半仙懂不懂算卦我不知道,但他肯定是懂历史的,自古以来,中原政权完蛋,自己把自己折腾死的除外,大多数外来的什么匈奴、蒙古,都在西北一带。\\

但就崇祯而言,肯定是不信的。因为对明朝威胁最大的,是后金。而后金的位置是东北,就算是被灭了,也是始于东南,终于东北。\\

但事实告诉我们,算卦这种事,有时是很准的。\\

西北很早就有人闹事了,但原先并不大,最多就是几十个人,抢个商铺,拿几把菜刀,闹完后上山当匪,杀掉的最高官员,也就是个知县,如果混得好,没准将来还能招安,当正规军。\\

到崇祯元年,事情闹大了。\\

整个陕西、甘肃一带,民变四起,杀掉知县,只能算起步了。个别地方还干掉了巡抚,而且杀完抢完且不散伙,经常到处流窜,到哪抢哪。\\

这种团伙,史书上称之为流贼。\\

流贼的特点是,四处跑,抢完就走,打一枪换个地方。组织性不强。昨天抢完,今天就走,可以,昨天被抢,今天加入抢别人,也可以。成员流动性很大,但都有固定领导团队。\\

当时的西北,类似这种团队有很多,优秀的团队管理者也很多。但久而久之,问题出现了,由于成员流动性太大,且没有固定办公场所,团伙成员文化又低,天天跟着混,时间长了,很难分清谁是谁。\\

为妥善解决这个问题,团队首领们想出了一个绝招——取外号。\\

所以在崇祯元年,陕西巡抚呈交皇帝的报告上,有如下称呼:\\

飞天虎、飞山虎、混天王、王和尚、黑杀神、大红狼、小红狼、一丈青、上天龙、过天星。\\

全是外号。\\

取这样的外号,是很符合实际需要的。毕竟团队成员文化比较低,你要取个左将军、右都督之类的称号,他也不知道是啥意思,而且这种外号,大都是神魔鬼怪,叫起来相当威风。\\

至于这上面提到的诸位神魔到底是谁,别问我,我也不知道。\\

鉴于该行当风险很大,且从业者很多,要是运气不好,刚入行,把外号取好就被干掉,也很正常。而且许多外号由于过于响亮,使用率很高,经常是几个人共用一个外号,要搞清楚谁是谁,实在很难。\\

无论叫什么,姓甚名谁,其实都无所谓,你只需要知道,当时的西北,已经不可收拾。\\

按一般史书的说法,这种情况之所以出现,是因为明朝末年,朝廷腐败,经济萧条,贪官污吏,苛捐杂税数不胜数,民不聊生,于是铤而走险。\\

这种说法,就是传说中的套话,虽说不是废话,也差不多。\\

因为事实并非如此。\\

很多人并不知道,明朝末年的民间经济并没有萧条,比如东南沿海,经济实在太好,开生意做买卖,相当红火,大家齐心协力,正在搞资本主义萌芽,萧什么条?\\

赋税也没多少,以往两百多年,官田的赋税,只有百分之十,民间地主的赋税,最多也就收百分之二十。后来开征三饷也才到百分之四十。当然,个把地主恶霸除外。\\

西北之所以涌出这么多英雄好汉,只是因为崇祯运气不好,遇到了一件东西。\\

中庸有云:国之将兴,必有祯祥,国之将亡,必有妖孽。\\

其实遇到妖孽,倒也没什么,毕竟还有实体,实在不行,找人灭了它。\\

崇祯遇上的,叫做灾荒。\\

翻开史书,你会不禁感叹,崇祯同志的运气实在太差:\\

崇祯元年,陕西旱灾。崇祯二年,陕西旱灾,崇祯三年,陕西旱灾,崇祯四年,陕西旱灾……\\

灾荒之后,没有粮食吃,就是饥荒。\\

没有粮食吃,就吃人。\\

对受灾的人而言,吃人,并非童话。\\

据说当时西北各地的小孩,是不能四处乱跑的,如果没看住,跑了出去,基本就算没了。\\

注意,不是失踪,是没了。\\

失踪的意思,是被拐卖了,没了的意思,是被吃了。\\

据说,当时还有人肉市场,具体干什么买卖,看名字就知道。\\

说这么多,只是想说,这并不是童话,也不是神话,而是真话。\\

既然有灾荒,朝廷为什么不赈灾呢?\\

答案很简单,没钱。\\

此前有个经济学家对我说,明朝灭亡的真正原因,是没钱。\\

我表示同意,财政赤字太多,挣得没有花的多,最后垮台。\\

但他看了看我,说:我说的没钱,不是没有收入,是没钱。\\

有什么区别吗?\\

然后,他讲了一个小时,再然后,我翻了一个月的经济学,明白了区别。\\

我很想从头到尾,把我明白的事情告诉你们。但如果这样做,我会很累,你们也会很累,所以我决定,用几句话,把这个问题说清楚。\\

明朝灭亡,并非是简单的政治问题,事实上,这是世界经济史上的一个重要案例。\\

所谓没钱,是没有白银。\\

明朝,是当时世界上最先进的国家之一,到崇祯接班的时候,商品经济已经十分发达,而商品经济十分发达的标志,就是货币。\\

明朝的货币,是白银。\\

简单地说,没钱的意思,就是没有白银,没有白银,无论你有多少经济计划,有多少财政报表,都是胡扯淡。\\

举个例子,陕西受灾,朝廷估算,要赈灾,必须一百万两白银,但是就算你把皇帝的圣旨拿到陕西,也换不来一两银子,因为没有白银,所以无法赈灾。\\

好了,下一个问题,为什么没有白银。\\

先纠正一下,不是没有白银,而是白银不够。\\

为什么白银不够?\\

这是个很复杂的经济学问题,我不太想讲,估计人也不太想听。但不讲似乎也不行,简单说两句。\\

用大家都能明白的话说,就是白银有限,朝廷用掉了一两白银,未必能挣回来一两,加上我国人民,素来以勤俭节约闻名,许多人拿到真金白银,不喜欢花,要么存在家里,要么溶掉,做几个香炉、人像之类的,还能美化环境,所以市场的白银越来越少。\\

更重要的是,明朝的商品经济实在太过发达,经济越发达,需要的白银就越多,可是白银就那么多,所以到最后,白银就不够用了。这种现象,在经济学上有一个通称——通货紧缩。\\

我知道,有人会提出这样的问题——为什么不用纸币?\\

很好,如果你提出这个问题,说明你很聪明。\\

但我要告诉你,在你之前的六百多年,有人问过这个问题。这个人的名字,叫朱元璋。\\

六百多年前,他就想到了这个问题,所以开始发行纸币。\\

在经济学中,有这样一句谚语:棍棒打不垮经济理论。\\

这句话的通俗意思是,无论你多牛,都要照规矩来。\\

朱元璋就是牛人,也要按规矩来。虽然他发行了纸币,一千、一万都印过,可惜的是,几百年来,大家还是认白银,就不认纸币,再牛都没用。\\

这个问题到此为止,多余的话就不说了,你只要知道,崇祯同志是想赈灾的,之所以赈灾不成,是因为没有钱,之所以没有钱,是因为没有白银,之所以没有白银……\\

当然,之所以西北先闹起来,除去天灾、银祸外,还有点地方特色。\\

西北一带,向来比较缺水,比较穷困,比较没人理,外加地方官比较扯淡,所以这个地方的人,过得比较苦。\\

生活艰苦,饭都没处吃,自然没条件读书。\\

没条件读书,自然考不上功名,考不上功名,自然没官做。\\

没官做,也得找事做。\\

而西北一带人,最主要的工作,就是当兵。\\

生活艰苦,民风自然彪悍,当兵是最合适的工作。\\

除了当兵之外,还有一份更为合适的工作——驿站。\\

驿站虽说比较小,但好歹是官办的,也算是吃皇粮的,而且各省都有拨款,搞点潜规则,多少能捞点油水,养活自己,是不成问题的。\\

据统计,光是甘肃陕西,就有几万人指着驿站过日子。\\

崇祯二年(1629),驿站没了。\\

之前我说过,被裁掉了,裁掉它的,是一个叫做刘懋的好人。\\

崇祯同志的运气实在太差,灾荒、钱荒、又夺了人家的饭碗,如果不闹,就不正常了。\\

他不是故意的。\\

所有的一切,都是偶然。偶然的灾荒,偶然裁掉驿站,偶然的地点。\\

如果其中任意一个偶然没有发生,也许就不会有最后的灭亡。\\

可惜,全都偶然了。\\

我曾经百思不得其解,因此我认定,在这些偶然的背后,隐藏着一个必然,一个真正的,决定性的原因。\\

就是这个原因,导致了明朝的灭亡。\\

我想了很久,终于想出了这个最终的原因,四个字——气数已尽。\\

这个世界上的一切,大致都是有期限的。一个人能红两年,很可能是偶然的,能红十年,就是有道行的,能红二十年,那是刘德华。\\

公司也一样,能开两年,很正常,能开二十年,不太正常,能开两百年的,自己去数。\\

封建王朝跟公司差不多,只开个几年就卷铺盖的,也不少。最多也不过三百年,明朝开了二百多年,够意思了。\\

\subsection{抚战}
当然,崇祯是不会这样想的,无论如何,他都要撑下去,否则将来到地下,没脸见开铺的朱元璋。\\

所以他派出了杨鹤。\\

杨鹤,湖广武陵人(湖南常德),时任都察院左副都御史。经朝廷一致推荐,杨鹤被任命为兵部侍郎,三边总督,接替之前总督武之望的职务。\\

工作交接十分简单。应该说,基本不用交接,因为杨鹤到任的时候,武之望已经死了。\\

不是他杀,是自杀。\\

武总督是个很有责任感的人,鉴于西北民变太多,估计回去也没什么好果子吃,索性自杀。\\

而杨鹤之所以接替这个职务,是因为一次偶然的谈话。\\

杨鹤是一个进步比较慢的人,在朝廷里混三十多年,才当上佥都御史,混成这样,全靠他那张嘴。\\

皇帝喜欢魏忠贤,他骂魏忠贤;皇帝讨厌熊廷弼,他为熊廷弼辩护。想什么说什么,几起几落,该怎么来还怎么来。\\

崇祯元年,他被重新委任为御史,当时民变四起,大家都在商议对策。\\

有一次,几个人聚到一起,聊天。聊的就是这个,杨鹤就在其中。\\

杨鹤是都察院的,这事跟他本无关系,他之所以掺和进来,还是两个字——嘴欠。\\

反正是吹牛,不用动真格的,就瞎聊。这个说要打,那么说要杀,如此热闹,杨鹤终于忍不住了,他说,不能打,也不能杀。\\

然后他提出了自己的理论——元气说。\\

在他看来,造反的人,说到底,也还是老百姓。如果杀人太多,就是损伤元气,国家现在比较困难,应该培养元气,不能乱杀。\\

几句话,就把大家彻底说懵了,对于他的观点,大家有着相同的评价——胡说八道。\\

不杀人,怎么平乱?\\

这是一个不为绝大多数人接受的理论,不要紧,有一个人接受就行。\\

不久之后,崇祯知道了这个理论,十分高兴,召见了杨鹤。\\

好事一件接着一件。很快吏部主动提出,鉴于杨鹤同志的理论很有实用价值,正好前任三边总督武之望死了,正式提名杨鹤同志升任该职务。\\

杨鹤不想去。原因很简单,本来就是吹吹牛的,压根不会打仗,去了干啥?被人打?\\

但是牛都吹了,外加吏部支持,皇帝支持,如此重任在肩,咬咬牙就去了。\\

可是杨同志不知道,吏部之所以支持他,是因为讨厌。皇帝之所以支持他,是因为省事。\\

和杨鹤不同,吏部的同志们都是见过世面的,知道平乱是要砍人的,砍人是要死人的,死人是要流血的。杨鹤这套把戏,也只能忽悠人,为达到前事不忘、后事之师的效果,让后来的无数白痴书呆子明白,乱讲话要倒霉,才着力推荐他去。\\

死在那边最好,就算不死,也能脱层人皮。\\

相比而言,崇祯的用心是比较善良的。他之所以喜欢杨鹤,是因为杨鹤提出了很好的理论——省钱的理论。\\

不花钱,不杀人,不用军饷,不用调兵,就能平息叛乱,太省了。\\

就算是忽悠人的,最多把杨鹤拉回来砍了,很省成本,如此生意,不做白不做。\\

就这样,一脑袋浆糊的杨鹤去陕西上任,至少在当时,他的自我感觉很好。\\

杨鹤理论之中,最核心的一条,叫做和气。\\

用他自己的话说,杀人是伤和气的。所以能救活一个,就是一个,毕竟参加民变的,原先就是民。\\

这个理论,一年前,应该是对的。\\

杨鹤同志到任后,就发现不对了。\\

有一次,农民军进攻县城,被击退,抓住了几个俘虏,由杨鹤审问。\\

但还没问,杨鹤就发现一件极为诡异的事——他似乎见过这几个人。\\

确实见过,阅兵的时候见过。\\

没错,这几个人曾经站在阅兵的队伍里,曾经是他的部下。\\

\subsection{强,弱,之间}
农民军的战斗力很强吗?\\

对于这个疑问,最好的答案,应该是个反问——农民军的战斗力怎么会强呢?\\

在中国历史上,造反这类活,从来都是被动式。闲着没事干,但凡有口饭吃,是不会有人造反的,成本高,门槛也高。\\

但遗憾的是,造反这份工作,除了成本、门槛高外,技术含量还高。\\

要知道,明朝参加这项活动的,主要是农民。农民的基本工作,是种地,基本工具,是锄头。\\

而阻止他们参与这项活动的,是明军士兵。士兵的基本工作,是杀人,基本工具,是刀剑。\\

所以在明末大多数情况下,几百个农民军跟几百个明军对战,是不太可能发生的。据史料记载,大部分情况,是几万农民军,战胜了几百明军,或是几百农民军,搞定十几个看衙门的捕快。\\

而更大多数情况,是几千明军追着几万、甚至十几万农民军跑。\\

没办法,毕竟打仗是个技术活。圣贤曾经说过,把武器交给没有受过训练的民众,让他们去打仗,就是让他们送死。\\

没有训练,没有武器,没有兵法,没有指挥,就没有胜利。\\

但杨鹤先生惊奇地发现,他面对的情况,是完全不同的。\\

西北的民军里,除了业余造反的以外,还有很多专业造反的人士——明军,而且数量很多。\\

他们精通战术,作战狡猾,懂得明军的弱点,非常难以对付,且数量是越来越多,民变越来越大。\\

出现此类情况,归根结底,原因就两个字——没钱。\\

之前我说过,朝廷没有钱。没有钱的结果,除了没钱赈灾外,还没钱发军饷。\\

据统计,当时全国的部队,大致有上百万人,而能够按时领军饷的,只有辽东军区的十余万人。\\

而且就连辽东军,也不能保证按时发工资,拖几个月,也是经常的事。袁崇焕同志就曾经处理过相关事务。\\

辽东是前线,尚且如此,其他地方就别提了。西北一带,既然不是前线,自然没钱。有的人几年都没拿到工资,穷得叮当响,据说连武器都卖了,只求换顿饭吃。\\

没钱赈灾,老百姓吃苦,也没辙,没钱发饷,当兵的吃苦,就有辙了。\\

兜里没钱,手里有刀,怎么办?\\

凉拌,抢!\\

情况就是如此,官兵越来越少,民军越来越多,局势越来越撑不住。\\

杨鹤面对的形势大致如此,大家都明白,就他不明白,等他明白了,跑也跑不掉了。\\

如果换个会打仗的,能用兵的,多少还能撑几天,但杨鹤同志的主要特长,是招抚理论,这就比较麻烦了。据说当时朝廷里,有些人开玩笑,说杨鹤如果能撑一年,就倒着爬出去。\\

就当时的情况看,这位仁兄爬出去的可能性,大致是零。杨鹤同志的下岗日期,指日可待。\\

一年后,杨鹤向崇祯呈交了名单,在这份名单上,有这样十几个名字:\\

神一魁、王左桂、王嘉胤、红狼、小红狼、点灯子、过天星、独头虎……(以下略去XX字)\\

以上人等,全部归降。\\

这些人是干嘛地,看名字就能猜到,但这些人有什么分量,估计你就不知道了。\\

在当时的起义军中,最能打的,就是神一魁。此人具体情况不详,但应该受过军事训练,作战十分强悍,属于带头大哥级人物。\\

王左桂、王嘉胤,如果你不知道,那不怪你。对这二位兄弟,只提几句话就够了:当时,在王左桂的手下,有个小头目,叫做李自成。王嘉胤营门口站岗的,叫做张献忠。\\

至于后面那几位,就不说了,说了也没人知道,你只要明白,他们都是当时一等一的牛人,随便一个摆出来,都能搅得天翻地覆。\\

都投降了。\\

除这些人之外,当时陕西、甘肃境内几乎所有的农民军,都投降了。\\

他们投降的对象,就是那个一脑袋浆糊,啥也不懂,不会打仗的杨鹤。\\

奇迹就这样发生了,发生在所有人的眼前。\\

杨鹤不懂兵法,不熟军事,但他有一样别人没有的武器——诚意。\\

作为一个不折不扣的好人,杨先生很有诚意地寻找叛军,很有诚意地进行谈判,很有诚意地劝说投降,最后,他的诚意得到了回报。\\

事实证明,农民军之所以造反,并不是吃饱了撑的,只是因为吃不饱。现在既然朝廷肯原谅他们,给他们饭吃,自然愿意投降,毕竟造反这事,要经常出差,东跑西跑风险太大。\\

而对于杨总督,他们也是很客气的,很有点宋江喜迎招安的意思。\\

比如神一魁投降,约好地点,杨鹤打开城门,派出群众代表,热烈欢迎。众多民军头目大部到场,在杨总督的率领下,前往关帝庙,在关老爷面前,宣誓投降(关老爷靠得住)。\\

虽然此前双方素未谋面(可能在往城下射箭时看过几眼),但双方都表现出了相当的热情。特别是杨总督,获得了民军的一致推崇,他们赶走了杨鹤的轿夫,坚持一定要亲自把他抬到总督府,并以此为荣。\\

一时间,西北喜讯接连,朝廷奔走相告,杨鹤跟各民军领袖的关系也相当好,逢年过节,还互相送礼,致以节日的问候。\\

局面大好,大好。有效期,半年。\\

杨鹤同志读过很多书,干过许多工作,明白很多道理,但是他并不知道,从招抚的第一天开始,他就已经失败了。\\

因为有一个问题,他始终没弄明白。\\

正是这个问题,注定了他的悲惨结局。\\

这个问题是,他们为什么要造反?\\

答案是:为了活下去。\\

怎样才能活下去呢?\\

有钱,有粮食。\\

要说明这个问题,可以用一个三段论:\\

造反,是因为没钱、没粮食;投降,是因为有钱,有粮食。\\

杨鹤有钱,有粮食吗?\\

没有。\\

所以停止投降,继续造反。\\

在招降之前,杨鹤曾经认为,只要民军肯投降,事情就结束了,可是投降之后,他才明白,事情才刚开始。\\

光是神一魁的部队,就有三万多人,这么多人,怎么安置?\\

招来当兵,就别扯了,连自己手下那点人的军饷都解决不了,招来这些人,喝西北风?\\

赶回家种地,似乎也是白扯,年年灾荒,要能回家种地,谁还造反?\\

对于这个悖论,崇祯同志是知道的,也想了办法。\\

他先找了几万两银子,安排发放。然后又从自己的私房钱(内库)里,拿出了十万两,交给杨鹤,让他拿去花。\\

应该说,这一招还是很有效果的,民军们拿到钱,确实消停了相当长的时间。\\

具体是多长呢?\\

我前面说过了,半年。\\

半年,把钱都花完了,自然就不投降了,该怎么着还怎么着,继续反!\\

为了活下去。\\

\subsection{猛人出场}
崇祯四年(1631),领了半年工资后,神一魁再次反叛,西北群起响应,而且这次阵势更大,合计有三十多万人。\\

搞到这个地步,朝廷极为不满,许多大臣纷纷上告。\\

杨鹤很委屈,他本来就不是武将。之所以跑来办这事,实在是被人弄来的,原来是吹吹牛而已,你偏认真。来了之后,都没闲着,天天忙活这事,钱花完了,人家又反了,我有什么办法?\\

崇祯更委屈,原本看你吹得挺好,觉得你能办事,才把你派过去。这么信任你,你招降了人,我立马就给你十几万两银子,连老子的私房钱都拿出来了,你把钱花完了,这帮人又反了,十万两都打了水飘,你干什么吃的?\\

杨鹤委屈,就写信给崇祯,说我本不想干,你硬要我干,我要招抚,也是没有办法。\\

崇祯委屈,就写了封命令:锦衣卫,把杨鹤抓起来。\\

崇祯四年(1631)九月,杨鹤被捕,后发配袁州。\\

鉴于杨鹤的黑锅实在太重,由始至终,朝廷没人替他说话。\\

例外总是有的。\\

命令传出后,一个山海关的参政主动上书,要求替杨鹤承担处罚。\\

如此黑锅都敢背,是不正常的,但这个人帮杨鹤背锅,就是再正常不过了。\\

这位参政,是杨鹤的儿子,叫做杨嗣昌。\\

崇祯没有理睬,杨鹤先生的命运未能改变,依然去了袁州。\\

帮父亲背锅,看起来,是一件微不足道的小事,却导致了两个重大后果。\\

从这份奏疏上,崇祯看到了一个忠于父亲的人。按照当时的逻辑,忠臣,必定就是孝子,所以他记住了杨嗣昌的名字。他认定,此人将来必可大用。\\

而杨嗣昌背黑锅不成,父亲被发配了,对他而言,莫过于奇耻。从此,他牢牢记住了那些降而复叛的人,此仇,不共戴天。\\

杨鹤离开了,但这场大戏刚刚开幕,真正的猛人,即将出场。\\

一年前,招抚失败后,民军首领王左桂派出起义军,进攻军事重镇韩城,韩城派人去找杨鹤,告急。\\

杨鹤很急,因为他的政策是招抚,手中实在没有兵,但到这节骨眼上,就是自己拿菜刀,也不能不去了。\\

但他终究没有掌握菜刀技术,无奈,他想起了一个人。\\

这个人的手上也没有兵,但杨鹤相信,这个人是有办法的。\\

第一个猛人登场,他的名字,叫做洪承畴。\\

洪承畴接到了求援的命令,从某种程度上说,这是个相当扯淡的命令,你是总督都没办法,我怎么办?\\

但他并未抱怨,召集了自己的下人和亲兵,并就地招募了一些人,踏上了前往韩城的道路。\\

这是文官、陕西参政洪承畴的第一次出征,这年,他三十七岁。\\

洪承畴,字彦演,号亨九。福建南安人。\\

根据记载,此人的家世,可谓显赫一时:\\

曾祖父洪以诜,字德谦,中宪大夫,太傅兼太子太师、武央殿大学士。\\

曾祖母林氏、一品夫人。\\

祖父洪有秩,资政大夫、兵部尚书兼都察院右副都御史。\\

祖母戴氏,夫人。\\

有这么一份简历,基本就可以吃闲饭了。\\

可惜,洪承畴没能吃闲饭,事实上,他连饭都吃不上。\\

因为所有的这些简历,都是后来封的,换句话说,是他挣回来的。\\

洪承畴出生时,他的父亲因为家境贫寒,外出打工去了,他的母亲虽然穷,却比较有文化,从小就教他读书写字。\\

洪承畴很聪明,据说7岁就能背三字经,这是很了不起的。比如说我,27的时候,还只能背人之初,性本善。\\

万历四十三年(1615年),洪承畴23岁,参加全省统考(乡试),他的成绩很好,全省第19名。\\

第二年,他到北京参加全国统考,成绩更好,全国第17名,二甲。\\

然后分配工作,他被分配到刑部。\\

这个结果对他而言,是比较倒霉的。\\

原因我说过,在明代,要想将来入阁当大学士,必须当庶吉士,进翰林院。以洪承畴的成绩,应该能进,可是偏就没进。\\

此后的十几年,洪承畴混得还可以,当上了刑部郎中,又被外放地方,当了参政。\\

参政这个官,说大不大说小不小,通常是混到最后,光荣地退休。\\

从没考上翰林的进士,混饭吃的小参政,到历史留名,骂声不绝,余音绕柱的大人物,只是因为,他外放的地方,是陕西。\\

刚去陕西的时候,洪承畴带了很多书。\\

所以洪承畴带兵去救韩城的时候,只是一个书生,他没有打过仗,也没有杀过人。\\

据说在世界上,有这样一种人,他们天生就会打仗,天生就会杀人。\\

这是事实,不是据说。\\

洪承畴是一个真正的天才,军事天才,他带着临时拼起来的家丁、仆人、伙夫,就这么上了战场,却没有丝毫的胆怯。\\

面对优势敌军,他凭借卓越的指挥,轻易击败了起义军,斩杀五百余人,解围韩城。\\

在洪承畴的人生中,有过无数次战役,有过无数个强大的对手,最重要的,是这一次。\\

这个微不足道的胜利,让洪承畴明白,他是多么的强大,强大到可以力挽狂澜,可以改变无数人的命运。\\

他要凭借着自己的努力,挽救这个末落的王朝,创造太平的盛世。\\

讽刺的是,他最终做到了,却是以一种他做梦也未曾想到的方式。\\

洪承畴是一个务实的人,具体表现在,他正确地意识到,杨鹤是一个蠢货。\\

招抚是没有用的,钱是不够用的,唯一有用的方式,是镇压。\\

来陕西上任之前,洪承畴带来了很多书。三十年以来,书,是他仅有的寄托。\\

战后,他丢掉了书,做出了一个新的抉择——开战。\\

奇迹就是这样发生的,此后的两个月里,洪承畴率领这支纯粹的杂牌部队,连战连胜,民军闻之色变,望风而逃。\\

在历史上,他的这支军队,有一个专门的称呼——“洪兵”。\\

洪承畴是文官,杨鹤也是文官,这是两个人的共同点,也是他们唯一的共同点。\\

对待民军,杨鹤是很客气的,投降前,他好言好语招抚,投降后,他好吃好喝招待。\\

而洪承畴的态度有点差别。投降前,他说,如果不投降,就杀掉你们;投降后,他说,你们投降了,所以杀掉你们。\\

对于这件事情,我始终很疑惑,读圣贤书,就读出这么个觉悟?\\

自古以来,杀人放火之类的事,从来没断过,但公认最无耻的事,就是“杀降”,人家都投降了,你还要干掉他,太过缺德。\\

但更让我疑惑的是,这种缺德事,洪承畴同志非但干了,还经常干。\\

比如那位曾经围过韩城,被洪承畴打跑的王左桂,后来也投降了。洪承畴听说后,决定请他吃饭。\\

还没吃完,一群人冲进来,把王左桂剁了。\\

我始终觉得,这事干得相当龌龊,就算动手,起码也得等人家吃完饭。\\

落在他手上的民军头领,不是抵挡到底被杀,就是不抵抗投降被杀。总之,无论抵抗到底,还是不抵抗到底,都得被杀。\\

但事实告诉我们,在某些时候,这种方法是有效的,至少对某些人很有效。\\

这个某些人,是指张献忠之类的人。\\

关于张献忠的具体情况,这里先不讲;关于他后来有没有在四川干过那些事,也不讲;只讲一个问题——投降的次数。\\

我曾经在图书馆翻过半个月的史料,查询张献忠先生投降的相关问题,我知道他是经常投降的,但我不知道,他能经常到这个份上。\\

简单地说,他的投降次数,用一只手,是数不过来的,两只手都未必,而且他投降的频率也很高。有一次,从投降到再反,只用了十几天。\\

这是难能可贵的。一般说来,投降之后,也得履行个程序,吃个饭,洗个澡,找个地方定居,以上工作全部忙完,至少也得个把月。但张先生效率之高,速度之快,实在令人咂舌。\\

相比而言,李自成就好得多了。虽然他也投降,但还是很有几分硬气的,说不投降,就不投降,属于硬汉型人物。\\

大体而言,当时许多民军的行为程序是,起兵、作战、被官军包围,投降,走出包围圈,拿起武器,继续作战。\\

此类表演,基本都是固定节目,数不胜数。很快,你就会看到两个典型案例。\\

洪承畴跟杨鹤不同,他是一个现实主义者。在他看来,要彻底扭转形势,不能招抚,不能受降,只有一个办法——赶尽杀绝。\\

这种方式的效果相当明显,短短几个月内,西北局势开始稳定,各路民军纷纷受挫,首领被杀。\\

他的优异表现得到了很多人的关注,包括崇祯。对他而言,高升是迟早的事。\\

但他毕竟太年轻,资历太浅,还要继续等。\\

两个月后,一件事情的发生,缩短了洪承畴的等待时间。\\

崇祯四年(1631),估计是有心脏病,或是胆囊炎,起义军进攻延绥巡抚镇守城池的时候,这位巡抚大人竟然被活活吓死。\\

没胆的人死了,就让有胆的人上,洪承畴接替了他的位置。\\

进步是没有止境的,又过了两个月,他的顶头上司杨鹤被抓了,总督的位置空了出来。没人能顶替,也没人愿意顶替,除了洪承畴。\\

崇祯四年(1631)十月,洪承畴正式接任三边总督。\\

噩梦开始了。\\

当时的起义军,已经遍布西北,人数有几十万。虽说其中许多都是凑人数的,某些部队还携家带口,什么八十老母,几岁小孩都带上,但看起来,确实相当吓人。\\

比如宁夏总兵贺虎臣,有一次听说起义军到境内观光,立即带了两千精兵,准备出战。到地方后,他看到了起义军的前锋队伍。\\

然而他没有动手,就在那里看着,静静地看着,看了会,就走了。\\

因为他始终没有看到这支队伍的尾巴。\\

这是一列长队,从前到后,长几十里。\\

对这样的起义军,看看就行了,真要动手,就傻了。\\

问题在于,当时的西北,到处都是这样的队伍,穿街过巷,比游行还壮观,见着就发怵。\\

然后,洪承畴来了。\\

在这个世界上,洪承畴害怕的东西,大致还不多。\\

在给朝廷的报告里,他天才地解决了这个问题:\\

西北民变,人数虽多,但大都是胁从,且老幼俱在,并无战力,真正精壮之人,十之一二而已,击其首,即可大破之。\\

这意思是,虽然闹事的人多,但真正能打仗的,十个人里面,最多也就一两个,把这几个人干掉,事情就结了。\\

实践证明,他的理论非常正确,所谓几十万义军,真正能打仗的,也就几万人而已。\\

而这几万人中,最强悍的,是三个人:王左桂、王嘉胤、神一魁。\\

只要除掉这三个人,大局必定。\\

这三个人中,王左桂已经被杀掉了,所以下一个目标,是王嘉胤。\\

然而就在此时,洪承畴得知了一个惊人的消息——王嘉胤死了。\\

王嘉胤是被杀的,杀掉他的人,是他的部下。\\

他的部下之所以要杀他,实在是被人逼得没办法。\\

逼他们的人,叫做曹文诏。\\
\ifnum\theparacolNo=2
	\end{multicols}
\fi
\newpage
\section{第二个猛人}
\ifnum\theparacolNo=2
	\begin{multicols}{\theparacolNo}
\fi
对曹文诏这个人,洪承畴曾经有过一个评价:世间良将,天下无双。\\

曹文诏,山西大同人,和洪承畴不一样,他没有履历,没读过书,没有背景,出人头地之前,他只是个小兵。\\

十年前,他在一个人的手下当兵,跟着此人去了辽东。这个人叫做熊廷弼。\\

九年前,广宁兵败,明军溃败,他没有逃跑,而是坚持留了下来,见到了他第二个上司——孙承宗。\\

六年前,孙承宗走了,他还是留了下来,此时,他已经当上了游击,而他的新上司,就是袁崇焕。\\

两年前,他跟着袁崇焕到了京城,守护北京,结果袁崇焕被抓,他依然留了下来。\\

一年前,他跟随孙承宗前往遵化,在那里,他奋勇作战,击退后金贝勒阿敏,并最终收复关内四城。\\

然后,他来到了西北。\\

对于这个人,我想就没必要多说了,从熊廷弼、孙承宗到袁崇焕,他都跟过,从努尔哈赤、皇太极到阿敏,他都打过。\\

什么世面都见过,什么牛人都跟过,现在把他调回来,打农民军。\\

而且他不是一个人回来的,跟着他回来的,还有一千人。\\

这一千人,是他的老部下,他们隶属于一支特殊的部队——关宁铁骑。\\

关宁铁骑,是明朝最精锐的特种部队,但人数并不多,大致在六千人左右,其中一半,在祖大寿的手中,曹文诏带回来的,只是六分之一。\\

而他的对手王嘉胤,手下的民军主力,在三万人左右。\\

王嘉胤什么来历,说法很多,靠谱的不多,但在当时那一拨人里,他是很牛的。之前我说过,在他手下,有个叫张献忠的小喽罗。顺便再说句,后来威震天下、被称为“闯王”的高迎祥(李自成是闯王2.0版本),都是他的人,给他打工。\\

而且这人很难得,很有点组织才能,连个县都没占住,就开始搞政府机构。但最搞笑的是,他还大胆地搞了机构改革,突破常规,明朝有的,他有;明朝没有的,他也有,不但有六部都察院,还有宰相。\\

当然,对于这些,曹文诏是没有兴趣的,到任后一个月,他就动手了。\\

按通常的说法,他率数倍于民军的官兵,以压倒性的优势,发动了进攻。\\

但事实是有点区别的,王嘉胤的兵力前面说过,是三万人,而曹文诏带去的人,是三千。\\

估计王嘉胤原先没在部队混过,也不大知道曹文诏何许人也,对曹总兵的来访,他倒不是很紧张,毕竟就三千人,还能咋样。\\

王嘉胤认为,就算曹文诏再强,就算他手下有关宁铁骑,但毕竟是十个打一个,无论如何,都是不会输的。所以他摆好了阵势,准备迎敌。\\

他太单纯了。\\

要知道,打了十几年仗,换了三任领导,从努尔哈赤打到皇太极,还能混到现在,光凭勇猛,十条命都是不够的。\\

曹文诏之所以出名,不是因为勇猛,而是因为耍诈。\\

此人身经百战,通晓兵法,到地方后,压根没动手,先断了王嘉胤的粮道。\\

王嘉胤慌了,要坚守,没有粮食,要突围,又没法冲出去。\\

就这样,王嘉胤冲了两个月,终于,在他即将放弃时,奇迹出现了。\\

曹文诏的包围圈,竟然出现了漏洞,王嘉胤终于找到机会,冲出重围。\\

王嘉胤感觉很幸运,虽说被困了两个月,但好歹还是出来了。换个地方,还能接着干。\\

可惜他并不知道,曹文诏是一个没有漏洞的人,他所有的失误,都是故意的。\\

把人围起来,然后死磕,是可以的,但是损失太大,最好的方法,是把他们放出来,然后一路追着打。\\

在这个思想的指导下,王嘉胤逃了出来,逃出来后,就后悔了。\\

因为从他逃出来那天起,曹文诏就跟在他屁股后面,紧追不放,追上就是一顿猛捶,五天之内打了五仗,王嘉胤一败涂地。\\

更可气的是,曹文诏似乎不打算一次把他玩死,每次打完就撤,等你跑远点,下次再打,反正他的部队是骑兵。对此,王嘉胤极为郁闷。\\

其实曹文诏也很郁闷,谁让你有三万人,我只有三千,只能慢慢打。\\

打了两个月,王嘉胤崩溃了,王嘉胤的部下也崩溃了。在某个混乱的夜晚,王嘉胤被部下杀死,部分投降了曹文诏。\\

王左桂死了,王嘉胤也死了,剩下的,还有神一魁。\\

在所有的起义军中,最能打的,最能坚持的,就是神一魁。\\

为了彻底铲除这个心腹之患,洪承畴决定,跟曹文诏合作。\\

所谓合作,就是客气客气。就官职而言,洪承畴是总督,曹文诏是总兵,洪承畴是进士,曹文诏是老粗。基本上,洪承畴怎么说,曹文诏就怎么做,相当听话。\\

几年后的那场悲剧,即源自于此。\\

其实这个时候,神一魁已经挂了,真正掌控军权的,是四个人:红军友、李都司、杜三、杨老柴。\\

虽说头头死了,但势头一点没消停,光主力部队,就有五万人,聚集在甘肃镇原,准备进攻平凉。\\

所以洪承畴决定,一次性彻底解决问题。\\

除曹文诏之外,他还调来了王承恩、贺虎臣等人,基本上西北最能打的几个总兵,都到齐了。\\

到齐了,就是群殴。\\

群殴之后,民军撑不住了,决定向庆阳撤退。\\

想法是好的,可惜做不了。特别是曹文诏,由于他率领的关宁铁骑,每人都有两匹马,骑累一匹就换一匹,机动性极强,民军往哪跑,他就等在哪。跑来跑去,没能跑出去。\\

经过两个月的僵持,双方终于在镇原附近的西濠决战,史称西濠之战。\\

整个战役的过程,大致相当于一堂生动的骑兵训练课。刚开打,还没缓过劲,曹文诏就率军冲入了敌军,乱砍乱杀,大砍大杀,基本上是怎么砍怎么有。\\

砍完了,退回来,歇会,歇完了,再冲进去,接着砍。所谓如入无人之境,大致就是这个状态。\\

民军的阵脚大乱,与此同时,洪承畴派出了他的主力洪兵,连同贺虎臣的宁夏兵,王承恩的甘肃兵,发动总攻,敌军就此彻底崩溃。\\

此战,民军损失近万人,首领杜三、杨老柴被生擒(曹文诏抓的)。\\

残余部队全部逃散。\\

通常状态下,都打残了,也就拉倒了。\\

洪承畴不肯拉倒,打残是不够的,打死是必须的。\\

神一魁的四个头领,抓了两个,还剩两个——红军友、李都司。\\

这个艰巨的任务,由曹文诏接手,他率领自己的两千骑兵,开始了追击。\\

接下来,是曹文诏的表演时间。\\

面对曹文诏的追击,几万军队几乎无法抵抗,连战连败,死伤近万,主要原因,还是曹文诏太猛。\\

曹总兵是见过大世面的,最猛的八旗军他都没怕过,打半业余的民军,自然没问题。每次进攻,他都带头冲锋,打得民军头目胆战心惊,时人有云:“军中有一曹,西贼闻之心胆摇”。\\

这种说法是客观的,却是不全面的,因为曹总兵不但玩硬的,还玩阴的。\\

在追击的路上,曹文诏的手下报告,他们抓住了一个叫李宫用的敌军将领。\\

按日常惯例,处理方法都是拉出去砍了,但曹文诏想了想,对手下说,放了这个人。\\

此后的事情,用史书上的话说,“文诏乃纵反间,绐其党,杀红军友。”\\

这句话的意思是,曹文诏放走了这个人,并利用他使了个反间计,忽悠了他的同党,杀掉了四大首领中的红军友。\\

其实我也很想告诉你,这个反间计到底怎么使的,只是我查了很多史料,也没查个明白。\\

有一点是肯定的,对民军而言,曹文诏,是最为恐惧的敌人。\\

人恐惧了,就会逃跑,逃无可逃,就不逃了。\\

神一魁剩下的,只有李都司了。\\

他很恐惧,所以他逃跑,但残酷的事实告诉他,继续跑,是没有前途的。\\

所以他决定,不跑了,回头,决战曹文诏!\\

等等,再想想。\\

想明白了,不跑了,回头,伏击曹文诏!\\

没办法,对付这样的猛人,还是伏击比较靠谱。\\

他们伏击的地点,叫做南原。\\

为保证圈套成功,他们围住了附近的一群明军,吸引曹文诏前来救援。\\

曹文诏来了,但在这里,他看到了敌军上千名骑兵,二话不说就追。\\

追到了南原,穿进了圈套,伏兵四起。\\

应该说,伏兵还是有点作用的,受到突然袭击,曹文诏的部队被打乱,曹文诏被冲散。\\

李都司估计是读过史书的,至少看过淝水之战,他当即派人在军中大喊:曹文诏已死!\\

很快,就喊成了口号,鉴于曹文诏不知被冲到哪去了,所以这个谣言很有点用,明军开始动摇。\\

然后,曹文诏就开始辟谣了,不用话筒,用长矛。\\

精彩表演开始,按史书上的说法,是“持矛左右突,匹马萦万众中。诸军望见”。\\

拿着长矛,左冲右突,单枪匹马在万军之中,如入无人之境,然后,大家都看见了他。\\

遇上这么个人,谣言是不管用了,伏击也别扯了,所以最后的结果,只能是“大败,僵尸蔽野”。\\

数过来,这应该是第二次大败了。但对于洪承畴和曹文诏而言,还没完。\\

残余部队的残余继续逃跑,曹文诏继续追击,然后是大败、复大败,又复大败。一路败到平凉,李都司终于不用败了,洪承畴杀掉了他。神一魁的四大头领,最终无人幸免。\\

但到这份上,曹总兵还没消停,他继续追击残敌,竟然追到了甘肃、宁夏,连续几战,把残敌赶尽杀绝,至此,神一魁的势力彻底退出历史舞台。西北之内,反军所剩无几。\\

王左桂、王嘉胤、神一魁,崇祯元年的三大民军领袖,就此结束他们的戏份,在这个舞台上,他们注定只是个配角。\\

\subsection{主角}
配角死光了,但龙套并没死,因为活不下去的人,终究还是活不下去,头头死了,就另找活路。\\

秉持这个原则,王左桂、王嘉胤、神一魁的残部,以及所有无法活下去的人,为了生存,继续战斗。\\

但鉴于陕西、甘肃打得太狠,他们跑到了山西。\\

虽说是半业余组织,但吃了这么大的亏,总结总结经验是应该的。于是,在王嘉胤部将王自用的号召下,所有剩下来的民军领袖,聚集在一块,开了个会。\\

会议的内容,是检讨教训,互相学习,互相促进,顺便再选领导。\\

其实也不用选,一般这种事,都是论资排辈。经过群众推举,王自用以资历最多,工龄最长,顺利当选新任头头。\\

鉴于曹文诏、洪承畴之类猛人的出现,大家共同认为,必须团结起来,协同作战。\\

当时去开会的,共有三十六支部队,史称“三十六营”。\\

跟以往一样,这三十六位头目,有三十六个外号,大致如下:\\

紫金梁、闯王、八大王、曹操、闯塌天、闯将、扫地王、黑煞神……\\

就外号水平而言,跟水浒传还没在一个档次上,梁山好汉们的文化程度,估计是够格的,什么急先锋、拼命三郎、花和尚,都是现代的流行用语,相比而言,扫地王之类的外号,实在让人不知所谓。\\

而且就人数而言,也差点,水浒好汉们,总共是一百单八个,这次只有三十六个,也就够个天罡。\\

但在某一点上,他们跟梁山好汉是很相似的,不可思议地相似。\\

你应该还记得,梁山好汉排队时,排在第一的,并不是及时雨宋江,而是托塔天王晁盖。\\

然而晁盖并不是真正的主角,因为后来他被人给挂了。\\

这次的三十六位老大也一样,排在第一的紫金梁,就是王自用,他是当时的首领,后来倒没被人挂,自己挂了。\\

真正的主角,是后面的五位,外号你要不知道,那就对个号吧:\\

{\footnotesize \begin{quote}
	闯王——高迎祥;\\
	八大王——张献忠;\\
	曹操——罗汝才;\\
	闯塌天——刘国能;\\
\end{quote}}

最后,是最牛的一位,闯将——李自成。\\

这是极为有趣的五个人,他们性格不同,关系不同,有的是上下级,有的是战友,有的是老乡,为了生存,揭竿而起。\\

然而在此后的十几年里,他们终将因为各自的原因,选择各自的道路,或互相猜忌,或者互相排挤,互相残杀,直至人生的终点。\\

终点太远了,从起点说起吧。\\

开完这次会后,各位老大纷纷表示,要统一思想,集中力量,共同行动。\\

这次开会的起义军,总兵力,近二十万人,开完后就分开了。\\

分开去打仗。\\

他们兵分几路,开始向山西各地进军。\\

崇祯得知,立即下令山西巡抚,全力围剿。\\

当时的山西巡抚,是个水货。\\

这位仁兄调兵倒很有一套,听说敌人来了,马上四处拉人,陕西、甘肃、宁夏的兵都被他拉了过来,光是总兵,就有三个。\\

但这人有个毛病,喜欢排兵布阵,把人调来调去,指挥乱七八糟,还没等他布出个形状,几路民军连续攻克多地,闹得天翻地覆。\\

于是崇祯恼火了,他决定换人,换一个能让这三十六位首领做噩梦的人——曹文诏。\\

曹文诏算是出头了。原先在辽东系,也就是个游击,荣归故里后,短短一年时间,就升了副总兵,现在是总兵。\\

山西总兵,大致相当于军区司令员,但按崇祯的意思,这个总兵,大致相当于总司令,因为根据命令,所有追剿军,都要服从曹文诏的指挥。\\

对于这个安排,三十六位头头是有准备的,所以他们决定,以太原一带为基地,协同合作,集中优势兵力,击溃曹文诏。\\

崇祯六年(1633),曹文诏正式上任,积极备战,准备进攻。\\

大战即将开幕,但在开幕之前,这场戏又挤上来一个人。\\

对这个人,曹文诏是比较熟悉的,因为在到西北之前,他经常见到这个人。\\

此人之所以上场,是被崇祯临时硬塞进来的。一般说来,但凡在历史舞台上混的,除个别猛人外(如朱元璋),艺术生涯都比较短,混个几年就得下场。\\

但这位仁兄,上场的时间实在很长,曹文诏下去了,他没下去,明朝亡了,他都没下去,直到死在场上,都是主角。\\

隆重介绍,第三个猛人——左良玉。\\

就知名度而言,左良玉是比较高的,在很大程度上,他要感谢孔尚任,因为这位仁兄把他写进了自己的戏里(《桃花扇》),虽然不是啥正面角色,但好歹是露了脸。\\

左良玉,字昆山,无学历,文盲。\\

左良玉的身世,是非常秘密的,秘密到连他自己都不知道。从小父母双亡,由叔父抚养长大,就这么个出身,你让他饱读诗书,就是拿他开涮。\\

没书读,也得找工作,长大以后,左良玉去当了兵,小兵。\\

他的成长经历,跟曹文诏类似,但他混得比曹文诏好,到崇祯元年的时候,就已经混到了都司。\\

顺便说一句,他之所以混得好,跟个人努力关系不大,只是因为一个偶然的机会。\\

天启年间,他还是个小兵时,有一次机缘巧合,遇到了一个人。\\

当时的左良玉,实在没啥特点,谁都瞧不上,但这个人算是例外。看见左良玉后,惊为天人,说他很好,将来很强大,就说了几句话,建议朝廷给他提了个游击。\\

这位慧眼识才的仁兄,叫做侯恂,希望你还记得他,因为天启二年,他还曾经提拔过另一个人——袁崇焕。\\

按侯恂的说法,左良玉是个难得的人才,很快就会出人头地。\\

但事情跟他所说的,似乎还是有点差距,左良玉一直到崇祯元年,还是个小人物。\\

但不负侯恂所望,左良玉终究还是出名了,只是出名的方式,比较特别。\\

这事之前也提过,崇祯元年,宁远兵变,巡抚毕自肃自尽,袁崇焕来收拾残局,收拾来收拾去,就把左良玉给收拾了。\\

当兵的没拿到工资,才兵变,左良玉有工资,自然不参加,但手下的兵哗变,他负领导责任,就这么被赶回了家。\\

回家呆了几天,又回来了。\\

袁崇焕死后,孙承宗又把他召了回来,去打关内四城,就是在那里,他开始暂露头角,和曹文诏并肩作战,收复了遵化。\\

恰好,这段时间侯恂也混得不错,顺道给他提了副将,从此顺风顺水。\\

客观地讲,左良玉同志的进步,基本上是靠侯恂的。但后来的事情告诉我们,侯恂是个眼光很准的人。袁崇焕,他没有看错;左良玉,也没有。\\

根据史料记载,左良玉身材很高,作战很猛,且足智多谋。虽说没文化,但很懂兵法,每次打仗都给人下套挖坑,此外,他个人的战斗技术也相当厉害。\\

除作战外,左良玉还有点个人技术,他使用的兵器,不是长矛,而是弓箭。据说百发百中,而且左右手都能射箭,速度极快。\\

到山西后,果然不同凡响。\\

先在涉县打了一仗,大败之,然后在辉县打了一仗,大败之,最后到了武安,被大败之。\\

这是个比较奇怪的事,当时左良玉的手下,有七八千人,竟然被农民军全歼,他自己带着几个手下好不容易才跑回来,实在很没有名将风采。\\

不过不要紧,就算名将,也有发挥失常的时候,何况还有个不会发挥失常的名将。\\

曹文诏的发挥从未失常,对于皇帝的信任,他很感动。\\

猛人被感动,反映在行动上,就是猛打,猛杀。\\

崇祯六年(1633)二月,曹文诏开始攻击。\\

他追击的敌人,有二十万,而他的兵力,是三千人。\\

无须怀疑,你没有看错,这就是曹文诏所有,且仅有的兵力。\\

他的追击之旅,第一站是霍州。在这里,他遇上了自己的第一个对手——上天龙。\\

上天龙究竟是谁,就别问了。我只知道,他是死在曹文诏手下的第一个首领。\\

上天龙手下,有上万人,摆好阵势,曹文诏率军冲锋。\\

这位兄弟抵抗的时间,也就是那一冲的瞬间——一冲就垮。\\

垮得实在太快,所以头头也没来得及跑,就被曹文诏杀了。\\

他的第二站,是孟县。\\

孟县,离太原没多远,在这里等待着他的,是混世王。\\

混世王这个外号,是很有点哲学意味的,毕竟在世上,也就是个混。但曹文诏用实际行动生动地告诉他,混是容易的,混成王是很难的。\\

双方在孟县相遇,混世王的兵力,大致是曹文诏的六倍。\\

六十倍都没用。\\

曹文诏毫无费力,就击溃了混世王,混世王想跑,没跑掉,被曹文诏斩杀。\\

当时的太原,算是民军的天下,因为这里是三十六营首领,紫金梁王自用的老巢。此外,如闯王高迎祥、闯将李自成等猛人,也都在那一带混。\\

曹文诏来后,就没法混了。\\

在他到任几个月后,史书上出现了这样的记载,“五台、盂县、定襄、寿阳贼尽平。”\\

曹文诏实在太猛,他连续作战,连续获胜,先后击溃十几支民军,但凡跟他作战的,基本都撑不过一天。此后,他又在太谷、范村、榆社连续发起攻击,“贼几消尽。”\\

其实打到这个份上,就算够意思了,但曹文诏是个比较较真的人,非要干到底,因为那个最终的目标,就在他的眼前——紫金梁。\\

曹文诏是明白人,他知道,就凭对方这二十多万人,即使站在那里不动,让他砍,三千人,也得砍上十天半月。\\

所以最快,最方便的办法,就是干掉紫金梁。\\

为实现这个目标,他发动了连续攻击,关于这段时间的经历,史书上的记载,大致是时间、地名、斩杀人数——曹文诏斩杀的人数。\\

短短十五天内,曹文诏率军七战七胜,打得紫金梁到处乱跑。先到泽州、再到润城、沁水,每到一地,最多一天,曹文诏就到,到了就打,打了就胜。\\

紫金梁原本的想法,是集中兵力,跟曹文诏死磕。\\

死磕未必能行,死是肯定的。\\

一个月,紫金梁的兵力已经损失了近三分之一,这么下去,实在赔不起了。\\

于是他做出决定,分兵。\\

紫金梁现在的想法是,曹文诏再猛,也没法分身,分兵之后,就看运气了,谁运气不好,被逮着,命苦不能怨政府。\\

就这么办了,紫金梁分工。他去榆社,老回回(三十六营之一)去武乡,过天星(三十六营之一)去高泽。\\

关于结局,史书上记载如下:“文诏皆击败”。\\

到底怎么办到,我到今天也没弄明白。\\

但紫金梁、八大王们明白了,混到今天,再不躲就没命了。\\

曹文诏是山西总兵,山西是没法呆了,往外跑。\\

跑路的方向,有两个,一个是直隶(河北),另一个是河南。\\

紫金梁去了河南,至少在那里,他还是比较安全的。\\

这个想法再次被证明,是错误的。因为曹文诏同志是很负责的,别说中国河南,就算欧洲的荷兰,估计照去。\\

在曹文诏的追击下,紫金梁王自用吃了大亏,好不容易跑到河南济源,终于解脱了。\\

人死了,就解脱了。\\

所幸,他还算是善终,在被曹文诏干掉之前,就病死了。\\

崇祯六年五月,紫金梁死去了,三十六营联盟宣告结束。\\

紫金梁结束了他的使命,接替他的,将是一个更为强大的人。\\

\subsection{合谋}
当然,对当时的起义军而言,这并不重要,重要的是,曹文诏还在追。\\

紫金梁死后,曹文诏继续攻击。在林县,他遇上了滚地龙率领的民军主力,一晚上功夫,全灭敌军,杀死滚地龙。此后又攻下济源,在那里,他杀死了三十六营的重要头领老回回。\\

洪承畴在陕西,陕西消停了,曹文诏在山西,山西也消停了。虽然河南也不安全,但对于众位头领而言,能去的地方,也只有河南了,具体的地点,是河南怀庆。\\

河南怀庆,位于河南北部,此地靠近山西五台山地区,地段很好,想打就打,不想打就钻山沟,是个好地方。\\

于是,崇祯六年(1633)六月,山西、陕西的民军基本消失——全跑去河南了。\\

河南的日子还算凑合,虽说曹文诏经常进来打几圈,但时不时还能围个县城,杀个把知县,混得还算凑合。到崇祯六年六月,来这里的民军,已经有十几万人。\\

但好日子终究到头了,因为另一个猛人,来到了河南——左良玉。\\

三年前,孙承宗收复关内四城的时候,最能打的两个,就是左良玉和曹文诏。\\

就军事天赋而言,两人水平相当,也有人说,左良玉还要厉害点,之所以打仗成绩不好,说到底还是个人员素质问题。\\

曹文诏率领的,是关宁铁骑,所谓天下第一强军,战斗力极强,打起来也顺手。\\

但左良玉估计是跟袁崇焕关系不好,来的时候,没有分到关宁铁骑(大多数在祖大寿的手上),只能在当地招兵。\\

这就比较麻烦了,倒不是说当地人不能打仗,关键在于,参加民军闹事的,大都也是当地人。\\

老乡见老乡,两眼泪汪汪,都是苦人家,闭只眼就过去了,官军也好,民军也罢,都是混饭吃,何必呢?\\

而这一次,左良玉得到了一支和以往不同的军队——昌平兵。\\

明代的军队,就战斗力而言,一般是北方比南方强。北方的军队,最能打的,自然是辽东军。问题在于,辽东军成本太高,给钱不说,还要给地,相对而言,昌平兵性价比很高,而且就在京城附近,也好招。\\

带着这拨人,左良玉终于翻身了,他连续出击,屡战屡胜,先后斩杀敌军上万人,追着敌军到处跑。\\

到崇祯六年(1633)九月,不再跑了。\\

民军主力被他赶到了河南武安,估计是跑得太辛苦,大家跑到这里,突然想到了一个问题:我们有十几万人,还跑什么?就在这里,跟左良玉死磕。\\

这是一个极为错误的抉择。\\

敌人不跑了,左良玉也不跑了,他开始安静下来,不发动进攻,也不撤退。\\

对左良玉的反常举动,民军首领们很纳闷,但鉴于左总兵向来彪悍,他们一致决定等几天,看这位仁兄到底想干什么。\\

左良玉想干的事情,就是等几天。\\

他虽然很猛,也很明白,凭自己这点兵力,追着在屁股后面踹几脚还可以,真卷袖子上去跟人拼命,是万万不能地。\\

在对手的配合下,左良玉安心地等了半个月,终于等来了要等的人。\\

根据崇祯的统一调派,山西总兵曹文诏、京营总兵王朴、总兵汤九州以及河南本地军队,日夜兼程,于九月底抵达武安,完成合围。\\

对首领们而言,现在醒悟,已经太晚了。\\

下面,我们介绍这个包围圈里的诸位英雄。据史料记载,除了知名人物高迎祥、张献忠、罗汝才、李自成外,还有若干历史人物,如薛仁贵、刘备(都是外号)以及某些新面孔:比如鞋底光(一直没想明白这外号啥意思,估计是说他跑得快),逼上路(这个外号很有觉悟)、一块云(估计原先干过诗人)、三只手(这个……);某些死人,比如混世王、上天龙……(应该之前已经被曹文诏干掉了)。\\

大抵而言,所有你知道,或是不知道的,都在这个圈里。\\

对诸位首领而言,崇祯六年的冬天应该是过不去了。\\

因为除被围外,他们即将迎来另一个相当可怕的消息。\\

按规定,但凡跨省调动,应该指认一名前线总指挥,根据级别,这个包围圈的最高指挥者,必定是曹文诏。\\

当然,如果真是曹文诏管这摊子事,历史估计就要改写了,因为以他老人家的脾气,逮住这么个机会,诸位首领连全尸都捞不着。\\

可是,不是曹文诏。\\

因为一个偶然的事件。\\

崇祯六年九月,曹文诏被调离,赴大同任总兵。\\

关于这次任命,许多史书上都用了一个词来形容——自毁长城。\\

打得好好的,偏要调走,纯粹是找抽。\\

而这笔帐,大都算到了御史刘令誉的头上。\\

因为据史料记载,曹文诏当年在山西的时候,跟刘御史住隔壁,曹总兵书读的少,估计也不大讲礼貌,欺负了刘御史,两人结了梁子。\\

后来刘御史到河南巡视,曹总兵跟他聊天,聊着聊着不对劲了,又开始吵,刘御史可能吃了点亏,回去就记住了,告了一黑状,把曹文诏告倒了,经崇祯批准,调到大同。\\

史料是对的,说法是不对的。\\

因为按照明代编制,山西总兵和大同总兵,算是同一级别,而且崇祯对曹文诏极为信任,别说一状,一百状都告不倒。\\

真正的答案,在半年后揭晓。\\

崇祯七年(1634)初,皇太极率军进攻大同。\\

崇祯是个很苦的孩子,上任时年纪轻轻,小心翼翼地装了两年孙子,干掉了死太监,才算正式掌权,掌权之后,手下那帮大臣又斗来斗去,好不容易干了几件事(比如裁掉驿站),又干出来个李自成。辛辛苦苦十几年,最后还是没辙。\\

史料告诉我们,崇祯很勤奋,他每天只睡几个小时,天天上朝,自己和老婆穿的衣服都打着补丁,也不好色(估计没时间),兢兢业业这么多年,没享受权利,尽承担义务。这样的皇帝,给谁谁都不干。\\

很可怜。\\

可怜的崇祯同志之所以要把曹文诏调到大同,是因为他没有办法。\\

家里的事要管,外面的事也得管,毕竟手底下能打仗的人就这么多,要有两个曹文诏,这事就结了。\\

对于皇太极的这次进攻,崇祯是有准备的,但当进攻开始的时候,才发现准备不足。\\

皇太极进攻的兵力,大致在八万人左右,打宁远没指望,但打大同还是靠谱的。\\

自进攻发起之日,一个月内,大同防线被全面击破,各地纷纷失守,曹文诏虽然自己很猛,盖不住手下太弱,几乎毫无还手之力。\\

击破周边地区后,皇太极开始集结重兵,攻击大同。\\

大同是军事重镇,一旦失陷,后果不堪设想。就兵力对比而言,曹文诏手下只有两万多人,而主力关宁铁骑,只有一千多人,失陷只是时间问题。\\

于是崇祯也玩命了,在他的调派下,吴襄率关宁铁骑主力,日夜兼程赶往大同,参与会战。\\

曹文诏也确实厉害,硬扛了十几天,等来了援兵。\\

皇太极眼看没指望,抢了点东西也就撤了。\\

崇祯七年(1634)的风波就此平息,手忙脚乱,终究是搞定了。\\

但曹文诏同志就惨了,虽然他保住了大同,但作为最高指挥官,责任是跑不掉的,好在朝廷里有人帮他说几句话,才捞了个戴罪立功。\\

但皇太极这次进攻,导致的最严重后果,既不是抢了多少东西,杀了多少人,也不是让曹总兵背黑锅,而是那个包围圈的彻底失败。\\

其实在崇祯十七年的统治中,有很多次,他都有机会将民军彻底抹杀。\\

这是第一次。\\
\ifnum\theparacolNo=2
	\end{multicols}
\fi
\newpage
\section{突围}
\ifnum\theparacolNo=2
	\begin{multicols}{\theparacolNo}
\fi
事实证明,那个包围圈相当结实,众位头领人多势众,从九月被围时起,就开始突围,突了两个月,也没突出去。\\

到十一月,连他们自己都认定,完蛋的日子不远了。\\

当时已是冬季,天气非常地冷,几万人被围在里面,没吃没喝,没进没退,打也打不过,跑也跑不掉。\\

然而不要紧,还有压箱底的绝技,只要使出此招,强敌即可灰飞烟灭——投降。\\

当然了,投降是暂时的,先投降,放下武器,等出了圈,拿起武器,咱再接着干。\\

但你要知道,投降也是有难度的。\\

为顺利投降,他们凑了很多钱,找到了京城总兵王朴,向他行贿。\\

没有办法,因为你要投降,还要看人家接不接受你投降。为了共同的目标,适当搞搞关系,也是应该的。\\

而且按很多人的想法,首领们应该是很穷的,总兵应该是很富的,事实上,这句话倒过来说,也还恰当。比如后来的张献忠,在谷城投降后,行贿都行到了朝廷里,上到大学士、下到知县,都收过他的钱。\\

人不认人,钱认人,这个道理,很通用。\\

问题在于,参与包围的人那么多,为什么偏偏行贿王朴呢?\\

这是一个关键问题,而这个问题的答案充分说明,诸位头领的脑袋,是很好使的。\\

只能行贿王朴,没有别的选择。\\

因为王朴同志,是京城来的。\\

在包围圈的全部将领中,他是最单纯的,最没见过世面。\\

王朴同志虽然来自京城,见惯大场面,但西北的场面,实在是没有见过,而在这群头领面前,他也实在比较单纯。\\

他知道,打仗有两种结果,投降就投降,不投降就打死,却不知道还有第三种——假投降。\\

他也不知道,在这个包围圈里的诸位头领,都有投降的经历,且人均好几次,某些层次高点的,如张献忠,那都是投降的专业人士。\\

再加上无知单纯的王总兵,也有点不单纯,还是收了头领们的钱,他还算比较地道,收钱就办事。\\

崇祯六年(1634)十一月十八日,首领们派了代表,去找王朴(钱已经送过了),表示自己的投降诚意,希望大家从此放下屠刀(当然,主要是你们),立地成佛。\\

王朴非常高兴,他的打算是完美的,受降,自己发点财,还能立功受奖,善莫大焉。\\

他随即下令,接受投降,并催促众首领早日集结队伍,交出武器。\\

当然他并没有撤除包围,那种蠢事他还是干不出来的。\\

但既然投降了,就是内部矛盾了,没必要兴师动众,可以原地休息,要相信同志。\\

你要说王朴没有丝毫提防,那也不对,他限令头头们十日之内,必须全部缴械投降。\\

不用十天,四天就够了。\\

二十四日,十余万民军突破王朴的防线,冲出了包围圈。\\

大祸就此酿成。\\

鉴于所有的军队都在搞包围,河南基本是没什么兵,所以诸位头领打得相当顺手,很是逍遥了几天。\\

也就几天。\\

十二月三日,左良玉就追来了。\\

包围圈被破后,崇祯极为恼火,据说连桌子都踹了,当即下令处罚王朴,并严令各部追击。\\

左良玉跑得最快。\\

之所以最快,倒不是他责任心有多强,只是按照行政划分,河南是他的防区,如果闹起来,他是要背黑锅的。\\

摆在面前的局势,是非常麻烦的,十几万民军涌入河南,遍地开花,压根没法收拾。\\

左良玉收拾了,他收拾了河南境内的所有民军——只用了二十天。\\

实践证明,左总兵是不世出的卓越猛人,他率领几千士兵,连续出击,在信阳、叶县等地先后击溃大量民军,肃清了所有民军,从头至尾,二十天。\\

左良玉同志工作成绩如此突出,除了黑锅的压力,以及他本人的努力外,还有一个更为重要的原因:他所肃清的,只是河南境内的民军,那些头领的主力,已经跑了。\\

跑到湖广了,具体地点,是湖广的郧阳(今湖北郧阳)。\\

我认为,他们跑到这个地方,是经过慎重考虑的。\\

跟河南接壤的几个省份,陕西是不能去的,洪承畴在那里蹲着,而且这人专杀投降的,去了也没前途。\\

山西也不能去,虽说曹文诏调走了,但几年来,广大头领们基本被打出了恐曹症,到了山西地界,就开始发怵,不到万不得已,也不要去。\\

那就去湖广吧。\\

最早进去的是高迎祥和李自成,且去的时候,随身带着几万人。郧阳巡抚当时就晕菜了,因为郧阳属于山区,平时都没什么人跑来,也没什么兵,这回大发了,一来,就来几万人,且都是闹事的。各州各县接连失陷,完全没办法,只好连夜给皇帝写信,说敌人太多,我反正是没办法了,伸长脖子,等着您给一刀。\\

这段日子,对高迎祥和李自成而言,是比较滋润的,没有洪承畴,没有曹文诏,没有左良玉,在他们看来,郧阳是山区,估摸着也没什么猛人,自然放心大胆。\\

这个看法是错误的。\\

事实上,这里是有猛人的,第四个猛人。\\

说起来这位猛人所以出山,还要拜高迎祥同志所赐,他要不闹,估计这人还出不来。\\

但值得庆幸的是,在此人正式露面之前,高迎祥和李自成就跑了。\\

具体跑到哪里,就不知道了,反正是几个省乱转悠,看准了就打一把,其余头领也差不离,搞得中原各省翻天覆地,连四川也未能幸免。\\

事情闹到这个地步,只能用狠招了。\\

崇祯七年,崇祯正式下令,设置一个新职务。\\

明代有史以来最大的地方官,就此登场。\\

在此之前,明代最大的地方官,就是袁崇焕,他当蓟辽督师时,能管五个地区。\\

光荣的记录被打破了,因为这个新职位,能管五个省。\\

这个职务,在历史中的称谓,叫做五省总督。包括山西、陕西、河南、湖广、四川,权力极大,也没什么管辖范围,反正只要是流贼出没的地方,都归他管。\\

职位有了,还要有人来当,按照当时的将领资历,能当这个职务的,只有两个选择:A:洪承畴,B:曹文诏。\\

答案是C,两者皆不是。\\

任职者,叫做陈奇瑜。\\

陈奇瑜,万历四十四年进士,历任都察院御史、给事中,后外放陕西任职。\\

在陕西,他的职务是右参政,而左参政,是我们的老朋友洪承畴。\\

但为什么要选他干这份工作,实在是个让人费解的事。\\

就资历而言,他跟洪承畴差不多,而且进步也慢点,崇祯四年的时候,洪承畴已经是三边总督了,他直到一年后,才干到延绥巡抚,给洪承畴打工。\\

就战绩而言,他跟曹文诏也没法比。\\

无论如何,都不应该是他,但无论如何,偏就是他了。\\

所以对于这个任命,许多人都有异议,认定陈奇瑜有背景,走了后门。\\

但事实上,陈奇瑜并非等闲之辈。\\

崇祯五年的时候,由于民军进入山西,主力部队都去了山西,陕西基本是没人管,兵力极少。\\

兵力虽少,民变却不少,据统计,陕西的民军,至少有三万多人。\\

这三万多人,大都在陈奇瑜的防区,而他的手下,只有两千多人。\\

一年后,这三万多人都没了——全打光了。\\

因为陈奇瑜,是一个近似猛人的猛人。\\

作为大刀都扛不起来的文官,陈奇瑜同志有一种独特的本领——统筹。\\

他是一个典型的参谋型军官,善于谋划、组织,而当时的民军,只能到处流窜,基本无组织,有组织打无组织,一打一个准。\\

凭借着突出的工作成绩,陈奇瑜获得了崇祯的赏识,从给洪总督打工,变成洪总督给他打工。\\

对于领导的提拔,陈奇瑜是很感动的,也很卖力,准备收拾烂摊子。\\

这是一个涉及五个省,几十万人的烂摊子,基本上,已经算是烂到底了,没法收拾。\\

陈奇瑜到任后,第一个命令,是开会。\\

各省的总督、总兵,反正是头衔上带个总字的,都叫来了。\\

然后就是分配任务,你去哪里,打谁,他去哪里,打谁,打好了,如何如何,打不好,如何如何,一五一十都讲明白,完事了,散会。\\

散会后,就开打。\\

崇祯七年(1634)二月,陈奇瑜上任,干了四个月,打了二十三仗。\\

全部获胜。\\

陈奇瑜以无与伦比组织和策划能力告诉我们,所谓胜利,是可以算出来的。\\

\begin{quote}
	\begin{spacing}{0.5}  %行間距倍率
		\textit{{\footnotesize
				\begin{description}
					\item[\textcolor{Gray}{\FA }] 多算胜,少算不胜,而况于无算乎?——孙子兵法
				\end{description}
		}}
	\end{spacing}
\end{quote}

陈总督最让人惊讶的地方,倒不是他打了多少胜仗,而在于,他打这些胜仗的目的。\\

打多少仗,杀多少人,都不是最终目的,最终的目的是,再打一仗,把所有人都杀光。\\

而要实现这个目标,他必须把所有的首领和民军,都赶到一个地方,并在那里,把他们全都送进地府。\\

他选中的这个地方,叫做车厢峡。\\

车厢峡位于陕西南部,长几十里,据说原先曾被当作栈道,地势极为险要。\\

所谓险要,不是易守难攻,而是易攻难守。\\

此地被群山环绕,通道极其狭窄,据说站在两边的悬崖上,往下扔石头,一扔一个准。\\

更要命的是,车厢峡的构造比较简单,只有一个进口,一个出口,没有其他小路,从出口走到进口,要好几天。这就意味着,如果你进了里面,要么回头,要么一条路走到黑,没有中场休息。\\

几万民军,就进了这条路。\\

这几万民军,是民军的主力,据说里面还有李自成和张献忠。\\

为什么走这条路,没有解释,反正进去之后,苦头就大了去了。\\

陈奇瑜的部队堵住了后路,还站在两边的悬崖上,往下射箭、扔石头,没事还放把火玩,玩了十几天,彻底玩残了。\\

想跑是跑不掉的,想打也打不着,众头领毫无办法,全军覆没就在眼前,实在熬不住了。\\

使用杀手锏的时候到了。\\

我说过,他们的杀手锏,就是投降,准确地说,是诈降。\\

\begin{quote}
	\begin{spacing}{0.5}  %行間距倍率
		\textit{{\footnotesize
				\begin{description}
					\item[\textcolor{Gray}{\FA }] 没条件,谁投降啊?——春节晚会某小品
				\end{description}
		}}
	\end{spacing}
\end{quote}

很有道理,很现实,但在这里,应该加上两个字:\\

没条件,谁让你投降啊?\\

所以在投降之前,必须先送钱,就如同上次送给王朴那样。\\

于是头领们凑了点钱,送给了陈奇瑜。\\

然而陈奇瑜没有收。\\

崇祯没看错人,陈奇瑜同志确实是靠得住的,他没有收钱。\\

麻烦了,不收钱,我们怎么安心投降,不,是诈降呢?\\

但事实证明,头领们的智商是很高的,他们随即使出了从古至今,百试不爽的绝招——买通左右。\\

陈奇瑜觉悟很高,可是扛不住手下人的觉悟不高,收了钱后,就开始猛劝,说敌人愿意投降,就让他们投降,何乐不为?\\

陈奇瑜没有同意。\\

陈奇瑜并不是王朴,事实上,他对这帮头领,那是相当了解,原先当延绥巡抚时,都是老朋友,知道他们狡猾狡猾地,所以没怎么信。\\

我之前曾经说过,陈奇瑜是一个近似猛人的猛人。\\

所谓近似猛人的猛人,就是非猛人。\\

他跟真正的猛人相比,有一个致命的弱点。\\

拿破仑输掉滑铁卢战役后,有人曾说,他之所以输,是因为缺少一个人——贝尔蒂埃。\\

贝尔蒂埃是拿破仑的参谋长,原先是测绘员,此人极善策划,参谋能力极强,但凡打仗,只要他在,基本都打赢了,当时,他不在滑铁卢。\\

但最后,有人补充了一句:\\

如果只有他(贝尔蒂埃)在,但凡打仗,基本都是要输的。\\

陈奇瑜的弱点,就是参谋。\\

和贝尔蒂埃一样,陈总督是个典型的参谋型军官,他很会参谋,很能参谋,然而参来参去,把自己弄残了。\\

军队之中,可以没有参谋,不能没有司令,因为在战场上,最关键的素质,不是参谋,而是决断。\\

陈奇瑜同志只会参谋,不会决断。\\

面对手下的劝说和胜利的诱惑,他妥协了。\\

陈奇瑜接受了投降,在他的安排下,近五万民军走出了车厢峡。\\

其实陈奇瑜也很为难,既要他们投降,又不能让他们诈降,要找人看着,但如果人太多,会引起对方疑虑。为了两全其美,他动脑筋,想出了一个绝妙的方法:每一百降军,找一个人看着,监督行动。\\

注意,是一个人,看守一百个人。\\

想出这个法子,只能说他的脑袋坏掉了。\\

跟上次不同,这次张献忠毫不拖拖拉拉,很有工作效率,走出车厢峡,到了开阔地,连安抚金都没拿,反了。\\

我很同情那些看守一百个人的人。\\

事情到这里,就算是彻底扯淡了,崇祯极为愤怒,朝廷极为震惊,陈奇瑜极为内疚,最终罢官了事。\\

了事?那是没可能的。\\

各路头领纷纷焕发生机,四处出战,河南、陕西、宁夏、甘肃、山西,烽烟四起。\\

估计是历经考验,外加焕发第二次生命的激动,民军的战斗力越来越强,原本是被追着跑,现在个把能打的,都敢追着官兵跑。比如陕西著名悍将贺人龙,原本是去打李自成,结果被李自成打得落花流水,还围了起来,足足四十多天,断其粮食劝他投降,搞得贺总兵差点去啃树皮,差点没撑过来。\\

到崇祯八年(1635),中原和西北,基本是全乱了,这么下去,不用等清兵入关,大明可以直接关门。\\

好在崇祯同志脑子转得快,随即派出了王牌——洪承畴。\\

在当时,能干这活的,也就洪承畴了,这个人是彻头彻尾的实用主义者,手狠且心黑,对于当前时局,他的指导思想只有一字——杀。\\

杀光了,就没事了。\\

就任五省总督之后,他开始组织围剿,卓有成效,短短几个月,民军主力又被他赶到了河南,各地民变纷纷平息。\\

接下来的程序,应该是类似的,民军被逼到某个地方,被包围,然后被逼无奈,被迫诈降。\\

所谓事不过三,玩了朝廷两把,就够意思了,再玩第三把,是不可能的。\\

洪承畴已经磨好刀,等待投降的诸位头领,这一次,他不会让历史重演。\\

是的,历史是不会重演的。\\

这次被逼进河南的民军,算是空前规模,光是大大小小的首领,就有上百人,张献忠、李自成、高迎祥、罗汝才、刘国能等大腕级人物,都在其中。民军的总人数,更是达到了创纪录的三十万。\\

为了把这群人一网打尽,崇祯也下了血本,他调集了近十万大军,包括左良玉的昌平兵,曹文诏的关宁铁骑、洪承畴的洪兵,总而言之,全国的特种部队,基本全部到齐。\\

但凡某个朝代,到了最后时刻,战斗力都相当之差,但明朝似乎是个例外。几十年前,几万人就能把十几万日军打得落花流水,几十年后,虽说差点,但还算凑合。\\

和以往一样,面对官军的追击,民军节节败退,到崇祯八年(1635),他们被压缩到洛阳附近,即将陷入重围,历史即将重演。\\

但终究没有重演。\\

因为在最关键的时刻,他们开了个会。\\

开会的地点,在河南荥阳,故史称“荥阳大会”。\\

这是一次极为关键的会议,一次改变了无数人命运的会议。\\

参与会议者,包括所有你曾经听说过,或者你从未听说过,或者从未存在过的著名头领。用史书上的说法,是“十三家”和“七十二营”。\\

家和营都是数量单位,但具体有多少人,实在不好讲。某些家,如高迎祥,有六七万人,某些营,兴许是皮包公司,只有几个人,都很难讲,但加起来,不会少于二十五万人。\\

当然,开会的人也多,十三加上七十二,就算每户只出个把代表,也有近百人。\\

简而言之,这是一次空前的大会,人多的大会。\\

根据史料留下的会议记录,会议是这样开始的,曹汝才先说话,讲述当前形势。\\

形势就别讲了,虽说诸位头领文化都低,还是比较明白事情的,敌人都快打上来了,还讲个屁?\\

有人随即插话,提出意见,一个字——逃。\\

此人认为,敌人来势很猛,最好是快跑,早跑,跑到山区,保命。\\

在场的人,大都赞成这个意见。\\

然后,一人大喝而起:“怯懦诸辈!”\\

说话的人,是张献忠。\\

张献忠,陕西延安府人,万历三十四年出生。\\

历史上,张献忠是一个有争议的人,夸他的人实在不多,骂他的人实在不少。\\

反映在他的个人简历上,非常明显。\\

但凡这种大人物,建功立业之后,总会有人来整理其少年时期的材料,而张献忠先生比较特殊,他少年时期的材料,似乎太多了点。\\

就成分而言,有人说,他家世代务农;有人说,他家是从商的;也有人说,他是世家后代;还有人说,他是读书出身;最后有人说,他给政府打工,当过捕快。\\

鉴于说法很多,传说很多,我就不多说了,简单讲下,这几种说法的最后结果:\\

务农说:务农不成,歉收,去从军了。\\

从商说:从商不成,亏本,去从军了。\\

世家说:世家破落,没钱,去从军了。\\

读书说:读书没谱,落第,去当兵了。\\

打工说:没有前途,气愤,去当兵了。\\

史料太多,说法太多,但所有的史料都说,他是一个不成功的人。\\

无论是务农、读书、从商、世家、打工,就算假设全都干过,可以确定的是,都没干好。\\

为什么没干好,没人知道,估计是运气差了点,最后只能去从军。\\

从军在当时,并非什么优秀职业,武将都没地位,何况苦大兵。\\

当兵,无非是拿饷。可是当年当兵,基本没有饷拿,经常拖欠工资,拖上好几个月,日子过得比较艰苦。\\

但奇怪的是,张献忠不太艰苦。据史料记载,他的小日子过得比较红火,有吃有喝,相当滋润,家里还很有点积蓄。\\

这是个奇怪的现象,而唯一的解释,就是他有计划外收入。\\

而更奇怪的是,他还经常被人讹,特别是邻居,经常到他家借钱,借了还不还,他很气愤,去找人要,人家不给,他没辙。\\

这是更为奇怪的一幕,作为手上有武器的人,还被人讹,只能说明,这些计划外收入,都是合法外收入。\\

据说,张献忠先生除了当兵之外,还顺便干点零活,打点散工,具体包括强盗、打劫等等。\\

这种兼职行为,应该是比较危险的,常在河边走,毕竟要湿鞋。张献忠同志终于被揭发了,他被关进监狱,经过审判,可能是平时兼职干得太多,判了个死刑。\\

关键时刻,一位总兵偶尔遇见了他,觉得他是个人才,就求了个情,把他给放了。\\

应该说这位总兵的感觉,还是比较准的,张献忠确实是个人才,造反的人才。\\

据说平时在军队里,张献忠先生打仗、兼职之余,经常还发些议论,说几句名人名言,比如“燕雀安知鸿鹄之志”,“王侯将相,宁有种乎”等等。\\

而他最终走上造反道路,是在崇祯三年(1630),那时,王嘉胤造反,路过他家乡,张献忠就带了一帮人,加入了队伍。\\

张献忠起义的过程,是比较平和的,没人逼他去修长城,他似乎也没掉队,至于爹妈死光,毫无生路等情况,跟他都没关系,而且在此之前,他还是吃皇粮的,实在没法诉苦。\\

所以这个人造反的动机,是比较值得怀疑的。\\

参加起义军后,张献忠的表现还凑合,跟着王嘉胤到处跑,打仗比较勇猛,打了一年,投降了。\\

因为杨鹤来了,大把大把给钱,投降是个潮流,张献忠紧跟时代潮流,也投了降。\\

当然,后来他花完钱后,顺应潮流,又反了。\\

此后的事情,只要是大事,他基本有份。三十六营开会、打进山西、打进河南、被人包围、向王朴诈降、又被人包围、向陈奇瑜诈降,反正能数得出来的事,他都干过。\\

但在这帮头领里,他依然是个小人物,总跟着别人混,直至这次会议。\\

他驳斥了许多人想逃走的想法,是很有种的,但除了有种外,就啥都没有了。因为敌人就在眼前,你要说不逃,也得想个辙。然而张献忠没辙。\\

于是,另一个人说话了,一个有辙的人:\\

“一夫犹奋,况十万众乎!官兵无能为也!”\\

李自成如是说。\\

李自成,陕西米脂人,万历三十四年生人。\\

比较凑巧的是,李自成跟张献忠,是同一年生的。\\

而且这两人的身世,都比较搞不清楚,但李自成相对而言,比较简单。\\

根据史料的说法,他家世代都是养马的。在明代,养马是个固定职业,还能赚点钱,起码混口饭吃,生活水准,大致是个小康。\\

所以李自成是读过书的,他从小就进了私塾,但据说成绩不好,很不受老师重视,觉得这孩子没啥出息。\\

直到有一天。\\

这天,老师请大家吃饭,吃螃蟹。\\

当然,老师的饭没那么容易吃,吃螃蟹前,让大家先根据螃蟹写首诗,才能开吃。\\

李自成想了想,写了出来。\\

老师看过大家的诗,看一首,评一首,看到他写的诗,没有说话。\\

因为在这首诗里,有这样一句话:一身甲胄任横行。\\

这位老师是何许人也,实在没处找,但可以肯定的是,他是一个比较厉害的人物,因为在短暂犹豫之后,他说出了一个准确的预言:\\

你将来必成大器,但始终是乱臣贼子,不得善终!\\

但李自成同学的大器之路,似乎并不顺利,吃过饭不久,他就退学了,因为他的父亲去世了。\\

没有经济基础,就没有上层建筑,李自成决定,先去打基础,但问题是,他家并不是农民,也没地,种地估计是瞎扯,所以他唯一能够选择的,就是给人打工。\\

这段时间,应该是李自成比较郁闷的时期。因为他年纪小,父亲又死了,经常被人欺负,有些地主让他干了活,还不给钱,万般无奈之下,他托了个关系,去驿站上班了。\\

李自成的职务是驿卒,我说过,驿站大致相当于招待所,驿卒就是招待所服务员,但李自成日常服务的,并不是人,而是马。\\

由于世代养马,所以李自成对马,是比较有心得的,他后来习惯于用骑兵作战,乃至于能在山海关跟吴三桂的关宁铁骑打出个平手,估计都是拜此所赐。\\

李自成在驿站干得很好,相比张献忠,他是个比较本分的人,只想混碗饭吃。\\

崇祯二年,饭碗没了。\\

我说过很多次,是刘懋同志建议,全给裁掉了。\\

刘懋认为,驿站纰漏太多,浪费朝廷资源,李自成认为,去你娘的。\\

你横竖有饭吃,没事干了,来砸我的饭碗。\\

但李自成还没有揭竿而起的勇气,他回了家,希望打短工过日子。\\

我也说过很多次,从崇祯元年,到崇祯六年,西北灾荒。\\

都被他赶上了,灾荒时期,收成不好,没人种地,自然没有短工的活路。此时,李自成听说,有一个人正在附近招人,去了的人都有饭吃。\\

他带着几个人去了,果然有饭吃。\\

这位招聘的人,叫做王左桂。\\

王左桂是干什么的,之前也说过了,作为与王嘉胤齐名的义军领袖,他比较有实力。\\

当时王左桂的手下,有几千人,分为八队,他觉得李自成是个有料的人,就让他当了八队的队长。\\

这是李自成担任的第一个职务,也是最小的职务,而他的外号,也由此而生——八队闯将。\\

一年后,王左桂做出了一个决定,他要攻打韩城。\\

他之所以要打这里,是经过慎重考虑的,因为韩城的防守兵力很少,而且当时的总督杨鹤,没有多少兵力可以增援,攻打这里,可谓万无一失。\\

判断是正确的,正如之前所说的,杨鹤确实没有兵,但他有一个手下,叫洪承畴。\\

这次战役的结果是,洪承畴一举成名,王左桂一举完蛋,后来投降了,再后来,被杀降。\\

王左桂死掉了,他的许多部属都投降了,但李自成没有,他带着自己的人,又去投奔了不沾泥。\\

不沾泥是个外号,他的真名,叫做张存孟(也有说叫张存猛)。但孟也好,猛也罢,这人实在是个比较无足轻重的角色,到了一年后,他也投降了。\\

然而李自成没有投降,他又去投了另一个人。这一次,他的眼光很准,因为他的新上司,就是闯王高迎祥。\\

这是极其有趣的一件事,王左桂投降了,李自成不投降,不沾泥投降了,他也没投降。\\

虽说李自成也曾经投降过,比如被王朴包围,被陈奇瑜包围等等,但大体而言,他是没怎么投降的。\\

这说明,李自成不是痞子,他是有骨气的。\\

相比而言,张献忠的表现实在不好。\\

他投降的次数实在太多,投降的时机实在太巧,每次都是打不过,或是眼看打不过了,就投降,等缓过一口气,立马就翻脸不认人,接着干,很有点兵油子的感觉。\\

史料记载,张献忠的长相,是比较魁梧的,他身材高大,面色发黄(所以有个外号叫黄虎),看上去非常威风。\\

而李自成就差得多了,他的身材不高,长得也比较抱歉,据说不太起眼(后来老婆跑路了估计与此有关),但他很讲义气,很讲原则,且从不贪小便宜。\\

历史告诉我们,痞子就算混一辈子,也还是痞子,滑头,最后只能滑自己。长得帅,不能当饭吃。\\

成大器者的唯一要诀,是能吃亏。\\

吃亏就是占便宜,原先我不信,后来我信,相当靠谱。\\

李自成很能吃亏,所以开会的时候,别人不说,他说。\\

第八队队长,不起眼的下属,四处寻找出路的孤独者,这是他传奇的开始。\\

他说,一个人敢拼命,也能活命,何况我们有十几万人,不要怕!\\

大家都很激动,他们认识到,李自成是对的,到这个份上,只能拼了。\\

但问题在于,他们已经被重重包围,在河南呆下去,死路,去陕西,还是死路,去山西,依然是死路,哪里还有路?\\

有的,还有一条。\\

李自成以他卓越的战略眼光,和无畏的勇气,指出那条唯一道路。\\

他说,我们去攻打大明的都城,那里很容易打。\\

他不是在开玩笑。\\

当然,这个所谓的都城,并不是北京,事实上,明代的都城有三个。\\

北京,是北都,南京,是南都,还有一个中都,是凤阳。\\

打北京,估计路上就被人干挺了,打南京,也是白扯,但打凤阳,是有把握的。\\

凤阳,位于南直隶(今属安徽),这个地方之所以被当作都城,只是因为它是朱元璋的老家。事实上,这里唯一与皇室有关的东西,就是监狱(宗室监狱,专关皇亲国戚),除此以外,实在没啥可说,不是穷,也不是非常穷,而是非常非常穷。\\

但凤阳虽然穷,还特喜欢摆谱,毕竟老朱家的坟就在这,逢年过节,还喜欢搞个花灯游行,反正是自己关起门来乐,警卫都没多少。\\

这样的地方,真是不打白不打。\\

而且进攻这里,可以吸引朝廷注意,扩大起义军的影响。\\

话是这么说,但是毕竟洪承畴已经围上来了,有人去打凤阳,就得有人去挡洪承畴,这么多头领,谁都不想吃亏。\\

所以会议时间很长,讨论来讨论去,大家都想去打凤阳,最后,他们终于在艰苦的斗争中成长起来,领悟了政治的真谛,想出了一个只有绝顶政治家,才能想出的绝招——抓阄。\\

抓到谁就是谁,谁也别争,谁也别抢,自己服气,大家服气。\\

抓出来的结果,是兵分三路,一路往山西,一路往湖广,一路往凤阳。\\

但这个结果,是有点问题的,因为我查了一下,抓到去凤阳的,恰好是张献忠、高迎祥、李自成。\\

没话说了。\\

但凡是没办法了,才抓阄,但有的时候,抓阄都没办法。\\

真没办法。\\

抓到好阄的一干人等,向凤阳进发了,几天之后,他们将震惊天下。\\

在洪承畴眼里,所谓民军,都是群没脑子的白痴,但一位哲人告诉我们,老把别人当白痴的人,自己才是白痴。\\

检讨\\

很巧,民军抵达凤阳的时候,是元宵节。\\

根据惯例,这一天凤阳城内要放花灯,许多人都涌出来看热闹,防守十分松懈。\\

就这样,数万人在夜色的掩护下,连大门都没开,就大摇大摆地进了凤阳城。\\

慢着,似乎还漏了点什么——大门都没开,怎么能够进去?\\

答:走进去。\\

因为凤阳根本就没有城墙。\\

凤阳所以没有城墙,是因为修了城墙,就会破坏凤阳皇陵的风水。\\

就这样,连墙都没爬,他们顺利地进入了凤阳,进入了老朱的龙兴地。\\

接下来的事情,是比较顺理成章的,据史料记载,带军进入凤阳的,是张献忠。\\

如果是李自成,估计是比较文明的,可是张献忠先生,是很难指望的。\\

之后的事情,大致介绍一下,守卫凤阳的几千人全军覆没,几万多间民房,连同各衙门单位,全部被毁。\\

除了这些之外,许多保护单位也被烧个干净,其中最重要的单位,就是朱元璋同志的祖坟。\\

看好了,不是朱元璋的坟(还在南京),是朱元璋祖宗的坟。\\

虽说朱五一(希望还记得这名字)同志也是穷苦出身,但张献忠明显缺乏同情心,不但烧了他的坟,还把朱元璋同志的故居(皇觉寺)也给烧了。\\

此外,张献忠还很有品牌意识,就在朱元璋的祖坟上,树了个旗帜,大书六个大字:“古元真龙皇帝”。\\

就这样,张献忠在朱元璋的祖坟上逍遥了三天,大吃大喝,然后逍遥而去。\\

事大了。\\

从古至今,在骂人的话里,总有这么一句:掘你家祖坟。\\

但一般来讲,若然不想玩命,真去挖人祖坟的,也没多少。\\

而皇帝的祖坟,更有点讲究,通俗说法叫做龙脉,一旦被人挖断,不但死人受累,活人也受罪,是重点保护对象。\\

在中国以往的朝代里,除前朝被人断子绝孙外,接班的也不怎么挖人祖坟,毕竟太缺德。\\

真被人刨了祖坟的,也不是没有,比如民国的孙殿英,当然他是个人行为,图个发财,而且当时清朝也亡了,龙脉还有没有,似乎也难说。\\

朝代还在,祖坟就被人刨了的,只有明朝。\\

所以崇祯听到消息后,差点晕了过去。\\

以崇祯的脾气,但凡惹了他的,都没有好下场。崇祯二年,皇太极打到北京城下,还没怎么着,他就把兵部尚书给砍了,现在祖坟都被人刨了,那还了得。\\

但醒过来之后,他却做出了一个让人意外的决定——做检讨。\\

请注意,不是让人做检讨,而是自己做检讨。\\

皇帝也是人,是人就会犯错误,如皇帝犯错误,实在没法交代,就得做检讨。这篇检讨,在历史上的专用名词,叫做“罪己诏”。\\

崇祯八年(1635)十月二十八日,崇祯下罪己诏,公开表示,皇陵被烧,是他的责任,民变四起,是他的责任,用人不当,也是他的责任,总而言之,全部都是他的责任。\\

这是一个相当奇异的举动,因为崇祯同志是受害者,张献忠并非他请来的,受害者写检讨,似乎让人难以理解。\\

其实不难理解,几句话就明白了。\\

根据惯例,但凡出了事,总要有人负责,县里出事,知县负责,府里出事,知府负责,省里出事,巡抚负责。\\

现在皇帝的祖坟出了事,谁负责?\\

只有皇帝负责。\\

对崇祯而言,所谓龙脉,未必当真。要知道,当年朱元璋先生的父母死了,都没地方埋,是拿着木板到处走,才找到块地埋的,要说龙脉,只要朱元璋自己的坟没被人给掘了,就没有大问题。\\

但祖宗的祖宗的坟被掘了,毕竟影响太大,必须解决。\\

解决的方法,只能是自己做检讨。\\

事实证明,这是一个相当高明的方法。自从皇帝的祖坟被掘了后,上到洪承畴,下到小军官,人心惶惶,唯恐这事拿自己开刀,据说左良玉连遗书都写了,就等着拉去砍了,既然皇帝做了检讨,大家都放心了,可以干活了。\\

当然,皇帝背了大锅,小锅也要有人背,凤阳巡抚和巡按被干掉,此事到此为止。\\

崇祯如此大度,并非他脾气好,但凡是个人,刨了他的祖坟,都能跟你玩命,更何况是皇帝。\\

但没办法,毕竟手下就这些人,要把洪承畴、左良玉都干掉了,谁来干活?\\

对于这一点,洪承畴、左良玉是很清楚的,为保证脑袋明天还在脖子上,他们开始全力追击起义军。\\

说追击,是比较勉强的,因为民军的数量,大致有三十万,而官军,总共才四万人。就算把一个人掰开两个用,也没法搞定。\\

好在,还有一个以一当十的人,曹文诏。\\
\ifnum\theparacolNo=2
	\end{multicols}
\fi
\newpage
\section{一个文雅的人}
\ifnum\theparacolNo=2
	\begin{multicols}{\theparacolNo}
\fi
为保证能给崇祯同志个交代,崇祯八年六月,曹文诏奉命出发,追击民军。\\

曹文诏的攻击目标,是十几万民军,而他的手下,只有三千人。\\

自打开战起,曹文诏就始终以少打多,几千人追几万人,是家常便饭。\\

但上山的次数多了,终究会遇到老虎的。\\

曹文诏率领骑兵,一口气追了几百里,把民军打得落花流水,斩杀数千人。\\

但自古以来,人多打人少,不是没有道理的。\\

跑了几百里后,终于醒过来了,三千人而已,跑得这么快,这么远,至于吗?\\

于是一合计,集结精锐兵力三万多人,回头,准备跟曹文诏决战。\\

崇祯四年起,曹文诏跟民军打过无数仗,从来没输过,胆子特大,冲得特猛,一猛子就扎了进去。\\

进去了就再没出来。\\

民军已走投无路,这次他们没打算逃跑,只打算死拼。\\

而曹文诏由于太过激动,只带了先锋一千多人,就跑过来了。\\

三万个死拼的人,对一千个激动的人,用现在的编制换算,基本相当于一个人打一个排,能完成这个任务的,估计只有兰博。\\

曹文诏不是兰博,但他实在也很猛,带着骑兵冲了十几次,所至之处,死伤遍地,从早上一直打到下午,斩杀敌军几千人。\\

眼看快到晚上,杀得差不多了,曹文诏准备走人。\\

这并非玩笑,曹总兵是骑马来的,就算打不赢,也能跑得赢。\\

在混乱的包围圈中,他集结兵力,发动突击,很快就突出了缺口,准备回家洗澡睡觉。\\

当时场面相当混乱,谁都没认出谁,在民军看来,跑几个也没关系,所以也不大有人去管这个缺口。\\

但关键时刻,出情况了。\\

曹文诏骑马经过大批民军时,有一个小兵正好被俘,又正好看见了曹文诏,就喊了一句:\\

“将军救我!”\\

当时的环境,应该是很吵的,有多少人听见很难说,但很不巧,有一个最不该听见的人,听见了。\\

这个人是民军的一个头目,而在不久之前,他曾在曹文诏的部队里干过。\\

作为一个敬业的人,他立即对旁人大喊:\\

“这就是曹总兵!”\\

既然是曹总兵,那就别想跑了。\\

民军集结千人,群拥而上围攻曹文诏。\\

曹文诏麻烦了,此时,他的手下已经被打散,跟随在他身边的,只有几个随从。\\

必死无疑。\\

必死无疑的曹文诏,在他人生的最后时刻,诠释了勇敢的意义。\\

面对上千人的围堵,他单枪匹马,左冲右突,亲手斩杀数十人,来回冲杀,无人可挡。\\

没人上前挑战,所有的人只是围着他,杀退一层,再来一层。\\

曹文诏是猛人,猛人同样是人,包围的人越来越多,他的伤势越来越重,于是,在即将力竭之时,他抽出了自己的刀。\\

在所有人的注视下,他举刀自尽。\\

曹文诏就这样死了,直到生命的最后一刻,他依然很勇敢。\\

无论如何,一个勇敢的人,都是值得敬佩的。\\

崇祯极其悲痛,立即下令追认曹文诏为太子太保,开追悼会,发抚恤金,料理后事等等。\\

从某个角度讲,曹文诏算是解脱了,崇祯还得接着受苦,毕竟那几十万人还在闹腾,这个烂摊子,必须收拾。\\

所以,曹文诏死后不久,崇祯派出了另一个人。\\

当时的局势,已经是不能再坏了,凤阳被烧了,曹文诏被杀了,皇帝也做了检讨,原先被追着四处跑的民军,终于到达了风光的顶点。\\

据史料记载,当时的将领,包括左良玉、洪承畴在内,都是畏畏缩缩,遇上人了,能不打就不打,非打不可,也就是碰一碰,只求把人赶走,别在自己防区里转悠,就算万事大吉。\\

对此,诸位头领大概也是明白的,经常带着大队人马转来转去,有一次,高迎祥带着十几万人进河南,左良玉得到消息,带人去看了看,啥都没说就回来了。\\

照这么下去,估计高迎祥就算进京城,大家也只能看看了。\\

然而一切都变化了,从那个人到任时开始。\\

对这个人,崇祯给予了充分的信任,给了一个绝后而不空前的职务——五省总督。\\

这个职务,此前只有陈奇瑜和洪承畴干过,但这人上来,并非是接班的,事实上,他是另起炉灶,其管辖范围包括江北、河南、湖广、四川、山东。\\

当时全国,总共只有十三个省,洪承畴管五个,他管五个,用崇祯的说法是:洪承畴督师西北,你去督师东南,天下必平!\\

这个人就是之前说过的第四个猛人,他叫卢象升。\\

对大多数人而言,卢象升是个很陌生的名字,但在当时,这是一个相当知名的名字,而在高迎祥、李自成的嘴里,这人有个专用称呼:卢阎王。\\

就长相而言,这个比喻是不太恰当的,因为所有见过卢象升的人,第一印象基本相同:这是个读书人。\\

卢象升,字建斗,江苏宜兴人。明代的江苏,算是个风水宝地,到明末,西北打得乌烟瘴气,国家都快亡了,这边的日子还是相当滋润,雇工的雇工,看戏的看戏。\\

鉴于生活条件优越,所以读书人多,文人多,诗人也多,钱谦益就是其中的优秀代表。\\

但除此外,这里也产猛人——卢象升。\\

所谓猛人,是不恰当的,事实上,他是猛人中的猛人。\\

但在十几年前,他跟这个称呼,基本是八杆子打不着,那时,他的头衔,是卢主事。\\

天启二年(1622),江苏宜兴的举人卢象升考中了进士,当时吏部领导挑中了他,让他在户部当主事。\\

据史料说,卢主事长得很白,人也很和气,所以人缘混得很好,没过两年,就提了员外郎,只用了三年时间,又提了知府。\\

到崇祯二年,卢象升已经是五品正厅级干部了,就提拔速度而言,相当于直升飞机,而且卢知府人品确实很好,从来没有黑钱收入,群众反应很好。\\

总之,卢知府的前途是很光明的,生活是很平静的,日子是很惬意的,直到崇祯二年。\\

这年是比较闹腾的,基本都是大事,比如皇太极打了进来,比如袁崇焕被杀死,当然,也有小事,比如卢象升带了一万多人,跑到了北京城下。\\

当时北京城下的援兵很多,有十几路,卢象升这路并不起眼,却是最有趣的一路,因为压根没人叫他来。\\

卢象升是文官,平时也没兵,但他听说京城危急,情急之下,自己招了一万多人,就跑过来了。\\

明末的官员,是比较有特点的,最大的特点,就是推卸责任,能不承担的,绝不承担,能承担的,也不承担,算是彻头彻尾的王八蛋。\\

卢象升负责任,起码他知道,领了工资,就该办事。\\

但遗憾(或者是万幸)的是,卢象升同志没能打上仗,他在城下呆了一个多月,后金军就走了。\\

当然,这未必是件坏事,因为以他当时的实力,要真跟人碰上,十有八九是个死。\\

但这无所事事的一个月,却永远地改变了卢象升的命运,因为这段时间里,他亲眼目睹,一个叫袁崇焕的统帅,如何在一夜之间,变成了囚犯。\\

这件事情,最终影响了他的一生,并让他在九年之后,做出了那个关键性的抉择。\\

朝廷的特点,一向是能用就使劲用,既然卢知府这么积极,干脆就让他改了行。\\

崇祯三年,卢象升提任参政,专门负责练兵。\\

当时最能打仗、最狠的兵,除辽东,就是西北,这两个地方的人相当彪悍,战斗力很强,敢于玩命,就算打到最后一个人,也不投降,是明朝主要的兵源产地。\\

卢象升练兵的地方是北直隶,就单兵作战能力而言,算是二流。\\

然而事实证明,只有二流的头头,没有二流的兵。\\

明朝的精锐部队,大都有自己的名字。比如袁崇焕的兵,叫做关宁铁骑,洪承畴的兵,叫做洪兵,而卢象升的兵,叫天雄军。\\

就战斗力而言,明末的军队中,最强的,当属关宁铁骑,天雄军的战斗力,大致排在第三(第二还没出场),比洪兵强。\\

据高迎祥和李自成讲,他们最怕的明军,就是天雄军。\\

比如关宁铁骑,虽然战斗力强,但都是骑兵,冲来冲去,死活好歹都是一下子,但天雄军就不同了,比膏药还讨厌,贴上就不掉,极其顽固,只要碰上了,就打到底,不脱层皮没法跑。\\

天雄军的士兵,大都来自大名、广平当地,并没有什么特别,之所以如此强悍,只是因为卢象升的一个诀窍。\\

两百多年后,有一个人使用了他的诀窍,组建了一支极为强悍的部队,这个人的名字,叫做曾国藩。\\

没错,这个诀窍的名字,叫做关系。\\

和曾国藩的湘军一样,卢象升的天雄军,大都是有关系的,同乡、同学、兄弟、父子,反正大家都是熟人,随便死个人,能愤怒一堆人,很有战斗力。\\

但这种关系队伍,还有个问题,那就是冲锋的时候,一个人冲,就会有很多人跟着冲,但逃跑的时候,有一个人跑,大家也会一起跑。\\

比如曾国藩同志,有次开战,就遇到这种事,站在后面督战,还划了条线,说越过此线斩,结果开打不久,就有人跑路,且一跑全跑,绕着线跑,追都没追上,气得投了河。\\

卢象升没有这个困惑,因为每次开战,他都站在最前面。\\

事实上,卢先生被称为卢阎王,不是因为他很能练兵,而是因为他很能杀人——亲手杀人。\\

之前我说过,卢象升长得很白,但我忘了说,他的手很黑。\\

卢象升是个很有天赋的人。据史料记载,他天生神力,射箭水平极高,长得虽然文明,动作却很粗野,每次作战时,都拿着大刀追在最前面,赶得对方鸡飞狗跳。\\

他最早崭露头角,是一次激烈的战斗。\\

崇祯六年,山西流寇进入防区,卢象升奉命出击,对方情况不详,以骑兵为主力,战斗力很强,人数多达两万。\\

卢象升只有两千人,刚开战,身边人还没反应过来,他就一头扎进了敌营。\\

他的这一举动,搞得对方也摸不着头脑,被他砍死了几个人后,才猛然醒悟,开始围攻他。\\

卢象升的大刀水平估计相当好,敌人只能围住,无法近身,万般无奈,开始玩阴的,砍他的马鞍(刃及鞍)。\\

马鞍被干掉了,卢象升掉下了马,然后,他站了起来,操起大刀,接着打(步战)。\\

接下来的事情,就比较骇人听闻了,卢象升就这么操着大刀,带着自己的手下,把对方赶到了悬崖边。\\

没办法了,只能放冷箭。\\

敌人的箭法相当厉害,一箭射中了卢象升的额头,又一箭,射死了卢象升的随从。\\

这两箭的意思大致是,你他娘别欺人太甚,逼急了跟你玩命。\\

这两箭的结果大致是,卢象升开始玩命了。而且他玩命的水平,明显要高一筹。\\

他提着大刀,越砍越有劲,几近疯狂(战益疾)。这下对方被彻底整懵了,感觉玩命都玩不过他,只好乖乖撤退,以后再没敢到他的地界闹事。\\

虽然卢象升的水平很高,但在当时,他还不怎么出名,也没机会出头,然而帮助他进步的人出现了,这人的名字叫做高迎祥。\\

崇祯七年,高迎祥等人跑出了包围圈,就进了郧阳,郧阳被折腾得够呛,巡抚也下了课,这事说过了。\\

但这件事,对卢象升而言,有着决定性的意义,因为接替郧阳巡抚的人,就是他。\\

如果高迎祥知道这件事情的后果,估计是死都不会去打郧阳的。\\

卢象升是个聪明人,聪明在他很明白,凭借目前的兵力,要把民军彻底解决,是绝不可能的。\\

作为五省总督(后来变成七省),他手下能够作战的精锐兵力,竟然只有五万人,但在这几省地界上转来转去的诸位头领,随便拉出来一个,都有好几万人,总计几十万,还满世界转悠,没处去找。\\

但他更明白,彻底解决民军的头领,是绝对可能的。\\

民军虽然人多势众,但大都是文盲,全靠打头的领队,只要把打头的干掉,立马就变良民。\\

而在所有的头头里,最有号召力,最能带队的,就是闯王。\\

强调,现在的闯王是高迎祥,不是李自成。\\

在所有的头领中,高迎祥是个奇特的人,他的奇特之处,就是他一点也不奇特。\\

明末的这帮头领,都是比较特别的,用今天的话说,就是很有个性。\\

但凡古代干这行的,基本是两种人,吃不上饭的,和混不下去的。文化修养,大都谈不上,所以做事一般都不守规矩,想怎么来就怎么来,军队也是一样,今天是这帮人,没准明天就换人了,指望他们严守纪律,按时出操,没谱。\\

但高迎祥是个特例,他没什么个性,平时不苟言笑,打赢了那样,打输了还那样。\\

许多头领打仗,明天究竟怎么走,不管,也懒得管,打到哪算哪。\\

高迎祥的行军路线,都是经过精心设计的,并表明路标,引导部队行进。\\

更吓人的是,高迎祥的部队,是有统一制服的——铠甲。\\

一般说来,盔甲这种玩意,只有官军才用(费用比较高,民军装备不起),大部都是皮甲,而高迎祥部队的盔甲,是铁甲。\\

所谓重甲骑兵,就是这个意思,更吓人的是,他的骑兵,每人都有两三匹马,日夜换乘,一天可以跑几百里,善于奔袭作战。\\

就这么个人,连洪承畴这种杀人不眨眼的角色,看见他都发怵。打了好几次,竟然是个平手。\\

所以一直以来,高迎祥都被朝廷列为头号劲敌。\\

卢象升准备解决这个人。\\

当然,他很明白,光凭他手下的天雄军,是很难做到的,所以,他上书皇帝,几经周折,要来了一个特殊的人。\\

这个人的名字,叫做祖宽。\\

祖宽,不是祖大寿的亲戚,具体点讲,他是祖大寿的佣人。\\

但祖大寿同志实在太过厉害,一个佣人跟着他混了几年,也混出来了,还当上了宁远参将。\\

其实对于祖宽,卢象升并不了解,他最了解的,是祖宽手下的三千部队——关宁铁骑。\\

作为祖大寿的亲信,祖宽掌管三千关宁军,卢象升明白,要战胜高迎祥,必须把这个人拉过来,必须借用这股力量。\\

现在,他终于成功了,他认定,高迎祥的死期已然不远。\\

此时的高迎祥,正在为攻打汝宁做准备,还没完事,祖宽就来了。\\

高迎祥到底是有点水平,他从没见过祖宽,但看架势,似乎比较难搞,毅然决定跑路。\\

但他之所以跑路,不是为逃命,而是为了进攻。\\

高迎祥的战略思想十分清晰,敌人弱小,就迎战,敌人强大,就先跑路,多凑几个人,人多了再打。\\

一年前,曹文诏就是被这种战法报销的。\\

这一次,他的目的地,是陕州,在这里,有两个人正等待着他——李自成、张献忠。\\

民军最豪华的阵容,也就这样了,高迎祥集结兵力,等待着祖宽的到来。\\

以现有的兵力,高闯王坚信,如果祖宽来了,就回不去了。\\

祖宽果然来了,也果然没有回去,因为高迎祥、李自成、张献忠又跑路了。\\

高迎祥的这次选择,是极为英明的,因为祖宽过来的时候,队伍里多了个人——左良玉。\\

高迎祥的这套策略,对付像王朴那样的白痴,估计还是有点用的,但祖宽这种老兵油子,那就没招了,他立马看穿了这个诡计,拉上了左良玉,一起去找高迎祥算帐。\\

接下来是张献忠先生的受难时间。\\

其实这事跟张献忠本没有关系,只是高迎祥让他过来帮忙,顺道挣点外快,可惜不巧的是,碰上了硬通货。\\

跑路的时候,根据惯例,为保证都能跑掉,是分头跑的,高迎祥、李自成是一拨,张献忠是另一拨。\\

所以官军的追击路线,也是两拨,左良玉一拨,祖宽一拨。\\

不幸的是,祖宽分到的,就是张献忠。\\

我说过,祖宽手下的,是关宁铁骑,跑得很快,所以他只用了一个晚上,就追上了张献忠,大破之。\\

张献忠逃跑了,他率领部队,连夜前行,一天一夜,跑到了九皋山。\\

安全了,终于安全了。\\

然后,他就看到了祖宽。\\

估计是等了很久,关宁军很有精神,全军突击,大砍大杀,张献忠主力死伤几千人,拼死跑了出去。\\

又是一路狂奔,奔了几百里,张献忠相信,无论如何,起码暂时是安全了。\\

然后,祖宽又出现了。\\

我说过,他的速度很快。\\

此后的结果,是非常壮观的,用史书的话说——伏尸二十余里。\\

张献忠出离愤怒了,而这一次,他做出了违反常规的决定,比较有种,回头跟祖宽决战。\\

是的,上面这句话是不靠谱的,张献忠先生从来不会违反常规,他之所以回头跟祖宽决战,因为在逃跑的路上,遇上了两个人——李自成、高迎祥。\\

人多了,胆就壮了,张献忠集结数万大军,在龙门设下埋伏,等待祖宽的到来。\\

张献忠的这个埋伏,难度很大,因为祖宽太猛,手下全是关宁铁骑,久经沙场,“发一声喊,伏兵四起”之类的场景,估计吓不住,就算用几万人围住,要冲出来,也就几分钟时间。\\

面对困境,张献忠同志展现了水平,他决定,攻击中间。\\

利用突袭,把敌军一分为二,分而击破,这是唯一的方法。\\

单就质量而言,他的手下实在比较一般,但正如一位名人所说,有数量,就有质量,他集结了十倍于祖宽的兵力,开始等待。\\

不出所料,祖宽出现了,依然不出所料,他没有丝毫防备,带领所有的兵力,进入了埋伏圈。\\

张献忠不出所料地发动了攻击,数万大军发动突袭,不出所料地把关宁军冲成了两截。\\

接下来,就是出乎意料的事了。\\

他惊奇地发现,虽然自己的人数占绝对优势,虽然自己出现得相当突然,但从这些被包围的敌人脸上,他看不到任何慌张。\\

其实张先生这一招,用在大多数官军身上,是很有效果的,对关宁军,是无效的。这帮人在辽东,主要且唯一的工作,就是打仗,见惯大场面,所谓伏兵,无非是出来的地方偏点,时间突然点,队伍分成两截,照打,有啥区别?\\

特别是祖宽,伏兵出现后,他非但没往前跑,反而亲自断后,就地组织反击,而他手下的关宁军,似乎也没有想跑的意思,左冲右突,大砍大杀,战斗从早上开始,一直打到晚上,伏兵打成了败兵,进攻打成了防守,眼看再打下去就要歇菜,撒腿就跑。\\

前后三战,张献忠损失极为惨重,死伤无数,被打出了毛病,据说听到卢象升、祖宽的名字就打哆嗦。\\

河南不能呆了,他率领军队,转战安徽。\\

相比而言,高迎祥、李自成的遭遇,可以用八个字来形容——只有更惨,没有最惨。\\

高迎祥第一次遇见卢象升,是在汝阳城外。\\

据史料记载,当时他的手下,有近二十万人,光是营帐,就有数百里(连营百里),浩浩荡荡,准备攻城,看起来相当吓人。\\

而他的对手,赶来救援的卢象升,只有一万多人。\\

其实一直以来,官军能够打败民军,原因在于官军骑马,而民军只能撒脚丫跑。\\

但高迎祥是个例外,我说过,他的军队,是重甲骑兵,而且每人有两匹马,机动性极强,而卢象升手下能跟他打两把的,只有关宁铁骑,且就一两千人。\\

更麻烦的是,当卢象升到达汝阳的时候,军需官告诉他,没粮食。\\

没粮食的意思,就是没饭吃,没饭吃的意思,就是没法打仗。\\

一般说来,军中断粮一天,军队就会失去一半战斗力,断粮两天以上,全军必定崩溃。\\

卢象升的军队断粮三天,没有一个逃兵。\\

这个看似没有可能的奇迹,之所以成为可能,只是因为卢象升的一个举动——他也断粮。\\

他非但不吃饭,连水都不喝(水浆不入口),此即所谓身先士卒。\\

所以结果也很明显——得将士心,同仇敌忾。\\

其实很多时候,群众是好说话的,因为他们所需要的并非粮食,而是公平。\\

公平的卢象升,是个很聪明的人,经过几天的观察,他敏锐地发现,高迎祥的部队虽然强悍,但是比较松散,选择合适的突破点,还是可以打一打的。\\

卢象升选择的突破点,是城西,鉴于自己步兵太多,骑兵太少,硬冲过去就是找死,他想到了一个办法。\\

一千多年前,诸葛亮同志鉴于实在干不过魏国的骑兵(蜀国以步兵为主),想到了同样的方法。\\

没错,对付骑兵,成本最低,老少咸宜的方式,就是弓箭,确切地说,是弩。\\

诸葛亮用的,叫做连弩,卢象升用的,史料上说,是强弩,具体工艺结构不太清楚,但确实比较强,因为历史告诉我们,高迎祥的重甲骑兵,在开战后仅仅几个小时里,就得到了如下结果——强弩杀贼千余人。\\

其实城西的部队被击破,死一千多人,对高迎祥而言,并不是啥大事,毕竟他的总兵力,有几十万人之多,但他的军阵中,有一个致命的弱点,导致了汝阳之战的失败。\\

这个弱点,就是人太多。\\

几十万人,连营百里,而据卢象升给皇帝的报告,高迎祥的主力骑兵,有五六万人,其余的大都是步兵以及部队家属。\\

步兵倒还好说,家属就麻烦了,这拨人没有作战能力,又大多属于多事型,就爱瞎咋呼,看到城西战败,便不遗余力地四处奔走,大声疾呼,什么敌人很多,即将完蛋之类。而最终的结果,就是真的完蛋了。\\

汝阳之战结束,高迎祥的几十万大军就此土崩瓦解,纷纷四散逃命,但高迎祥实在有点军事水平,及时布置后卫,阻挡卢象升的追击。\\

其实卢象升也没打算追击,一万人去追二十万人,脑子有问题。\\

但今天不追不等于明天也不追,卢象升看准机会,跟踪追击,在确山再次击败高迎祥,杀敌军数千人。\\

卢象升的亮相就此谢幕,自崇祯八年五月至十一月,他率绝对劣势兵力,先后十余战,每战必胜,斩杀敌军总计三万余人,彻底扭转了战略局势。\\

当然,高迎祥并不这么想,他依然认为,失败只是偶然,他所有的兵力,是卢象升的几十倍,战略的主动权,依然在他的手中,今年灭不了你,那就明年。\\

这个想法,让他最终只活到了明年。\\

十一月过去了,接下来的一个月,是很平静的,卢象升没有动,高迎祥也没有动,原因非常简单——过年。\\

无论造反也好,镇压也罢,都是工作,工作就是工作,遇到法定假日,该休息还是得休息。\\

休息一个月,崇祯九年正月,接着来。\\

最先行动的,是卢象升,他行动的具体方式,是开会。\\

开会内容,自然是布置作战计划,研究作战策略,讨论作战方案。\\

相对而言,高迎祥的行动要简单得多,只有两个字——开打。\\

从心底里,卢象升是瞧不上高迎祥的,毕竟是草寇,没读过书,没考过试,没有文化,再怎么闹腾,也就是个草寇,所以对于高迎祥的动向,卢象升是很有把握的:要么到河南开荒,要么去山西刨土,或者去湖广钻山沟,还有什么出息?\\

为此,他做了充分的准备,还找到了洪承畴,表示一旦高迎祥跑到西北五省,自己马上跑过去一起打。\\

然而高迎祥的举动,却是他做梦都想不到的。\\

闯王同志之所以叫闯王,就是因为敢闯,所以这一次,他决定攻击一个卢象升绝对想不到的地方——南京。\\

当然,在刚开始的时候,这个举动并不明显,他会合张献忠,从河南出发,先打庐州,打了几天,撤走。\\

接下来,他开始攻击和州,攻陷。\\

攻陷和州后,他开始攻击江浦,江浦距离南京,只有几十公里。\\

如果你有印象的话,就会发现,两百多年前,曾经有人以几乎完全相同的路线,发起了攻击,并最终取得天下——朱元璋。\\

高迎祥同志估计是读过朱重八创业史的,所以连进攻路线,都几乎一模一样,可惜他不知道,真正的成功者,是无法复制的。\\

朝廷大为震惊,南京兵部尚书立即调集重兵,对高迎祥发动反攻击,经过几天激战,高迎祥退出江浦。\\

退是退了,偏偏没走。\\

他集结几十万人,开始攻打滁州。\\

至此可以断定,他应该读过朱重八传记,因为几百年前,朱元璋就是从和州出发,攻占滁州,然后从滁州出发,攻下了南京。\\

滁州只是个地级市,人不多,兵也不多,而攻击者,包括李自成、张献忠等十几位头领,三十万人,战斗力最强,最能打的民军,大致都来了。\\

所有的头领,所有的士兵,都由高迎祥指挥。\\

高闯王终于爬上了人生山峰的顶点。\\

他决定,进攻滁州,继续向前迈步。\\

山峰的顶点,再迈一步,就是悬崖。\\

\subsection{惨败}
但至少在当时,形势非常乐观,滁州城内的兵力还不到万人,几十万人围着打,无论如何,是没问题的。\\

几天后,他得知卢象升率领援军,赶到了。\\

但他依然不怵,因为卢象升的援兵,也只有两万多人。此前虽说吃过卢阎王的亏,但现在手上有三十万人,平均十五个人打一个,就算用脚算,也能算明白了。\\

卢象升率领总兵祖宽、游击罗岱,向滁州城外的高迎祥发动了进攻。\\

双方会战的地点,是城东五里桥。\\

在讲述这场战役之前,有必要介绍一下滁州的地形,在滁州城东,有一条很宽的河流,水流十分汹涌。\\

我再重复一遍,河流很宽,水流很汹涌。\\

这场会战的序幕,是由祖宽开始的,关宁铁骑担任先锋,冲入敌阵,发动了进攻。\\

战斗早上开始,下午结束。\\

下午结束的时候,那条很宽,水流很汹涌的河流,已经断流了,断流的原因,史料说法如下——积尸填沟委堑,滁水为不流。\\

通俗点的说法,就是尸体填满了河道,水流不动。\\

尸体大部分的来源,是高迎祥的部下,在经历近七年的光辉创业后,他终于等来了自己最惨痛的溃败。\\

关宁铁骑实在太猛,面对城东两万民军,如入无人之境,乱砍乱杀。\\

高迎祥很聪明,他立即反应过来,调集手下主力骑兵,准备发动反击,毕竟有三十万人,只要集结反攻,必定反败为胜。\\

红楼梦里的同志们曾告诉我们这样一句话:大有大的难处。\\

高迎祥的缺点,就是他优点——人太多。\\

人多,嘴杂,外加刚打败仗,通讯不畅,也没有高音喇叭喊话,乱军之中,谁也摸不清怎么回事,所以高闯王折腾了半天,也没能集中自己的部队。\\

但高闯王还是很灵活的,眼看兵败如山倒,撒腿就往外跑,他相信,自己很快就能脱离困境。\\

这是很正确的,因为根据以往经验,官军都是拿工资的,而拿工资的人,有一个最大的特点——拿多少钱,干多少事。无论是洪承畴,还是左良玉,只要把闹事的赶出自己管辖范围就算数了,没人较真。所谓跟踪追击这类活动,应该属于加班行为,但朝廷历来没有发加班费的习惯,所以向来是不怎么追的,追个几里,意思到了,也就撤了。\\

但是这一次,情况发生了变化。\\

我说过,卢象升是一个好人,一个负责任的官员。这一点反映在战斗上,就是认死理,凡是都往死了办。\\

按照这个处事原则,他追了很远——五十里。\\

之前我还说过,卢象升的外号,是卢阎王,虽然长得很白,但手很黑,无论是民军,还是民军家属,只要被他追上,统统都格杀勿论,五十里之内,民军尸横遍野,保守估计,高迎祥的损失,大致在五万人以上。\\

追到五十里外,停住了。\\

不追,不是因为不想追,也不是不能追,而是不必追。\\

摆脱了追击的高迎祥很高兴,现在的局势并不算坏,三年前,他被打得只剩下几千人,逃到湖广郧阳,避避风头,二十天后出山,又是一条好汉,何况手上有几十万人乎?\\

但安徽终究是呆不下去了,他转变方向,向寿山进发,准备在那里渡过黄河,去河南打工。\\

黄河岸边,他就遇到了明军总兵刘泽清。\\

刘泽清用大刀告诉他,此路不通。\\

刘泽清并非猛人,并非大人物,也没多少兵,但是,他有渡口。\\

他就堵在河对岸,封锁渡口,烧毁船只,高迎祥只能看看,掉头回了安徽。\\

无所谓,到哪儿都是混。\\

但在回头的路上,他又遇见了祖大乐。\\

祖大乐也是辽东系的著名将领,遇上了自然没话说,又是一顿打,高迎祥再次夜奔。\\

好不容易奔到开封,又遇见了陈永福。\\

陈永福是个当时没名,后来有名的人,五年后,他坚守城池,把一个人变成了独眼龙——独眼李自成。\\

这种人,自然不白给,在著名地点朱仙镇跟高迎祥干了一仗,大败了高迎祥。\\

高迎祥终于发现,事情不大对劲了,自己似乎掉进了圈套。\\

他的感觉,是非常正确的。\\

得知高迎祥攻击滁州时,卢象升曾极为惊慌,但惊慌之后,他萌生了一个计划——彻底消灭高迎祥的计划。\\

高迎祥的想法,是非常高明的,学习朱重八同志,突袭南直隶,威胁南京,但遗憾的是,他忽略了一个重要的问题——他没有在这里混过。\\

没有混过的意思,就是人头不熟,地方不熟,什么都不熟。\\

所以这个计划的关键在于,绝不能让高迎祥离开,把他困在此地,就必死无疑。\\

刘泽清挡住了他的去路,祖大乐把他赶到了开封,陈永福又把他赶走,但这一切,只是序幕,最终的目的地,叫做七顶山。\\

七顶山,位于河南南阳附近,被祖大乐与陈永福击败后,高迎祥逃到了这里,就在这里,他看到了一个等候已久的熟人——卢象升。\\

当然,除了卢象升外,还有其余一干人等,比如祖大乐、祖宽、陈永福等等。\\

此时的高迎祥,手下还有近十万人,就兵力而言,大致是卢象升的两倍,更关键的是,他的主力重甲骑兵,依然还有三万多人。\\

然而战争的结果,却让人大跌眼镜,号称“第一强寇”的高迎祥,竟然毫无还手之力,主力基本被全歼,仅带着上千号人夺路而逃。\\

这是一个比较难以理解的事,最好的答案,似乎还是四个字——气数已尽。\\

十几万士兵、下属打得干干净净,兵器、家当丢得一干二净,高迎祥同志这么多年,折腾一圈,从穷光蛋,又变成了穷光蛋,基本算是白奋斗了,应该说,他很倒霉。\\

但我个人认为,有个人比他更倒霉——李自成。\\

这个世界上,还有什么事情,比变成光杆司令更倒霉呢?\\

有的,比如,变成光棍司令。\\

李自成的麻烦在于,他的老婆给他戴了绿帽子。\\

这位给李自成送帽子的老婆,叫邢氏,虽然不能肯定李自成有多少老婆,但这个老婆,是比较牛的。\\

按史料的说法,这位老婆基本不算家庭妇女,估计也不是抢来的,相当之强悍,打仗杀人毫无含糊,更难得的是,她还很有智谋,帮李自成管账,据说私房钱都管。\\

在管账的时候,她见到了高杰。\\

高杰,米脂人,李自成的老乡。据说打小时候就认识,后来李自成造反,他毫不犹豫,搭伙一起干,从崇祯二年开始,同生共死,是不折不扣的铁哥们。\\

铁哥们,也是会生锈的。\\

李自成第一次怀疑高杰,是因为一件偶然的事。\\

崇祯七年八月,时任五省总督陈奇瑜,派出参将贺人龙进攻李自成。\\

贺人龙是个相当猛的人,此人战斗力极强,且杀人如麻,每次上战场,都要带头冲锋,被称为贺疯子。\\

贺疯子气势汹汹地到了地方,看到了李自成,打了一仗,非但没打赢,还被人给围住了,且一围就是两个月。\\

但李自成并不想杀掉贺人龙,因为贺人龙是他的老乡,而且他正在锻炼队伍阶段,需要人才,就写了封信,让高杰送过去,希望贺人龙投降。\\

这个想法是比较幼稚的,贺人龙同志说到底是吃皇粮的,有稳定的工作,要他跟着李自成同志四处乱跑,基本等于胡扯,所以信送过去后,毫无回音,说拿去擦屁股也有可能。\\

按说这事跟高杰没关系,贺人龙投不投降,是他自己的事,可是意外发生了。\\

去送信的使者,从贺人龙那里回来后,没有直接去找李自成,而是找了高杰。\\

这算是个事吗?\\

在这个世界上,很多事,说是事,就是事,说不是,就不是。\\

而李自成明显是个喜欢把简单的问题搞复杂的人,加上贺人龙同志守城很厉害,他打了两个月,连根毛都没拔下来,所以他开始怀疑,贺人龙和高杰,有不同寻常的关系,就把高杰撤了回来。\\

无论是铁哥们,还是钛哥们,在利益面前,都是一脚蹬。\\

对李自成同志的行为,高杰相当不爽,但这事说到底,还是高杰的责任。\\

因为他回来之后,就跟邢氏勾搭上了。\\

到底是谁勾搭谁,什么时候勾搭上的,基本算是无从考证,但史料上说,是因为高杰长得很帅,而邢氏是管账的,高杰经常跑去报销,加上邢氏的立场又不太坚定,一来二去,就勾搭上了。\\

关于这件事情的严重性,高杰同志是有体会的,在回顾了和李自成十几年的交情、几年的战斗友谊,以及偷人老婆的内疚后,他决定,投奔官军。\\

当然,他是比较够意思的,临走时,把邢氏也带走了。\\

对李自成而言,这是一个极为沉重的打击,老婆跑了,除面子问题外,更为严重的是,他的很多秘密,老婆都知道(估计包括私房钱的位置)。\\

除了老婆损失外,还有人才损失。\\

在当时李自成的部下里,最能打仗的,就是高杰,此人极具天赋,投奔了官军后,就一直打,打到老主顾李自成都歇菜了,他还在继续战斗。\\

高杰投降的对象,是洪承畴,洪总督突然接到天上掉下来的馅饼,自然高兴异常,立刻派兵出击,连续击败李自成,斩杀万人。\\

总而言之,对各位头领而言,崇祯九年算是个流年,老婆跑了,手下跑了,跑来跑去,就剩下自己了。\\

对高迎祥而言,更是如此。\\

老婆跑了,再找一个就是,十几万大军都跑光了,就只能钻山沟了。\\

所以高闯王毅然决定,跑进郧阳山区。\\

两年前,就是在那里,被打得只剩半条命的高迎祥捡了条命,东山再起。\\

卢象升闻讯,立刻找到祖宽和祖大乐,吩咐他们,立即率军出发,追击高迎祥。\\

祖宽回答:不干。\\

卢象升无语。\\

之所以无语,因为他们从来就没干过。\\

\subsection{关宁铁骑}
很久以前,我以为所谓战争,大都是你死我活,上了战场,管你七大姑八大姨,都往死里打,特别是明末,但凡开打,就当不共戴天,不共戴地,不共戴地球,打死了算。\\

后研读历史多年,方才知道,以上皆为忽悠是也。\\

按史料的说法,当时的作战场景大致如下:\\

比如一支官军跟民军相遇,先不动手,喊话,喊来喊去,就开始聊天,聊得差不多,民军就开始丢东西,比如牲口,粮食等等,然后就退,等退得差不多了,官军就上前,捡东西,捡得差不多,就回家睡觉,然后打个报告给朝廷,说歼敌多少多少,请求赏赐云云。\\

应该肯定的是,在当时,有这种行为的官军,是占绝大多数,认认真真打仗的,只占极少数,所谓“抛生口,弃辎重,即纵之去”。\\

现象也好理解,当时闹事的,大都是西北一带人,而当兵的,也大都是关中人。双方语言相通,说起来都是老乡,反正给政府干活,政府也不发工资(欠饷),即使发了工资,都没必要玩命,这么打仗,非但能领工资,还能捞点外快,最后回去了还能领赏,非常有利于创收。在史料中,这种战斗方式有个专用名词:打活仗。\\

因为活仗好打,且经济效益丰富,所以大家都喜欢打,打来打去,敌人越打越多,局势越来越恶化,直到关宁铁骑的到来。\\

其实关宁铁骑的人数没多少,我算了一下,入关作战的加起来,也就五千来人,卢象升、洪承畴手下最能打的,基本就是这些人,最厉害的几位头领,都是被他们打下去的。\\

之所以能打,有两个原因,首先,这帮人在辽东作战,战斗经验丰富,而且装备很好,每人均配有三眼火铳,且擅长使用突袭战术,冲入敌阵,势不可挡。\\

而第二个原因,相当地搞笑,却又相当地真实。\\

我说过,每次打仗时,民军都要喊话,所谓喊话,无非就是谈条件,我给你多少钱,你就放我走,谈妥了就撤,谈不妥再打。\\

但每次遇到关宁铁骑,喊话都是没用的,经常是话没喊完,就冲过来了,完全不受收买,忠于职守。\\

我此前曾以为,如此尽忠职守,是因为他们很有职业道德,后来看的书多了才明白,这是个误会。套用史料上的话,是“边军无通言语,逢贼即杀”,意思是,辽东军听不懂西北方言,喊话也听不懂,所以见了就砍。\\

所以我一直认为,多学点语言,是会用得着的。\\

高迎祥就是吃了语言的亏,估计是屡次喊话没成,也没机会表达自己的诚意,所以被人穷追猛打了几个月,也没接上头。\\

在众多的民军中,高迎祥的部队,算是战斗力最强的,手下骑兵,每人两匹马,身穿重甲,也算是山寨版的关宁铁骑。虽说战斗力还是差点,但山寨版有山寨版的优势,比如……钻山沟。\\

高迎祥钻了郧阳山区,祖宽是不钻的,因为他的部队,大部都是骑兵,且待遇优厚,工资高,要让他们爬山,实在太过困难,卢象升协调了一个多月,也没办法。\\

照这个搞法,估计过几个月,闯王同志带着山寨铁骑出来闹腾,也就是个时间问题。\\

在这最为危急的时刻,更危急的事情发生了。\\

崇祯九年(1636)四月,当卢象升同志正在费尽口水劝人进山时,辽东的皇太极做出了一个重大的决定——建国。\\

皇太极建都于沈阳,定国号为清,定年号为崇德。\\

这一举动表明,皇太极同志正式单飞,另立分店,准备单干。\\

通常来讲,新店开张,隔壁左右都要送点花圈花篮之类的贺礼,很明显,明朝没有这个打算,也没这个预算。\\

不要紧,不送,就自己去抢。\\

崇祯九年(1636)六月,清军发起进攻。\\

这次进攻的规模很大,人数有十万人,统兵将领是当时清军第一猛将阿济格,此人擅长骑兵突击,非常勇猛。\\

难得的是,他不但勇猛,脑子也很好用,关宁防线他是不去碰的,此次进关,他选择的路线,是喜峰口。\\

此后的战斗没有悬念,明朝的主力部队,要么在关宁防线,要么在关内,所以阿济格的抢掠之旅相当顺利,连续突破明军防线,只用了半个月,就打到了顺义(今北京市顺义区)。\\

我认为,阿济格是个很能吃苦的人,具体表现为不怕跑路,不怕麻烦,到了北京城下,没敢进去,就开始围着北京跑圈,从顺义跑到了怀柔(今北京怀柔区),又从怀柔跑到了密云(今北京密云区),据说还去了趟西山(今北京西山),圆满完成了画圈任务。\\

当然,他也没白跑。据统计,此次率军入侵,共攻克城池十二座,抢掠人口数十万,金银不计其数。\\

鉴于明朝主力无法赶到,只能坚壁清野,所以阿济格在北京呆了很长时间,而且,他还是个很有点幽默感的人,据说他抢完走人时,还立了块牌子,上写四个字——各官免送!\\

我始终认为,王朝也好,帝国也罢,说穿了,就是个银行,这边收钱,那边付钱,总而言之,拆东墙,补西墙。\\

不补不行,几百年里,跑来拆墙的人实在太多,国家治不好,老百姓闹事,国防搞不好,强盗来闹事,折腾了这边,再去折腾那边,边拆边补,边补边拆。\\

但国家也好,银行也罢,都怕一件事——银行术语,叫做挤兑,政治术语,叫内忧外患,街头大妈术语,叫东墙西墙一起拆。\\

明朝大致就是这么个状况,客观地看,如果只有李自成、张献忠闹事,是能搞定的,如果只有清军入侵,也是能搞定的,偏偏这两边都闹,就搞不定了。\\

于是一个月后,卢象升得知了一个惊人的消息,他被调离前线,等待他的新岗位,是宣大总督。\\

对于这个任命,无数后人为之捶腿、顿足、吐唾沫,说什么眼看内患即将消停,卢象升却走了,以至于局势失去控制,崇祯昏庸等等等等。\\

在我看来,这个任命,无非是挖了东墙的砖,往西墙上补,不补不行,如此而已。\\

卢象升走了,两年后,他将在新的岗位上,完成人生最壮烈的一幕。\\

\subsection{接班}
听说卢象升离开的消息后,高迎祥非常高兴,因为他很清楚,像卢阎王这样的猛人,不是量产货,他擦亮眼睛,等待着下一个对手的出现。\\

他等来的接班人,叫做王家桢。\\

王家桢,直隶人,时任兵部侍郎,此人口才极佳,善读兵法,出谋划策,滔滔不绝。\\

行了,直说吧,这是个废柴。\\

他之所以被派来干这活,实在是因为嘴太贱,太喜欢谈兵法,太引人注目,最终得到了这份光荣的工作。\\

但王总督对自己的实力还是很明白的,刚到不久就上书皇帝,说自己身体比较弱,当五省总督太过勉为其难,干巡抚就成。\\

崇祯还是很体贴的,让他改行当了河南巡抚。\\

但王巡抚刚上任没几天,就遇上了一件千载难逢的倒霉事。\\

这件倒霉事,叫做兵变,兵变并不少见,之所以说是千载难逢,是因为参与兵变的,是王巡抚的家丁。\\

连家丁都兵变,实在难能可贵,连崇祯同志都哭笑不得,直接把他赶回家卖红薯。\\

有这样的好同志来当总督,高迎祥的好日子就此开张,没过多久,他就出了山区,先到河南,拉起了几万人的队伍,连战连胜,此后又转战陕西,气势逼人,洪承畴拿他都没办法。\\

四大猛人里,曹文诏死了,洪承畴没辙,左良玉固守,高迎祥最怕的卢象升,又去了辽东,现在而今眼目下,高闯王可谓天下无敌。\\

然后,第五位猛人出场了。\\

在这人出场前,高先生跟四大猛人打了近七年,越打越多,越打越风光,从几千打到几万、几十万,基本是没治了。当时朝廷上下一致认为,隔几天跟他打一仗,能让他消停会,就不错了。至于消灭他,大致是个梦想。\\

在这人出场后,梦想变成了现实。\\

他没有用七年,连七个月都没用。事实上,直到崇祯九年(1636)三月,他才出山,只用了四个月,就搞定了高迎祥。\\

在历代史料里,每到某王朝即将歇业的时候,经常看到这样一句话,XX死而X亡矣。\\

前面的XX,一般是指某猛人的名字,后面的X,是朝代的名字,这句话的意思是,某猛人,是某王朝最后的希望,某猛人死了,某王朝也就消停了。\\

在明代完形填空里,这句话全文如下:\\

传庭死,而明亡矣。\\

传庭者,孙传庭也。\\
\ifnum\theparacolNo=2
	\end{multicols}
\fi
\newpage
\section{孙传庭}
\ifnum\theparacolNo=2
	\begin{multicols}{\theparacolNo}
\fi
孙传庭是个相当奇怪的人,因为在杀死高迎祥之前,他从未带过兵,从未打过仗,过去三十多年里,他从事的主要工作,是人事干部。\\

孙传庭,字伯雅,山西代县人,万历四十七年进士。在崇祯九年之前,历任永城、商丘知县,吏部主事。\\

其实他的运气不错。我查了查,万历四十七年的进士,到天启初年,竟然就当上了吏部郎中,人事部正厅级干部,专管表彰奖励。\\

六部之中,吏部最大,而按照惯例,吏部尚书,一般都是从吏部郎中里挑选的。孙传庭万历二十一年(1593)出生,照这个算法,他当郎中的时候,还不到三十岁,年轻就是资本,照这个状态,就算从此不干,光是熬,都能熬到尚书。\\

然而没过两年,孙传庭退休了,提前三十年退休。\\

他丢弃了所有的前途和官位,毅然回到了家乡,因为他看不顺眼一个人——魏忠贤。\\

看不顺眼的人,很多,愿意辞官的,不多。\\

崇祯元年,魏忠贤被办挺了,无论在朝还是在野,包括当年给魏大人鞠躬、提鞋的人,都跳出来对准尸体踩几脚,骂几句,图个前程。\\

但孙传庭依然毫无动静,没有人来找他,他也不去找人,只是平静地在老家呆着,生活十分平静。\\

八年后,他打破了平静,主动前往京城,请求复职。\\

出发之前,他说出了自己复出的动机:\\

“待天下平定之日,即当返乡归隐。”\\

朝廷很够意思,这人没打招呼就跑了,也没点组织原则,十年之后又跑回来,依然让他官复原职,考虑到他原先老干人事工作,就让他回了吏部,接着搞人事考核。\\

对他而言,这份工作的意思,大致就是混吃等死,但他没有提出异议,平静地接受,然后,平静地等待。\\

一年后,机会出现了,在陕西。\\

当时的陕西巡抚,是个非常仁义的人,具体表现为每次在城墙上观战,都不睁眼。据他自己说,是不忍心看,但大多数人认为,他是不敢,这号人在和平时期,估计还能混混,这年头,就只能下岗。\\

巡抚这个职务,是个肥缺,平时想上任是要走后门的,但陕西巡抚,算是把脑袋别在裤腰带上混饭吃的,没准哪天就被张某某、高某某剁了,躲都没处躲,孙传庭就此光荣上任,因为主动申请的人,只有他一个。\\

孙传庭出发之前,皇帝召见了他。\\

对于孙巡抚的勇敢,崇祯非常欣赏,于是给了孙传庭六万两白银,作为军费。\\

除此之外,一无所有。\\

按崇祯的说法,国家比较困难,经费比较紧张,也就这么多了,你揣着走吧,省着点用。\\

当年杨鹤拿了崇祯十万两私房钱,招抚民军,也就用了几个月,孙传庭拿着六万两,也就打个水漂。\\

但人和人是不一样的。\\

自古以来,要人办事,就得给钱,如果没钱,也行,给政策。\\

孙传庭很干脆,他不要钱,只要政策,自己筹饷,自己干活,朝廷别管,反正干好了是你的,干不好我也跑不掉。\\

就这样,孙传庭拿着六万两白银,来到了陕西。\\

当时陕西本地的军队,战斗力很差,按照当时物价,六万两白银,大致只够一万人半年的军饷,最能打的将领,如曹变蛟(曹文诏的侄子)、左光先、祖宽,要么在洪承畴手下,要么跟着卢象升。总之,孙传庭算是个三无人员:无钱、无兵、无将。\\

但凡这种情况,若想咸鱼翻身,大都要经过卧薪尝胆、励精图治、艰苦奋斗、奋发图强等过程,至少也得个两三年,才闪亮登场,大破敌军。\\

孙传庭上任的准确时间,是崇祯九年(1636)三月,他全歼高迎祥的时间,是崇祯九年(1636)七月。\\

从开始,到结束,从一无所有,到所向披靡,我说过,四个月。\\

他到底是怎么完成的,到今天,也没想明白。\\
\ifnum\theparacolNo=2
	\end{multicols}
\fi
\newpage
\section{子午谷}
\ifnum\theparacolNo=2
	\begin{multicols}{\theparacolNo}
\fi
此时的高迎祥,已经来到陕西。\\

他之所以来陕西,是因为此时的陕西比较好混。\\

虽说洪承畴一直都在陕西,而他手下的洪兵也相当厉害,但他最近正在陕北对付另一位老冤家李自成。不知是李自成让他来帮忙,还是听说陕西巡抚比较软,高迎祥义无反顾地来了,单程。\\

自古以来,从下至上,要想进入陕西,必先经过汉中,所以当年刘备占据四川,要攻击曹操的长安,必占据汉中,此后诸葛亮六次北伐,都经过汉中出祁山作战。\\

高迎祥也不例外,但在进军汉中的路上,有一支队伍挡住了他。\\

率领这支队伍的,是孙传庭。\\

对于孙传庭,高迎祥并不熟悉,也不在乎,而且这支队伍只有万把人,似乎也不难打,他随即率领军队发起攻击,打了几次,损失上千人,没打动。\\

兵力占据优势,但多年的战斗经验告诉高迎祥,这是一支比较邪门的军队,不能再打了,他决定绕道。\\

他的直觉非常正确,那支镇守汉中,只有万把人的部队,在历史上,却有一个专门的称呼——秦军。\\

之前我说过,明末的军队,战斗力最强的,是关宁铁骑,排第三的,是天雄军,排在第二的,是秦军。\\

关宁铁骑强悍,因为机动,天雄军善战,因为团结,而秦兵的战斗力,因为个性。\\

我曾查阅明代兵部资料,惊奇地发现,秦兵的主力,大都来自同一个地方——陕西榆林。\\

榆林,是个非常奇特的地方,据说每次打仗的时候,压根不用动员,只要喊两嗓子,无论男女老幼,抄起家伙就上,而且说砍就砍,绝无废话。\\

因为这里只有士兵,没有平民。\\

榆林,明朝九边之一,自打朱元璋时起,就不怎么种地,传统职业就是当兵。平时街坊四邻聊天,说的也不是今年种了多少地,收了多少粮食,大都是打了哪些地方,砍了多少人头(按人头收费)。几百年下来,形成独特个性,具体表现为,进攻时,就算只有一个,都敢冲锋,撤退时,就算只剩一个,都不投降。\\

而且这里的人跟民军相当有缘分。听说民军来了,就算只是路过,都极其兴奋,冲出去就打,男女老幼齐上阵,估计是当兵的人多,什么张大叔李大伯,上次就死在民军手里,喊一嗓子,能动员一群亲戚,后来李自成攻打榆林,全城百姓包括大妈大爷在内,都没一个投降,就凭这个县,足足跟李自成死磕了八天,实在太过强悍。\\

孙传庭的兵,大致就是这些人。所以高迎祥没办法,是很正常的。\\

但高迎祥同志是要面子的,来都来了,还让我空手回去?无论如何,都要闯进去。\\

人有的时候,不能太执着。\\

执着的高迎祥经过深刻思考,多方查找,终于想到了一个方法。\\

他找到了一条隐蔽的小路,从这条小路,可以绕开汉中,直逼西安,只要计划成功,他就能一举攻克西安,占领陕西,大功告成。\\

一千多年前,有两个人在几乎相同的地方,陷入了相同的困局,他们都发现了这条路。一个人说,由此地进攻,必可大获全胜;另一个人说,若设伏于此,必定全军覆没!\\

没错,这两个人,一个叫诸葛亮,一个叫魏延,而他们发现的这条小路,叫做子午谷。\\

至于结局,地球人(看过《三国演义》〈演义上说的,别当真,看看就行〉的地球人)都知道,魏延想打,诸葛亮不让打,最后司马懿跳出来说,就知道你不敢打。\\

对于这个故事,许多人都说,诸葛亮过于谨慎,要按照魏延的搞法,早就打到长安了(魏延自己也这么说)。\\

而在高迎祥的故事里,只有魏延,没有诸葛亮。\\

所以一千年后,他在同样的地方,做出了不同的选择——出兵子午谷。\\

崇祯九年(1636)七月,高迎祥率领全部主力,冲入了子午谷,从这里,他将迅速到达西安。\\

但他不知道,这条路还通往另一个地点——地狱。\\

子午谷之所以是小路,是因为很小,对高迎祥而言,这句话绝对不是废话。\\

由于道路狭窄,而且天降大雨,他的几万大军,走了好几天,才走了一半,人困马乏,物资损失严重。\\

但高迎祥毫不沮丧,因为他相信,这个出乎许多人意料的举动,几天之后,必将震惊天下。\\

许多人确实没料到,但许多人里,并不包括孙传庭。\\

七月十六日,经过艰苦行军,高迎祥终于到达黑水峪,只要通过这里,前方就是坦途。\\

然后,满怀憧憬的高迎祥,看见了满怀愤怒的孙传庭。\\

愤怒是可以理解的,因为他已经在这里,等了十五天。\\

孙传庭的军事嗅觉极为敏锐,从高迎祥停止进攻的那一刻,他就意识到,这兄弟要玩花样了。\\

而他唯一可能的选择,只有子午谷。\\

所以在撤离汉中,在子午谷的黑水峪耐心等待,因为他知道,艰苦跋涉之后,出现在他面前的高迎祥,是十分脆弱的。\\

总攻随即开始,就人数对比而言,高迎祥的手下,大约在五万人以上,孙传庭兵力无法考证,估计在两万人左右,狭路相逢。\\

无论是高迎祥,还是孙传庭,都很清楚,玩命的时刻到了。\\

生命的最后时刻,高迎祥展现了他令人生畏的战斗力,虽然极为疲劳,但他依然率军发动多次突击,三次击破孙传庭的包围圈。\\

但他终归没能跑掉,原因很简单,这是一条小路。\\

在小路里打仗,就好比在胡同里打架,就算拿着青龙偃月刀,都没有板砖好使,而且道路太窄,没法跑开,所以他每次冲出去,没过多久,又被围住。\\

孙传庭的部队也着实厉害,抗击打能力极强,每次被冲垮,没过多久就又聚拢,充分发挥榆林的优良传统,作战到底,毫不退让。\\

以死相拼,死不后退,激战四天。\\

孙传庭取得了最后的胜利。\\

崇祯九年(1636)七月二十日,负伤的高迎祥在山洞中被俘,与他一同被俘的,还有他的心腹将领刘哲、黄龙,他的几万大军,已在此前彻底崩溃。\\

纵横世间七年的闯王高迎祥,就此结束了他的一生,在过去的七年中,他曾驰骋西北,扫荡中原,但终究未能成功。毫无疑问,他是一个了不起的人物,然而终究到此为止。\\

科学点的说法,是运气不好,迷信点的说法,这就是命。\\

高迎祥被捕的消息传到京城时,崇祯皇帝没信,不是不信,是不敢信,等人到了面前,才信。\\

处死高迎祥的那一刻,崇祯开始相信,自己能力挽狂澜。\\

\subsection{最后的帅才}
高迎祥被杀了,对崇祯而言,是利好消息,而对某些头领而言,似乎也不利空。\\

高迎祥死后,许多头领纷纷投降,比如蝎子块、冲破天等等,原先跟着高闯王干,闯王都没闯过去,自己也就消停了。\\

但有某些人,是比较高兴的,比如张献忠。\\

张献忠跟高迎祥似乎有点矛盾,原先曾跟着打凤阳,但后来分出去单干,也不在一个地界混,算是竞争关系,高迎祥死后,论兵力,他就是老大。\\

还有一个人,虽然很悲伤,却很实惠。\\

一直以来,李自成都跟着高迎祥干,高迎祥的外号,叫做闯王,而李自成,是闯将。据某些史料上说,李自成是高迎祥的外甥,这话估计不怎么靠谱,但关系很铁,那是肯定的。\\

高迎祥的死,给了李自成两样东西。\\

第一样是头衔,从此,闯王这个名字,只属于李自成。\\

第二样是兵力,高迎祥的残部,由他的部将率领,投奔了李自成。\\

在这个风云变幻的乱世,离去者,是上天抛弃的,留存者,是上天眷顾的。\\

对张献忠和李自成而言,他们的天下之路,才刚迈出第一步。\\

第一步,是个坑。\\

我说过,对民军头领而言,崇祯九年(1636)是个流年,卢象升来了,打得乱七八糟,好不容易跑进山区,人都调走了,又来了个孙传庭,还干掉了高迎祥。\\

按说坏事都到头了,可是事实告诉我们,所谓流年,是一流到底,绝不半流而废。\\

一个比孙传庭更可怕的对手,即将出现在他们的面前,与之前的洪承畴、曹文诏、卢象升不同,他并非一个能够上阵杀敌的将领。\\

他是统帅。\\

崇祯九年(1636),阿济格率领大军打进来时,崇祯非常紧张,但最紧张的人并不是他,而是张凤翼。\\

张凤翼,时任兵部尚书,他之所以紧张,是因为按惯例,如果京城(包括郊区)被袭,皇帝会不高兴,皇帝不高兴,就要拿人撒气,具体地说,就是他。\\

更要命的是,崇祯老板撒气的途径,是追究责任,具体地说,是杀人,比如七年前,皇太极打到京城,兵部尚书王洽就被干掉了,按照这个传统,他是跑不掉的。\\

但张部长还算识相,眼看局面没法收拾,就打了个报告,说清军入侵,是我的责任,我想戴罪立功,到前方去,希望批准。\\

崇祯当即同意,打发他去了前线。\\

但张尚书到前线后,似乎也没去拼命,每天只干一件事——吃药。\\

他吃的,是毒药。\\

这是一种比较特别的毒药,吃了不会马上死,必须坚持吃,每天吃,饭前饭后吃,锲而不舍地吃,才能吃死。\\

对于张尚书的举动,我曾十分疑惑,想死解腰带就行了,实在不行操把菜刀,费那么大劲干甚?\\

过了好几年,才想明白,高,水平真高。\\

如果自杀,按当时的状况,算是畏罪,死了没准抚恤金都没有,但要上阵杀敌,似乎又没那个胆,索性慢性自杀,就当自然死亡了,还算是牺牲在工作岗位上,该享受的待遇,一点不少,老狐狸。\\

这兄弟不但死得慢,算得也准,清军九月初退兵,他九月初就死,连一天都没耽误。\\

他死了,也就拉倒了,可是崇祯同志不能拉倒,必须继续招工。\\

但榜样在前面,岗位风险太高,说了半天,也没人肯干。\\

左右为难之际,崇祯想到了一个人。\\

这个人很孝顺,曾三次上书,请求让自己代替父亲受罚,那是在他决心处罚杨鹤的时候。\\

他还清楚地记得这个人的名字——杨嗣昌。\\

杨嗣昌,字文弱,湖广武陵人,万历三十八年进士。\\

崇祯见到杨嗣昌时,很忧虑。\\

局势实在太差,民军闹得太凶,清军打得太狠,两头夹攻,东一榔头西一棒,实在难于应付,如此下去,亡国是迟早的事,怎么办?\\

杨嗣昌只说了一句,一句就够了:\\

“大明若亡,必亡于流贼!”\\

如果你仔细想想,就会发现这句话实在准得离谱。\\

按照杨嗣昌的说法,清军或许很强,但短时间内,并没有太大威胁,但如果不尽快解决民军,大明必定崩溃。\\

简单地说,就是先解决内部矛盾,再解决外部矛盾。\\

为了实现这个意图,杨嗣昌还提出了一个计划,这个计划在历史上的名字,是八个字:四正六隅,十面张网。\\

四正,包括湖广、河南、陕西、凤阳,六隅,是指山东、山西、应天、江西、四川、延绥。简单地说,这个优秀计划的大致内容,是一部垃圾电影的名字——十面埋伏。\\

它的大致意思是,全国范围内,设置十个战区,四个主要,六个次要,只要发现民军出现,各地将联合围剿。简而言之,就是划定管辖范围,在谁的地方出事,就让谁去管,出事的主管,没出事的协管。\\

听完杨嗣昌的计划,崇祯只说了一句话:\\

“我用你太晚了!”\\

对于这句话,朝廷的许多大臣都认为,是彻彻底底的胡扯,无论是杨嗣昌,还是他的那个什么十面埋伏,都是空口白说,毫无价值,在他们看来,杨嗣昌同志将是第三个被干掉的兵部尚书。\\

然而他们错了,如果说在当时的世界上,还有一个人能够拯救危局,那么这个人,只能是杨嗣昌。\\

两年后,只剩十八个人的李自成,和束手投降的张献忠,可以充分说明这一点。\\

所有的转变,都从这一刻开始,魏忠贤、清军入侵、民变四起,朝廷争斗,紧张,痛苦,毫无生机,但始终未曾放弃。\\

或许崇祯本人并不知道,经过长达八年暗无天日的努力,他即将迎来大明的曙光。\\

放他去!\\

崇祯死前,曾说过这样一句话:诸臣误我!\\

对于这句话,大多数人认为,是在推卸责任。\\

但考证完崇祯年间的朝政,我认为,这句话比较正确。确切地说,给崇祯打工的这帮大臣,除部分人外,大多数可以分为两种,一种叫混蛋,一种叫混帐。\\

这个世界上,有两种人最痛苦,第一种是身居高位者,第二种是身居底层者,第一种人很少,第二种人很多。第一种人叫崇祯,第二种人叫百姓。\\

而最幸福的,就是中间那拨人,主要工作,叫做欺上瞒下,具体特点是,除了好事,什么都办,除了脸,什么都要。\\

崇祯每天打交道的,就是这拨人,比如崇祯三年(1630)西北灾荒,派下去十万石粮食赈灾,从京城出发的时候,就只剩下五万,到地方,还剩两万,分到下面,只剩一万,实际领到的,是五千。\\

这事估计是办得太恶心了,崇祯也知道了,极为愤怒,亲自查办。\\

案情查明:先动手的,是户部官员,东西领下来,不管好坏,先拦腰切一刀,然后到了地方,巡抚先来一下,知府后来一下,剩下的都发到乡绅手里,美其名曰代发,代着代着就代没了。\\

结合该案,综合明代史料,崇祯时期的官员,比较符合如下规律:脸皮的厚度,跟级别职务,大致成反比例增长。\\

这是比较合理的,位高权重的,几十年下来,有身份,也要面子,具体办事的就不同了,树不要皮,必死无疑,人不要脸,天下无敌,好欺负的,就往死了欺负,能捞钱的,就往死了捞,啥名节、脸面,都顾不上,捞点实惠才是最实在的,正如马克思所说,资本的积累,那是血淋淋的。\\

而且这拨人,还有个特点,什么青史留名、国家社稷,那都太遥远了,跟他们讲道理,促膝谈心都是没用的,用今天的话说,就是吃硬不吃软。教育没有用的,骂也没有用,往脸上吐唾沫都没用,相对而言,比较合适的方式是,把唾沫吐到眼里,再说上一句:孙子,我能治你!\\

比如当年追查阉党,就那么几个人,研究来研究去,连亲手干掉杨涟的许显纯,都研究成过失杀人,撤职了事,还是崇祯亲自上阵,才把这人干掉。\\

再比如这事,案发后,崇祯非常生气,下令严查,查到户部,户部研究半天,拉出来几个人,说是失职,给撤了,准备结案。\\

崇祯生气了,重装上阵,找出来几个主犯,杀了,剩下的,充军。\\

总之,崇祯年间的朝廷,是比较混账的,而带头混账的,是温体仁。\\

温体仁这个人,历史上的评价不高:奸臣,彻头彻尾的奸臣。\\

我之前说过,温体仁是个很有能力的人,精明强干,博闻强记,善于处理政务。\\

所以综合起来,温体仁先生的最终评价应该是,一个很有能力、精明强干、博闻强记、善于处理政务的彻头彻尾的奸臣。\\

温体仁,是个很复杂的坏人,复杂在无论你怎么看,都会发现,这是个真正的好人。\\

在工作中,温体仁是个很勤劳的人。据史料记载,他兢兢业业,每天从早干到晚,很能工作,别人几年干不了的事,他几天就能搞定。\\

在生活中,他是廉政典范。据说他当首辅时,给他送礼的人从门口排到街上,等几天,他一个都不见,所有的礼品都退,退不了的就扔,比海瑞还海瑞。\\

在处理与同事间的关系上,他非常谦虚,从不说别人坏话,而且很能听取他人意见。比如有个叫文震孟的人,是他的晚辈,刚入内阁,他却非常尊敬,遇事都要找来商量,一点架子没有。\\

综上所诉,温体仁同志在过去的几年里,在工作上、生活上严格要求自己,团结同事,评定应为优秀。\\

那么接下来,我们就温体仁同志的评定问题,进行鉴定:\\

在工作中,他反应敏捷,很有能力,但历史告诉我们,要成为一个青史留名的坏人,没有能力,是不行的。\\

在生活上,他严格要求自己,不受贿赂,是因为他的仇人太多,要被人抓住把柄,是很麻烦的。\\

在跟同事相处时,他确实很和善,比如对文震孟,相当地客气,但原因在于,文震孟是崇祯的老师,后台很硬,而且当时他正在挖坑,等着文老师跳下去。\\

如果纵观温体仁的经历,可以发现,他有个历史悠久的习惯——整人。\\

崇祯二年(1629),他跟周延儒合谋,整垮了钱龙锡,进了内阁,过了几年,他又整垮了周延儒,当了首辅,又过了两年,他整垮了前途远大的文震孟,维护了自己的地位。\\

而且他整人的方式相当地高明。比如文震孟有个亲信,因为犯了事,要受处分。顺便说句,这人的事比较大,按情节,至少也是撤职。\\

文震孟和皇帝关系好,名声很好,势力很大,且刚进内阁,对温体仁而言,是头号眼中钉,但面对如此难得的整人机会,他毅然放弃了,非但没有落井下石,反而帮忙找了人,只给了个降职处分,很够意思,文震孟很感激。\\

大坑就是这样挖成的。\\

温体仁很清楚,崇祯是个眼睛不揉沙子的人,处分官员,是只有更重,没有最重,如果从轻处理,皇帝大人是不会答应的,肯定会加重,而文震孟同志比较正直,脾气也大,肯定要跟皇帝死磕,下场是比较明显的。\\

事情如他所料,皇帝大人听说后,非常震怒,把那人直接撤职,赶回家种田了,而文震孟不愧硬汉本色,跟皇帝吵了好几天,加上温体仁煽风点火,竟然也被免了。\\

其实这些倒无所谓,在道上混的,整个把人,搞点阴谋,也没什么,这种事,当年张居正也没少办,之所以是奸臣,是因为他不办事。\\

崇祯登基以来,干过很多事,平乱、抗金、整顿,忙完这边又忙那边,而温体仁上台以来,就只干一件事——个人进步。\\

为了个人进步,他很精明,坑了钱龙锡,坑了周延儒,坑了文震孟,坑了所有的挡路或可能挡路的人。\\

为了个人进步,他除了精明外,有时还很傻——装傻。\\

有一次,崇祯把他找来,有件事情要问他的看法,温体仁当即回答:不知道。\\

崇祯随即追问,为何不知道。\\

温体仁回答:臣本愚笨(原话),只望皇上圣裁。\\

为了个人进步,他很团结同志,很合群,为了整倒钱龙锡,他拉拢了周延儒,两人齐心合力,还把钱谦益同志送回了家。\\

当然,为了个人进步,他有时也不合群,很孤独,比如他对老朋友周延儒下手时,很干脆,没有丝毫犹豫,整人太多,多年家里鬼都不上门,还经常跟崇祯说,我不结党,所以孤独。\\

明明很阴险,很狡猾,很恶心人,还动不动就说我很耿直,我很愚蠢,很能促进食欲。\\

能人,兼职奸人,最奸的能人,是奸人,最能的奸人,还是奸人。\\

鉴定完毕。\\

在当时朝廷里,只要混过几年的,大致都知道温体仁同志的本性,换句话说,都知道他是个奸人。\\

可是知道没用,因为温体仁先生是个能干的奸人,而且深得皇帝信任,谁都告不倒,时人有云:崇祯遭瘟(温)。而且他本人心黑手狠兼皮厚,在朝廷混了多年,就快修炼成妖了,实在无人可比。\\

俗语有云,占着茅坑,不拉屎。客观地说,在内阁大臣的位置上,温体仁的行为并不符合这句话,确切地说,他占着茅坑,只拉屎。\\

外敌入侵,内乱不止,诚此危急存亡之秋,温体仁同志孜孜不倦,为了自己而奋斗,整人、挖坑,忙得不亦乐乎,如果让他继续折腾,大明可以提前关门。\\

但不知是气数未尽,还是坟里的朱重八发威,天下无敌的温体仁,终究还是等来了敌人——一个他曾战胜过的敌人。\\

自打辩论会上掉进温体仁的大坑,被赶回家,钱谦益已经在家呆了八年。八年里,除了看人种地(他是地主),主要的娱乐,就是写诗。\\

这些诗大都收入他的文集,可以找来看看,心理效果明显,心情好时看,可以抑郁,心情不好时看,可以去自杀。\\

诗的主要意思,基本比较雷同,什么我很后悔,我要归隐,我白活了,我没意思,反正一句话,我这一辈子,是走了黑道。\\

毕竟家里蹲了七八年,有点怨气很是正常,但钱谦益同志还是说错了,他走的黑道,还没有黑到头。\\

崇祯十年(1637),在家看人种地的钱谦益突然听说,有一个叫张汉儒的当地师爷,写了份状子告他。\\

要知道,钱大人虽说在上面混得很差,但到地方,还是比较恶霸的,小小师爷闹事,容易摆平。\\

然而没过几天,他就迎来了几位从京城来的客人——几位来抓他的客人。\\

在被押解的路上,钱谦益才搞明白,原来那位师爷的状子,是告御状。\\

这个世上,但凡有人的地方,就有斗争,但凡斗争,就有谱,包括政治斗争。一般说来,把对手弄到偏远山区,回家养老,也就够本了,没必要赶尽杀绝,但这事,也因人而异,比如温体仁,就是个没谱的人。\\

要么是他太过得意,或者太恨钱谦益,总之他没打算按着谱走,某天突然心血来潮,想起在那遥远的江南,还有个没被整死的钱谦益。\\

没整死,就往死里整。\\

但他毕竟位高权重,如果要自己动手,传出去实在太丢面子,而且容易留下把柄,所以他决定,借刀杀人。\\

他借到的刀,就是张汉儒。\\

之所以找到张汉儒,因为这人是个衙门师爷,小人物,无论如何,跟内阁首辅,都是扯不上关系的,而且张师爷长期在法律界工作,对拍黑砖之类的工作非常熟悉,且乐此不疲。\\

果然,接到工作指示后,张师爷连夜工作,写出了一份状子。\\

所谓小人物,在写状子这点上,是不恰当的。当年大人物杨涟告魏忠贤,总共二十四条大罪,而张师爷告钱谦益的罪状,有五十八条。\\

这五十八条罪状,堪称经典之作,包括贪污、受贿、走私、通敌、玩权、结党,总而言之,只要你能想到的罪状,他都写了。\\

但钱谦益倒没怎么慌,因为这份状子写得实在太过扯淡,都赶回家当老百姓了,还贪污个甚?玩权、掌控朝政,基本就是胡话,崇祯这么精明的人,是不会信的。\\

可是他到北京,就真慌了,因为他在朝廷的朋友告诉他,他的罪状,皇帝已经批了,即将定罪。\\

其实钱谦益同志应该有点思想准备,要明白,温体仁是首辅,所有的公文,都是他票拟的,底下送上来,他签个字,皇帝都未必看,要收拾你小子,小菜。\\

钱谦益不愧是当过东林党领导的,虽然回家消停几年,威望依然很大,他被抓过来,很多人出面,什么给事中、郎中、尚书,包括大学士,都帮他说话,说他很冤枉,情节很曲折。\\

全无作用,皇帝知道了,也没理。\\

因为温体仁要的,就是这个效果。\\

八年前,兵强马壮的钱谦益,没能干过势单力孤的温体仁,是因为温体仁同志精通心理学。\\

他很清楚,说话人再多都没用,说了能算的只有崇祯,而崇祯最讨厌的事情,就是拉帮结派,帮忙的人越多,就越坏事。都八年了,钱大人还没明白这个道理,实在毫无长进。\\

所以外面越是起哄,皇帝就越不买账,钱谦益同志的脑袋,就离鬼头刀越来越近。\\

温体仁已做好庆祝准备,等待着钱谦益被杀的那一天。\\

对此,钱谦益颇有共识,他虽在牢里,消息很灵通,感觉事情不太对劲,就亲自写了几封信,托人直接交给皇帝,为自己辩解。\\

但结果很不幸,皇帝大人压根没看,很明显,他对钱谦益同志,是比较厌恶的。\\

钱谦益终于走到了绝路,帮忙没用,辩解没用,找皇帝都没用,找什么人似乎都没用了。\\

等着他的,只有喀嚓一刀。\\

有句俗语:万事留一线,将来好见面。这句俗语,用比较通俗的话说,就是没必要逼人太甚。\\

被逼得太甚的钱谦益,在阴暗的牢房里,终于使出了杀手锏。\\

关于钱谦益同志,之前介绍的时候,漏了一点,这位仁兄除了是东林党的头头外,还有个关系——他中进士的时候,录取他的老师,叫做孙承宗。\\

孙承宗同志,大家都很熟悉了,很有本事,除了能打仗外,也能搞关系,魏忠贤在的时候,都拿他没办法。\\

但问题是,孙承宗已经退休好几年了,说话也不好使,让他出面,估计也很麻烦。\\

钱谦益并没有幻想,他所以找到孙承宗,只是希望孙老师帮他找另一个人,这个人的名字叫做曹化淳。\\

曹化淳,是知名人士,我依稀记得,在金庸的小说《碧血剑》里,他是个死跑龙套的,且跑过好几回。\\

但在崇祯十年的时候,他是司礼监秉笔太监,崇祯的亲信。\\

在当时,能跟温体仁较劲的,也就只有他了。\\

但问题是,这位太监同志跟温体仁无仇,钱谦益也并非他的亲戚,犯不上较这个劲。\\

但钱谦益认定,这个人,能帮他的忙,救他的命。\\

凭什么呢?\\

就凭十年前,他曾经写过一篇文章。\\

其实这篇文章,跟曹化淳并没有丝毫关系,但钱谦益相信,看着这篇文章的份上,曹化淳是会帮忙的。\\

因为这篇文章是王安的墓志铭。\\

我讲过,很久以前,魏忠贤是王安的亲信,但我没有讲过,当时王安的亲信,还有一个曹化淳。\\

这似乎是个比较复杂的关系。大致是这么回事。\\

钱谦益去找曹化淳帮忙,因为他曾经帮王安写过墓志铭,而曹化淳是王安的亲信,所以看在死人的面子上,多少要帮点忙。外加他的老师孙承宗,面子比较广,托他出面,还有点活人的面子,死人活人双管齐下,务必成功。\\

成功了。\\

曹化淳得知消息,非常吃惊,加上这人跟着王安,还有点良心,感觉温体仁太过分,就答应帮个忙。\\

当然,找完了人还得听消息,钱谦益找了个人,天天去朝廷找人打听情况,连续找了三天,都没人理会,毫无消息,第四天,他得到准确的口信:可安心矣。\\

可安心矣的意思,就是这事已经搞定,收拾行李,准备出狱。\\

钱谦益也是这么理解的,他相信曹化淳已经解决了一切。\\

曹化淳原本也这么认为,他上下活动,估计再过几天,事情就结了。\\

可是偏就没有结。\\

因为温体仁又来了。\\

温首辅以为钱谦益必死,没想到过了几天,竟然连曹化淳都折腾进来了,这样下去,事情就黄了,既然干了,就干到底,所以他决定,连曹化淳一起整。\\

他先散布消息,说钱谦益跟曹化淳合伙,然后还找了个证人,让他出面,指证钱谦益给曹化淳行贿,最后为万无一失,他还请了假。\\

每次但凡要整人时,温体仁就会请假,回家呆着,这意思是在我请假期间,发生的任何事情,我既不知道,也不在场,事完了,拍拍屁股再去上班。\\

其实对温体仁而言,钱谦益死,还是不死,都没多大关系,反正就政治地位而言,钱地主已经是个死龙套。\\

可做可不做的好事,最好做,可做可不做的坏事,最好不做。可惜,温体仁同志没有这个觉悟。\\

在他看来,钱谦益已经是个平民,而袒护钱谦益的曹化淳,不过是个司礼太监,作为内阁首辅,要办这两个人,是很容易的。\\

可惜他不知道,曹化淳这个人的复杂程度,远远超出他的想象。\\

因为曹化淳非但是太监,还有特务背景,他原本在东厂干过,到司礼监后,跟现任东厂提督太监王之心是哥们,关系很铁。\\

而今温大人竟拿他开刀,实在是搞错了码头。曹公公勃然大怒,立刻跑到东厂,找到王之心,商量对策,毕竟温体仁老奸巨猾,无懈可击,要彻底搞倒他,必须想个办法。\\

商量半天,办法有了。\\

先去找皇帝,主动报告此事,说事情很复杂,后果很严重,于是皇帝大人也震惊了,下令严查,事情闹大了。\\

接下来,就是去抓人,温体仁是没法抓的,但张汉儒一干人等,随便抓,抓回来,就直接丢进东厂。\\

据说东厂的刑罚,总共有上百种,花样繁多,能够让人恨自己生出来,比什么测谎仪好用多了,所以但凡丢进这里的人,都很诚实。\\

张汉儒之流,似乎也不是什么钢铁战士。按史料的说法,进来的头天晚上,曹化淳去审了一次,就审出来了,除了交代本人作案情况外,连幕后主使温体仁先生的诸多言行,也一起交代了。\\

曹化淳拿到口供,立马就奔了崇祯,崇祯看过之后,沉默了很久,然后,他说了四个字:\\

“体仁有党!”\\

这四个字的意思,用江湖术语解释:温体仁,是有门派的。\\

崇祯是不喜欢门派的,作为武林盟主,任何门派他都不喜欢,像温体仁这种人见狗嫌的家伙,虽然讨厌,但用着放心。\\

然而这件事清楚地告诉他,温体仁同志也有门派,虽然门派比较小,但再小都是门派。\\

然后,他拿来了一封奏疏。\\

这封奏疏是温体仁的辞职信,按照他的传统,为了彻底表示自己的清白,他写了这封文书,说自己身体不好,估计也帮不了皇帝了,希望让自己回家养老。\\

类似这种客气信件,崇祯也会客气客气,写几句挽留的话,然后该怎么干还怎么干。\\

然而这一次,在这封奏疏上,他只写了三个字。\\

奏疏送到温体仁家时,他正在吃饭,他停了下来,等待着以往听过许多次的客套话。\\

然而这一次,他只听到了三个字——放他去。\\

放他去的意思,大致有以下几种:滚、快滚、从哪里来,滚哪里去。\\

据说他当时就晕了过去。\\

温体仁终于倒了,这位聪明绝顶的仁兄,从顶上摔了下来,他落寞地回了家,第二年,死在家乡。\\

明代最后的一位权奸,就此落幕,确实,最后一个。\\
\ifnum\theparacolNo=2
	\end{multicols}
\fi
\newpage
\section{天才的计划}
\ifnum\theparacolNo=2
	\begin{multicols}{\theparacolNo}
\fi
温体仁下台,最受益的人,应该是杨嗣昌。我查了一下,他崇祯十年(1637)三月当兵部尚书,温体仁是六月走人的,按照温先生的脾气,像杨嗣昌这种牛人,不踩下去,是不大可能的。\\

温体仁走了,杨嗣昌来了,不久之后,他就将进入内阁,实践自己天才的计划。\\

按照杨嗣昌的计划,要实现十面张网,现在的人是不够的,必须再增兵十二万。\\

要增兵,就得给钱,按杨嗣昌的算法,必须增加饷银二百八十万两以上。\\

这个计划极为冒险,因为这笔钱杨嗣昌是不出的,崇祯也是不出的,唯一的来源,只能是找老百姓要,具体说来,就是加租。\\

比如原先你一年交一百多斤粮食,全家还能丰衣足食,张献忠、李自成打过来的时候,你可能会出门看热闹,然后回家吃饭。然后官府告诉你,加租,每年交两百斤,结果全家只能吃糠,再打过来的时候,你就会出门,帮李自成叫声好,让他们往死里打,帮你出口气。\\

再后来,官府告诉你,再加租,每年交四百斤,结果全家连糠都没法吃,不用人家打上门,你就会打好包袱,出门去找闯王同志。\\

为了搞定西北民变,崇祯已经加过几次租了,如果再加,后果不堪设想。所以很多大臣坚决反对。\\

但是崇祯仍旧同意了,因为他相信,杨嗣昌的计划,能够挽救危局。\\

最后,杨嗣昌说,要实现这个计划,我必须用一个人。\\

崇祯同意了。\\

杨嗣昌推举的这个人,叫熊文灿。\\

熊文灿,贵州永宁卫人。万历三十五年进士。历任礼部主事、布政使、两广总督。\\

杨嗣昌之所以推举熊文灿,只是因为一个误会。\\

不久前,两广总督的熊文灿得知了这样一个消息,崇祯的一名亲信太监来到广东探访,干啥不知道,虽说来意不明,但对这种特派员之类的人物,熊总督心里是有数的,专程请过来吃饭。\\

既然是吃饭,就要喝酒,吃饱喝足,再送点礼,这位太监也很上道,非常高兴,一来二去,也就熟了。\\

既然是熟人,也就好说话了,双方无话不谈,从国内形势到国际风云,什么都说,但只有一件事,熊总督始终没有套出来。\\

你到底来干什么的?\\

几天后,这位太监要走,熊总督决定再请他吃顿饭,最后套口风。\\

这顿饭吃得很满意,双方临别,喝得也多,喝着喝着,就开始说起民变的事。\\

熊总督估计是喝多了,外加豪气干云,当时拍着桌子大喝一声:\\

“诸臣误国,如果我去,怎么会让他们闹到如此地步(令鼠辈至是哉)!”\\

他万没想到,有个人比他还激动。\\

太监立即站了起来,他流露出多年卧底终于找到同志的表情,热烈地握住了熊总督的手,说出了熊总督套了很多天,都没有套出来的话:\\

“我到这里来,就是来考察你的!回去我就禀报皇上,让您去平乱,除了你,谁还能扫清流贼(非公不足办此贼)!”\\

酒醒了。\\

熊总督到底是多年的老官僚,听到这话,当时酒就醒了,脑筋急速运转后,凭借二十余年的功底,立即提出了五难,四不可。\\

所谓五难,四不可,大致就是九个条件,也就是说,只有满足了这些条件,熊总督才能勉为其难地上任。\\

大致说来,这就是一篇公文,就算让专职秘书写,也得写个一天两天,熊总督转眼就能完工,实在用心良苦。\\

然而太监也并非凡人,只用一句话,就打碎了熊总督的如意算盘:\\

“你放心,这些我回去都会禀报皇帝,但如果皇帝都答应,你就别推辞了。”\\

就这样,熊总督的一片报国之心穿越上千里路,来到了京城。\\

崇祯知道了,杨嗣昌也知道了,在那遥远的南方,有一个叫熊文灿的忠义之士,愿意为国付出一切。\\

当然了,熊总督的那些条件,自然不在话下,关键时刻,有人肯上,就难能可贵了,怎么能够吝惜条件呢?\\

所以在这关键时刻,杨嗣昌提出了熊文灿,而崇祯也欣然同意了,他们都相信,他能圆满实现这个天才的计划。\\

于是,远在千里之外的熊总督接到了调令,他即将前往中原,接替无能的前任总督王家桢。\\

熊文灿原先的辖区,是广东、广西两个省,而他现在的辖区,包括河南、山西、陕西、湖广、四川五省,按说,他应该很高兴,高兴得一头撞死。\\

两广总督,虽说管的都是不发达地区,盗贼也多,但好歹图个平安,也没人来闹,现在这五个省,动辄就是几十万人武装大游行,且都是巨寇、猛寇,没准哪天就被抓走,实在比较刺激。\\

但既然来了,再跟皇帝说,其实我是忽悠您的,那天是喝多了,估计也不行,想来想去,只能硬着头皮上了。\\

后世有很多人,对熊先生相当不屑,说他没有能力,没有气魄,但在我看来,熊总督并没有那么不堪。他自幼读书,当过地方官,也到过京城,还出过海(出使琉球),见过大世面,总体而言,他只有两样东西不会——这也不会,那也不会。\\

虽说熊文灿能力比较差,比较怕事,比较没有打过仗,但他能够升到两广总督,竟然是靠一项军功。\\

这项军功的具体内容是,他搞定了一个许多人都无法搞定的人,此人的名字,叫郑芝龙。\\

郑芝龙,是福建一带的著名海盗,有个著名的儿子——郑成功。\\

熊总督招降郑芝龙后,又用郑芝龙干掉了其他海盗,成功搞定福建沿海,最终搞定自己,获得提升。\\

但熊总督长年以来的表现有目共睹,骗得了上级,骗不了群众,所以他去上任的时候,许多人都认定,熊总督是壮官一去不复返了。\\

崇祯十年(1637)十月,熊文灿正式来到湖广上任,迎接他的,是下属左良玉。\\

刚开始的时候,左良玉对熊总督还比较客气(没摸清底细),过了几天,发现熊总督黔“熊”技穷,除了天天开会,啥本事都没有,索性就消失了。没办法,像熊总督这种熊人,左总兵是看不上的。\\

熊总督也急了,他本不想来,来了,将领又不听使唤,自己手下的兵力,加起来还不到一万人,又要完成业绩,无奈之下,只好使出老招数——招抚。\\

当时在他的辖区里,最大的两股民军,分别是张献忠和刘国能。其中张献忠有九万人,刘国能有五万。\\

熊文灿决定招抚这两个人。\\

虽然在朝廷混得还行,但论江湖经验,跟张献忠、刘国能比,熊总督还是很傻很天真,他不知道这二位的投降史,也不了解黑道的规矩,更何况,他的兵还不到人家的十分之一,要想招降,是很困难的。\\

但熊总督最头疼的问题,还不是上面这些,他首先要解决的,是另一个问题——发通知。\\

因为张献忠和刘国能从事特殊行业,平时也没住在村里,以熊总督的情报系统,要找到这两个人,似乎很难,情急之下,为了表示自己的诚意,熊总督派了几百个人,以今日张贴医治性病广告之决心,在村头乡尾四处贴告示,以告知朝廷招安之诚意。\\

对此,左良玉嗤之以鼻,连杨嗣昌听说后,也只能苦笑。\\

总之,在当时,熊总督在大家的眼里,大约是个笑话,笑完了,就该滚蛋了。\\

然而这个笑话,却以一种无人可以预料的方式,继续了下去。\\

过了不久,熊总督就得到消息,民军的同志们找来了。\\

先找上门的是张献忠,他表示,自己虽然兵强马壮,但是很想投降,很想为国效力,但鉴于投降程序很麻烦,所以需要准备几天。\\

这是鬼话。\\

类似这种话,张献忠说的次数,估计他自己都数不清,这也是张头领看熊总督是生人,专程忽悠一把,要换了洪承畴、卢象升等一干熟人,拉出去就剁了。\\

但张献忠派人上门,除了逗人玩,还有客观原因。\\

自打崇祯九年围剿风暴以来,经济形势是一天不如一天,高迎祥垮台了,众多头领环境都不好,随时可能破产裁员,包括李自成在内。\\

高迎祥死后,孙传庭就放出了话,只要搞定了李自成,他就退休回家。\\

李自成在陕北对付洪承畴,已经很吃力了,又来了这么个冤家,两下夹攻,连吃败仗,没办法,陕西没法呆了,只好掉头进了四川。\\

偏偏年景太差,又赶上杨嗣昌开始搞十面埋伏工程,只能接着往前跑,前有追兵,后有堵截,实在没办法,只能以掩耳盗铃之势窝在原地,动弹不得。\\

环境如此,张献忠混得也差,留个后路是必要的,所以找到了熊总督,当然,投降是不会的,先谈条件,过几年实在不行了,再投降。\\

但他万没想到,过几天,他就会乖乖投降。\\

因为几天后,一个消息传来,刘国能投降了。\\

刘国能,外号闯塌天,在当时的诸位头领中,他大概能排到前五名,是个相当棘手的人物。\\

他得知熊总督招降的消息后,也找上门来,表示自己虽兵强马壮,但是很想投降,鉴于投降程序很麻烦,需要准备几天。\\

其实刘国能同志的台词,跟张献忠的差不多,不同之处在于,他准备了几天,就真的投降了。\\

崇祯十年(1637)十一月,刘国能率五六万大军,向仅有一万人的熊文灿投降,服从改编。\\

小时候,我读《水浒传》的时候,曾经相当厌恶宋江,觉得他替天行道,开始造反,很是英雄,最后却又接受招安,去征讨方腊,很是狗熊,同样的一个人,怎么前后差别那么大呢?\\

后来我才明白,造反的宋江,和招安的宋江,始终是同一个人。\\

为什么要造反?\\

造反,就是为了招安。\\

当年的宋江,原本是给政府干活,而且还有职务,根据水浒的说法,日子过得很不错,除了拿工资,还勾结黑社会(如晁盖等人),吃点外快,还经常结交江湖兄弟,给钱从不小气(宋江:你当及时雨的名号是白给的?),只是一时手快,在被检举之前,干掉了自己的小妾,所以才被迫流落造反。\\

刘国能的情况比较类似,跟张献忠不一样,他原本是读过书的,据说还有个秀才的功名,但后来不知是一时冲动,还是懵懂无知,竟然造了反,好在运气不错,这么多年没被干掉,还混得不错。\\

但造反这活,混得不错是不够的,毕竟工作不太稳定,危险性大,刘国能又是个比较孝顺的人,希望在家孝敬父母,所以趁此机会,准备投降,换个工作。\\

刘国能这一投降,就把张献忠吓懵了:投降,还有抢生意的?\\

眼看问题严重,他立即派出使者,去找熊文灿,表示近期就投降了。\\

但是熊总督也硬气了,没有盛情挽留,反而表示,关于投降的问题,还要研究研究,才确定是否接受。\\

原本投降是供不应求,现在成了供大于求,卖方市场变成买方市场,麻烦了。\\

但张献忠不愧是在朝廷里混过的,非常机灵,立刻转变思路,决定,送礼。\\

而且张献忠明智地意识到,熊总督的道行很深(两广总督是个肥差),单是送钱估计没戏,所以他专程找了几件古董玉器(反正是抢来的),派人送了过去,只求一件事,让我投降。\\

捞钱之余还有政绩,如此好事,对熊总督而言,不干就不是人。他马上接受了投降,并且命令张献忠等人就地安置。\\

张献忠投降的时候,手下有七八万人,而他的驻地,在谷城(今湖北谷城)。\\

消息传来,崇祯极为高兴,认定熊总督是旷世奇才,大加赞赏。\\

杨嗣昌也很高兴,高兴之余,他提出了一个想法。\\

客观地讲,这是个比较阴险的想法,以致于后来很多人认为,如果照这个想法办了,天下就消停了。\\

这个想法的具体内容,是让张献忠在投降之前,办一件事——打李自成。\\

这就好比黑帮团伙,每逢拉人入伙的时候,都要让新人干点缺德事,比如砍人放火之类,专用术语,叫沾点血,今后才好一起干。\\

但崇祯同志实在很讲道义,他表示人家刚来投降,就让人干这种事,似乎有点过分,所以也就这么算了。\\

对崇祯的信任,谷城的张献忠先生如果毫无感动,那也是很正常的。作为投降专业户,他所要考虑的,是什么时候再造反,以及造反之后,什么时候再投降。\\

实际情况,似乎也是如此,崇祯十一年(1638)十月,张献忠同志已经难能可贵地投降了十个月,很明显,他也不打算打破自己以往的投降记录,开始私下联系,蠢蠢欲动。而以熊总督的觉悟,估计只有张先生的砍刀砍到他的枕头上,才能反应过来。\\

然而,就在以往场景即将重播之际,一个消息,彻底地打乱了张献忠的计划。\\

三个月前,陕西的李自成呆不下去,跑到了四川,刚到四川时,李自成过得还可以,后来洪承畴调集重兵围剿,他就退往山区,双方僵持不下,李自成瞅了个空,又跑回了陕西。\\

以往每次李自成跑路的时候,洪承畴都礼送出境,送出去就行,确保他别回来,并不多送,但这次李自成发现,洪承畴开始讲礼貌了。\\

李自成从四川出来的时候,屁股后面跟着一群送行的人,比如关宁军的主要将领祖大弼、左光先以及曹文诏的侄子曹变蛟等等。\\

而且这帮人很有诚意,一直跟在后面,且玩命地打,比如曹变蛟,带着三千骑兵,跟了二十多天,连衣服都没换(未卸甲),连续击败李自成,直接把人赶出了陕西。\\

洪承畴之所以如此卖力,是因为挨了骂。按照防区划定,陕西归孙传庭管,四川归洪承畴管,照孙传庭的想法,李自成进了四川,就别让他再出去了,可是洪承畴不知怎么回事,竟然又让李自成跑了。\\

孙传庭自然不干,认定是洪承畴玩花样,让自己背黑锅,气得不行,就告了一状。\\

这一状相当狠,崇祯极为愤怒,马上就批了个处分,那意思是,你想干就好好干,不想干我就干你,搞得洪承畴连觉都没法睡,连夜开会,准备跟李自成玩真的。\\

对方突然下猛招,李自成没有思想准备,连陕西都没呆住,只能往外跑了。\\

一路往西北跑,跑了几天几夜,到了甘肃,终于没人追了。\\

但过了几天,李自成才明白,不追是有理由的。\\

在明代,西北是比较荒凉的,陕西的情况还凑合,再往外跑,基本就没人了,所以压根没必要追,让他自己饿死就行。\\

洪承畴的想法大致如此,事情也正如他所料,李自成混得实在太惨,没人、没粮,一个多月,损失竟然过半,已经穷途末路。\\

然而出乎洪承畴意料的是,没过几天,李自成竟然穿越严密封锁,又回来了——从他的眼皮底下。\\

据说这件事情吓得洪大人几天没睡着觉,毕竟刚刚作过检查,还出这么大的事,随即写信,向崇祯请罪。\\

但崇祯的领导水平实在是高,一句话都没说,只是让他戴罪立功。\\

感动得眼泪汪汪的洪大人决心,用行动来报答领导的信任,马上找到孙传庭,要跟他通力合作,彻底解决李自成。\\

孙传庭很够意思,啥也不说了,立即调兵,发动了总攻。在一个月里,跟李自成打了四仗。\\

四仗之后,李自成只剩一千人。\\

只剩一千人的李自成,躲进了汉中的深山老林。\\

原本几万精锐手下,被打得只剩一个零头,甚至连他最可靠的亲信祁总管,也带着人当了叛徒,在山沟里受冻的李自成,感到了刺骨的寒意。\\

如果是张献忠,到这个时候,估计早就投降洗了睡了,但李自成依然不投降,他依然坚定。\\

但再坚定,都要解决问题,李自成明白,老呆在山里,终究是不行的,必须走出去。\\

经过分析,他正确地认识到,四川是不能去了,陕西也不能去了,要想有所成就,唯一的目的地,是河南。\\

河南有人口,有灾荒,加上还有几个从前的老战友,所以,这是李自成最好的,也是唯一的选择。\\

而从汉中到河南,必须经过南原。\\

南原,位于潼关附近,是此去必经之路,为了交通安全,李自成在出发前,进行了长期侦查,摸清地形,为了麻痹敌人,他在山区蹲了一个多月,直到所有官军撤走,才正式上路。\\

一路上,李自成相当机灵,数次避过官军,但终究有惊无险地到了南原。\\

南原是他的最后一站,只要通过这里,他的命运就将彻底改变。\\

一个月前,当李自成只剩一千余人,躲进山里的时候,孙传庭认为,这是歼灭李自成的最好时机,必须立刻进山围剿,至少也要围困。\\

然而洪承畴反对,他认为既不要围剿,也不用围困。\\

孙传庭很愤怒,他判定,李自成必定会再次出山,而且他的进攻方向,一定是河南。\\

这一次,洪承畴没有反对,他说,确实如此。\\

既然确实如此,为什么不全力围剿呢?\\

因为最好的围剿地点,是潼关南原,无论他从何处出发,那里是他的必经之路。\\

所以当李自成全军进入南原之后,他才发现,自己落入了陷阱。\\

据史料记载,为了伏击李自成,孙传庭集结了三万以上的兵力,每隔数十里,就埋伏一群人,山沟、丛林,只要能塞人的地方,都塞满。\\

如此架势,别说突围,就算是挤,估计都挤不出去。\\

所以从战斗一开始,就毫无悬念,蜂拥而上的明军开始猛攻,挨了闷棍后,李自成开始突围,往附近的山里跑,然而跑进去才发现,明军比他进来得还早,于是又往外跑,跑了一天,没能跑出去。\\

李自成部余下的一千多人,是他的精锐亲军,九年来,南征北战,无论是四川、陕西、钻山沟,绕树林,都坚定不移地跟着走。\\

到了南原,就再也走不动了。\\

虽然经过拼死厮杀,终究没能突围出去,从白天打到晚上,一千个人,只剩下了十八个。\\

李自成也是十八个人之一,他趁着夜色,率领部将刘宗敏,逃出了包围圈,他的手下全军覆没,老婆孩子全部被俘。\\

在一片黑暗中,孤独的李自成逃入了商洛山,在那里,他将开始艰难的等待。\\

至此,西北民变基本平息,几位著名头领,基本都被按平,要么灭了,要么投降,没灭也没降的,似乎也很悲哀,毕竟连被没灭的价值都没有,是很郁闷的。\\

张献忠老实了,现在经济形势这么差,工作不好找,如果再去造反,吃饭都成问题,所以他收回了自己的再就业计划,开始踏踏实实当个地主。(谷城基本归他管)\\

消停了。\\

民变基本平息,朝廷基本安定,要走的走了,要杀的杀了,要招安的也招安了,经过长达十年的混乱,大明终于等来了曙光。\\

对目前的情况,崇祯很高兴,他忙活了十年,终于得到了喘息的机会,他曾对大臣说,再用十年,必将社稷兴盛,天下太平。\\

十年?\\

一年都没有。\\

看到光明的崇祯并不知道,他看到的,并不是曙光,而是回光,回光返照。\\

\subsection{其实问题很简单}
几乎就在李自成全军覆没的同时,一件事情的发生,再次改变了大明帝国的命运。\\

崇祯十一年(1638),皇太极决定,进攻明朝,清军兵分两路,多尔衮率左翼军,岳托率右翼军,越过长城,发动猛攻。\\

应该说,为了这次进攻,皇太极是很费心思的,他不去打关宁防线(也是实在打不过来),居然绕了个大圈,跑到了密云。\\

密云的守军很少,但几乎没人认为,清军会从这里进攻,因为这里山多,且险,要从这里过来,要爬很多山,而且很难爬,要爬很久。从这里打进来,那是绝无可能。\\

据说经常卖假古董的人,最喜欢听到的话,就是某位很懂行的顾客,很自信地表示,古董的某某特征,是绝对仿不出来的。\\

皇太极有没有卖过古董,那是无从考证,但他选择的地方,就是这里,他的战术非常简单,就是爬山。\\

清军到这里后,开始爬山,确实很多山,很难爬,足足爬了三天。\\

但终究是爬过来了。\\

清军爬过来的时候,蓟辽总督吴阿衡正在喝酒,且喝大了,脑袋比较晕,清军都到密云了,他才明白过来。\\

人喝醉之后,有两个后果,一、头疼,二、胆子大。\\

这两个后果,吴总督都有,最终后果是,头疼的吴总督,胆大无比,带着几千人,就奔着清军去了。\\

喝醉的人,要是一打一,仗着抗击打能力,还有点胜算,但要是群殴,也就只能被殴,没过多长时间,吴总督就被殴死了,清军突破长城防线,全线入侵,形势万分危急。\\

密云距离北京,今天坐车,如果没堵车,大致是两个钟头,当年骑马,如果没堵马,估计也就一两天。\\

离京城一两天,也就是离崇祯一两天,所以消息传到京城,大家都很恐慌,只有几个人不慌,其中之一,就是崇祯。\\

崇祯之所以不慌,是因为六个月前,他就知道清军会入侵,而且连入侵的时间,他都知道得一清二楚。\\

六个月前,有一个人将攻击的时间,方式都告诉了他,这个人并非间谍,也不是卧底,他的名字,叫皇太极。\\

半年前的一天,杨嗣昌曾在私下场合对崇祯说了一个故事,这个故事比较长,所以千言万语化为一句话:\\

在东汉,开国皇帝汉光武帝刘秀,跟匈奴议和了。\\

这个故事的意思很明白,就是让崇祯去跟清朝和谈。\\

客观地讲,这是唯一的方法。\\

就军事实力而言,当时的清朝虽然军队人数不多(最大兵力二十万),但战斗力相当强(某些西方军事学家跟着凑热闹,说是十七世纪最强的骑兵),明朝的军队人数,大致在六十万到八十万左右,但能打仗的(辽东系、洪兵、秦兵),也就是二十多万,要真拉开了打,估计也不太行。\\

好在地形靠谱,守着几个山口,清军也打不过来,所以按照常理,是能够维持的。\\

但要命的是后院起火,出了李自成等一干猛人,只能整天拆东墙补西墙,所以杨嗣昌建议,跟清朝和谈,先解决内部矛盾。\\

其实杨嗣昌的故事,还有下半段:刘秀跟匈奴和谈,搞定内部后,没过多少年,就派汉军出塞,把匈奴打得落荒而逃。\\

所谓秋后算账,虽然杨嗣昌没讲,但崇祯明白,所以他决定,先忍一口气,跟清朝和谈,搞定国内问题先。\\

当时知道这件事情的,只有三个人,包括崇祯、杨嗣昌、太监高起潜。\\

为保证万无一失,和谈使者是不能派的,杨嗣昌不知去哪里寻摸来个算命的,跑到皇太极那边,说要谈判。\\

皇太极的态度相当好,说愿意和谈,而且表示,如果和谈成功,就马上率军撤回原地。\\

当然,这位老兄一向不白给,末了还说了一句,如果和谈不成功,我就打过去,具体时间,是在今年的秋天。\\

崇祯愿意和谈,因为这是没有办法的办法。\\

过了几个月,在他的暗中指使下,杨嗣昌正式提出,建议与清朝和谈。\\

此后的事情,打死他都想不到。\\

建议提出后,按史料的说法,赞成的人很少,反对的人很多,事实上,是只有人反对,没有人赞成。\\

最先蹦出来的,是六部的几个官员,骂了杨嗣昌,然后是一拨言官,说杨嗣昌卖国,应该拉出去千刀万剐,全家死光光。\\

但把这件事最终搅黄的,是最后出场的人,一个人——黄道周。\\

黄道周同志的简历,我就不多说了,这位仁兄后来有个外号,叫“黄圣人”,后来跟清军死战到底,堪称名副其实。\\

黄圣人当着皇帝的面,直接跟杨嗣昌搞辩论,一通天理人欲,先把杨嗣昌说晕,然后发挥特长(他的专业是理学),从理论角度证明,杨嗣昌主张议和,是天理难容,违背人伦等等。\\

说了半天,杨嗣昌基本没有还手之力,崇祯虽然气不过,但黄先生理论基础太扎实,也没办法,等辩论完了,也不宣布结果,当场就下了令,黄道周连降六级,到外地去搞地方建设。\\

皇帝大人虽然出了气,但和谈是绝不可能了,杨嗣昌再也没提,大家都能等,皇太极例外,他在关外等了几个月,眼看没了消息,认定是被忽悠,就又打了进来。\\

对当时的崇祯而言,和谈是最好的出路,其实问题很简单,当年汉高祖如此英雄,还得往匈奴送人和亲,皇太极从来没要过人,无非是要点钱,弄点干货,也就完事了。\\

但如此简单的问题,之所以搞得这么复杂,如此多人反对,其实只是因为一件东西——心态。\\

\subsection{天朝上邦}
我曾研习过交通史(中外交往),惊奇地发现,国家与国家之间的关系,和人其实差不多,穷了,就瞧不起你,打你,富了,就给你面子,听话。\\

比如美国,说谁是流氓谁就是流氓,说打谁就打谁,盟友遍布天下,时不时还搞个会盟,弄个盟军,朋友遍天下,全世界人民都羡慕。\\

但这事你要真信了,那就傻了,要知道,那都是拿钱砸出来的,听话,就是友好邻邦,就给美元,给援助,很人道,不听话,就是流氓国家,给导弹,很暴力。\\

而且山姆大叔是真有钱,导弹那是贵,一百万美元一个,照扔,一扔就几十个,心眼太实在,我估摸着,要全换成手榴弹,从飞机上往下扔,也能扔个把月。\\

归根结底,就是两个字,实力。\\

谁有实力,谁就是大爷,没实力,就是孙子,美国有实力,其实也就一百多年,趁着英国老大爷跟德国老大爷干仗,奋发图强,终成超级大爷。\\

相比而言,中国当大爷的时间,实在是比较长,自打汉朝起,基本就是世界先进国家,虽然中途闹腾过,后来唐朝时又起来了,也是全世界人民羡慕,往死了派留学生,相对而言,欧洲除了罗马帝国挺得比较久,大部分时间,都是一帮封建社会的职业文盲砍来砍去,直到明朝中期,都是世界领先。\\

鉴于时间太久,心态难免有点问题,比如后来英国工业革命,开始当大爷了,就派使者到中国,见到乾隆。本意大致是要跟中国通商。\\

然后,乾隆同志对他们说,回去给你们乔治(当时的英国国王)带个信,就说你的孝心我知道了,你的贡品我收到了(战舰模型),我天朝应有尽有,你就不要再费心了,给我送这些东西,是比较耽误事的,你们那里是蛮荒之地,生活很困难,好好种地,我这里东西很多,赏点给你,回家好好用吧。\\

几十年后,在蛮荒之地种地的英国农民们,驾驶着战舰打了进来。\\

这种毛病由来已久,毕竟牛了太多年,近的朝鲜、越南、日本且不说,最远的,能打到中亚、西伯利亚,自古以来,就是天朝上邦,四方来拜,外国使臣来访,表面上好吃好喝招待着,临走还捎堆东西,说天朝物产丰富,什么都有,只管拿,背地里说人家是蛮夷,没文化,落后,看你可怜,给你几个赏钱。\\

牛的时候,怎么干都行,等到不牛了,还想怎么干都行,那就不行了。\\

明朝官员的思维,大致就是如此,就军事实力而言,谈判是最好的选择,然而没有人选择。\\

这种行为,说得好听点,叫坚持原则,说得不好听,叫不识时务,明朝最后妥协的机会,就这样被一群不识时务的人拒绝了。\\

十年前,我读到这里的时候,曾经很讨厌黄道周,讨厌这个固执、不识时务的人,我始终认为,他的决策是完全错误的。\\

直到我知道了黄道周的结局。\\

七年后,当清军入关时,在家赋闲的黄道周再次出山,辅佐唐王。\\

唐王的地盘,大致在福建一带,他是个比较有追求的人,很想打回老家,可惜他有个不太有追求的下属——郑芝龙。\\

郑芝龙的打算,是混,无论清朝明朝,自己混好就行,唐王打算北伐,郑芝龙说你想去就去,反正我不去。\\

唐王所有的兵力,都在郑芝龙的手里,所以说了一年多,只打雷没下雨。\\

这时黄道周站出来,他说,战亦亡,不战亦亡,与其坐而待毙,何如出关迎敌。\\

唐王很高兴,说你去北伐吧,然后他说,我没有兵给你。\\

黄道周说,不用,我自己招兵。\\

然后他回到了家,找到了老乡、同学、学生,招来了一千多人,大部分人都是百姓。\\

隆武元年(唐王年号,1645),黄道周出师北伐,他的军队没有经验,从未上过战场,甚至没有武器,他们拥有的最大杀伤力武器,叫做锄头、扁担。所以这支军队在历史上的名字,叫做“扁担军”。\\

黄道周的妻子随同出征,她召集了许多妇女,一同前往作战,这支部队连扁担都没有,史称“夫人军”。\\

就算是最白痴的白痴,也能明白,这是自寻死路。\\

然而黄道周坚定地向前进发,明知必死无疑。正如当年他拒绝和谈,绝不妥协。\\

三个月后,他在江西婺源遭遇清军,打了这支队伍的第一仗,也是最后一仗。\\

结果毫无悬念,武器的批判没能代替批判的武器,黄道周全军覆没。黄道周被俘,被送到了南京,无数人轮番出面劝他投降,他严辞拒绝。\\

三个月后,他在南京就义,死后衣中留有血书,内容共十六字:\\

\begin{quote}
	\begin{spacing}{0.5}  %行間距倍率
		\textit{{\footnotesize
				\begin{description}
					\item[\textcolor{Gray}{\FA }] 纲常万古,节义千秋,天地知我,家人无忧。
				\end{description}
		}}
	\end{spacing}
\end{quote}

落款:\\

\begin{quote}
	\begin{spacing}{0.5}  %行間距倍率
		\textit{{\footnotesize
				\begin{description}
					\item[\textcolor{Gray}{\FA }] 大明孤臣黄道周
				\end{description}
		}}
	\end{spacing}
\end{quote}

正如当年的他,不识时务,绝不妥协。\\

有人曾对我说,文明的灭绝是正常的,因为麻烦太多,天灾人祸、内斗外斗,所以四大文明灭了三个,只有中国文明流传至今,实在太不容易。\\

我想想,似乎确实如此,往近了说,从鸦片战争起,全世界强国(连不强的都来凑热闹)欺负我们,连打带抢带烧带杀,还摊上个“量中华之物力”配合人家乱搞的慈禧,打是打不过,搞发展搞不了(洋务),同化也同不了(人家也有文明),软不行,硬也不行,识时务的看法,是亡定了。\\

然而我们终究没有亡,挺过英法联军,挺过甲午战争,挺过八国联军,挺过抗日,终究没有亡。\\

因为总有那么一群不识时务的人,无论时局形势如何,无论敌人有多强大,无论希望多么渺茫,坚持,绝不妥协。\\

所以我想说的是,当年的这场辩论,或许决定了大明的未来,或许黄道周并不明智,或许妥协能够挽回危局,但不妥协的人,应该得到尊重。\\

面对冷酷的世间、无奈的场景,遇事妥协,不坚持到底,是大多数人、大多数时间的选择,因为妥协,退让很现实,很有好处。\\

但我认为,在人的一生中,至少有那么一两件事,应该不妥协,至少一两件。因为不妥协、坚持虽然不现实,很没好处,却是正确的。\\

人,是要有一点精神的,至少有一点。\\
\ifnum\theparacolNo=2
	\end{multicols}
\fi
\newpage
\section{卢象升的选择}
\ifnum\theparacolNo=2
	\begin{multicols}{\theparacolNo}
\fi
明朝的道路就此确定,不妥协,不退让。\\

相应的结果也很确定,皇太极带着兵,再次攻入关内,开始抢掠。\\

这次入关的,可谓豪华阵容,清朝最能打的几个,包括阿济格、多尔衮、多铎、岳托,全都来了,只用三天,就打到密云,京城再度戒严。\\

要对付猛人,只能靠猛人,崇祯随即调祖大寿进京,同时,他还命令陕西的孙传庭、山东的刘泽清进京拉兄弟一把,总之,最能打仗的人,他基本都调来了。\\

但问题在于,祖大寿、孙传庭这类人,虽然能力很强,但有个问题——不大服管。特别是祖大寿,自从袁崇焕死后,他基本上就算是脱离了组织,谁当总督,都不敢管他,当然,他也不服管。\\

对这种无组织、无纪律的行为,崇祯很愤怒,后果不严重,毕竟能打的就这几个,你要把他办了,自己提着长矛上阵?\\

但不管终究是不行的,崇祯决定,找一个人,当前敌总指挥。\\

这个人必须有能力强,战功多,威望高,威到祖大寿等猛人服气,且就在京城附近,说用就能用。\\

满足以上条件的唯一答案,是卢象升。\\

崇祯十一年(1638),卢象升到京城赴任。\\

他赶到京城,本来想马上找皇帝报到,然而同僚打量他后,问:你想干嘛?\\

之所以有此一问,是因为这位仁兄来的时候,父亲刚刚去世,尚在奔丧,所以没穿制服,披麻戴孝,还穿着草鞋。如果这身行头进宫,皇帝坐正中间,他跪下磕头,旁边站一堆人,实在太像灵堂。\\

换了身衣服,见到了崇祯,崇祯问,现在而今,怎么办?\\

卢象升看了看旁边的两个人,只说了一句话:主战!\\

站在他身边的这两人,分别是杨嗣昌、高起潜。\\

这个举动的意思是,知道你们玩猫腻,就这么着!\\

据说当时杨嗣昌的脸都气白了。\\

崇祯倒很机灵,马上出来打圆场,说和谈的事,那都是谣传,是路边社,压根没事。\\

卢象升说,那好,我即刻上阵。\\

第二天,卢象升赴前线就任,就在这一天,他收到了崇祯送来的战马、武器。\\

其实崇祯送来这些东西,只是看他远道而来,意思意思。\\

然而卢象升感动了,他说,以死报国!\\

就如同九年前,没有命令,无人知晓,他依然率军保卫京城。\\

他始终是个单纯的人!\\

几天后,卢象升得知,清军已经逼近通州,威胁京城。\\

当时他的手下,只有三万多人,大致是清军的一半,而且此次出战的,都是清军主力,要真死磕,估计是要休息的,所以大多数识时务的明军将领都很消停,能不动就不动。\\

然而卢象升不识时务,他分析形势后,决心出战。\\

卢象升虽然单纯,但不蠢,他明白,要打,白天是干不动的,只能晚上摸黑去,夜袭。\\

在那个漆黑的夜晚,士兵出发前,他下达了一条名垂青史的军令:\\

\begin{quote}
	\begin{spacing}{0.5}  %行間距倍率
		\textit{{\footnotesize
				\begin{description}
					\item[\textcolor{Gray}{\FA }] 刀必见血!人必带伤!马必喘汗!违者斩!
				\end{description}
		}}
	\end{spacing}
\end{quote}

趁着夜色,卢象升向着清军营帐,发起了进攻。\\

进攻非常顺利,清军果然没有提防,损失惨重,正当战况顺利进行之时,卢象升突然发现了一个严重的问题。\\

他的后军没有了。\\

按照约定,前军进攻之后,后军应尽快跟上,然而他等了很久,也没有看到后军,虽然现在还能打,但毕竟是趁人不备,打了一闷棍,等人家醒过来,就不好办了,无奈之下,只能率前军撤退。\\

卢象升决定夜袭时,高起潜就在现场。\\

作为监军太监,高起潜并没有表示强烈反对,他只是说,路途遥远,很难成功,卢象升坚持,他也就不说了。\\

但这人不但人阴(太监),人品也阴,暗地里调走了卢象升的部队,搞得卢总督白忙活半天。\\

差点把命搭上的卢象升气急败坏,知道是高起潜搞事,极为愤怒,立马去找了杨嗣昌。\\

这个举动充分说明,卢总督虽然单纯,脑袋还很好使,他知道高起潜是皇帝身边的太监,且文化低,没法讲道理,要讲理,只能找杨嗣昌。\\

在杨嗣昌看来,卢象升是个死脑筋,没开窍,所以见面的时候,他就给卢象升上了堂思想教育课,告诉他,议和是权宜之计,是伟大的,是光荣的。\\

卢象升只说了一句话,就让杨嗣昌闭上了嘴。\\

这句话也告诉我们,单纯的卢象升,有时似乎也不单纯。\\

“我手领尚方宝剑,身负重任,如果议和,当年袁崇焕的命运,就要轮到我的头上!”\\

袁崇焕这辈子最失败的地方,就是不讲政治,相比而言,卢象升很有进步。\\

九年前,他在北京城下,亲眼看到了袁崇焕的下场。那一幕,在他的心里,种下了难以磨灭的印象。他很清楚,如果议和,再被朝里那帮言官扯几句,汉奸叛徒的罪名,绝对是没个跑。\\

与其死在刑场,不如死在战场,他下定了决心。\\

杨嗣昌也急了,当即大喝一声:\\

你要这么说,就用尚方宝剑杀我!\\

卢象升毫不示弱:\\

要杀也是杀我,关你何事?如今,只求拼死报国!\\

杨嗣昌沉默了,他明白,这是卢象升的最后选择。\\

卢象升想报国,但比较恶搞的是,崇祯不让。\\

事实上,卢象升对形势的分析是很准确的,因为夜袭失败,朝廷里那帮吃饱了没事干的言官正准备弹劾他,汉奸、内奸之类的说法也开始流传,如果他同意和谈,估计早就被拉出去一刀了。\\

更麻烦的是,崇祯也生气了,因为卢象升上任以来,清军依然嚣张,多处城池被攻陷,打算换个人用用。\\

此时,一位名叫刘宇亮的人站了出来,说,我去。\\

刘宇亮,时任内阁首辅,朝廷重臣,国难如此,实在看不下去,极为激动,所以站了出来。\\

崇祯非常高兴,大大地夸奖了刘大人几句。\\

等皇帝大人高兴完了,刘大人终于说出了话的下半句:我去,阅兵。\\

崇祯感觉很抑郁,好不容易站出来,搞得这么激动,竟然是涮我玩的?\\

其实这也不怪刘首辅,毕竟他从没打过仗,偶尔激动,以身报国,激动完了,回家睡觉,误会而已。\\

但崇祯生气了,生气的结果就是,他决定让刘首辅激动到底,一定要他去督师。\\

关键时刻,杨嗣昌出面了。\\

杨嗣昌之所以出头,并非是他跟刘首辅有什么交情,实在是刘首辅太差,太没水平,让这号人去带兵,他自己死了倒没啥,可惜了兵。\\

所以他向皇帝建议,刘首辅就让他回去吧。目前在京城里,能当督师的,只有一个人。\\

崇祯知道这个人是谁,但他不想用。\\

杨嗣昌坚持,这是唯一人选。\\

崇祯最终同意了。\\

三天后,卢象升再次上任。\\

此时,清军的气势已经达到顶点,接连攻克城池,形势非常危急。\\

然而卢象升没有行动,他依然按兵未动。\\

因为此时他的手下,只有五千人,杨嗣昌讲道理,高起潜却不讲,阴人阴到底,调走了大部主力,留下的只有这些人。\\

打,只能是死路一条,卢象升很犹豫。\\

就在这时,他得知了一个消息——高阳失陷了。\\

高阳,位处直隶(今河北),是个小县城,没兵,也没钱,然而这个县城的失陷,却震惊了所有的人。\\

因为有个退休干部,就住在县城里,他的名字叫孙承宗。\\

他培养出了袁崇焕,构建了关宁防线,阻挡了清军几十年,熬得努尔哈赤(包括皇太极)都挂了,也没能啃动。无论怎么看,都够意思了。\\

心血、才华、战略、人才,这位举世无双的天才,已经奉献了所有的一切,然而,他终将把报国之誓言,进行到人生的最后时刻。\\

清军进攻的时候,孙承宗七十六岁,城内并没有守军,也没有将领,更没有粮草,弹丸之地,不堪一击。\\

很明显,清军知道谁住在这里,所以他们并没有进攻,派出使者,耐心劝降,做对方的思想工作,对于这位超级牛人,可谓是给足了面子。\\

而孙承宗的态度,是这样的,清军到来的当天,他就带着全家二十多口人,上了城墙,开始坚守。\\

在其感召之下,城中数千百姓,无一人逃亡,准备迎敌。\\

每次看到这里,我都会想起黄道周,想起后来的卢象升,想起这帮顽固不化的人,正如电影集结号里,在得知战友战死的消息后,男主角叹息一声的那句台词:\\

\begin{quote}
	\begin{spacing}{0.5}  %行間距倍率
		\textit{{\footnotesize
				\begin{description}
					\item[\textcolor{Gray}{\FA }] 老八区教导队出来的,有一个算一个,都他妈死心眼。
				\end{description}
		}}
	\end{spacing}
\end{quote}

黄道周和孙承宗应该不是教导队出来的,但确实是死心眼。\\

这种死心眼,在历史中的专用称谓,叫做——气节。\\

失望的清军发动了进攻,在坚守几天后,高阳失守,孙承宗被俘。\\

对于这位俘虏,清军给予了很高的礼遇,希望他能投降,当然,他们自己也知道,这基本上是不可能的。\\

所以在被拒绝之后,他们毫无意外,只是开始商量,该如何处置此人。\\

按照寻常的规矩,应该是推出去杀掉,成全对方的忠义,比如文天祥等等,都是这么办的。\\

然而清军对于这位折磨了他们几十年的老对手,似乎崇拜到了极点,所以他们决定,给予他自尽的权利。\\

孙承宗接受了敌人的敬意,他整顿衣着,向北方叩头,然后,自尽而死。\\

这就是气节。\\

消息很快流传开来,举国悲痛。\\

崇祯十一年(1638)十二月二十日,听说此事的卢象升,终于下定了决心。\\

此前,他曾多次下令,希望高起潜部向他靠拢,合兵与清军作战,但高起潜毫不理会。而从杨嗣昌那里,他得知,自己将无法再得到任何支援。他的粮草已极度缺乏,兵力仅有五千,几近弹尽粮绝。\\

而清军的主力,就在他的驻地前方,兵力是他的十倍,锋芒正锐。\\

弄清眼前形势的卢象升,走出了大营。\\

和孙承宗一样,他向着北方,行叩拜礼。\\

然后,他召集所有的部下,对他们说了这样一番话:\\

\begin{quote}
	\begin{spacing}{0.5}  %行間距倍率
		\textit{{\footnotesize
				\begin{description}
					\item[\textcolor{Gray}{\FA }] 我作战多年,身经几十战,无一败绩,今日弹尽粮绝,敌众我寡,而我决心已定,明日出战,愿战者随,愿走者留,但求以死报国,不求生还!
				\end{description}
		}}
	\end{spacing}
\end{quote}

十二月二十一日,卢象升率五千人,向前进发,所部皆从,无一人留守。\\

出发的时候,卢象升身穿孝服,这意味着,他没有打算活着回来。\\

前进至巨鹿时,遭遇清军主力部队,作战开始。\\

清军的人数,至今尚不清楚,根据史料推断,至少在三万以上,包围了卢象升部。\\

面对强敌,卢象升毫无畏惧。他列阵迎敌,与清军展开死战,双方从早上,一直打到下午,战况极为惨烈,卢象升率部反复冲击,左冲右突,清军损失极大。\\

在这天临近夜晚的时候,卢象升明白,败局已定了。\\

他的火炮、箭矢已经全部用尽,所部人马所剩无几。\\

但他依然挥舞马刀,继续战斗,为了他最后的选择。\\

然后,清朝官员编写的史料告诉我们,他非常顽强,他身中四箭、三刀,依然奋战。他也很勇敢,自己一人,杀死了几十名清兵。\\

但他还是死了,负伤力竭而死,尽忠报国而死。\\

相信很多人并不知道,卢象升虽然位高权重,却很年轻,死时,才刚满四十岁。\\

他死的时候,身边的一名亲兵为了保住他的尸首,伏在了他的身上,身中二十四箭而死。\\

他所部数千人,除极少数外,全部战死。\\

我再重复一遍,这就是气节。\\

在明末的诸位将领中,卢象升是个很特殊的人,他虽率军于乱世,却不扰民、不贪污,廉洁自律,坚持原则,从不妥协。\\

中庸有云:\\

\begin{quote}
	\begin{spacing}{0.5}  %行間距倍率
		\textit{{\footnotesize
				\begin{description}
					\item[\textcolor{Gray}{\FA }] 国有道,不变塞焉,国无道,至死不变。
				\end{description}
		}}
	\end{spacing}
\end{quote}

无论这个世界多么混乱,坚持自己的信念。\\

我钦佩这样的人。\\

\subsection{幽默}
记得不久前,我去央视对话节目做访谈,台下有问观众站起来,说,之前一直喜欢看你的书,但最近却发现了个问题。\\

什么问题?\\

之前喜欢看,是因为你写的历史很幽默,很乐观,但最近发现你越来越不对劲,怎么会越来越惨呢?\\

是啊,说句心里话,我也没想到会这样,应该改变一下,这么写,比如崇祯没有杀袁崇焕,皇太极继位的时候,心脏病突发死了,接班的多尔衮也没蹦几天,就被孝庄干掉了,然后孤儿寡母在辽东过上了安定的生活。李自成进入山林后,没过几天,由于水土不服,也都过去了。\\

然后,伟大的大明朝终于千秋万代,崇祯和他的子孙们从此过着幸福的生活。\\

是的,现在我要告诉你的是,历史的真相。\\

历史从来就不幽默,也不乐观,而且在目前可知的范围内,都没有什么大团圆结局。\\

所谓历史,就是过去的事,它的残酷之处在于,无论你哀嚎、悲伤、痛苦、流泪、落寞、追悔,它都无法改变。\\

它不是观点,也不是议题,它是事实,既成事实,拉到医院急救都没办法的事实。\\

我感觉自己还是个比较实诚的人,所以在结局即将到来之前,我想,我应该跟您交个底,客观地讲,无论什么朝代的史书,包括明朝在内,都不会让你觉得轻松愉快,一直以来,幽默的并不是历史,只是我而已。\\

虽然结局未必愉快,历史的讲述终将继续,正如历史本身那样,但本着为人民服务的精神,我将延续特长,接着幽默下去,不保证你不难受,至少高兴点。\\

\subsection{忽悠}
正如以往,清军没有长期驻守的打算,抢了东西就跑了,回去怎么分不知道,但被抢的明朝,那就惨了。\\

首先是将领,卢象升战死,孙传庭、洪承畴全都到了辽东,准备防守清军,我说过,这是拆了东墙补西墙,没办法,不拆房子就塌了。\\

其次是兵力,能打仗的兵,无论是洪兵,还是秦兵,都调到辽东了。\\

所以最后的结果是,东墙补上了,西墙塌了。\\

说起忽悠这个词,近几年极为流行,有一次我跟人聊天,说起这个词,突然想起若有一天,此词冲出东北,走向世界,用英文该怎么解释,随即有人发言,应该是cheat(欺骗)。\\

我想了一下,觉得似乎对,但不应该这么简单,毕竟如此传神的词,应该有一个传神的翻译,苦思冥想之后,我找到了一个比较恰当的翻译:here and there。\\

回想过去十几年,自打学习英语以来,我曾翻译过不下两篇英语文章,虽然字数较少(三百字左右),但回望短暂的翻译生活,我认为这个词是最为恰当的。\\

这个词语的灵感,主要来自于熊文灿先生。作为一个没有兵力,没有经验的高级官员,他主要的武器,就是先找这里,再找那里,属于纯忽悠型。\\

但值得夸奖的是,他的忽悠是很有效果的,在福建的时候,手下只有几个兵,对面有一群海盗,二话不说,先找到了郑芝龙,死乞白赖地隔三差五去找人家(所以后来有的官员弹劾他,说他是求贼),请客送礼,反复招安,终于招来了郑芝龙。\\

虽然后来证明,郑大人是不大可靠的,但在当时,是绝对够用了,后来他借助郑大人的力量,杀掉了不肯投降的海盗刘香,平定了海乱。\\

这种空手道的生意,估计熊大人是做上瘾了,所以到中原上任的时候,他也玩了同一套把戏,先here招降了刘国能,再用刘国能,there招降了张献忠,here and there,无本生意,非常高明。\\

但这种生意有个问题,因为熊大人本人并无任何实力,只要here不行,或者there不行,他就不行了。\\

张献忠就是个不行的人,按照他的习惯,投降的时候,就要想好几时再造反,所以刚开始,他就不肯缴械,当然,这也有个说法,之所以不肯缴械,是因为他认为自己罪孽深重,要留着自己这几杆枪,为朝廷效力。\\

熊文灿倒是很高兴,表扬了好几次,后来他果真缺兵,去找张献忠要几千人帮忙,张献忠又说还没安顿好,先休整几天。\\

张献忠住的地方,就在今天襄樊的谷城地区,他老人家在此,基本就是县长了,想干什么就干什么,每天都要去县城里转一圈,算是视察,他手下的兵也没消停,每天都要刻苦操练。\\

与此同时,张县长也开始意识到,自己以前的行为是有错误的,比如,每次打仗的时候,都用蛮力,很少动脑子,且部队文化太低,没有读过兵法。为了加强理论教育,保证将来再造反的时候,有相当的理论基础,他找来了一个叫做潘独鳌的秀才,给他当军师。\\

这位潘独鳌到底何许人也,待查,估计是个吴用型的人物,应该是几次举人没考上,又想干点事,就开始全心全意地给张献忠干活,具体说就是教书,每天晚上,在张县长的统一带领下,大大小小的头目们跑去听课,课程有好几门,比如孙子兵法等等。学习完后,张县长还要大家写出学习心得,结合实际(比如再次造反后,该怎么打仗),分析讨论,学习气氛非常浓烈。\\

但他所干过最猖狂的事,还是下面这件事。\\

崇祯十二年(1639)年初的一天,谷城县令阮之钿接到报告,说谷城来了个人,正在和张献忠见面。\\

阮县令的职责是监视张献忠,加上他还比较尽责,就派了个人去打探看看到底是谁来了,谈了些什么。\\

没过多久,那人就回来了,他说谈了些什么,就不太知道了,但来的那个人,他认出来了。\\

谁?\\

李自成。\\

阮知县差点晕过去。\\

按照常理,自从一年前被打垮后,李自成应该躲在山沟里艰苦朴素,怎么会出来呢?还这么大摇大摆地见张献忠。\\

让人难以想象,这个来访者确实是李自成,他是来找张献忠要援助的。\\

更让人想不到的是,李自成就这么在谷城呆了几天,都没人管,又大摇大摆地走了。\\

其实不是没人管,是没法管。\\

张献忠之所以嚣张,是因为他手下还有几万人,而熊大人,我说过,他的主要能力,就是这里、那里的忽悠,要真拿刀收拾张县长,就没辙了。\\

而且更麻烦的是,他还收了张献忠的钱。\\

在明末农民起义的许多头领中,张头领是个异类,异就异在他不太像绿林好汉,反而很像官僚。\\

比如他在投降后,就马上马不停蹄地开始送礼,从熊文灿开始,每个月都要去孝敬几趟,而且他还喜欢串门,联络感情,连远在京城的诸位大人,他也没忘了,经常派人去送点孝敬,所以每次有什么事,他都知道得比较早。\\

此外,张县长还很讲礼数,据某些史料讲,他去见上级官员时,还行下跪礼,且非常周到,具有如此天赋,竟然干了这个,实在选错了行。\\

古语有云,司马昭之心,路人皆知,而张县长的心,似乎也差不多了,从上到下,都知道他要反,只不过迟早而已,比如左良玉,曾多次上书,要求解决张献忠,还有阮知县,找熊文灿讲了几次,熊大人没理他,结果气得阮大人回家自尽了。\\

总之,无论谁说张献忠要反,熊文灿都表示,这是没可能的,张献忠绝不会反。\\

对此,许多史料都奋笔疾书,说熊大人是白痴,是智商有问题。\\

我觉得这么说,是典型的人身攻击,熊大人连忽悠都能玩,绝非白痴。他之所以始终不相信张献忠会反,是因为他不能相信。\\

我相信,此时此刻,熊文灿的脑海里,经常出现这样一番对话,对话的时间,是两年前,熊大人刚刚接到调令,在以找死的觉悟准备赴任之前。\\

对话的地点,是庐山。对话的人,是个和尚,叫做空隐。\\

熊文灿找到空隐,似乎是想算卦,然而还没等他说话,空隐和尚就先说了:\\

“你错了(公误矣)!”\\

怎么个错法呢?\\

“你估量估量,你有能搞定流贼的士兵吗(自度所将兵足制贼死命乎)?”\\

“不能。”\\

“有能够指挥大局,独当一面的将领吗(有可属大事、当一面、不烦指挥而定者乎)?”\\

“没有。”\\

按照上下文的关系,下一句话应该是:\\

那你还干个屁啊!\\

但空隐毕竟是文明人,用了比较委婉的说法(似乎也没太委婉):\\

“你两样都无,上面(指皇帝)又这么器重你,一旦你搞不定,要杀头的!”\\

熊文灿比较昏,等了半天,才想出一句话:\\

“招抚可以吗?”\\

然而空隐回答:\\

“我料定你一定会招抚,但是请你记住,海贼不同流贼,你一定要慎重!”\\

这段对话虽然比较玄乎,但出自正统史料,并非杂谈笔记,所以可信度相当高,空隐提到的所谓海贼,指的就是郑芝龙,而流贼,就不用多说了。\\

他的意思很明确,熊大人你能招降海上的,却未必能招降地上的,可问题是,熊大人只有忽悠的能耐,就算海陆空一起来,他也只能招抚。外加他还收了张献忠的钱,无论如何,死撑都要撑下去。\\

死撑的结果,就是撑死。\\

张献忠之所以投降,不过是避避风头,现在风头过去,赶巧清军入侵,孙传庭和洪承畴两大巨头都到了辽东,千载难逢,决不能错过。\\

于是,崇祯十二年(1639)五月,正当崇祯兄收拾清军入侵残局的时候,张献忠再次反叛,攻占谷城。\\

谷城县令阮之钿真是好样的,虽然他此前服毒自尽,没有死成,又抢救过来了,但事到临头,很有点士大夫精神,张献忠的军队攻入县城,大家都跑了,他不跑,非但不跑,就坐在家里等着,让他投降,不降,杀身成仁。\\

很明显,张献忠起兵,是有着充分准备的,因为他第一个目标,并非四周的州县,而是曹操。\\

以曹操作为外号,对罗汝才而言,是比较贴切的,作为明末三大头领之一,他很有点水平,作战极狡猾,部下精锐,所以张献忠在起兵之前,先要拉上他。\\

罗汝才效率很高,张献忠刚反,他就反,并与张献忠会师,准备在新的工作岗位上继续奋斗。\\

顺道说一句,张献忠同志在离开谷城前,干的最后一件事,是贴布告,布告的内容,是一张名单,包括这几年他送出去的贿赂,金额,以及受贿人的名字,全部一清二楚,诏告天下。\\

不该收的,终究要还。\\

我没有看到那份布告,估计熊文灿同志的名字,应该名列前茅。但此时此刻,受贿是个小问题,渎职才是大问题。\\

熊文灿还算反应快,而且他很幸运,因为当时世上,能与张献忠、罗汝才匹敌的人,不会超过五个,而在他的手下,就有一个。\\

在众多头领中,左良玉最讨厌,也最喜欢的,就是张献忠。\\

他讨厌张献忠,是因为这个人太闹腾,他喜欢张献忠,是因为这个人虽然闹腾,却比较好打,他能当上总兵,基本就是靠打张献忠,且无论张头领状态如何,心情好坏,只要遇到他,就是必败无疑。\\

所以左总兵毅然决定,虽说熊大人很蠢,但看在张献忠份上,还是要去打打。\\

几天后,左良玉率军,与张献忠、罗汝才在襄阳附近遭遇,双方发生激战,惨败——左良玉。\\

所谓惨败,意思是,左良玉带着很多人去,只带着很少人跑回来。\\

之所以失败,是因为他太过嚣张,瞧不上张献忠,结果被人打了埋伏。\\

这次失败还导致了两个后果:\\

{\footnotesize \begin{quote}
	一、由于左良玉跑得太过狼狈,丢了自己的官印,当年这玩意丢了,是没法补办的,所以不会刻公章的左总兵很郁闷。\\
	二,熊文灿把官丢了,纵横忽海几十年,终于把自己忽了下去。\\
\end{quote}}

一个月后,崇祯下令,免去熊文灿的职务,找了个人代替他,将其逮捕入狱,一年后,斩首。\\

代替熊文灿的人,是杨嗣昌,逮捕熊文灿的人,是杨嗣昌,如果你还记得,当年推举熊文灿的人,是杨嗣昌。\\

从头到尾,左转左转左转左转,结果就是个圈,他知道,事到如今,他只剩下一个选择。\\

崇祯十二年(1639)九月,杨嗣昌出征。\\

明朝有史以来,所有出征的将领中,派头最大的,估计就是他了,当时他的职务,是东阁大学士,给他送行的,是皇帝本人,还跟他喝了好几杯,才送他上路。\\

崇祯是个很容易激动的人,激动到十几年里,能换几十个内阁大学士,此外,他的疑心很重,很难相信人。\\

而他唯一相信,且始终相信的人,只有杨嗣昌。在他看来,这个人可信,且可靠。\\

可信的人,未必可靠。\\

对于崇祯的厚爱,杨嗣昌很感动,据史料说,他当时就哭了,且哭得很伤心,很动容,表示一定完成任务,不辜负领导的期望。\\

当然,光哭是不够的,哭完之后,他还向崇祯要了两样东西,一样给自己的:尚方宝剑,另一样是给左良玉的:平贼将军印。\\

然后,杨嗣昌离开了京城,离开了崇祯的视线,此一去,即是永别。\\

崇祯十二年(1639)十月,杨嗣昌到达襄阳,第一件事,是开会。与会人员包括总督以及所有高级将领。杨嗣昌还反复交代,大家都要来,要开一次团结的大会。\\

人都来了,会议开始,杨嗣昌的第一句话是,逮捕熊文灿,押送回京,立即执行。\\

然后,他拿出了尚方宝剑。\\

明白?这是个批斗会。\\

总督处理了,接下来是各级军官,但凡没打好的,半路跑的,一个个拉出来单练,要么杀头,要么撤职,至少也是处分,当然,有一个人除外——左良玉。\\

左良玉很慌张,因为他的罪过很大,败得太惨,按杨大人的标准,估计直接就拉出去了。\\

但杨嗣昌始终没有修理他,直到所有的人都处理完毕,他才叫了左良玉的名字,说,有样东西要送给你。\\

左良玉很激动,因为杨嗣昌答应给他的,是平贼将军印。\\

在明代,将军这个称呼,并非职务,也不是级别,大致相当于荣誉称号,应该说,是最高荣誉,有明一代,武将能被称为将军的,不会超过五十个人。\\

对左良玉而言,意义更为重大,因为之前他把总兵印丢了,这种丢公章的事,是比较丢人的,而且麻烦,公文调兵都没办法,现在有了将军印,实在是雪中送火锅,太够意思。\\

杨嗣昌绝顶聪明,要按照左良玉的战绩,就算砍了,也很正常,但他很明白,现在手下能打仗的,也就这位仁兄,所以必须笼络。先用大棒砸别人,再用胡萝卜喂他,恩威并施,自然服气。\\

效果确实很好,左良玉当即表示,愿意跟着杨大人,水里水里去,火里火里去,干到底。\\

对于杨嗣昌的到来,张献忠相当紧张,紧张到杨大人刚来,他就跑了。\\

因为他知道,熊文灿只会忽悠,但杨嗣昌是玩真格的,事业刚刚起步,玩不起。\\

张献忠对局势有足够的判断,对实力有足够的认识,可惜,跑得不足够快。\\

他虽然很拼命地跑,但没能跑过左良玉,心情激动的左大人热情高涨,一路狂奔,终于在四川截住了张献忠。\\

战斗结果说明,如果面对面死打,张献忠是打不过的,短短一天之内,张献忠就惨败,败得一塌糊涂,死伤近万人,老婆孩子,连带那位叫做潘独鳌的军师,都给抓了,由于败得太惨,跑得太快,张献忠连随身武器都丢了(大刀),这些东西被左良玉全部打包带走,送给了杨嗣昌。\\

消息传来,万众欢腾,杨嗣昌极为高兴,当即命令左良玉,立即跟踪追击,彻底消灭张献忠。\\

左良玉依然积极,马上率军,尾随攻击张献忠。\\

局势大好。\\

士为知己者死\\

十几天后,左大人报告,没能追上,张献忠跑了。\\

杨嗣昌大怒,都打到这份上了,竟然还让人跑了,干什么吃的,怎么回事?\\

左良玉回复:有病。\\

按左大人的说法,是因为他进入四川后,水土不服,结果染了病,无力追赶,导致张献忠跑掉。\\

但按某些小道消息的说法,事情是这样的,在追击过程中,张献忠派人找到左良玉,说你别追我了,让我跑,结果左良玉被说服了,就让他跑了。\\

这种说法的可能性,在杨嗣昌看来,基本是零,毕竟左良玉跟张献忠是老对头,而且左大人刚封了将军,正在兴头上,残兵败将,拿啥收买左良玉?无论如何,不会干这种事。\\

然而事实就是这样。\\

左良玉很得意,张献忠很落魄,左良玉很有钱,张献忠很穷,然而张献忠确实收买了左良玉,没花一分钱。\\

他只是托人,对左良玉说了一句话。\\

这句话的大意是,你之所以受重用,是因为有我,如果没有我,你还能如此得意吗?\\

所谓养寇自保,自古以来都是至理名言,一旦把敌人打光了,就要收拾自己人,左良玉虽说是文盲,但这个道理也还懂。\\

然而就凭这句话,要说服左良玉,是绝无可能的,毕竟在社会上混了这么多年,一句话就想蒙混过关,纯胡扯。\\

左良玉放过张献忠,是因为他自己有事。\\

因为一直以来,左良玉都有个问题——廉政问题。文官的廉政问题,一般都是贪污受贿,而他的廉政问题,是抢劫。\\

按史料的说法,左良玉的军队纪律比较差,据说比某些头领还要差,每到一地都放开抢,当兵的捞够了,他自己也没少捞,跟强盗头子没啥区别。\\

对他的上述举动,言官多次弹劾,朝廷心里有数,杨嗣昌有数,包括他自己也有数,现在是乱,如果要和平了,追究法律责任,他第一个就得蹲号子。\\

所以,他放跑了张献忠。\\

这下杨嗣昌惨了,好不容易找到个机会,又没了,无奈之下,他只能自己带兵,进入四川,围剿张献忠。\\

自打追缴张献忠开始,杨嗣昌就没舒坦过。\\

要知道,张献忠他老人家,原本就是打游击的,而且在四川一带混过,地头很熟,四川本来地形又复杂,这里有个山,那里有个洞,经常追到半路,人就没了,杨大人只能满头大汗,坐下来看地图。\\

就这么追了大半年,毫无结果,据张献忠自己讲,杨嗣昌跟着他跑,离他最近的时候,也有三天的路,得意之余,有一天,他随口吟出一首诗。\\

这是一首诗,一首打油诗,一首至今尚在的打油诗(估计很多人都听过),打油诗都能流传千古,可见其不凡功力,其文如下:\\

\begin{quote}
	\begin{spacing}{0.5}  %行間距倍率
		\textit{{\footnotesize
				\begin{description}
					\item[\textcolor{Gray}{\FA }] 前有邵巡抚,常来团转舞。
					\item[\textcolor{Gray}{\FA }] 后有廖参军,不战随我行。
					\item[\textcolor{Gray}{\FA }] 好个杨阁部,离我三尺路。
				\end{description}
		}}
	\end{spacing}
\end{quote}

文采是说不上了,意义比较深刻,所谓邵巡抚,是指四川巡抚邵捷春,廖参军,是指监军廖大亨。据张献忠同志观察,这二位一个是经常来转转,一个是经常跟着他走,只有杨嗣昌死追,可是没追上。\\

这首诗告诉我们,杨嗣昌很孤独。\\

所有的人,都在应付差事,出工不出力,在黑暗中坚持前行的人,只有他而已。\\

在史书上,杨嗣昌是很嚣张的,闹腾这么多年,骂他的口水,如滔滔江水,延绵不绝,然而无论怎么弹劾,就是不倒。就算他明明干错了事,崇祯却依然支持他,哪怕打了败仗,别人都受处分,他还能升官。\\

当年我曾很不理解,现在我很理解。\\

他只是信任这个人,彻底地相信他,相信他能力挽狂澜,即使事实告诉他,这或许只能是个梦想。\\

毕竟在这个冷酷的世界上,能够彻底地相信一个人,是幸运的。\\

崇祯并没有看错人,杨嗣昌终将回报他的信任,用他的忠诚、努力,和生命。\\

崇祯十三年(1640)十二月,跟着张献忠转圈的杨嗣昌得到了一个令他惊讶消息:张献忠失踪。\\

对张献忠的失踪,杨嗣昌非常关心,多方查找,其实如张头领永远失踪,那也倒好,但考虑到他突遭意外(比如被外星人绑走)的几率不大,为防止他在某地突然出现,必须尽快找到这人,妥善处理。\\

张献忠去向哪里,杨嗣昌是没有把握,四川、河南、陕西、湖广,反正中国大,能藏人的地方多,钻到山沟里就没影,鬼才知道。\\

但张献忠不会去哪里,他还有把握,比如京城、比如襄阳。\\

京城就不必说了,路远坑深,要找死,也不会这么个死法。而襄阳,是杨嗣昌的大本营,重兵集结,无论如何,绝不可能。\\

下次再有人跟你说,某某事情绝无可能,建议你给他两下,把他打醒。\\

张献忠正在去襄阳的路上。\\

对张献忠而言,去襄阳是比较靠谱的,首先,杨嗣昌总跟着他跑,兵力比较空虚,其次,他的老婆孩子都关在襄阳,更重要的是,在襄阳,有一个人,可以置杨嗣昌于死地。\\

为了达到这个目的,他创造了跑路的新纪律,据说一晚上跑了三百多里,先锋部队就到了,但人数不多——十二个。\\

虽然襄阳的兵力很少,但十二个人估计还是打不下来的,张献忠虽然没文凭,但有常识,这种事情他是不会做的。\\

所以这十二个人的身份,并不是他的部下,而是杨嗣昌的传令兵。\\

他们穿着官军的衣服,趁夜混入了城,以后的故事,跟特洛伊木马计差不多,趁着夜半无人,出来放火(打是打不过的),城里就此一片浆糊,闹腾到天明,张献忠到了。\\

他攻下了襄阳,找到了自己的老婆孩子,就开始找那个能让杨嗣昌死的人。\\

找半天,找到了,这个人叫朱翊铭。\\

朱翊铭,襄王,万历皇帝的名字,是朱翊钧,光看名字就知道,他跟万历兄是同辈的,换句话说,他算是崇祯皇帝的爷爷。\\

但这位仁兄实在没有骨气,明明是皇帝的爷爷,见到了张献忠,竟然大喊:千岁爷爷饶命。\\

很诡异的是,张献忠同志非常和气,他礼貌地把襄王同志扶起来,让他坐好。\\

襄王很惊慌,他说,我的财宝都在这里,任你搬用,别客气。\\

张献忠笑了,他说,你有办法让我不搬吗?\\

襄王想想也是,于是他又说,那你想要什么?\\

张献忠又笑了:我要向你借一样东西。\\

什么东西?\\

脑袋。\\

在杀死襄王的时,张献忠说:如果没有你的脑袋,杨嗣昌是死不了的。\\

此时的杨嗣昌,刚得知张献忠进入湖广,正心急火燎地往回赶,赶到半路,消息出来,出事了,襄阳被攻陷,襄王被杀。\\

此后的事情,按很多史料的说法,杨嗣昌非常惶恐,觉得崇祯不会饶他,害怕被追究领导责任,畏罪自杀。\\

我个人认为,这种说法很无聊。\\

如果是畏罪,按照杨嗣昌同志这些年的工作状况,败仗次数,阵亡人数,估计砍几个来回,都够了,他无需畏惧,只需要歉疚。\\

真实的状况是,很久以前,杨嗣昌就身患重病,据说连路都走不了,吃不下饭,睡不着觉,按照今天的标准,估计早就住进高干病房吊瓶了。\\

然而他依然坚持,不能行走,就骑马,吃不下,就少吃或不吃,矢志不移地追击张献忠。我重复一遍,这并非畏惧,而是责任。\\

许多年来,无论时局如何动荡,无论事态如何发展,无论旁人如何谩骂,弹劾,始终支持,保护,相信,相信我能挽回一切。\\

山崩地裂,不可动摇,人言可畏,不能移志,此即知己。\\

士为知己者死。\\

所以当他得知襄王被杀时,他非常愧疚,愧疚于自己没有能够尽到责任,没有能够报答一个知己的信任。\\

一个身患重病的人,是经不起歉疚的,所以几天之后,他就死了,病重而亡。\\

他终究没能完成自己的承诺。\\

他做得或许不够好,却已足够多。\\

对于杨嗣昌的死,大致有两种态度,一种是当时的,一种是后来的,这两种态度,都可以用一个字来形容——活该。\\

当时的人认为,这样的一个人长期被皇帝信任,实在很不爽,应该死。\\

后来的人认为,他是刽子手,罪大恶极,应该死。\\

无论是当时的,还是后来的,我都不管,我只知道,我所看到的。\\

我所看到的,是一个人,在绝境之中,真诚,无条件信任另一个人,而那个人终究没有辜负他的信任。\\
\ifnum\theparacolNo=2
	\end{multicols}
\fi
\newpage
\section{选择,没有选择}
\ifnum\theparacolNo=2
	\begin{multicols}{\theparacolNo}
\fi
杨嗣昌死了,崇祯很悲痛,连他爷爷辈的亲戚(襄王)死了,他都没这么悲痛,非但没追究责任,还追认了一品头衔,抚恤金养老金,一个都没少。知己死了,没法以死相报,以钱相报总是应该的。\\

其实和崇祯比起来,杨嗣昌是幸运的,死人虽说告别社会,但毕竟就此解脱,彻底拉倒。\\

而崇祯是不能拉倒的,因为他还要解决另一个问题,一个更麻烦的问题。\\

崇祯十三年(1640),崇祯正忙着收拾张献忠的时候,皇太极出兵了。\\

虽然此前他曾多次出兵,但这一次很不寻常。\\

因为他的目标,是锦州。\\

自打几次到关宁防线挖砖头未果,皇太极就再也没动过锦州的心思,估计是十几年前被袁崇焕打得太狠,打出了恐x症,到锦州城下就打哆嗦。\\

所以每次他进攻的时候,都要不远万里,跑路、爬山、爬长城,实在太过辛苦,久而久之,搏命精神终于爆发,决定去打锦州。\\

但实践证明,孙承宗确实举世无双,他设计的这条防线,历经近二十年,他本人都死了,依然在孜孜不倦地折腾皇太极。\\

皇太极同志派兵打了几次,毫无结果,最后终于怒了,决定全军上阵。\\

同年四月,他发动所部兵力,包括多尔衮、多铎、阿济格,甚至连尚可喜、孔有德的汉奸部队,都调了出来,同时,还专门造了上百门大炮,对锦州发动了总攻。\\

守锦州的,是祖大寿。\\

事情的发展告诉皇太极,当年他放走祖大寿,是比较不明智的。因为这位仁兄明显没有念他的旧情,还很能干,被围了近三个月,觉得势头危险,才向朝廷求援。\\

而且据说祖大寿的求援书,相当地强悍,非但没喊救命,还说敌军围城,若援军前来,要小心敌人陷阱,不要轻敌冒进,我还撑得住,七八月没问题。\\

但崇祯实在够意思,别说七八月,连七八天都没想让他等,他当即开会,商量对策。\\

开会的问题主要是两个,一、要不要去,二、派谁去。\\

第一个问题很快解决,一定要去。\\

就军事实力而言,清军的战斗力,要强于明军,辽东能撑二十多年,全靠关宁防线,如果丢了,就没戏了。\\

第二个问题,也没什么疑问,卢象升死了,杨嗣昌快死了。\\

只有洪承畴。\\

问题解决了,办事。\\

崇祯十三年(1640)五月,洪承畴出兵了。\\

得知他出兵后,皇太极就懵了。\\

打了这么多年,按说皇太极同志是不会懵的,但这次实在例外,因为他虽然料定对方会来,却没有想到,会来得这么多。\\

洪承畴的部队,总计人数,大致在十三万左右。属下将领,包括吴三桂、白广恩等,参与作战部队除本部洪兵外,还有关宁铁骑一部,总之,最能打的,他基本都调来了。\\

本来是想玩玩,对方却来玩命,实在太敞亮了。\\

考虑到对方的战斗能力和兵力,皇太极随即下令,继续围困锦州,不得主动出战,等待敌军进攻。\\

但是接下来的事情,却让他很晕。\\

因为洪承畴来后,看上去没有打仗的打算,安营、扎寨,每天按时吃饭,睡觉,再吃饭,再睡觉,再不就是朝城里(锦州)喊喊话,兄弟挺住等等。\\

晕过之后,他才想明白,这是战术。\\

洪承畴的打算很简单,他判定,如果真刀真枪拼命,要打败清军,是很困难的,所以最好的方法,就是守在这里,慢慢地耗,把对方耗走了,完事大吉。\\

这是个老谋深算的计划,也是最好的计划。对这一招,皇太极也没办法,要走吧,人都拉来了,路费都没着落,就这么回去,太丢人。\\

但要留在这里,对方又不跟你开仗,只能耗着。\\

耗着就耗着吧,总好过回家困觉。\\

局势就此陷入僵持,清军在祖大寿外面,洪承畴在清军外面,双方就隔几十里地,就不打。\\

当然,清军也没完全闲着,硬攻不行,就开始挖地道,据说里三层、外三层,赛过搞网络的,密密麻麻。\\

但事实告诉我们,祖大寿,那真是非一般的顽强,而且他还打了埋伏,之前跟朝廷说,他可以守八个月,实际满打满算,他守了两年。\\

就这样,从崇祯十三年(1640)五月到崇祯十四年(1641)五月,双方对峙一年。\\

六月底,开战了。\\

洪承畴突然打破平静,出兵,向松山攻击挺进。\\

这个举动大大出乎清军的意料,清军总指挥多尔衮(皇太极回家)没有提防,十万人突然扑过来,被打了个措手不及,战败。\\

消息传来,皇太极晕了,一年都没动静,忽然来这么一下,你打鸡血了不成?\\

多年的作战经验告诉他,决战的时刻即将到来,于是他立即上马,率领所有军队,前往松山。\\

但是,有个问题。\\

当时皇太极,正在流鼻血。\\

一般说来,流鼻血,不算是个问题,拿张手纸塞着,也还凑合。\\

但皇太极的这个鼻血,据说相当之诡异,流量大,还没个停,连续流了好几天,都没办法。\\

但军情紧急,在家养着,估计是没辙了,于是皇太极不顾流鼻血,带病工作,骑着马,一边流鼻血,一边就这么去了。\\

让人难以理解的是,他没有找东西塞鼻孔,却拿了个碗,就放在鼻子下面,一边骑马一边接着,连续两天两夜赶到松山,据说到地方时,接了几十碗。\\

反正我是到今天都没想明白,拿这碗干什么用的。\\

会战地点,松山,双方亮出底牌。\\

清军,总兵力(包括孔有德等杂牌)共计十二万,洪承畴,总兵力共计十三万,双方大致相等。\\

清军主将,包括多尔衮、多铎、济尔哈朗等精锐将领,除个把人外,都很能打。\\

洪承畴方面,八部总兵主将,除吴三桂外,基本都不能打。\\

至于战斗力,就不多说了,清军的战斗力,大致和关宁铁骑差不多,按照这个比率,自己去想。\\

换句话说,要摊开了打,洪承畴必败无疑。\\

但洪承畴,就是洪承畴。\\

崇祯十四年(1641)七月二十八日,洪承畴突然发动攻击,率明军抢占制高点乳锋山,夺得先机。\\

他十分得意,此时他的军中的一个武官对他说了一件事:\\

占据高地固然有利,但我军粮少,要提防清军抄袭后路。\\

然而洪承畴似乎兴奋过度,把那个人训了一顿,说:\\

我干这行十几年,还需要你提醒?\\

大多数历史学者认为,这句话,就是他失败的最终原因。\\

因为就战略而言,固守是最好的方法,进攻是最差的选择,而更麻烦的是,当时的洪承畴,在进攻之前,只带了三天的粮食。\\

无论如何,只带三天的粮食,是绝对不够的。\\

所以结论是,一贯英明的洪承畴,犯了一个愚蠢的错误,最终导致了战败。\\

我原本认为,这个结论很对,洪承畴很蠢,起码这次很蠢。\\

后来我想了想,才发现,洪承畴不蠢,起码这次不蠢。在他看似荒谬的行动背后,隐藏着一个极为精明的打算。\\

其实洪承畴并不想进攻,他很清楚,进攻极为危险,但他没有办法。\\

因为有个人一直在催他,这个人的名字叫陈新甲,时任兵部尚书,而这位陈尚书的外号,叫小杨嗣昌。\\

杨嗣昌同志的特点,是风风火火,玩命了干,能得这个外号,可见陈大人也不白给。\\

自打洪承畴打持久战,他就不断催促出战,要洪督师赶紧解决问题,是打是不打,多少给个交代。\\

但洪承畴之所以出战,不仅因为陈尚书唠叨,像他这样的老油条,是不会怕唐僧的。\\

他之所以决定出战,最根本的原因,就是两个字——没钱。\\

我查过资料,明末时期的军饷,以十万人计,吃喝拉撒外加工资、奖金,至少在三十万两白银以上。\\

要在平时,这也是个大数,赶巧李自成、张献忠都在闹腾,要是洪承畴再耗个几年,崇祯同志的裤子,估计都要当出去。\\

所以不打不行。\\

但洪承畴不愧为名将,所以在出发前,他想出了一个绝招:只带三天粮食。\\

要还没明白,我就解释一遍:\\

带上三天粮食出征,如果遇上好机会,就猛打一闷棍,打完就跑,也不怕对手断后路。\\

如果没有机会,看情形不妙,立马就能跑,而且回来还能说,是粮食不够了,才跑回来的,对上面有了个交代,又不怕追究政治责任,真是比猴还精。\\

精过头,就是蠢。\\

如果换了别人,这个主意没准也就成了,可惜,他的对手是皇太极。\\

皇太极不愧老牌军事家,刚到松山,还在擦鼻血,看了几眼,就发现了这个破绽。\\

八月二十日,就在洪承畴出发的第二天,他派遣将领突袭洪军后路,占领锦州笔架山粮道。\\

“欲战,则力不支;欲守,则粮已竭。”洪承畴彻底休息了。\\

当然,当然,在彻底休息前,洪承畴还有一个选择——突围。\\

毕竟他手里还有十几万人,要真玩命,还能试试。\\

于是他找来了手下的八大总兵,告诉他们事态紧急,必须通力合作,然后,他细致分配了工作,从哪里出发,到哪里会合,一切安排妥当,散会。\\

我忘了说,在这八个总兵里,有一个人,叫做王朴。\\

第二天,突围开始。\\

按照洪承畴的计划,突围应该是很有秩序的,包括谁进攻,谁佯攻,谁殿后,大家排好队,慢慢来。\\

可还没等洪承畴同志喊一二三,两个人就先跑了。\\

那两个先跑的人,一个是王朴。\\

如果没有重名,这位王朴兄,应该就是八年前,在黄河边上收钱,放走诸位头领的总兵同志。\\

照此看来,他还是有进步的,八年前,收钱让别人跑,现在撒腿就跑,也没想着找皇太极同志拿钱,实在难得。\\

而另一位带头逃跑的,史料记载有点争议,但大多数人认为,是吴三桂。\\

无论如何,反正是散了,彻底散了,全军溃败,无法收拾,十余万人土崩瓦解,被人杀的,被踩死的,不计其数,损失五万多人。\\

洪承畴还算是镇定,关键时刻,找到了曹变蛟、丘民仰,还聚了上万人,占据松山城,准备伺机撤退。\\

可是皇太极很不识相,非要解决洪承畴,开始围城,劝降。\\

洪承畴拒不投降,派使者向京城求救。\\

可他足足等了半年,也没有等来救兵,他很纳闷,为什么呢?\\

因为他糊涂了,就算用脚趾头想,也能明白,援兵是绝不会到的。\\

要知道,他老人家来,就是救援锦州的,能带的部队都带了,可现在他也被人围住,再去哪里找人救他?\\

其实洪承畴同志不知道,皇帝陛下也在等,不过他等的,不救兵,而是洪承畴的死亡通知书。\\

按史料的说法,洪承畴同志被围之后不久,京城这边追悼会什么的都准备好了,家属慰问,发放抚恤,追认光荣,基本上程序都走了,就等着洪兄弟为国捐躯。\\

其实洪承畴原本也这么盘算来着,死顶,没法顶了,就捐躯。做梦都没想到,他连捐躯都没捐成。\\

崇祯十五年(1642)二月二十日,在这个值得纪念的日子,松山副将夏承德与清军密约,打开了城门,洪承畴被俘。\\

几个月后,无计可施的祖大寿终于投降,这次,他是真的投降了。\\

自崇祯十三年(1641)至崇祯十五年(1643),明朝和清朝在松山、锦州一带会战,以明军失利告终,史称“松锦大战”。\\

除宁远外,辽东全境陷落,从此,明朝在关外,已无可战。\\

消息传到北京,照例,崇祯很悲痛,虽然这几年他经常悲痛,但这次,他尤其激动,连续几天都泪流满面,因为他又失去了一位好同志——洪承畴。\\

按目击者的说法,洪承畴同志被抓之后,非常坚强,表示啥也别说了,给我一刀就行,后来英勇就义,眼睛都没眨,很勇敢,很义气。\\

所以崇祯很是感动,他亲自主持了洪承畴同志的追悼会,还给他修了坛(明朝最高规格葬礼),以表彰他英勇就义的精神。\\

洪承畴没有就义,他投降了。\\

当然,他刚被俘的时候,还是比较坚持原则的,没有投降,结果过了几天,由于平时没有注意批评和自我批评,关键时刻没能挺住,还是投降了。\\

至于他投降后的种种传奇,就不说了,可以直接跳过,说说他的结局。\\

清朝统一中原时,洪承畴由于立下大功,干了很多工作,有很大的贡献,被委以重任,担任要职。\\

清朝统一中原后,洪承畴由于立下大功,干了很多工作,有很大的贡献,被剥夺一切官职,光荣退休。\\

后来他死了,死后追封爵位,三等阿达哈哈番,这是满语,汉语翻译过来,是三等轻车都尉。\\

如果你不清楚清朝爵位制度,我可以解释,高级爵位分为公、侯、伯、子、男五级,每个爵位,又分一到三等,一等为最高。\\

男爵再往下一等,就是轻车都尉,三等轻车都尉,是轻车都尉中的最低等。我查了一下,大致是个从三品级别。\\

我记得洪承畴活着给明朝打工时,就是从一品太子太保,死了变从三品,有性格。\\

后来又过了几十年,乾隆发话,要编本书,叫做贰臣传。\\

所谓贰臣,通俗点说,就是叛徒,洪承畴同志以其光辉业绩,入选叛徒甲等。\\

在此之前,似乎就是乾隆同志,还曾发话,说抗清而死的黄道周,堪称圣人,说史可法是英雄,要给他立碑塑像。\\

我又想起了陈佩斯那个经典小品里的台词:\\

叛徒,神气什么!\\

好像还是这个小品,另一句话是:\\

你说我当时要是咬咬牙,不就挺过来了吗?\\

絮絮叨叨说这几句,只是想说:\\

{\footnotesize \begin{quote}
	一、历史证明,叛徒是没有好下场的。同志瞧不起的人,敌人也瞧不起。\\
	二、黄道周挺过来了,我敬佩,卢象升挺过来了,我景仰,洪承畴没挺过来,我鄙视,但理解。\\
\end{quote}}

咬牙挺过来,是不容易的。\\

所以,我不接受,但我理解。\\

\subsection{气数}
现在的崇祯,基本已经焦了,里面打得一塌糊涂,外面打得糊涂一塌,没法混了。\\

但他还是要撑下去,直到撑死,因为最能折腾他的那位仁兄还没出场。\\

据说打崇祯十二年起,崇祯同志经常做梦,梦见有一个人,在他的手上,写了一个字——有。\\

这是个很奇怪的梦,而且还不止一次,所以他把这个梦告诉文武大臣,让他们帮忙解释。\\

大家听说,都说很好,说很吉利,我想了想,有道理,因为有,总比没有好。\\

然而有一个人却大惊失色,这个人叫王承恩,是崇祯的贴身太监。\\

散朝后,他找到了崇祯,对他说出了这个梦境的真实意义,可怕的寓意——大明将亡。\\

按照王承恩的解释,这个有,实际上是两个字。上面,是大字少一捺,下面,是明字少半边。\\

所以这个字的意思,就是大明,要少一半。\\

崇祯不信,不敢信,大明江山,自打朱重八起,二百多年,难道要毁在自己手上?\\

个人认为,崇祯同志过于忧虑了,因为毁不毁,这事不由他。\\

但这个梦实在比较准,我查了一下,他做梦的时间,大致就是那个毁他江山的人,出现的时间。\\

崇祯十二年(1639),一个人从深山中走出。\\

他的随从很少,很单薄,且很不起眼,无论是张献忠,还是皇太极,他都望尘莫及。但命中注定,他才是最终改变一切的人,五年之后。\\

这人我不说,你也知道是李自成。\\

李自成在山里蹲了一年多,干过什么,没人知道,只知道他出来之后,进步很快。\\

一年多时间,他又有了几千人,占了几个县城。\\

但就全国而言,他实在排不上,有时经济困难,还得找张献忠拉兄弟一把。\\

鉴于生计困难,崇祯十三年(1640)初,他率军进入河南,新年新气象,他准备到那里碰碰运气。\\

通常来讲,这个想法没啥搞头,因为之前他经常全国到处出差,河南也是出差地之一,跑来跑去,没什么意外惊喜。\\

但这次不一样。\\

崇祯十三年(1640),河南大旱。\\

这场大旱,史料上说,是两百多年未遇之大旱,河南的景象,借用古人的话:白骨露於野,千里无鸡鸣。\\

大旱也好,没有鸡叫也罢,没有牛,没有猪都罢,有一样东西,是终究不会罢的——征税。\\

不征税,就没钱打张献忠,没钱防皇太极,必须征。\\

这么个环境,让人不造反,真的很难。\\

至于结局,不用想也知道,劳苦大众,固然劳苦,也是大众,劳苦久了,大众就要闹事,就要不交税,不纳粮,于是接下来,就是那句著名的口号:\\

\begin{quote}
	\begin{spacing}{0.5}  %行間距倍率
		\textit{{\footnotesize
				\begin{description}
					\item[\textcolor{Gray}{\FA }] 吃他娘,穿他娘,开了大门迎闯王,闯王来时不纳粮。
				\end{description}
		}}
	\end{spacing}
\end{quote}

之前我说什么来着?气数。\\

没错,就是气数。\\

其实气数这玩意,说穿了,就是个使用年限,好比饼干,只能保质三天,你偏三年后吃,就只能拉肚子。\\

又好比房子,只能住三十年,你偏要住四十年,就只能住危房,没准哪天上厕所的时候,被埋进去。\\

什么东西,都有使用年限,比如大米,比如王朝,比如帝国。\\

不同的是,大米的年限看得见,王朝的年限看不见。\\

看不见,却依然存在。\\

对于气数,崇祯是不信的,开始不信。\\

等到崇祯十四年,怕什么来什么,后院起火,前院也起火,卢象升死了,辽东败了,中原乱了,信了。\\

在一次检讨会上,他紧绷了十四年的神经,终于崩溃了。\\

他嚎啕大哭,一边哭,一边说:\\

我登基十四年,饱经忧患,国家事情多,灾荒多,没有粮食,竟然人吃人,流寇四起,这都是我失德所致啊,这都是我的错啊。\\

他不停地哭,不停地哭。\\

我同情他。\\

大臣们似乎也很同情,纷纷发言,说这不是您的错。\\

但不是皇帝的错,是谁的错呢?\\

气数。\\

几乎所有的人,众口一词,说出了这两字。\\

崇祯终于认了,他承认这是气数。但他终究是不甘心的:\\

“就算是气数,人力也可补救,这么多年了,补救何用?”\\

然后接着大哭。\\

崇祯大哭的时候,李自成正在前进,在属于他的气数上,大踏步地前进。\\

在河南,他毫不费力地招募了十几万人,只用了两年时间,就占领了河南全境,所向披靡,先后杀死陕西总督傅宗龙、汪乔年,以及我们的老熟人福王朱常洵。\\

鉴于崇祯同志的倒霉史,已经太长,鉴于他受的苦,实在太多,鉴于不想有人说我拿崇祯同志混事,还鉴于我比较乐观,不太喜欢落井下石,所以,我决定简单点,至少保证你不至于看得太过郁闷。\\

李自成同志依然在前进,一年后,他进入陕西,击败了明朝的最后一位猛人孙传庭,占领西安。明军就此再无还手之力。\\

崇祯十六年(1643),李自成在西安,集结所有兵力,准备向京城出发,他将终结这已延续二百七十多年的帝国。\\

在出发前,他发出了一道檄文,文中有八个字:\\

\begin{quote}
	\begin{spacing}{0.5}  %行間距倍率
		\textit{{\footnotesize
				\begin{description}
					\item[\textcolor{Gray}{\FA }] 嗟尔明朝,气数已尽。
				\end{description}
		}}
	\end{spacing}
\end{quote}

\subsection{嗟尔,明朝}
对于上述八个字,崇祯应该是认账的,因为不认账不行。\\

上台以前,憋足了劲要干掉那个死人妖,死人妖干掉了,又出来党争,后金入侵,看准了袁崇焕,要他出来上岗,一顿折腾,后金没能折腾回去,袁督师倒给折腾没了,本想着卧薪尝胆,忍几年,搞好国内经济建设,再去收复大好河山,结果出了天灾,又出来若干人等造反。\\

调兵,干掉若干人等,若干人等被干掉,又出来了若干更狠的人(比如张献忠、李自成),再调兵,把若干更狠的人,又打下去,投降的投降,跑的跑,正准备一鼓作气……\\

清军打进来了。\\

好吧,那就去打清军,全部主力调到辽东,打个一年半载,好不容易把人熬走,后院又起火了,投降的不投降,跑进去的又跑出来。\\

很巧,又是灾荒,大荒,没法活,于是大家跟着一起造反。\\

这种编剧思路,很类似于早些年的经典电视剧《渴望》,按当时编剧思路,就是找个弱女子,什么坏事、孬事、恶心人到死的事,都让她碰上,整体流程大致是,一棍子打过来,挺住,再一棍子打过来,继续挺住,挺到最后,就好人一生平安了。\\

崇祯的故事就是这样,他挨棍子的数量,估计比渴望女主角要多得多,抗击打能力更强,但不同的是,他的故事没有一个好的结局。\\

因为他的故事,是真实的,而真实的东西,往往都很残酷。\\

崇祯并非一个温和的人,他很急躁,很用力,用今天的话说,叫用力过猛,但那个烂摊子,不用力过猛,只能收摊。\\

崇祯很节俭,他的衣服、袜子,都打了补丁,请注意,打补丁的,并不一定很节俭,往往很浪费,比如后来清朝的道光同志,衣服破了,让人去打了个补丁,五十两白银,这哥们全然是败家的,还说特便宜。\\

而崇祯的补丁,是他找老婆打的,免费。\\

此外,崇祯还有个特点:走路慢,因为走得快,里面的破衣服就会飘出来——节俭是节俭,脸面还是要的。\\

他工作很努力,每天白天上朝,晚上加班,据史料记载,大致要干七八个时辰(十四到十六个小时),累得半死不活,第二天接着干。\\

简单地说,崇祯同志干的,是这样一份工作,没有工作范围,没有工作界限,什么都要管,每天上班,不是跟人吵架(言官),就是看人吵架(党争),穿得破烂,吃得也少,跟老婆困觉较少,只睡五六小时,时不时还有噩耗传来,什么北边打过来,西边打过去,祖坟被人烧了,部将被人杀了,东西被人抢了等等。\\

这工作,谁干?\\

最不幸的是,崇祯同志以上所有的不幸,都无法换来一个幸福的结局——他的努力,终究失败。\\

但比最不幸更不幸的是(简称最最不幸),崇祯知道这点。\\

知道结局(注:悲剧),也无法改变,却依然要继续,这就是人生的最大悲哀。\\

史料告诉我们,崇祯同志应该知道自己的结局,他多次谈到命数,气数,经常对人哀叹:大明天下,奈何亡于朕手!\\

然而他依然尽心尽力、全力以赴、日以继夜、夜以继日、勤勤恳恳、任劳任怨、不到长城心不死,撞了南墙不回头,往死了干,直到最后结局到来,依然没有放弃,直到兵临城下的那一天,依然没有放弃。\\

一个了不起的人。\\

结局到来的具体过程,就没必要细说了,我说过,我是个有幽默感的人,很明显,至少对于崇祯而言,这段并不幽默。\\

我还说过,我是个不喜欢写废话的人,同样,对崇祯而言,这段是废话。\\

当然,对李自成同志而言,这段很幽默,也不是废话,他从陕西出发,只用了三个月时间,就到了北京。\\

三月十七日,李自成的军队到达西直门(他从西边来),开始攻城。\\

崇祯同志有句名言,诸臣误我,还有一句,是文臣人人可杀,三月十七日,事实证明,这两句话很正确。\\

内阁大臣拿不出主意,连话都没几句,且不说了,守城的诸位亲信,什么兵部尚书、吏部侍郎,压根就没抵抗,全部打开城门投降。\\

当天,外城失陷,第二天,内城失陷。\\

崇祯住在紫禁城,就是今天的故宫,故宫有多大,去过的地球人都知道。\\

这里,就是他的最后归宿。\\

\subsection{三月十八日的夜晚}
在这个夜晚,发生了很多事,都是后事。\\

其实后事处理起来,也很简单,就几句话,后妃上吊,儿子跑掉(对于后患,大多数人都不留),料理完了,身边还有个女儿。\\

这个女儿,叫做长平公主,关于她的前世今生,金庸同志已经说过了,虽然相关内容(包括后来跟韦小宝同志的际遇),百分之九十以上都是胡扯,但有一点是正确的,他确实砍断了女儿的手臂。\\

这个举动在历史上非常有名,实际情况,却比许多人想象中复杂得多,但无论如何,原因很简单,他不希望这个女儿落入敌人的手中,遭受更大的侮辱。\\

不是残忍,而是慈爱。\\

我知道,许多人永远无法理解,那是因为,他们永远无需去理解。\\

处理完一切后,崇祯决定,去做最后一件事——自尽。\\

自尽,是一件比较有勇气的事,按照某位哲学家的说法,你敢死,还不敢活吗?没种。\\

但现实是残酷的,而今这个世界,要活下去,比死需要更大的勇气。\\

但崇祯的死,并非懦弱,而是一种态度,负责任的态度。\\

我说过,所谓王朝,跟公司单位差不多,单位出了事,领导要负责任,降级、扣工资、辞退,当然,也包括自尽。\\

崇祯决定自尽,他打算用这种方式,表达他的如下观点:\\

{\footnotesize \begin{quote}
	一、绝不妥协。\\
	二、绝不当俘虏。\\
	三、尊严。\\
\end{quote}}

于是,在那天夜里,崇祯登上了煤山(今天叫做景山),陪在他身边的,还有一个叫做王承恩的太监。\\

就这样吗?\\

就这样吧。\\

他留下了最后的遗言:\\

\begin{quote}
	\begin{spacing}{0.5}  %行間距倍率
		\textit{{\footnotesize
				\begin{description}
					\item[\textcolor{Gray}{\FA }] 诸臣误朕,朕死,无面目见祖宗,自去冠冕以发覆面,任贼分尸,勿伤百姓一人。
				\end{description}
		}}
	\end{spacing}
\end{quote}

所有的一切,都结束了。\\

他走向了那颗树。\\

应该结束了。\\

按照惯例,每个人的讲述结束时,会有一句结束语,而当这个王朝结束的时候,也会有一句话,最后一句话。\\

是的,这句话我已经写过了,不是昨天,也不是前天,而是几年以前,在我的第一本书里,朱元璋登基那一段的最后,有一句话,就是那句,几年前,我就写好了。\\

还记得吗?\\

所有的王朝,他的开始,正如他的结束,所以才有了这句结束语,没错,就是下面这句:\\

\begin{quote}
	\begin{spacing}{0.5}  %行間距倍率
		\textit{{\footnotesize
				\begin{description}
					\item[\textcolor{Gray}{\FA }] 走上了这条路,就不能再回头。
				\end{description}
		}}
	\end{spacing}
\end{quote}
\ifnum\theparacolNo=2
	\end{multicols}
\fi
\newpage

\section{结束了}
\ifnum\theparacolNo=2
	\begin{multicols}{\theparacolNo}
\fi
从理论上说……结束了\\

结束了吗?\\

结束了。\\

真的结束了吗?\\

没有。\\

从理论上说,文章结束了,但从实践上说,还没有。\\

废话。\\

其实历史和小说不一样,因为历史的答案,所有人都知道,崇祯同志终究是要死的,而且肯定是吊死,他不会撞墙,不会抹脖子,不会喝敌敌畏,总而言之,我不说,你们都知道。\\

所以结局应该是固定的,没有支线。\\

但是,我的结局,并不是这个。换句话说,我的文章,有两个结局,这只是第一个。\\

我读了十五年历史,尊重历史,所以这篇文章从头至尾,不能说无一字无来历,但大多数,都是有出处的。我不敢瞎编。\\

所以第二个结局,也是真实的,只不过比较奇特,它一直在我的脑海里,最后,我决定把这个比较奇特的结局写出来。\\

\subsection{第二个结局}
徐宏祖出生的时候,是万历十五年。\\

在这个特定的年龄出生,真是缘分,但外面的世界,跟徐宏祖并没有多大关系,他的老家在江阴,山清水秀,不用搞政治,也不怕被人砍,比较清净。\\

当然,清净归清净,在那年头,要想出人头地,青史留名,只有一条路——考试(似乎今天也是)。\\

徐宏祖不想考试,不想出人头地,不想青史留名,他只想玩。\\

按史籍说,是从小就玩,且玩得比较狠,比较特别,不扔沙包,不滚铁环,只是四处瞎转悠,遇到山就爬,遇到河就下,人极小,胆子极大。\\

此外,他极其讨厌考试,长大后,让他去考科举,死都不去。该情节,放在现在,大致相当于抗拒高考。\\

这号人,当年跟今天的下场,估计是差不多,被拉回家打一半死不活,绝无幸免。\\

然而徐宏祖的父母没有打他,非但没有打他,还告诉他,你要想玩,就玩吧,做自己喜欢做的事情就行。\\

这种看似惊世骇俗的思想,似乎很不合理,但对徐家人而言,很合理。\\

对了,应该介绍一下徐宏祖同志的家世,虽然他的父母,并非什么大人物,也没名气,但他有一位祖先,还算是很有名的,当然,不是好名。\\

在徐宏祖出生前九十年,徐家的一位先辈进京赶考,路上遇到了一位同伴,叫做唐寅,又叫唐伯虎。\\

没错,他就是徐经。\\

后来的事情,之前讲过,据说是徐经作弊,结果拉上了唐伯虎,大家一起完蛋,进士没考上,连举人都没了,所以徐经同志痛定思痛,对坑害了无数人(主要是他)的科举制度深恶痛绝,教育子孙,要与这个万恶的制度决裂,爱考不考,去他娘的。\\

对这段百年恩怨,徐宏祖是否了解,不清楚,但他会用,那是肯定的。更重要的是,徐家虽说没有级别,还有点钱,所以他决定,索性不考了,出去旅游。\\

刚开始,他旅游的范围,主要是江浙一带,比如紫金山、太湖、普陀山等等。后来愈发勇猛,又去了雁荡山、九华山、黄山、武夷山、庐山等等。\\

但这里,存在着一个问题——钱。\\

旅行家和大侠的区别在于,旅行家是要花钱的,列一下,大致包括以下费用:交通费、住宿费、导游费、餐饮费、门票费,如果地方不地道,还有个挨宰费。\\

我说过,徐家是有钱的,但只是有点钱,没有很多钱,大约也就是个中产阶级。按今天的标准,一年去旅游一次,也就够了,但徐宏祖的旅行日程是:一年休息一次。\\

他除了年底回家照顾父母外,一年到头都在外面,但就这么个搞法,他家竟然还过得去。\\

原因很简单,比如交通费,他不坐火车、也不坐汽车(想坐也没),少数骑马,多靠步行(骑马爬山试试)。\\

住宿费,基本不需要,徐宏祖去的地方,当年大都没有人去,别说三星级,连孙二娘的黑店都没有,树林里、悬崖上,打个地铺,也就睡了。\\

餐饮费,也没有,我考察过,徐宏祖同志去的地方,也没什么餐馆,每次他出发的时候,都是带着干粮,而且他很扛饿,据说能扛七八天,至于喝水,山里面,那都是矿泉水。\\

门票费也是不用了,当年谁要能在徐宏祖同志去的地方,设个点收门票,那只能说明,他比徐宏祖还牛,该收。\\

挨宰费是没有的,但挨宰是可能的,且比较敞亮,从没有暗地加价坑钱,都是拿刀,明着来抢。要知道,没门票的地方,固然没有奸商,却很可能有强盗。\\

据本人考证,徐宏祖最大的花销,是导游费用。作为一个旅行家,徐宏祖很清楚,什么都能省,这笔钱是不能省的,否则走到半山腰,给你挖个坑,让你钻个洞,那就休息了。\\

就这样,家境并不十分富裕的徐宏祖,穿着俭朴的衣服,没有随从,没有护卫,带着干粮,独自前往名山大川,风餐露宿,不怕吃苦,不怕挨饿,一年只回一次家,只为攀登。\\

从俗世的角度,徐宏祖是个怪人,这人不考功名,不求做官,不成家立业,按很多人的说法,是毁了。\\

我知道,很多人还会说,这种生活荒谬,是不符合常规的,是不正常的,是缺根弦的,是精神有问题的。\\

我认为,说这些话的人,是吃饱了,撑的,人只活一辈子,如何生活,都是自己的事,自己这辈子浑浑噩噩地没活好,厚着脸皮还来指责别人,有多远,就去滚多远。\\

徐宏祖旅行的唯一阻力,是他的父母。他的父亲去世较早,只剩他的母亲无人照料。圣人曾经教导我们:父母在,不远游。\\

所以在出发前,徐宏祖总是很犹豫,然而他的母亲找到他,对他说了这样一番话:\\

“男儿志在四方,当往天地间一展胸怀!”\\

就这样,徐宏祖开始了他伟大的历程。\\

他二十岁离家,穿着布衣,没有政府支持,没有朋友帮助,独自一人,游历天下二十余年,他去过的地方,包括湖广、四川、辽东、西北,简单地说,全国十三省,全部走遍。\\

他爬过的山,包括泰山、华山、衡山、嵩山、终南山、峨眉山,简单地说,你听过的,他都去过,你没听过的,他也去过。\\

此外,黄河、长江、洞庭湖、鄱阳湖,金沙江、汉江,几乎所有江河湖泊,全部游历。\\

在游历的过程中,他曾三次遭遇强盗,被劫去财物,身负刀伤,还由于走进大山,无法找到出路,数次断粮,几乎饿死。最悬的一次,是在西南。\\

当时,他前往云贵一带,结果走到半路,突然发现交通中断,住处被当地土著围,过了几天,外面又来了明军,又开始围,围了几天,就开始打,打了几天,就开始乱。徐宏祖好歹是见过世面的,跑得快,总算顺利脱身。\\

在旅行的过程中,他还开始记笔记,每天的经历,他都详细记录下来,鉴于他本人除姓名外,还有个号,叫做霞客,所以后来,他的这本笔记,就被称为《徐霞客游记》。\\

崇祯九年(1636),五十岁的徐宏祖决定,再次出游,这也是他的最后一次出游,虽然他自己没有想到。\\

正当他考虑出游方向的时候,一个和尚找到了他。\\

这个和尚的法号,叫做静闻,家住南京,他十分虔诚,非常崇敬鸡足山迦叶寺的菩萨,还曾刺破手指,血写过一本法华经。\\

鸡足山在云南。\\

当时的云南鸡足山,算是蛮荒之地,啥也不通,要去,只能走着去。\\

很明显,静闻是个明白人,他知道自己要一个人去,估计到半路就歇了,必须找一个同伴。\\

徐宏祖的名气,在当时已经很大了,所以他专门找上门来,要跟他一起走。\\

对徐宏祖而言,去哪里,倒是个无所谓的事,就答应了他,两个人一起出发了。\\

他们的路线是这样的,先从南直隶出发,过湖广,到广西,进入四川,最后到达云贵。\\

不用到达云贵,因为到湖广,就出事了。\\

走到湖广湘江(今湖南),没法走了,两人坐船准备渡江。\\

渡到一半,遇上了强盗。\\

对徐宏祖而言,从事这种职业的人,他已经遇到好几次了,但静闻大师,应该是第一次。此后的具体细节不太清楚,反正徐宏祖赶跑了强盗,但静闻在这场风波中受了伤,加上他的体质较弱,刚撑到广西,就圆寂了。\\

徐宏祖停了下来,办理静闻的后事。\\

由于路上遭遇强盗,此时,徐宏祖的路费已经不足了,如果继续往前走,后果难以预料。\\

所以当地人劝他,放弃前进念头,回家。\\

徐宏祖跟静闻,是素不相识的,说到底,也就是个伴,各有各的想法,静闻没打算写游记,徐宏祖也没打算去礼佛,实在没有什么交情。而且我还查过,他此前去过鸡足山,这次旅行对他而言,并没有太大的意义。\\

然而他说,我要继续前进,去鸡足山。\\

当地人问:为什么要去。\\

徐宏祖答:我答应了他,要带他去鸡足山。\\

可是,他已经去世了。\\

我带着他的骨灰去。答应他的事情,我要帮他做到。\\

徐宏祖出发了,为了一个逝去者的愿望,为了实现自己的承诺,虽然这个逝去者,他并不熟悉。\\

旅程很艰苦,没有路费的徐宏祖背着静闻的骨灰,没有任何资助,他只能住在荒野,靠野菜干粮充饥,为了能够继续前行,他还当掉了自己所能当掉的东西,只是为了一个承诺。\\

就这样,他按照原定路线,带着静闻,翻阅了广西十万大山,然后进入四川,越过峨眉山,沿着岷江,到达甘孜松潘。\\

渡过金沙江,渡过澜沧江,经过丽江、经过西双版纳,到达鸡足山。\\

迦叶寺里,他解开了背上的包裹,拿出了静闻的骨灰。\\

到了。\\

我们到了。\\

他郑重地把骨灰埋在了迦叶寺里,在这里,他兑现了承诺。\\

然后,他应该回家了。\\

但他没有。\\

从某个角度讲,这是上天对他的恩赐,因为这将是他的最后一次旅途,能走多远,就走多远吧。\\

他离开鸡足山,又继续前行,行进半年,翻越了昆仑山,又行进半年,进入藏区,游历几个月后,踏上归途。\\

回去没多久,就病了。\\

喜欢锻炼的人,身体应该比较好,天天锻炼的人(比如运动员),就不一定好,旅游也是如此。\\

估计是长年劳累,徐宏祖终究是病倒了,没能再次出行。崇祯十四年(1641),病重逝世,年五十四。\\

他所留下的笔记,据说总共有两百多万字,可惜没有保留下来,剩余的部分,大约几十万字,被后人编成《徐霞客游记》。\\

在这本书里,记载了祖国山川的详细情况,涉及地理、水利、地貌等情况,被誉为十七世纪最伟大的地理学著作,翻译成几十国语言,流传世界。\\

好的,总结应该出来了,这是一个伟大的地理学家的故事,他为了研究地理,四处游历,为地理学的发展做出了突出贡献,是中华民族的骄傲。\\

是这样吗?\\

不是的。\\

其实讲述这人的故事,只想探讨一个问题,他为何要这样做。\\

没有资助,没有承认(至少生前没有),没有利益,没有前途,放弃一切,用一生的时间,只是为了游历?\\

究竟为了什么?\\

我很疑惑,很不解,于是我想起另一个故事。\\

新西兰登山家希拉里,在登上珠穆朗玛峰后,经常被记者问一个问题:\\

你为什么要爬?\\

他总不回答,于是记者总问,终于有一次,他答出了一个让所有人都无法再问的答案:\\

因为它(指珠峰),就在那里!\\

因为它就在那里。\\

其实这个世上很多事,本不需要理由,之所以需要理由,是因为很多人喜欢找抽,抽久了,就需要理由了。\\

正如徐霞客临终前,所说的那句话:\\

“汉代的张骞,唐代的玄奘,元代的耶律楚材,他们都曾游历天下,然而,他们都是接受了皇帝的命令,受命前往四方。”\\

“我只是个平民,没有受命,只是穿着布衣,拿着拐杖,穿着草鞋,凭借自己,游历天下,故虽死,无憾。”\\

说完了。\\

我要讲的那样东西,就在这个故事里。\\

我相信,很多人会问,你讲了什么?\\

用如此之多的篇幅,讲述一个王朝的兴起和衰落,在终结的时候,却说了这样一个故事,你到底想说什么?\\

我重复一遍,我要讲的那样东西,就在这个故事里,已经讲完了。\\

所以后面的话,是讲给那些不明白的人,明白的人,就不用继续看了。\\

此前,我讲过很多东西,很多兴衰起落、很多王侯将相、很多无奈更替、很多风云变幻,但这件东西,我个人认为,是最重要的。\\

因为我要告诉你,所谓千秋霸业,万古流芳,以及一切的一切,只是粪土。先变成粪,再变成土。\\

现在你不明白,将来你会明白,将来不明白,就再等将来,如果一辈子都不明白,也行。\\

而最后讲述的这件东西,它超越上述的一切,至少在我看来。\\

但这件东西,我想了很久,也无法用准确的语言,或是词句来表达,用最欠揍的话说,是只可意会,不可言传。\\

然而我终究是不欠揍的,在遍阅群书,却无从开口之后,我终于从一本不起眼,且无甚价值的读物上,找到了这句适合的话。\\

这是一本台历,一本放在我面前,不知过了多久,却从未翻过,早已过期的台历。\\

我知道,是上天把这本台历放在了我的桌前,它看着几年来我每天的努力,始终的坚持,它静静地,耐心地等待着终结。\\

它等待着,在即将结束的那一天,我将翻开这本陪伴我始终,却始终未曾翻开的台历,在上面,有着最后的答案。\\

我翻开了它,在这本台历上,写着一句连名人是谁都没说明白的名人名言。\\

是的,这就是我想说的,这就是我想通过徐霞客所表达的,足以藐视所有王侯将相,最完美的结束语:\\

成功只有一个——按照自己的方式,去度过人生。\\
\ifnum\theparacolNo=2
	\end{multicols}
\fi
\chapter*{后记}
\addcontentsline{toc}{chapter}{后记}
\ifnum\theparacolNo=2
	\begin{multicols}{\theparacolNo}
\fi
本来没想写,但还是写一个吧,毕竟那么多字都写了。\\

记得前段时间,去央视面对面访谈,主持人问我,书写完的时候,你有什么感觉?\\

其实这个问题,我曾经问过我自己很多次,高兴、兴奋、沮丧,什么都有可能。\\

但当这刻来到的时候,我只感觉,没有感觉。\\

不是矫情。\\

怎么说呢,因为我始终觉得写这玩意,是个小得没法再小的事。然而很快有人告诉我,你的书在畅销排行榜蹲了几天,几月,几年,然后是几十万册、几百万册,直到某天,某位仁兄很是激动地对我说,改革开放三十年,这本书发行量,可以排进前十五名。\\

有意思吗?说实话,有点意思。\\

雷打不动的,还有媒体,报纸、期刊、杂志、电视台,从时尚到社会,从休闲到时局,从中央到地方,从中国到外国,借用某位同志的话,连宠物杂志都上门找你。平均一天几个访问,问的问题,也大致雷同,翻来覆去,总也是那么几个问题,每天都要背几遍,像我这么乏味的人,谁愿意跟我聊,那都是交差,我明白。\\

外型土得掉渣,也硬拽上若干电视讲坛,讲一些相当通俗,相当大众,相当是人就能听明白的所谓历史(类似故事会),当然,该问的还得问下去,该讲的可能还得讲下去。\\

这个没意思,没意思,也得接着混。\\

我始终觉得,我是个很平凡的人,扔人堆里就找不着,放在通缉令上,估计都没人能记住,到现在还这么觉得,今天被人记住了,明天就会被人忘记,今天很多人知道,明天就不知道,所以所谓后记,所谓感想,所谓获奖感言之类的无聊的,乱扯的,自欺欺人的,胡说八道的,都休息吧。\\

那么接下来,说点有必要说的话。\\

首先,是感谢,非常之感谢。\\

记得马未都同志有次对我说,这世上很多人都有不喜欢你的理由。因为你成名太早,成名太盛,太过年轻,人家不喜欢你,那是有道理的,所以无论人家怎么讨厌你,怎么逗你,你都得理解,应该理解。\\

我觉得这句话很有道理,所以一直以来,我都无所谓。\\

但让我感动的是,广大人民群众应该还是喜欢我的,一直以来,我都得到了许多朋友的帮助,没有你们,我撑不到今天,谢谢你们,非常真诚地谢谢你们。\\

谢谢。\\

然后是心得,如果要问我,有个什么成功心得,处世原则,我觉得,只有一点,老实做人,勤奋写书,无它。\\

几年来,我每天都写,没有一天敢于疏忽,不惹事,不闹事,即使所谓盛名之下,我也从未懈怠,有人让我写文章推荐商品,推荐什么就送什么,还有的希望我做点广告,费用可以到六位数,顺手就挣。\\

我没有理会。因为我不是商人。\\

出版商亲自算给我听,由于我坚持把未出版部分免费发表,因此每年带来的版税损失,可以达到七位数,这还不包括盗版,以及各种未经许可的文本。\\

我依然坚持,因为我相信,这是个自由的时代,每个人有看与不看的自由,也有买和不买的自由,任何人都不应该被强迫。\\

这是我的处世原则,我始终坚持,或许很多人认为这么干很吃亏,但结果,相信你已经看到。\\

好的,还有历史,既然写了历史,还要说说对历史的看法。\\

就剩几句了,虚的就算了,来点实在的吧。\\

很多人问,为什么看历史,很多人回答,以史为鉴。\\

现在我来告诉你,以史为鉴,是不可能的。\\

因为我发现,其实历史没有变化,技术变了,衣服变了,饮食变了,这都是外壳,里面什么都没变化,还是几千年前那一套,转来转去,该犯的错误还是要犯,该杀的人还是要杀,岳飞会死,袁崇焕会死,再过一千年,还是会死。\\

所有发生的,是因为它有发生的理由,能超越历史的人,才叫以史为鉴,然而我们终究不能超越,因为我们自己的欲望和弱点。\\

所有的错误,我们都知道,然而终究改不掉。\\

能改的,叫做缺点,不能改的,叫做弱点。\\

顺便说下,能超越历史的人,还是有的,我们管这种人,叫做圣人。\\

以上的话,能看懂的,就看懂了,没看懂的,就当是说疯话。\\

最后,说说我自己的想法。\\

因为看得历史比较多,所以我这个人比较有历史感,当然,这是文明的说法,粗点讲,就是悲观。\\

这并非开玩笑,我本人虽然经常幽默幽默,但对很多事情都很悲观,因为我经常看历史(就好比很多人看电视剧一样),不同的是,我看到的那些古文中,只有悲剧结局,无一例外。\\

每一个人,他的飞黄腾达和他的没落,对他本人而言,是几十年,而对我而言,只有几页,前一页他很牛,后一页就怂了。\\

王朝也是如此。\\

真没意思,没意思透了。\\

但我坚持幽默,是因为我明白,无论这个世界有多绝望,你自己都要充满希望。\\

人生并非如某些人所说,很短暂,事实上,有时候,它很漫长,特别是对苦难中的人,漫长得想死。\\

但我坚持,无论有多绝望,无论有多悲哀,每天早上起来,都要对自己说,这个世界很好,很强大。\\

这句话,不是在满怀希望光明时说的,很绝望、很无助、很痛苦、很迷茫的时候,说这句话。\\

要坚信,你是一个勇敢的人。\\

因为你还活着,活着,就要继续前进。\\

曾经有人问我,你怎么了解那么多你不应该了解的东西,你怎么会有那么多六七十岁的人才有的感受。我说我不知道。跟我一起排话剧的田沁鑫导演说,我是上辈子看了太多书,憋屈死了,这辈子来写。\\

我没话说。\\

还会不会写?应该会,感觉还能写,还写得出来,毕竟还很年轻,离退休尚早,尚能饭。\\

继续写之前,先歇歇,累得慌。\\

是的,这个世界还是很有趣的。\\

最后送一首食指的诗给大家,我所要跟大家讲的,大致就在其中了吧。\\

\begin{quote}
	\begin{spacing}{0.5}  %行間距倍率
		\textit{{\footnotesize
				\begin{description}
					\item[\textcolor{Gray}{\FA }] 当蜘蛛网无情地查封了我的炉台
					\item[\textcolor{Gray}{\FA }] 当灰烬的余烟叹息着贫困的悲哀
					\item[\textcolor{Gray}{\FA }] 我依然固执地铺平失望的灰烬
					\item[\textcolor{Gray}{\FA }] 用美丽的雪花写下:相信未来
					\item[\textcolor{Gray}{\FA }] 当我的紫葡萄化为深秋的露水
					\item[\textcolor{Gray}{\FA }] 当我的鲜花依偎在别人的情怀
					\item[\textcolor{Gray}{\FA }] 我依然固执地用凝霜的枯藤
					\item[\textcolor{Gray}{\FA }] 在凄凉的大地上写下:相信未来
					\item[\textcolor{Gray}{\FA }] 我要用手指那涌向天边的排浪
					\item[\textcolor{Gray}{\FA }] 我要用手掌那托住太阳的大海
					\item[\textcolor{Gray}{\FA }] 摇曳着曙光那枝温暖漂亮的笔杆
					\item[\textcolor{Gray}{\FA }] 用孩子的笔体写下:相信未来
					\item[\textcolor{Gray}{\FA }] 我之所以坚定地相信未来
					\item[\textcolor{Gray}{\FA }] 是我相信未来人们的眼睛
					\item[\textcolor{Gray}{\FA }] 她有拨开历史风尘的睫毛
					\item[\textcolor{Gray}{\FA }] 她有看透岁月篇章的瞳孔
					\item[\textcolor{Gray}{\FA }] 不管人们对于我们腐烂的皮肉
					\item[\textcolor{Gray}{\FA }] 那些迷途的惆怅、失败的苦痛
					\item[\textcolor{Gray}{\FA }] 是寄予感动的热泪、深切的同情
					\item[\textcolor{Gray}{\FA }] 还是给以轻蔑的微笑、辛辣的嘲讽
					\item[\textcolor{Gray}{\FA }] 我坚信人们对于我们的脊骨
					\item[\textcolor{Gray}{\FA }] 那无数次的探索、迷途、失败和成功
					\item[\textcolor{Gray}{\FA }] 一定会给予热情、客观、公正的评定
					\item[\textcolor{Gray}{\FA }] 是的,我焦急地等待着他们的评定
					\item[\textcolor{Gray}{\FA }] 朋友,坚定地相信未来吧
					\item[\textcolor{Gray}{\FA }] 相信不屈不挠的努力
					\item[\textcolor{Gray}{\FA }] 相信战胜死亡的年轻
					\item[\textcolor{Gray}{\FA }] 相信未来、热爱生命
				\end{description}
		}}
	\end{spacing}
\end{quote}

二十多岁写,写完还是二十多岁,有趣。\\

是的,这个世界还是很有趣的。\\

无需害怕。\\

无需绝望。\\

要相信自己。\\
\ifnum\theparacolNo=2
	\end{multicols}
\fi
